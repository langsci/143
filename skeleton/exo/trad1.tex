% -*- coding: utf-8 -*-
\begin{exo}\label{e:versionLO}
Traduisez dans {\LO} les phrases françaises ci-dessous.  
\pagesolution{crg:versionLO}
Vous choisirez
les noms de prédicats et de constantes comme cela vous arrange, mais
vous indiquerez à chaque fois ce qu'ils traduisent du français.  Si
une phrase contient une présupposition, ne traduisez que 
son  contenu asserté (\ie\ non présupposé).  On ne tiendra pas compte
de valeur sémantique des temps verbaux (on les néglige pour l'instant).
\begin{enumerate}
\item Antoine n'est plus barbu.
\item Tout est sucré ou salé.
\item Soit tout est sucré, soit tout est salé.
\item Le chien qui aboie ne mord pas. (proverbe)
\item C'est Marie que Jérôme a embrassée.
\item Il y a des hommes et des femmes qui ne sont pas unijambistes.
\item Tout le monde aime quelqu'un.
\item Si tous les homards sont gauchers alors Alfred aussi est
  gaucher. 
\item Quelqu'un a envoyé une lettre anonyme à Anne.
\item Seule Chloé est réveillée.
\end{enumerate}
\begin{solu} (p.~\pageref{e:versionLO})\label{crg:versionLO}
%Traduction Fr $\leadsto$ \LO.

\begin{enumerate}
\item Antoine n'est plus barbu.\\ $\leadsto$
\(\Xlo\neg\prd{barbu}(\cns{a})\)
\item Tout est sucré ou salé.\\ $\leadsto$
\(\Xlo\forall x [\prd{sucré}(x) \vee \prd{salé}(x)]\)
\item Soit tout est sucré, soit tout est salé.\\ $\leadsto$
\(\Xlo [\forall x\, \prd{sucré}(x) \vee \forall x\, \prd{salé}(x)]\)
\item Le chien qui aboie ne mord pas. (proverbe) \\ $\leadsto$
\(\Xlo\forall x [[\prd{chien}(x) \wedge \prd{aboyer}(x)] \implq
  \neg\prd{mordre}(x)]\) \\ou
\(\Xlo\forall x [\prd{chien}(x) \implq [\prd{aboyer}(x) \implq
  \neg\prd{mordre}(x)]]\)
\item C'est Marie que Jérôme a embrassée. \\$\leadsto$
\(\Xlo\prd{embrasser}(\cns{j},\cns{m})\)
\item Il y a des hommes et des femmes qui ne sont pas
  unijambistes. \\$\leadsto$ 
\(\Xlo\exists x \exists y [[\prd{homme}(x) \wedge \prd{femme}(y)] \wedge
  [\neg\prd{unij}(x) \wedge \neg\prd{unij}(y)]]\)
\item Tout le monde aime quelqu'un. 

Cette phrase est ambiguë.  Elle peut signifier que pour chaque personne, il y a une personne que la première aime (et donc possiblement autant de personnes aimées que de personnes aimantes), cela correspond à la traduction (a) ci-dessous ; mais elle peut signifier aussi qu'il existe une personne aimée de tout le monde, ce qui correspond à la traduction (b). 
  \begin{enumerate}
    \item $\leadsto$
      \(\Xlo\forall x \exists y\, \prd{aimer}(x,y)\)\\ou
      \(\Xlo\forall x [\prd{hum}(x) \implq \exists y [\prd{hum}(y) \wedge
    \prd{aimer}(x,y)]]\)
    \item $\leadsto$
       \(\Xlo\exists y\forall x\, \prd{aimer}(x,y)\)\\ou
      \(\Xlo\exists y [\prd{hum}(y) \wedge\forall x [\prd{hum}(x) \implq 
    \prd{aimer}(x,y)]]\)
  \end{enumerate}
\item Si tous les homards sont gauchers alors Alfred aussi est
  gaucher.\\ $\leadsto$
\(\Xlo\forall x [\prd{homard}(x) \implq \prd{gaucher}(x)] \implq \prd{gaucher}(\cns{a})\)
\item Quelqu'un a envoyé une lettre anonyme à Anne. \\$\leadsto$
\(\Xlo\exists x [\prd{hum}(x) \wedge \exists y [[\prd{lettre}(y) \wedge
      \prd{anon}(y)] \wedge \prd{envoyer}(x,y,\cns{a})]]\)
\item Seule Chloé est réveillée.\\$\leadsto$
\(\Xlo\forall x [\prd{réveillé}(x) \ssi x=\cns{c}]\)\\
mais on peut aussi proposer \(\Xlo\forall x [\prd{réveillé}(x) \implq
  x=\cns{c}]\), qui n'a pas le même sens, et qui là exclut le présupposé.
\end{enumerate}
\end{solu}
\end{exo}

\begin{exo}\label{e:version2LO}
Même exercice. \pagesolution{crg:version2LO}
\begin{enumerate}
\item Il existe des éléphants roses.
\item Quelque chose me gratouille et me chatouille.
\item Quelque chose me gratouille et quelque chose me chatouille.
\item Nimes est entre Avignon et Montpellier.
\item S'il y a des perroquets ventriloques, alors Jacko en est un.
\item Anne a reçu une lettre de Jean, mais elle n'a rien reçu de
  Pierre.
\item Tout fermier qui possède un âne est riche.
\item Il y a quelqu'un qui a acheté une batterie et qui est en train
  d'en jouer.
\item Il y a un seul océan.
\item Personne n'aime personne.
\end{enumerate}
\begin{solu} (p.~\pageref{e:version2LO}) %Traductions.
\label{crg:version2LO}

\begin{enumerate}
\item Il existe des éléphants roses.\\$\leadsto$
\(\Xlo\exists x [\prd{éléphant}(x) \wedge \prd{rose}(x)]\)
\item Quelque chose me gratouille et me chatouille.\\$\leadsto$
\(\Xlo\exists x [\prd{gratouiller}(x,\cns l) \wedge
  \prd{chatouiller}(x,\cns l)]\)  (avec $\cns l$ pour le locuteur)
\item Quelque chose me gratouille et quelque chose me
  chatouille.\\$\leadsto$
\(\Xlo[\exists x\,\prd{gratouiller}(x,\cns l) \wedge \exists x\,
  \prd{chatouiller}(x,\cns l)]\) 
\item Nimes est entre Avignon et Montpellier.\\$\leadsto$
\(\Xlo\prd{être-entre}(\cns n,\cns a, \cns m)\)
\item S'il y a des perroquets ventriloques, alors Jacko en est
  un.\\$\leadsto$
\(\Xlo[\exists x [\prd{perroquet}(x) \wedge \prd{ventriloque}(x)] \implq
  [\prd{perroquet}(\cns j) \wedge \prd{ventriloque}(\cns j)]]\)
\item Anne a reçu une lettre de Jean, mais elle n'a rien reçu de
  Pierre. \\$\leadsto$
\(\Xlo\exists x [\prd{lettre}(x) \wedge \prd{recevoir}(\cns{a},x,\cns{j})]
  \wedge \neg\exists y \, \prd{recevoir}(\cns{a},y,\cns{p})\)
\item Tout fermier qui possède un âne est riche.\\$\leadsto$
\(\Xlo\forall x [[\prd{fermier}(x) \wedge \exists y [\prd{âne}(y) \wedge
    \prd{posséder}(x,y)] ] \implq \prd{riche}(x)]\)
\item Il y a quelqu'un qui a acheté une batterie et qui est en train
  d'en jouer.\\$\leadsto$
\(\Xlo\exists x [\prd{humain}(x) \wedge \exists y [\prd{batterie}(y)
  \wedge \prd{acheter}(x,y) \wedge \prd{jouer}(x,y)]]\)
\item Il y a un seul océan.\\$\leadsto$
\(\Xlo\exists x [\prd{océan}(x) \wedge \forall y [\prd{océan}(y) \implq y=x]]\)
\item Personne n'aime personne.
  \begin{enumerate}
  \item $\leadsto$
    \(\Xlo\neg \exists x [\prd{humain}(x) \wedge \exists y [\prd{humain}(y)
    \wedge \prd{aimer}(x,y)]]\)\\
    ou
    \(\Xlo\forall x [\prd{humain}(x) \implq \forall y [\prd{humain}(y) \implq
    \neg\prd{aimer}(x,y)]]\)
    \\ou
    \(\Xlo\forall x  \forall y [[\prd{humain}(x) \wedge \prd{humain}(y)] \implq
    \neg\prd{aimer}(x,y)]\)
  \item $\leadsto$
    \(\Xlo\neg\exists x [\prd{humain}(x) \wedge \forall y [\prd{humain}(y) \implq \neg\prd{aimer}(x,y)]]\)\\
ou
    \(\Xlo\forall x [\prd{humain}(x) \implq \neg\forall y [\prd{humain}(y) \implq \neg\prd{aimer}(x,y)]]\)
  \end{enumerate}
Il est important de noter que cette dernière phrase est ambiguë (d'où les deux séries de traductions).  Dans une première interprétation, elle signifie que pour chaque individu du modèle, celui-ci n'aime personne (autrement dit, il n'y a pas d'amour dans le modèle) ; c'est ce que donnent les traductions (a).  Dans la seconde interprétation, la phrase signifie qu'il est faux qu'il y a des gens qui n'aiment personne (autrement dit, tout le monde aime au moins une personne) ; elle est peut-être un peu moins spontanée, mais elle apparaît naturellement dans le dialogue \sicut{--- Albert n'aime personne. --- Mais non voyons, \textsc{personne} n'aime personne} (facilitée par l'accent intonatif sur le premier \sicut{personne}) ; c'est ce que donnent les traductions (b).
\end{enumerate}
\end{solu}
\end{exo}
