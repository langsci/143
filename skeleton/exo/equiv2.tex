% -*- coding: utf-8 -*-
\begin{exo}\label{exo:equivlog}
Montrez que les paires de formules suivantes sont chacune logiquement
\pagesolution{crg:equivlog}%
équivalentes : 
\begin{enumerate}
\item \(\Xlo\phi\) et \(\Xlo\neg\neg\phi\)
\label{Nex:2neg}
\item \(\Xlo[\phi \implq \psi]\) et \(\Xlo[\neg\phi \vee \psi]\) 
\label{Nex:implq}
\item \(\Xlo[\phi \implq \psi]\) et \(\Xlo[\neg\psi \implq \neg\phi]\) 
\label{Nex:contraposition}
\item \(\Xlo[\phi \implq [\psi \implq \chi]]\) et \(\Xlo[[\phi \wedge \psi]
  \implq \chi]\)
\label{Nex:2implq}
\item \(\Xlo[\phi \wedge [\psi \vee \chi]]\) et \(\Xlo[[\phi \wedge \psi] \vee
  [\phi \wedge \chi]]\)
 \item \(\Xlo[\phi \vee [\psi \wedge \chi]]\) et \(\Xlo[[\phi \vee \psi] \wedge
  [\phi \vee \chi]]\)
\end{enumerate}
Il n'est pas inutile de connaître par c\oe ur ces équivalences (la \numero
\ref{Nex:2neg} s'appelle d'ailleurs la \emph{loi de la double
  négation}\index[sbj]{loi!\elid\ de la double négation} 
et la \numero \ref{Nex:contraposition}  la \emph{loi de
  contraposition}%
\footnote{Un exemple d'équivalence via la double négation est
  donnée par \sicut{Marie fume} et \sicut{il est faux que
  Marie ne fume pas}.  La loi de contraposition peut s'illustrer
  par \sicut{s'il y a du feu, il y a de la fumée} et
  \sicut{s'il n'y a pas de fumée, il n'y a pas de feu}.
}%
).\index[sbj]{loi!\elid\ de contraposition} 
%
\begin{solu} (p.~\pageref{exo:equivlog}) %Équivalences logiques.
\label{crg:equivlog}

Par défaut, les équivalences se démontrent par la méthode des tables de vérité comme dans l'exercice précédent, mais par endroits il est possible de procéder autrement.
\begin{enumerate}
\item \(\Xlo\phi\) et \(\Xlo\neg\neg\phi\)
\small\[
\begin{array}{>{\columncolor[gray]{.9}}c||c|>{\columncolor[gray]{.9}}c}
\cellcolor{white}\Xlo\phi & \Xlo\neg\phi & \cellcolor{white}\Xlo\neg\neg\phi\\\hline\hline
1&0&1\\
0&1&0\\
\end{array}
\]\normalsize
En fait, c'est presque laborieux de faire la table de vérité, car la démonstration est triviale et immédiate : $\Xlo\neg$ inverse les valeurs de vérité, donc si on inverse deux fois (ou un nombre paire de fois), on retombe sur la valeur initiale (comme retourner deux fois une pièce de monnaie).

\item \(\Xlo[\phi \implq \psi]\) et \(\Xlo[\neg\phi \vee \psi]\)
\small\[
\begin{array}{c|c||>{\columncolor[gray]{.9}}c||c|>{\columncolor[gray]{.9}}c}
\Xlo\phi & \Xlo\psi & \cellcolor{white}\Xlo[\phi\implq\psi] & \Xlo\neg\phi& \cellcolor{white}\Xlo[\neg\phi\vee\phi]
\\\hline\hline
1 & 1 & 1 & 0 & 1\\
1 & 0 & 0 & 0 & 0\\
0 & 1 & 1 & 1 & 1\\
0 & 0 & 1 & 1 & 1\\ 
\end{array}
\]\normalsize

\item \(\Xlo[\phi \implq \psi]\) et \(\Xlo[\neg\psi \implq \neg\phi]\) 

Ici on pourrait assez rapidement et facilement faire la table de
vérité, mais on peut aussi démontrer l'équivalence très simplement,
en raisonnant.  On sait, par la démonstration précédente, que $\Xlo[\phi
  \implq \psi]$ équivaut à $\Xlo[\neg\phi \vee \psi]$. Pour la même
raison, on sait que $\Xlo[\neg\psi \implq \neg\phi]$ équivaut à  
$\Xlo[\neg\neg\psi \vee \neg\phi]$, ce qui équivaut à
$\Xlo[\psi \vee \neg\phi]$ en vertu de la loi de double négation démontrée
ci-dessus en 1.  Et comme la disjonction est {commutative}, 
% Attention : commutatif pas défini dans le chapitre
cela équivaut à $\Xlo[\neg\phi \vee \psi]$, et donc à $\Xlo[\phi
  \implq \psi]$.

\item \(\Xlo[\phi \implq [\psi \implq \chi]]\) et \(\Xlo[[\phi \wedge \psi]
  \implq \chi]\)

Là aussi on peut démontrer l'équivalence en utilisant celle démontrée
ci-dessus en 2, ainsi que d'autres démontrées précédemment. Par
l'équivalence 2, on sait que
$\Xlo[\phi \implq [\psi \implq \chi]]$ équivaut à
$\Xlo[\neg\phi \vee [\psi \implq \chi]]$, qui équivaut à
$\Xlo[\neg\phi \vee [\neg\psi \vee \chi]]$.  Or on a démontré que $\Xlo\vee$
est commutative, donc la formule équivaut aussi à 
$\Xlo[[\neg\phi \vee \neg\psi] \vee \chi]$.  Et par une des lois de
Morgan, on sait aussi que $\Xlo[\neg\phi \vee \neg\psi]$ équivaut à
$\Xlo\neg[\phi \wedge \psi]$.  Donc, par remplacement, notre formule
équivaut à $\Xlo[\neg[\phi \wedge \psi] \vee \chi]$. Or cette formule
répond au schéma $\Xlo[\neg X\vee Y]$ de l'équivalence 2. Donc on en
conclut que notre formule équivaut à \(\Xlo[[\phi \wedge \psi]
  \implq \chi]\).

\item \(\Xlo[\phi \wedge [\psi \vee \chi]]\) et \(\Xlo[[\phi \wedge \psi] \vee
  [\phi \wedge \chi]]\)
\small\[
\begin{array}{c|c|c||c|>{\columncolor[gray]{.9}}c||c|c||>{\columncolor[gray]{.9}}c}
\Xlo\phi & \Xlo\psi & \Xlo\chi & \Xlo[\psi \vee \chi] & \cellcolor{white}\Xlo[\phi \wedge [\psi \vee
  \chi]] & \Xlo[\phi \wedge \psi] & \Xlo[\phi \wedge \chi] & \cellcolor{white}\Xlo[[\phi \wedge\psi]\vee[\phi\wedge\chi]]
\\\hline\hline
1 & 1 & 1 & 1 & 1 & 1 & 1 & 1\\
1 & 1 & 0 & 1 & 1 & 1 & 0 & 1\\
1 & 0 & 1 & 1 & 1 & 0 & 1 & 1\\
1 & 0 & 0 & 0 & 0 & 0 & 0 & 0\\
0 & 1 & 1 & 1 & 0 & 0 & 0 & 0\\
0 & 1 & 0 & 1 & 0 & 0 & 0 & 0\\
0 & 0 & 1 & 1 & 0 & 0 & 0 & 0\\
0 & 0 & 0 & 0 & 0 & 0 & 0 & 0\\
\end{array}
\]\normalsize

\item \(\Xlo[\phi \vee [\psi \wedge \chi]]\) et \(\Xlo[[\phi \vee \psi] \wedge
  [\phi \vee \chi]]\)
\small\[
\begin{array}{c|c|c||c|>{\columncolor[gray]{.9}}c||c|c||>{\columncolor[gray]{.9}}c}
\Xlo\phi & \Xlo\psi & \Xlo\chi & \Xlo[\psi \wedge \chi] & \cellcolor{white}\Xlo[\phi \vee
  [\psi \wedge \chi]] & \Xlo[\phi \vee \psi] & \Xlo[\phi \vee \chi] & \cellcolor{white}\Xlo[[\phi \vee\psi]\wedge[\phi\vee\chi]]
\\\hline\hline
1 & 1 & 1 & 1 & 1 & 1 & 1 & 1\\
1 & 1 & 0 & 0 & 1 & 1 & 1 & 1\\
1 & 0 & 1 & 0 & 1 & 1 & 1 & 1\\
1 & 0 & 0 & 0 & 1 & 1 & 1 & 1\\
0 & 1 & 1 & 1 & 1 & 1 & 1 & 1\\
0 & 1 & 0 & 0 & 0 & 1 & 0 & 0\\
0 & 0 & 1 & 0 & 0 & 0 & 1 & 0\\
0 & 0 & 0 & 0 & 0 & 0 & 0 & 0\\
\end{array}
\]\normalsize
\end{enumerate}
\end{solu}
\end{exo}
