% -*- coding: utf-8 -*-
%%\chapter{Temps et mondes -- l'intensionnalité}
\chapter{Sémantique intensionnelle}
%%%%%%%%%%%%%%%%%%%%%%%%%%%%%%%%%%%
\label{Ch:t+m}
\Writetofile{solf}{\protect\section{Chapitre \protect\ref{Ch:t+m}}}



Dans ce chapitre nous allons mettre en \oe uvre  une première amélioration de notre système sémantique --~étape importante pour faire évoluer notre langage {\LO} vers ce qui, à terme, se rapprochera suffisamment du formalisme proposé par \citet[entre autres]{PTQ} et qui reste la matrice théorique de la sémantique formelle contemporaine.\Andex{Montague, R.}
La logique du calcul des prédicats, sur laquelle se fonde {\LO}, est insuffisante pour rendre compte de certaines propriétés sémantiques des langues naturelles, et l'un des développements cruciaux de la sémantique formelle dans les années 1960 (et notamment de la Grammaire de Montague)\is{grammaire!\elid\ de Montague} a été d'adopter une logique \emph{intensionnelle}, plus performante, issue des logiques modales déjà existantes. 
La première partie du chapitre va présenter les insuffisances de {\LO} dans son état actuel et va introduire la notion clé d'intensionnalité.  Ensuite nous verrons comment étendre formellement {\LO} pour le rendre plus adéquat et quelles sont les applications qui peuvent découler de ce perfectionnement. Nous verrons notamment que cela nous permettra de formaliser plus précisément la notion de sens. 



\section{Les limites de la sémantique extensionnelle}
%====================================================
\label{s:limitesExt}

Commençons par nous remettre en mémoire le principe d'extensionnalité\is{extensionnalité!principe d'\elid}\is{extensionnalité} que nous avions rencontré dans le chapitre~\ref{LCP} en
\S\ref{leibniz} p.~\pageref{leibniz}%
\footnote{Ce principe est également appelé «principe de substitution» ou «principe de Leibniz». La formulation originale de Leibniz, citée par
  \citet{Frege:SuB}, est : \alien{Eadem sunt qui substitui possunt
    salva veritate}, c'est-à-dire «sont identiques ceux que l'on
  peut substituer (l'un à l'autre) en préservant la vérité».\index{Leibniz, Gottfried W.}}.
Le voici dans une formulation plus précise :

\begin{princ}[Principe d'extensionnalité]\label{pr:ext}\is{extensionnalité!principe d'\elid}
Soit {\Modele} un modèle. % et $g$ une assignation.  
Si $\vrb\alpha$ est une
expression de {\LO}, $\vrb\beta$ une sous-expression de $\vrb\alpha$ et
$\vrb\gamma$ une expression telle que, pour toute assignation $g$,
$\denote{\vrb\beta}^{\Modele,g}=\denote{\vrb\gamma}^{\Modele,g}$ et
si $\vrb{\alpha'}$ est l'expression \vrb\alpha\ dans laquelle on a remplacé \vrb\beta\ par \vrb\gamma, alors $\denote{\vrb\alpha}^{\Modele,g}=\denote{\vrb{\alpha'}}^{\Modele,g}$ (pour toute assignation $g$).
\end{princ}


En bref, ce principe dit que tant que l'on ne change pas la dénotation d'une partie d'une expression, alors la dénotation de l'expression complète ne change pas. Le terme d'extensionnalité vient d'\emph{extension}\is{extension}, qui est un synonyme de \emph{dénotation}.  

Une manière de faire apparaître l'application de ce principe est de poser un raisonnement de la forme \ref{xPX1} où une des prémisses, \ref{xPX1b}, expose l'identité de dénotation entre deux sous-expressions et où la conclusion, \ref{xPX1c}, réalise la substitution de ces deux sous-expressions :

\ex. \label{xPX1}
\a. Aristote a écrit \emph{L'éthique à Nicomaque}. \label{xPX1a}
\b. Aristote était le précepteur d'Alexandre le Grand. \label{xPX1b}
\c. \juge{\small$\satisf\:$} Le précepteur d'Alexandre le Grand a écrit \emph{L'éthique à Nicomaque}. \label{xPX1c}

C'est bien en vertu du principe d'extensionnalité que \ref{xPX1c} est une conséquence logique de \ref{xPX1a} et \ref{xPX1b}. Et bien sûr, cette inférence se retrouve immédiatement dans la traduction en {\LO} du raisonnement : 


\ex. \label{xPX2}
\a.
\(\Xlo\prd{écrire}(\cnsi a0,\cns e)\) \label{xPX2a}
\b.
\(\Xlo\cnsi a0 = \atoi x\,\prd{précepteur}(x,\cnsi a1)\) \label{xPX2b}
\c. \juge{\small$\satisf\:$}
\(\Xlo\prd{écrire}(\atoi x\,\prd{précepteur}(x,\cnsi a1),\cns e)\) \label{xPX2c}

Quel que soit le modèle dans lequel nous nous plaçons, si \ref{xPX2b} est satisfaite dans ce modèle, alors, dans ce même modèle, le calcul de la dénotation de \ref{xPX2a} et \ref{xPX2c} donnera le même résultat.

Une sémantique qui respecte (toujours) ce principe est dite extensionnelle et un
langage est dit extensionnel si sa sémantique l'est.  Le langage {\LO}
que nous avons examiné jusqu'ici est extensionnel.  En revanche, nous
allons voir  qu'une langue comme le français ne respecte pas le
principe d'extensionnalité.  Les deux sous-sections qui suivent sont consacrées à la présentation de phénomènes sémantiques du français qui montrent les limites de {\LO} dans son état actuel. Le premier phénomène (\S\ref{ss:re/dicto}) illustre un échec clair et net du principe d'extensionnalité. Le second (\S\ref{ss:ModNonExt}) fait apparaître des limites de l'expressivité de {\LO} qui sont dues en partie à son caractère extensionnel.
%\fixme{***}
Ensuite les sections \S\ref{s:Prior}, \S\ref{s:mondes} et \S\ref{s:intension} montreront comment améliorer formellement {\LO} en le débarrassant de l'extensionnalité, et ce faisant nous le rendrons \kwo{intensionnel}.



\subsection{Lectures \textit{de re {\vs}\ de dicto}}
%-------------------------------------------------
\label{ss:re/dicto}


Nous allons examiner ici un nouveau type d'ambiguïtés qui a la particularité de mettre dramatiquement en échec le principe d'extensionnalité. Mais auparavant, faisons un rappel sur la manière dont les ambiguïtés sont traitées dans notre système sémantique.\is{ambiguïté}
Le principe d'extensionnalité  dans {\LO} est directement lié au fait que pour n'importe quel modèle $\Modele$ et n'importe quelle assignation $g$, toute expression \vrb\alpha\ de {\LO} a \emph{une et une seule} dénotation\footnote{Sauf dans les cas où $\vrb\alpha$ contient une présupposition qui est fausse par rapport à $\Modele$ et $g$, auquel cas \vrb\alpha\ n'a aucune dénotation. Mais ce n'est pas un problème ici ; ce qui importe c'est qu'\vrb\alpha\ n'aura jamais plus d'une dénotation.}.  Autrement dit, le calcul de la dénotation, que nous représentons par la notation \(\denote{\;}^{\Modele,g}\), est une fonction.  Cela est dû à l'univocité de {\LO} ; chaque expression du langage a un seul sens, et comme le sens est ce qui détermine la dénotation, on obtient une seule dénotation pour un modèle et une assignation donnés. 
Dans les langues naturelles, c'est un peu différent. Si, par exemple, $E$ est une expression ambiguë du français, cela veut dire qu'il existe au moins un modèle \Modele\ tel que \(\denote{E}^{\Modele,g}\) aura deux valeurs distinctes.  C'est d'ailleurs ainsi qu'au chapitre~\ref{Ch:1} (\S\ref{s:Ambiguïté}, p.~\pageref{d:ambig}), on avait établi la méthode qui démontre les ambiguïtés sémantiques.
Mais cette multiplicité de dénotations pour $E$ n'est normalement pas un problème pour notre système, car si $E$ est ambiguë, c'est qu'elle a (au moins) deux sens différents, qui correspondent donc à \emph{deux traductions} différentes en {\LO}, et chacune de ces traductions n'aura qu'une dénotation par rapport à ce modèle \Modele.
En règle générale, les ambiguïtés de la langue ne contrarient pas le principe d'extensionnalité ; mais avec celles que nous allons examiner à présent, les choses se compliquent un peu.

L'ambiguïté qui nous intéresse ici est typiquement illustrée par l'exemple suivant\footnote{La paternité de cet exemple, très récurrent dans la littérature philosophique et sémantique, revient à \citet{Linsky:67}.} : %\ref{x:Oedipe} :

\ex. \label{x:Oedipe}
{\OE}dipe$_1$ voulait épouser sa$_1$ mère.


Une première remarque s'impose : je prends soin de décorer la phrase d'indices référentiels «${}_1$»\is{indice!\elid\ referentiel@\elid\ référentiel} afin d'évacuer immédiatement une certaine ambiguïté, celle où le {\GN} \sicut{sa mère} pourrait valoir soit pour \sicut{la mère d'\OE dipe} soit pour \sicut{la mère de quelqu'un d'autre}. Les {\GN} possessifs comportent un composant «pronominal» qui, à ce titre, induit une ambiguïté selon comment se résout son antécédent. C'est entendu. Mais ce n'est pas du tout ce qui va nous occuper ici. Ici il s'agit de regarder le {\GN} \sicut{sa mère} comme correspondant toujours à \sicut{la mère d'\OE dipe}, ce que contraint l'indice «${}_1$». Et même avec cette consigne, \ref{x:Oedipe} est encore ambiguë.

Une première lecture que l'on peut attribuer à \ref{x:Oedipe} (peut-être la plus spontanée) est celle qui décrit une situation franchement et délibérément incestueuse, dans laquelle \OE dipe pense que la femme qu'il souhaite épouser est sa mère. Ou, si l'on veut, une situation dans laquelle \OE dipe se dirait : «moi, je veux me marier avec ma maman».
La seconde lecture est celle qui est plus conforme à la légende d'\OE dipe de la mythologie grecque (contrairement à la précédente). Elle décrit une situation où, par exemple, \OE dipe veut épouser Jocaste, et Jocaste se trouve être la mère d'\OE dipe mais \OE dipe ne le sait pas (ni Jocaste d'ailleurs). 

\label{p.re/dicto}%
Cette seconde interprétation est caractérisée en disant que le {\GN} \sicut{sa mère} y a une lecture \kwo{de re}\is{de re@\alien{de re}}. \alien{De re} signifie en latin\footnote{En philosophie, on fait traditionnellement remonter l'opposition et la terminologie \alien{de re} \vs\ \alien{de dicto} à Thomas d'Aquin.\index{Thomas d'Aquin}} «au sujet de la chose» ; et il faut comprendre cette lecture comme étant celle par laquelle \sicut{sa mère} dénote l'individu qui est \emph{réellement} ou en \emph{réalité}\footnote{\emph{Ré}ellement et \emph{ré}alité viennent de la racine \alien{res} que l'on retrouve dans \alien{de re}.} la mère d'\OE dipe.  Pour la première interprétation, on dit que \sicut{sa mère} a une lecture \kwo{de dicto}\is{de dicto@\alien{de dicto}}, qui signifie «au sujet du \alien{dictum}~» c'est-à-dire de ce qui est dit.  Autrement dit, la lecture \alien{de dicto} ne renvoie pas à la réalité mais au contenu propositionnel de la subordonnée infinitive \sicut{épouser sa mère}. Pour dire les choses plus simplement, sous la lecture \alien{de dicto}, \sicut{sa mère} dénote l'individu qu'\OE dipe croit être sa mère.

Cette ambiguïté peut sembler un peu artificielle, voire tirée par les cheveux, mais elle est bien réelle. Et cela vaut la peine de s'en assurer méthodiquement.  Pour ce faire, construisons donc un modèle $\Modele_1$ par rapport auquel nous pourrons juger \ref{x:Oedipe} à la fois vraie et fausse.  
Il est inutile de le décrire formellement (d'ailleurs nous ne saurions pas le faire, pour le moment), il suffit d'indiquer les informations importantes qui caractérisent l'état du monde qui nous intéresse. 
Voici un exemple de modèle qui fait l'affaire :

\ex. $\Modele_1$ contient les individus et les faits suivants :
\a. \Obj{\OE dipe}, un homme ;
\b. \Obj{Jocaste}, une femme, la mère d'\Obj{\OE dipe} ;
\b. \Obj{Hélène}, une autre femme (qui n'est donc pas la mère d'\Obj{\OE dipe}) ;
\b. \Obj{\OE dipe} aime \Obj{Hélène} et souhaite l'épouser ;
\b. \Obj{\OE dipe} croit qu'\Obj{Hélène} est sa mère ;
\b. \Obj{\OE dipe} ne sait pas que \Obj{Jocaste} est sa mère ;
\b. \Obj{\OE dipe} ne souhaite pas épouser \Obj{Jocaste}.

Par rapport à ce modèle $\Modele_1$, on peut dire que \ref{x:Oedipe} est vraie, car voulant épouser \Obj{Hélène} en croyant que c'est sa mère, ce que veut \Obj{\OE dipe} c'est bien : \sicut{épouser sa mère}. C'est le cas où on interprète \sicut{sa mère} avec la lecture \alien{de dicto}.  Mais on peut également juger \ref{x:Oedipe} fausse, car \Obj{Jocaste} est la mère d'\Obj{\OE dipe} et il ne veut pas l'épouser.  Cette fois, c'est le cas où \sicut{sa mère} a la lecture \alien{de re}. 

Par cette démonstration, nous voyons que l'ambiguïté de \ref{x:Oedipe} est bien  localisée au niveau du \GN\ \sicut{sa mère}. 
C'est d'ailleurs ce qui apparaît via les gloses suivantes, qui permettent d'expliciter précisément ce qui distingue les deux lectures :

\ex.
\a. \OE dipe voulait épouser \emph{celle que le locuteur «sait»\footnote{Je mets des guillemets ici, car le locuteur lui-même peut se tromper sur l'identité de la mère d'\OE dipe. Mais cela n'a pas d'importance, car en interprétant la phrase, nous nous interrogeons sur ses conditions de vérité, c'est-à-dire que nous examinons l'hypothèse où elle est vraie.} être la
  mère d'\OE dipe}.\hfill(\alien{de~re})
\b. \OE dipe voulait épouser \emph{celle qu'\OE dipe pensait être sa mère}.
\hfill(\alien{de dicto})


Et lorsque nous évaluons \ref{x:Oedipe} par rapport à $\Modele_1$, nous trouvons que \sicut{sa mère} dénote tantôt \Obj{Jocaste} (par la lecture \alien{de re}) tantôt \Obj{Hélène} (par la lecture  \alien{de dicto}).  Le {\GN} a deux dénotations distinctes pour un même modèle, il est donc ambigu. 

Mais... attendons une seconde. Si le {\GN} \sicut{sa mère} est ambigu, cela voudrait qu'il a (au moins) deux sens... Deux sens... Sérieusement ?!  C'est là une conclusion assez étrange. Car \sicut{sa mère} --~ou, pour être plus précis, \sicut{la mère d'\OE dipe}~-- n'a qu'un seul sens en français\footnote{Ici on ne s'intéresse pas à ce qui pourrait être un sens figuré de \sicut{mère}, comme dans \sicut{mère patrie} ou \sicut{mère de tous les vices}... Dans la phrase \ref{x:Oedipe}, \sicut{mère} n'est utilisé qu'au sens propre, pour les deux lectures. Et même si, avec cette acception, on choisit de voir de la polysémie dans ce terme (opposant par exemple \sicut{mère biologique} à \sicut{mère adoptive}), ce n'est pas du tout cette ambiguïté qui est à l'\oe uvre dans notre exemple.}. Son sens nous est donné par ses conditions de dénotation, et c'est \alien{grosso modo} «l'unique individu qui est de sexe féminin et qui a engendré \OE dipe», il n'y en a pas d'autre.  D'ailleurs, nous savons traduire ce sens dans {\LO} : c'est \(\Xlo\atoi x\,\prd{mère}(x,\cns{\oe})\). Et si nous calculons \(\denote{\Xlo\atoi x\,\prd{mère}(x,\cns{\oe})}^{\Modele_1,g}\), nous ne trouverons en fait qu'une seule valeur : \Obj{Jocaste}.


Cette observation nous montre deux choses importantes. D'abord que l'ambiguïté de \ref{x:Oedipe} est structurelle. Ce n'est pas parce qu'elle est localisée sur le {\GN} \sicut{sa mère} que ce {\GN} en est responsable. En fait, ce qui provoque l'ambiguïté \alien{de re} \vs\ \alien{de dicto} en \ref{x:Oedipe}, c'est le verbe \sicut{vouloir}.  De manière générale, les verbes comme \sicut{croire}, \sicut{penser}, \sicut{savoir}, \sicut{vouloir},
\sicut{souhaiter}, \sicut{décider}, \sicut{affirmer}, \sicut{regretter}, etc.\ vont avoir cet effet.  Il s'agit de ces verbes qui ont la particularité de prendre une proposition subordonnée complétive en guise de complément d'objet (ça peut être une proposition finie introduite par \sicut{que} ou une proposition infinitive). On les appelle des verbes d'\kw{attitude propositionnelle},\label{VAttProp} 
car ils dénotent une relation  entre un individu (le sujet du verbe) et un contenu propositionnel (donné par la subordonnée), et cette relation exprime une certaine attitude du sujet vis-à-vis de ce contenu ; cette attitude peut être une croyance, un désir, un regret, etc.\footnote{Il faut savoir que tout verbe qui sous-catégorise une complétive n'est pas nécessairement un verbe d'attitude propositionnelle. Certains verbes (certes peu nombreux) n'expriment pas d'attitude particulière du sujet vis-à-vis de la subordonnée ; c'est le cas, par exemple, de \sicut{mériter}, \sicut{risquer} ou \sicut{sembler}.} 
Ces verbes sont à l'origine de l'ambiguïté précisément parce qu'ils font intervenir, dans la phrase, un attitude autre que celle du locuteur ; autrement dit, ils permettent de changer de point de vue dans le cours de l'interprétation de la phrase. À cet égard, on dit que ces verbes créent un environnement \kw{opaque}\footnote{C'est essentiellement à \citet[][\S30--32]{Quine:60fr} que l'on doit la mise au jour et la qualification de \emph{positions opaques}. Ensuite, par métonymie,  les lectures \dedicto\ sont également parfois appelées \kwo{opaques}, et les lectures \dere\ \kwo{transparentes}.} (ou \kw{oblique}).

La deuxième chose importante à constater 
est que nous sommes, en quelque sorte,
\emph{coincés} dans $\Modele_1$.  
En effet, nous venons de voir que \sicut{sa$_1$ mère} a, dans $\Modele_1$, forcément la même dénotation que \sicut{Jocaste}. En dépit de toutes les informations que fournit ce modèle, il ne nous permettra jamais de donner à \sicut{sa$_1$ mère} la dénotation \Obj{Hélène} --~alors que c'est ce que nous voudrions pour rendre compte de la lecture \alien{de dicto}.  Nous n'avons pas le choix. 
Et d'une certaine manière, c'est normal, car la sémantique de \LO\ est faite comme ça :
elle respecte le principe de compositionnalité (et aussi  d'extensionnalité) qui dit que pour interpréter une phrase dans un modèle donné, il faut interpréter ses parties dans \emph{le même} modèle. 
Or ce dont nous aurions besoin, c'est la possibilité d'accéder à un autre modèle au cours de l'interprétation de la phrase.  À partir du moment où l'on émet ce souhait, on ouvre la porte de l'\kw{intensionnalité}. 
L'intensionnalité peut se caractériser simplement ainsi : c'est la capacité pour le langage d'envisager plusieurs modèles différents à la fois dans le calcul sémantique. 

Bien sûr {\LO}, tel qu'il est défini jusqu'à présent, dispose déjà d'une multitude de modèles différents de $\Modele_1$, et en particulier des modèles où le {\GN} \sicut{sa$_1$ mère} dénote \Obj{Hélène}. Ce sont les modèles dans lesquels {Hélène} est réellement la mère d'\OE dipe.  Mais il faut bien comprendre que 
l'enjeu de l'intensionnalité n'est pas d'interpréter la phrase \ref{x:Oedipe} dans un tel modèle (ça, nous savons déjà le faire), mais bien de se donner les moyens, au cours de l'interprétation de \ref{x:Oedipe} dans $\Modele_1$, d'aller visiter provisoirement un autre modèle, puis de revenir à $\Modele_1$ pour déclarer que la phrase est vraie ou fausse (dans $\Modele_1$). Nous verrons dans les sections \ref{s:mondes} et \ref{s:intension} comment intégrer cela formellement dans notre système sémantique, mais pour le moment continuons à explorer l'ambiguïté \alien{de re} {\vs}\ \alien{de dicto}, qui est une des manifestations les plus nettes du phénomène d'intensionnalité.\is{intensionnalité}


Nous avons vu que la lecture \alien{de dicto} de \ref{x:Oedipe} était liée aux croyances d'\OE dipe, c'est-à-dire d'un des protagonistes de la petite histoire décrite par la phrase. Mais ce n'est pas la prise en compte de croyances autres que celles du locuteur qui est déterminante pour déclencher l'ambiguïté ; ce qui compte, c'est vraiment la présence d'un élément qui installe un environnement opaque.  Ainsi la phrase {\Next}, elle, n'est en aucune manière ambiguë :


\ex. \OE dipe$_1$ a épousé sa$_1$ mère.

Bien sûr, on peut toujours distinguer les cas où \OE dipe {sait} que celle qu'il épouse est sa mère des cas où il l'ignore. Mais cette distinction n'intervient pas du tout dans le sens de la phrase.  Dans un modèle où \Obj{Jocaste} est la mère d'\Obj{\OE dipe}, {\Last} est vraie ssi \Obj{\OE dipe} a épousé \Obj{Jocaste}, point. Peu importe qu'ils soient au courant de leur lien de parenté.
On pourrait dire, si l'on veut, que \Last\ ne présente qu'une lecture {\dere} ; l'important c'est surtout que la phrase ne contient pas d'élément qui introduise un environnement opaque.  

Voici, à titre d'illustration, quelques exemples supplémentaires qui font apparaître l'ambiguïté avec d'autres éléments opacifiants.  Ceux-ci sont soulignés  dans les phrases qui suivent, et les \GN\ portant l'ambiguïté sont en italiques :


\ex. 
\a.
Sue \uline{croit} que c'est \emph{un républicain} qui va remporter l'élection.
\label{x:SueRep}
\b. Arthur \uline{cherche} \emph{une licorne}.
\label{x:ArtLic}
%\ex.  Jean \uline{croyait} que {Marie} était quelqu'un d'autre.
\b. Le procès de l'homme$_1$ qui a tué \uline{deux fois} \emph{sa$_1$
  femme}. 
\label{x:Libe}
\b. \emph{Miss France} est de plus en plus gourde \uline{chaque année}.
\label{x:MissF}


L'exemple \ref{x:SueRep} montre que l'ambiguïté peut porter aussi sur un {\GN} indéfini.  Selon la lecture \dedicto\ de \sicut{un républicain}, qui est probablement la plus spontanée, la phrase signifie que Sue se dit à elle-même «c'est un républicain qui va gagner l'élection».  Selon la lecture \dere, c'est le locuteur qui qualifie un individu de républicain et il ajoute que Sue pense que cet individu va remporter l'élection. Dans cette lecture, rien ne nous dit que Sue pense que ce candidat est républicain ; elle peut le penser, mais elle peut aussi penser qu'il est démocrate ou indépendant (et ainsi se tromper par rapport à ce que pense savoir le locuteur au sujet de l'affiliation du candidat).

\largerpage

La phrase \ref{x:ArtLic}, qui est adaptée d'un exemple de \citet{PTQ}, illustre  aussi l'ambiguïté sur un {\GN} indéfini, mais elle montre également que l'environnement opaque peut être déclenché par un verbe comme \sicut{chercher}, qui n'est pas à proprement parler un verbe d'attitude propositionnelle (il n'y a pas de proposition subordonnée dans la phrase). Mais \sicut{chercher} incorpore néanmoins un certain point de vue du sujet ; d'ailleurs le sens de \sicut{chercher} inclut une idée de \sicut{vouloir trouver}, qui contient un verbe d'attitude propositionnel.
Avec la lecture \dere\ de \sicut{une licorne}, \ref{x:ArtLic} nous parle d'une licorne et nous dit qu'Arthur est à sa recherche. La phrase implique alors que les licornes existent (en tout cas qu'il en existe au moins une) ; et dans un modèle similaire à notre monde, où les licornes n'existent pas, elle sera fausse\footnote{En vérité, c'est un peu plus complexe que cela. La lecture \alien{de re} ne s'engage pas tant sur l'existence réelle des licornes que sur la qualification par le locuteur que quelque chose est une licorne. C'est un peu subtil, mais nous reviendrons sur ce point plus tard dans le chapitre. Pour le moment, tenons-nous en à la présente simplification, elle n'est pas vraiment dommageable pour la bonne compréhension du phénomène.}.  
Avec la lecture \dedicto, la phrase \ref{x:ArtLic} peut être vraie dans notre monde, il suffit qu'Arthur croit à l'existence des licornes et qu'il désire en attraper une (c'est ce qui pourrait se passer si le nommé Arthur venait nous voir en disant «Dites, vous n'auriez pas vu une licorne ? Je suis à sa poursuite depuis ce matin» ; même si nous ne croyons pas à l'existence des licornes et que nous pensons qu'il délire, nous avons tout à fait le droit de rapporter cet épisode en utilisant la phrase \ref{x:ArtLic}).

L'exemple \ref{x:Libe} est le titre d'un fait divers (sordide) paru il y a quelques années dans \emph{Libération}. Il montre que l'environnement opaque qui déclenche l'ambiguïté peut aussi être instauré par un adverbial de localisation temporelle (\sicut{deux fois}). Dans ce cas, l'opposition {\dere} {\vs}\ {\dedicto} ne repose plus vraiment sur une distinction de points de vue entre des agents, mais sur une distinction de points de repère temporels.  Au-delà de cela, l'opposition de lectures fonctionne de la même manière. 
La lecture \dere\ de \sicut{sa femme} se situe par rapport au point de repère global de la phrase, c'est-à-dire celui adopté par le locuteur et que l'on peut plus ou moins assimiler au moment de l'énonciation: le locuteur fait référence à la femme de l'homme en question et nous dit que ce dernier l'a tuée deux fois.  À moins d'admettre que la malheureuse ait pu ressusciter entre temps, c'est une lecture assez absurde par rapport à ce que nous savons du monde. Évidemment la lecture visée par l'auteur est la lecture \dedicto. Celle-ci place l'interprétation de \sicut{sa femme} par rapport aux points de repère temporels qui sont les «fois» mentionnées dans la phrase. Autrement dit,  \ref{x:Libe} résume l'histoire d'un récidiviste qui a tué sa femme, puis s'est remarié et finalement a tué sa seconde femme.  Selon la lecture \dedicto, \sicut{sa femme} renvoie à celle qui est la femme de l'homme à «chaque fois».

L'exemple \ref{x:MissF} est tout à fait similaire au précédent. Selon la lecture {\dere} de \sicut{Miss France}, on nous parle d'une certaine personne, qui se trouve être Miss France, et on nous dit que sa bêtise empire d'année en année. Selon la lecture \dedicto, on nous dit que, depuis un certain temps, la Miss France de l'année $n$ est encore plus bête que celle de l'année $n-1$. 
L'exemple permet également de montrer que l'ambiguïté \dedicto\ \vs\ \dere\ peut s'appliquer sur des {\GN} comme \sicut{Miss France}, qui grammaticalement (au moins dans \ref{x:MissF}) fonctionnent comme des noms propres.\is{nom!\elid\ propre} Il s'agit d'un nom propre un peu particulier puisque c'est un titre et non un nom de baptême ; cependant nous allons voir (cf.\ \ref{x:Spiderman} \alien{infra}) que d'une certaine manière, et dans certaines circonstances, des noms propres plus ordinaires peuvent eux aussi faire les frais de l'ambiguïté (ou en tout cas de quelque chose qui y ressemble fort).

\medskip

L'ambiguïté \alien{de re} \vs\ \alien{de dicto} met en échec le principe d'extensionnalité, car la dénotation de certaines expressions (ayant un seul sens) n'apparaît plus déterminée univoquement pour un modèle donné.  Nous pouvons donc la détecter assez facilement en posant des formes de raisonnements comme celle que nous avons vu en \ref{xPX1} : 

\ex.
\a. \OE dipe$_1$  voulait épouser sa$_1$ mère.
\b. Jocaste était la mère d'\OE dipe.
\c. \juge{\small$\not\satisf\:$} \OE dipe voulait épouser Jocaste.

Ici $\not\satisf$ indique que la conclusion \Last[c] ne peut pas être tenue pour une véritable conséquence logique des phrases qui précèdent. 
La non validité du raisonnement est due au fait qu'il est formulé en français et que la phrase \Last[a] est ambiguë. Si nous l'interprétons avec une lecture \alien{de re} du {\GN} \sicut{sa$_1$ mère}, le raisonnement est valide, mais il ne l'est plus si nous retenons la lecture \alien{de dicto} (ce que montre le modèle $\Modele_1$). 

Les formes de raisonnements suivantes (adaptées de \citealt{Gamut:2}) illustrent encore l'échec du principe d'extensionnalité : 

% En fait ces formes de raisonnements, non valides, ne servent pas spécifiquement à faire apparaître l'opposition de dicto de re, leur rôle est plus général : ils illustrent explicitement l'échec de la substitution (principe d'extensionnalité). ça permet d'aller plus loin et de voir que l'échec se produit sans forcément avoir recours à l'ambiguïté : Tante May pense que Spiderman est un criminel / Peter Parker est Spiderman / Tante May pense que Peter Parker est un criminel.
%(plus tard)


\ex.
\a. L'inspecteur {sait} que {le voleur}  est passé par le toit.
\b. Le voleur est Arsène Lupin.
\c. \juge{\small$\not\satisf\:$} L'inspecteur sait qu'Arsène Lupin est passé par le toit.


\ex. \label{x:O1963}
\a. Le Président des USA a été assassiné à Dallas en 1963.
\b. Barack Obama est le Président des USA.
\c. \juge{\small$\not\satisf\:$} Barack Obama a été assassiné à Dallas en 1963.

\ex.  \label{x:Spiderman}
\a. Tante May pense que Spider-Man est un criminel.
\b. Peter Parker est Spider-Man.
\c. \juge{\small$\not\satisf\:$} Tante May pense que Peter Parker est un criminel.


Terminons par deux petites remarques. On a parfois la tentation d'assimiler la lecture \dere\ d'un indéfini à sa lecture spécifique (cf.~\ref{ss:specificite}). Mais il est probablement plus prudent de distinguer ces deux notions. 
D'abord, un indéfini peut très bien avoir une lecture \dere\ sans avoir l'interprétation ou l'emploi spécifique.  C'est par exemple ce que l'on retrouve dans la seconde phrase de \ref{x:antilope}\footnote{J'admets que cet exemple est un peu artificiel. Il joue sur le passage malicieux d'une lecture \dedicto\ dans la première phrase à une lecture \dere\ dans la seconde. De plus, en français, pour marquer le contraste, on aurait plus naturellement tendance à dire «sauf qu'en réalité c'est une antilope  qu'il cherche» ; mais là, l'indéfini ne se trouve plus dans l'environnement opaque et l'opposition \dere/\dedicto\ n'est plus tellement pertinente. Cependant il est assez facile de trouver d'autres exemples d'indéfinis à lecture \dere\ non spécifique et, en guise d'exercice, j'invite les lecteurs à en construire quelques uns.} :

\ex. \label{x:antilope}
Arthur cherche une licorne. Sauf que, il ne le sait pas, mais en réalité il cherche une antilope.

Par ailleurs, il n'est pas impossible d'envisager un indéfini avec une lecture spécifique et {\dedicto} --~c'est ce qui faisait l'objet de l'exercice \ref{exo:specdicto} du chapitre précédent, avec la phrase :

\ex. 
Alice croit qu'un vampire, à savoir Dracula en personne, l'a mordue
pendant la nuit.

Il y a au moins une lecture de cette phrase selon laquelle la spécificité de \sicut{un vampire} n'est pas tant assumée par le locuteur que par Alice (c'est elle qui est convaincue que c'est précisément Dracula qui l'a mordue, alors que le locuteur sait que les vampires n'existent pas). Notons que cet exemple a aussi le mérite de soulever la question du traitement sémantique des entités fictives (même si nous savons que Dracula n'existe pas, cela ne nous empêche pas de le considérer comme un individu bien particulier), nous y reviendrons plus tard dans ce chapitre.

L'autre remarque fait aussi une connexion avec le chapitre précédent. On peut légitimement se douter que l'opposition \dedicto\ \vs\ \dere\ se ramène à  une histoire de portées. D'ailleurs ce que montrent les exemples \ref{x:Libe}, \ref{x:MissF} et \ref{x:O1963} rappelle ce que nous avons vu au sujet des adverbes temporels et de quantification.  L'intuition que nous pouvons avoir est que les environnements opaques correspondent à la portée de certains éléments de la phrase (par exemple les verbes d'attitudes propositionnelles) et qu'un {\GN} a la lecture \dedicto\ s'il est interprété dans cette portée et la lecture \dere\ s'il est interprété en dehors. C'est ainsi que nous procéderons (dans un premier temps). Pour le moment, nous n'avons pas les moyens techniques de représenter l'opacité dans {\LO} (mais c'est pour très bientôt), mais nous pouvons d'ores et déjà esquisser l'analyse pour des indéfinis, en glosant les conditions de vérité comme suit :

\ex.[\ref{x:SueRep}]
Sue {croit} que c'est {un républicain} qui va remporter l'élection.
\a. {\dere} : il existe un individu qui est républicain et tel que Sue
pense qu'il va gagner l'élection.
\b.
{\dedicto} : Sue pense qu'il existe un individu qui est républicain et
qui va gagner l'élection.  

Cependant (sans tout de suite dévoiler la fin de l'histoire) nous verrons également  que cette stratégie, en termes de portées relatives, n'est pas sans poser problème pour l'analyse du phénomène.

\medskip

% -*- coding: utf-8 -*-
\begin{exo}\label{exo:derededicto}
Trouvez les quatre lectures (théoriquement) possibles de :\pagesolution{crg:derededicto}

\begin{enumerate}
\item Eugène$_1$ trouve que le Pape ressemble à son$_1$ arrière-grand-père.
\item \OE dipe$_1$ ne {savait} pas que {sa$_1$ mère} était {sa$_1$ mère}.
\end{enumerate}

\begin{solu}(p.~\pageref{exo:derededicto})\label{crg:derededicto} 

Pour illustrer les différentes lectures, je propose un scénario qui rend vrai chacune d'elles (mais cela ne veut pas dire nécessairement que chacune de ces lectures n'est vraie \emph{que} dans le scénario particulier qui lui est associé).  Il s'agit bien sûr d'alternances combinées \dedicto/\dere\ (cf. \S\ref{ss:re/dicto}, p.~\pageref{p.re/dicto}).

\begin{enumerate}
\item Eugène$_1$ trouve que le Pape ressemble à son$_1$ arrière-grand-père.
  \begin{enumerate}
  \item \sicut{le Pape}, \sicut{son arrière-grand-père} : \dedicto\\
  Eugène se dit : «c'est marrant, le Pape ressemble beaucoup à mon arrière-grand-père».
  \item \sicut{le Pape} : \dere, \sicut{son arrière-grand-père} : \dedicto\\
  Eugène voit une photo du Pape, sans savoir de qui il s'agit, et il se dit : «ce type ressemble à mon arrière-grand-père».
  \item \sicut{le Pape} : \dedicto, \sicut{son arrière-grand-père} : \dere\\
  Eugène voit une photo de son arrière-grand-père, sans savoir de qui il s'agit, et il se dit : «Le Pape, il ressemble à ce type».
  \item \sicut{le Pape}, \sicut{son arrière-grand-père} : \dere\\
  Eugène voit une photo du Pape, sans savoir de qui il s'agit, et une photo de son arrière-grand-père, sans le reconnaître non plus, et il se dit : «ces deux types se ressemblent beaucoup».
  \end{enumerate}

\item \OE dipe$_1$ ne {savait} pas que {sa$_1$ mère} était {sa$_1$ mère}.
\\
NB : comme déjà mentionné dans le chapitre, il convient ici de faire abstraction de la polysémie du nom \sicut{mère} (mère biologique {\vs} mère adoptive {\vs} mère sociale etc.) afin de ne pas surmultiplier les interprétations.

\begin{enumerate}
\item  \sicut{sa mère} : \dedicto\ $\times 2$\\
Quelqu'un dit à \OE dipe : «Ta mère est ta mère» (ce qui est une tautologie), et il répond (sans ironie)  : «Ah ? Je ne savais pas».  C'est évidemment une situation très absurde.

\item 1\ier\ \sicut{sa mère} : \dere, 2\ieme\ \sicut{sa mère} : \dedicto\\
Jocaste est la mère d'\OE dipe et quelqu'un dit à \OE dipe, en désignant Jocaste : «Cette femme est ta mère».  \OE dipe répond : «Ah ? Je ne savais pas».

\item 1\ier\ \sicut{sa mère} : \dedicto, 2\ieme\ \sicut{sa mère} : \dere\\
Jocaste est la mère d'\OE dipe, \OE dipe ne sait pas qui est sa mère (il ne l'a jamais connue et n'a aucune information sur elle) et quelqu'un lui dit, en désignant Jocaste : «Ta mère, c'est cette femme».  \OE dipe répond : «Ah ? Je ne savais pas».  Naturellement c'est similaire à la situation précédente.

\item  \sicut{sa mère} : \dere\ $\times 2$\\
Jocaste est la mère d'\OE dipe et quelqu'un dit à \OE dipe : «Jocaste est Jocaste».   \OE dipe répond : «Ah ? Je ne savais pas».  C'est une situation absurde similaire à la première.

\end{enumerate}

\end{enumerate}
\end{solu}
\end{exo}





\subsection{Modifieurs non extensionnels}
%-------------------------------------------
\label{ss:ModNonExt}

Le phénomène que je vais présenter ici est, linguistiquement, beaucoup plus marginal que l'ambiguïté {\dedicto/\dere} : il ne concerne que quelques mots de la langue. Mais il est assez souvent mentionné dans la littérature pour illustrer les limites d'expressivité d'un langage extensionnel comme \LO.
Il a trait à certains adjectifs employés comme épithètes (c'est-à-dire modifieurs de noms).

Jusqu'à présent, nous avons traité les adjectifs qualificatifs de la même manière que les noms communs, en les traduisant par des prédicats à une place\footnote{Tout en sachant qu'à l'instar de certains noms, il existe aussi des adjectifs relationnels\is{adjectif!\elid\ relationnel} qui se traduisent par des prédicats binaires ; par exemple : \sicut{fier}, \sicut{content}, \sicut{supérieur}...}. À ce titre, ils dénotent donc des ensembles d'individus, et on les appelle des adjectifs \kwo{intersectifs}\is{adjectif!\elid\ intersectif} car lorsqu'ils se combinent avec un nom, la dénotation du constituant formé s'obtient en faisant l'intersection de la dénotation du nom et de celle de l'adjectif. En effet, la dénotation de \sicut{tigre édenté} par rapport à \Modele\ (et une assignation $g$), c'est bien 
\(\denote{\prd{tigre}}^{\Modele,g} \cap \denote{\prd{édenté}}^{\Modele,g}\). 
Et dans {\LO}, nous obtenons correctement cette interprétation intersective, car nous traduisons la combinaison nom + adjectif au moyen d'une conjonction : \ref{x:Hbbs1} se traduit par \(\Xlo[\prd{tigre}(\cns h) \wedge \prd{édenté}(\cns h)]\), qui dit que la dénotation de \sicut{Hobbes} appartient aux deux ensembles dénotés par les prédicats, donc qu'elle appartient à leur intersection.



\ex.  \label{x:Hbbs}
\a. Hobbes est un tigre édenté.\label{x:Hbbs1}
\b. Hobbes est un tigre.\label{x:Hbbs2}
\c. Hobbes est édenté.\label{x:Hbbs3}


En observant les phrases de \ref{x:Hbbs}, il est très facile d'y voir les relations de conséquence logique suivantes :
\(\ref{x:Hbbs1} \satisf \ref{x:Hbbs2}\) et  \(\ref{x:Hbbs1} \satisf \ref{x:Hbbs3}\).
Et ces conséquences sont parfaitement prédites par {\LO}, grâce à un théorème élémentaire de logique qui dit que \(\xlo{[\phi \wedge \psi]} \satisf \vrb\phi\). 
C'est précisément parce que l'adjectif \sicut{édenté} est intersectif que ces inférences sont disponibles.



Mais avec les adjectifs suivants, en \ref{x:Adjnoninter}, ces
inférences ne tiennent plus. 



\ex. \label{x:Adjnoninter}
\a. \emph{ancien} étudiant
\b. %future maman / 
\emph{futur} Prix Nobel
\c. \emph{présumé} coupable
\d. \emph{faux} diamant

Par exemple, un ancien étudiant n'est pas
nécessairement un étudiant. Si je dis de Jean qu'il est un futur Prix Nobel, on ne va pas en inférer que Jean est un Prix Nobel.  Un présumé coupable ne doit pas être considéré comme un coupable, et un faux diamant n'est techniquement pas un diamant. 
Les inférences du type \(\ref{x:Hbbs1} \satisf \ref{x:Hbbs2}\) ne sont pas valides, mais celles du type \(\ref{x:Hbbs1} \satisf \ref{x:Hbbs3}\), quant à elles, sont en quelque sorte encore pire que cela : grammaticalement, ces adjectifs ne semblent pas pouvoir s'employer seuls, sans tête nominale, en position d'attribut (du moins dans le sens qu'ils ont en \ref{x:Adjnoninter}). On ne peut pas dire que Jean est futur ou qu'il est présumé. Si l'on peut, éventuellement, dire que Jean est ancien, ce sera, au mieux, avec une acception très différente de l'adjectif.

Par conséquent, il sera incorrect de traduire le sens d'une phrase comme \sicut{Jean est un ancien étudiant} par \(\Xlo[\prd{ancien}(\cns j) \wedge \prd{étudiant}(\cns j)]\). Et c'est parce que les adjectifs de \ref{x:Adjnoninter} ne sont pas intersectifs. D'ailleurs, on s'en rend compte assez facilement : s'ils étaient intersectifs, ils dénoteraient chacun un ensemble d'individus ; mais ça n'aurait guère de sens d'essayer de déterminer dans un modèle l'ensemble des individus anciens, des individus futurs, présumés ou faux. Certes le constituant \sicut{ancien étudiant} dénote lui un ensemble, mais cet ensemble n'a {a priori} aucun rapport avec celui dénoté par \sicut{ancien ministre} ou \sicut{ancien footballeur}.  En fait ces adjectifs ne \emph{semblent}  pas avoir  vraiment de dénotation autonome\footnote{En réalité, ils ont une dénotation propre, mais il ne s'agit pas d'un simple ensemble d'individus. Nous verrons comment cela peut se formaliser à partir du chapitre~\ref{ch:types}.}, et leur contribution sémantique dépend étroitement de celle de la tête nominale qu'ils accompagnent. Mais de quelle manière ?  Le fait est que la dénotation de \sicut{ancien étudiant} dans un modèle $\Modele$ \emph{ne dépend pas} de la dénotation de \sicut{étudiant} dans $\Modele$.  Elle dépend plutôt de la dénotation qu'a \sicut{étudiant} dans \emph{un autre} modèle, et plus précisément un modèle qui décrit un état du monde antérieur à celui décrit par $\Modele$.  On retrouve ici la caractéristique de l'intensionnalité\is{intensionnalité} que nous avons vu dans les pages précédentes, à savoir que pour interpréter une expression (ici \sicut{ancien étudiant}) dans un modèle donné nous avons besoin d'aller consulter un autre modèle. Et c'est pour cette raison que les adjectifs en \ref{x:Adjnoninter} sont également appelés \kwo{adjectifs intensionnels}\is{adjectif!\elid\ intensionnel}\is{intensionnel!adjectif \elid}. Informellement, le rôle sémantique de l'adjectif \sicut{ancien} est d'aller chercher dans un modèle du passé la dénotation de la tête nominale (par exemple \sicut{étudiant}) et de rapatrier cette dénotation dans le modèle actuel. Une expression qui fonctionne ainsi sémantiquement est intensionnelle.

Évidemment \sicut{ancien} et \sicut{futur} (ainsi, entre autres, que le préfixe \sicut{ex-}) annoncent la question de la temporalité,\is{temporalité} qui est en soi un phénomène intensionnel, et nous l'aborderons en \S\ref{ss:tempomod} et plus précisément en \S\ref{s:Prior} \alien{infra}. Mais les adjectifs intensionnels ne sont pas nécessairement liés à la temporalité. Des adjectifs comme \sicut{présumé}, \sicut{prétendu}, \sicut{supposé}, \sicut{soi-disant}, etc.\ fonctionnent de manière similaire, mais au lieu de visiter des modèles passés ou à venir, ils visitent des modèles qui décrivent des réalités hypothétiques. 

Pour démontrer que les adjectifs intensionnels mettent en échec le principe d'extensionnalité propre à {\LO}, la méthode peut paraître un peu moins naturelle que pour les cas des ambiguïtés \dedicto\ \vs\ \dere, car la substitution du principe d'extensionnalité \ref{pr:ext} ne porte pas ici sur des {\GN} pleins mais  seulement sur des prédicats nominaux. L'idée générale est la suivante. Il faut se placer dans un modèle où deux prédicats nominaux $X$ et $Y$ ont la même dénotation (attention, cela ne veut pas dire que $X$ et $Y$ sont synonymes, mais simplement qu'il se trouve que, dans le modèle observé, leurs dénotations coïncident), et on démontre ensuite que le fait qu'un individu soit, par exemple, un ancien $X$ n'implique pas nécessairement qu'il est aussi un ancien $Y$.  
Pour illustrer cela linguistiquement avec un exemple concret, on est obligé de passer par un mini-modèle ou un modèle-jouet qui comporte un domaine d'individus particulièrement restreint, simplement parce qu'autrement, il est difficile de trouver des prédicats de dénotations identiques. Supposons par exemple que nous assistons à une réception mondaine et que nous ne nous intéressons qu'aux personnes présentes. Et supposons également que nous savons que parmi les invités, tous les avocats, et seulement eux, sont alcooliques. Cela veut dire que, dans ce modèle particulier, \sicut{alcoolique} et \sicut{avocat} ont exactement la même dénotation. Dans ce cas, si nous savons que Pierre est un alcoolique, alors nous pouvons en conclure qu'il est un avocat. Autrement dit, le raisonnement \Next\ est valide.

\largerpage[-1]

\ex.
\a. Ici, ce soir, tous les avocats sont alcooliques, et tous les alcooliques %présents 
sont des avocats.
\b. Pierre est un alcoolique.
\c. \juge{\small$\satisf\:$}  Pierre est un avocat.


Ce raisonnement est valide notamment grâce au principe d'extensionnalité car \Last[c] est le résultat de la substitution de \sicut{alcoolique} par \sicut{avocat} à partir de la prémisse \Last[b], sachant que les deux expressions ont la même dénotation (du fait de \Last[a]). En revanche, en appliquant la même substitution, le raisonnement {\Next}, lui, n'est pas valide :

\ex.
\a. Ici, ce soir, tous les avocats sont alcooliques, et tous les alcooliques %présents 
sont des avocats.
\b. Pierre est un ancien alcoolique.
\c. \juge{\small$\not\satisf\:$}  Pierre est un ancien  avocat.


Les adjectifs que nous examinons ici sont donc eux aussi des symptômes du caractère intensionnel de la langue naturelle. Comme promis, nous allons, à partir de la section \ref{s:Prior}, commencer l'amélioration de {\LO} pour le rendre intensionnel. Cependant les perfectionnements apportés dans ce chapitre ne suffiront pas à poser une analyse tout à fait complète des adjectifs intensionnels ; pour cela nous devrons poursuivre le développement de {\LO} par le chapitre~\ref{ch:types}. %(et peut-être aussi le chapitre~\ref{Ch:temps2}).


\subsection{Temporalité et modalités}
%------------------------------------
\label{ss:tempomod}\is{temps}

Je ne vais pas énormément développer ici le dernier exemple des limites d'expressivité de {\LO} liées à son extensionnalité, d'une part parce qu'il s'agit d'un phénomène qui nous est apparu avec évidence dès le début de la présentation de {\LO}, et d'autre part parce que c'est ce que nous allons aborder en détail dans les \S\ref{s:Prior} et \ref{s:mondes}.  Le phénomène en question est, bien sûr, celui de l'expression du temps et des modalités dans la langue naturelle. Depuis le début, lorsque nous avons traduit des phrases dans {\LO}, nous avons toujours été obligés de négliger la contribution sémantique des temps verbaux\is{temps!\elid\ verbal} (\ie\ les temps de conjugaison) et des adverbiaux ou circonstanciels dits de temps. 
Autrement dit, {\LO} ne nous permet pas de donner des traductions différentes pour les phrases en \Next. Au mieux, on peut dire que {\LO} sait produire une traduction pour \Next[a], mais dans ce cas, il manque quelque chose pour traduire \Next[b,c].  Bien entendu, c'est une faiblesse considérable de l'expressivité de notre langage sémantique.

\ex.
\a. Alice regarde un dessin animé.
\b. Alice a regardé un dessin animé.
\b. Alice regardera un dessin animé.




Notons que j'emploie ici «temporalité»\is{temporalité}  dans un sens restreint pour désigner tout ce qui a trait à un renvoi au présent, au passé ou au futur et tout autre localisation chronologique. On sait par ailleurs que l'expression du temps dans les langues ne se réduit pas à cela. En particulier il y a aussi tout ce que l'on regroupe sous le terme général d'\emph{aspect} (j'évoquerai brièvement cette question en \S\ref{ss:PbTmps} et nous y reviendrons plus sérieusement au chapitre~\ref{Ch:temps2}, vol.~2). Autrement dit introduire la temporalité dans {\LO} ne suffit pas à un faire un sort complet à la contribution des temps verbaux dans un système sémantique.  Mais c'est un bon début, et c'est ce sur quoi nous allons nous concentrer dans les pages qui suivent.%en \S\ref{s:Prior}.
\label{TpsvsAsp} 

\sloppy
Comme nous l'avons entrevu précédemment avec les adjectifs  \sicut{ancien} et \sicut{futur}, la temporalité est un phénomène intensionnel. D'ailleurs, cela transparaissait aussi, par exemple, dans la forme de raisonnement \ref{x:O1963} en \S\ref{ss:re/dicto} (p.~\pageref{x:O1963}).  Parler du passé ou du futur c'est en quelque sorte aller visiter des états du monde (donc des modèles) différents de l'état actuel du monde.  Là encore nous retrouvons la caractéristique fondamentale de l'intensionnalité.

\fussy

Les modalités fonctionnent \emph{un peu} de la même manière. La notion de modalité en sémantique ne correspond pas exactement à ce que l'on appelle traditionnellement les modes\is{mode!\elid\ de conjugaison} en grammaire (comme l'indicatif, le subjonctif, le conditionnel)\footnote{Mêmes si les modes, en soi, contribuent à l'expression de modalités, mais la notion de modalité est plus large et ne se ramène pas uniquement aux modes.}, mais plutôt à tout ce qui a à voir avec l'hypothétique, le virtuel, l'incertain, etc.   Nous l'aborderons en \S\ref{s:mondes}.




\subsection{Conclusion : en route vers l'intensionnalité}
%-------------------------------------------------------
\is{intensionnalité}

Récapitulons brièvement. Notre langage {\LO} est extensionnel.
Cela implique que, pour un modèle {\Modele} donné, et une assignation
$g$ donnée, une 
expression interprétable $\Xlo\alpha$ de {\LO} a une et une seule valeur
sémantique :  \(\denote{\Xlo\alpha}^{\Modele,g}\).  
Nous avons vu que la langue naturelle n'est pas extensionnelle, mais
intensionnelle, et nous allons donc devoir rendre {\LO} lui aussi
intensionnel. 
Passer à l'intensionnalité c'est concevoir, au sein du calcul
interprétatif d'une phrase, que l'on puisse envisager qu'une expression donnée reçoive
différentes valeurs sémantiques (\ie\ différentes dénotations) en fonction
de certaines circonstances.   
Techniquement cela veut dire que nous allons avoir besoin de manipuler
une multitude de modèles concurrents dans notre système sémantique. 
Et voilà, d'une certaine manière, tout est dit sur l'idée de fond
qui sous-tend l'intensionnalité en sémantique ; il ne nous reste plus
qu'à attaquer la mise en \oe uvre formelle de ce principe.
Mais par la même occasion, nous verrons que «l'intensionnalisation»
de {\LO} va aussi nous permettre d'exprimer dans \LO\ le sens de
constructions linguistiques que nous n'étions pas en mesure de traiter
jusqu'à présent, ce qui va représenter un  
gain d'expressivité et d'efficacité considérable et indispensable pour
notre système.

%%\pagebreak

\section{Une sémantique temporelle intensionnelle}
%=================================================
\label{s:Prior}

Nous allons examiner dans cette section un système sémantique
temporel, inspiré des travaux du logicien
A. \citet{Prior:67}.\Andex{Prior, A.}
Ce système
va nous permettre d'introduire la dimension du temps dans notre
langage objet {\LO} et donc dans la théorie sémantique.  Disons-le
tout de suite, nous verrons que cette sémantique temporelle n'est pas
parfaitement appropriée  pour rendre compte correctement des
phénomènes linguistiques qui mettent en jeu le temps, et il nous
faudra l'abandonner en tant que telle pour la remplacer par une
formalisation plus adéquate (chapitre~\ref{Ch:temps2}, vol.~2).  Cependant,
l'approche qui va être décrite ici présente plusieurs intérêts.
D'abord, elle introduit des notions qui resteront utiles pour décrire
la sémantique de la temporalité.  Ensuite, elle étend considérablement
l'expressivité du système formel et constitue ainsi une approche
très explicative et pédagogique pour commencer à se familiariser avec
le principe d'intensionnalisation de la théorie sémantique (ce que
nous aborderons de façon plus étendue dans les sections \ref{s:mondes} et
\ref{s:intension}). 



\subsection{Modèle temporel}
%---------------------------
\label{s:ModeleTemp}
\is{temps|(}

L'enjeu pour nous est donc de nous donner les moyens de représenter
différemment les conditions de vérités de phrases comme, par exemple, 
\ref{x:Alicedorta} et \ref{x:Alicedormait} :

\ex. 
\a. Alice dort. \label{x:Alicedorta}
\b. Alice dormait.\label{x:Alicedormait} 


La traduction de \ref{x:Alicedorta} est
\(\Xlo\prd{dormir}(\cns{a})\). 
Et donc pour un
modèle \(\Modele = \tuple{\Unv{A},\FI}\), dans lequel nous savons,
entre autres, que la constante \cns a de {\LO} dénote l'individu
\Obj{Alice} (autrement dit $\FI(\cns a)=\Obj{Alice}$), 
les conditions de vérité de \ref{x:Alicedorta} 
sont les suivantes :
%se résument simplement ainsi :
\(%\denote{\sicut{Alice dort}}^{\Modele,g} =
\denote{\prd{dormir}(\cns{a})}^{\Modele,g} =1\) ssi
\(\denote{\cns{a}}^{\Modele,g} \in \denote{\prd{dormir}}^{\Modele,g}\) (\RSem\ref{RIprd2}) ; donc
ssi \(\FI(\cns{a}) \in \FI(\prd{dormir})\) ; donc ssi \Obj{Alice}
appartient à l'ensemble (ou la classe) des individus de {\Unv{A}} qui
dorment dans $\Modele$. 
Remarquons ici un sous-entendu un peu évident mais qui va prendre de
l'importance : si les conditions de vérité de \ref{x:Alicedorta} sont
celles-là, cela veut dire que le modèle d'évaluation, \Modele, est
pris comme décrivant l'état du monde actuel,  correspondant au moment
de l'énonciation.

Maintenant, pour expliciter les conditions de vérité de 
\ref{x:Alicedormait}, en nous inspirant de celles de
\ref{x:Alicedorta}, nous pouvons dire que \ref{x:Alicedormait} 
est vraie dans \Modele, si \Obj{Alice} appartient à
une classe de dormeurs, mais pas les dormeurs de maintenant, plutôt  les
dormeurs «d'auparavant», les dormeurs du passé.  Et bien sûr, nous ne disposons
pas d'une telle classe dans un modèle comme {\Modele}.




Nous  pourrions nous la donner de manière simple, en envisageant dans le
vocabulaire de {\LO} un prédicat
\prd{dormait} qui dénote tous les dormeurs du passé, \ie\ tel que
$\FI(\prd{dormait})$ est l'ensemble des individus de {\Unv{A}} qui
étaient dormeurs dans le passé.   Ce n'est d'ailleurs pas si éloigné de ce que nous
allons faire, 
mais nous allons procéder de manière plus systématique.  
Car le défaut de cette suggestion,
c'est que l'on perdrait le lien sémantique entre \sicut{dormait} et
\sicut{dort} : ces 
deux formes verbales ne correspondraient plus au même prédicat de \LO, ce qui
est assez contre-intuitif ; on aimerait plutôt avoir un prédicat \prd{dormir}
intemporel, mais qui a des dénotations différentes en fonction du
temps.   Car finalement dormir... c'est dormir, peu importe quand ; ce
qui change avec le temps c'est \emph{qui} dort, mais pas ce que
signifie ce verbe.  En d'autres termes : ce qui change avec le temps,
c'est la dénotation du prédicat, mais pas son sens. 
Et rappelons qu'un symbole de prédicat représente (et synthétise) le
sens d'un mot, pas sa dénotation.

Précédemment, nous avons vu que l'idée de base qui sous-tend l'intensionnalité est de démultiplier les modèles.  
À cet effet, ce que nous allons donc faire, c'est, en quelque sorte, découper  modèle en tranches
temporelles, et pour chaque 
tranche, nous allons définir une classe de dormeurs. Nous n'allons pas
le découper
en trois tranches qui seraient le passé, le présent et le futur.  Car
ça ne serait pas assez systématique ni réaliste.  
La classe des dormeurs (et toute autre classe) peut, au moins
virtuellement, changer à tout moment.   
Pour accéder à «tous ces moments»,
nous allons modéliser le temps
%(qui passe) 
sous la forme d'un ensemble, ou d'une série,  d'\kwi{instants}{instant}.  
Et cet ensemble va nous servir de patron pour effectuer
le découpage en tranches temporelles.



Nommons {\Tps} cet ensemble d'instants.\is{I@\Tps}  
Les instants que
contient {\Tps} seront notés par $i_1$, $i_2$, $i_3 \dotsc$ (et $i$, $i'$, $i''\dotsc$ quand nous voudrons parler d'instants quelconques).
Par exemple, nous pouvons nous donner \(\Tps = \set{i_0; i_1; i_2; i_3; i_4; \dotsc}\).  Les instants seront conçus comme des points de repère ou des jalons temporels, un peu à l'instar de marques sur une frise chronologique ou de ce que nous appelons ordinairement des dates. 
Ainsi {\Tps} modélise le cours du temps ; on peut alors choisir de le rendre infini ou non, mais nous pouvons au moins nous attendre à ce qu'il contienne un nombre immense d'éléments.  
Dans l'immédiat, nous ne ferons pas ici d'hypothèses plus avancées sur la
nature ontologique des instants.  Par exemple, nous laisserons de côté les questions de savoir si les instants ont une épaisseur (\ie\ une durée), ou si {\Tps} est un ensemble infini dénombrable ou dense\footnote{Informellement, un ensemble ordonné est dit dense si entre deux éléments quelconques on trouvera toujours une infinité d'éléments «intermédiaires».}.

Ce qui va nous importer ici,
c'est que les instants sont \kwi{ordonnés}{ordre} dans \Tps, \ie\ ils sont organisés selon une
\kwi{relation d'ordre total}{ordre!\elid\ total}. %linéaire ?.  
Nous utiliserons le symbole $\tprec$ pour
représenter cette relation d'ordre.  Il s'agit bien sûr d'un ordre
temporel, et donc d'un relation d'\emph{antériorité}.\is{relation!\elid\ d'anteriorité ($\tprec$)}
Ainsi la notation $i \tprec i'$
dit que l'instant $i$ est antérieur à l'instant $i'$, ou l'instant $i$ est avant l'instant $i'$, (ou encore $i'$ est postérieur à
$i$, etc.).  Et cet ordre est total, ce qui signifie que tout instant est chronologiquement situable par rapport à tous les autres ; autrement dit, si nous prenons deux instants quelconques $i$ et $i'$ de {\Tps}, nous saurons  que 
nécessairement soit $i \tprec i'$, soit $i' \tprec
i$, soit $i=i'$\footnote{De temps en temps, nous nous autoriserons à écrire $i\tpreceq i'$ pour dire, comme de coutume, que $i\tprec i'$ \emph{ou} $i=i'$.}.  %L'ordre {\tprec} est \kwo{total} sur \Tps.
Nous le saurons, car l'information nous sera donnée : l'ordre $\tprec$ est prédéfini et il est constitutif de la définition de {\Tps}\footnote{Même s'ils ne sont pas définis de la même manière, on voit que {\Tps} est très similaire aux ensembles classiques de nombres ($\mathbb N$, $\mathbb Z$, $\mathbb R$, etc.), sauf que pour les nombres, $<$ représente l'infériorité bien sûr. Nous aurions d'ailleurs pu prendre directement un de ces ensembles pour modéliser le temps, cela n'aurait pas changé grand chose formellement : les nombres ont les propriétés mathématiques nécessaires et suffisantes pour jouer le rôle de nos instants.}.
De la sorte, $\Tps$ est défini comme n'importe quel axe temporel qu'on a l'habitude de rencontrer sur les graphiques qui représentent des évolutions sur diverses périodes (Fig~\ref{fig:I<}).
Pour présenter {\Tps} comme un ensemble muni de l'ordre $\tprec$, on écrit habituellement \tuple{\Tps,\tprec}, ou éventuellement $\Tps_{\tprec}$ pour faire court.
\newpsstyle{instant}{dotstyle=*}

\begin{figure}[h!]
\begin{center}
\begin{pspicture}[shift=*](9,1)
\psline{->}(0.5,.5)(8.5,.5)\rput[B](8.5,.7){\Unv{I}}%
\multips(1.2,.5)(.5,0){14}{\psdots[style=instant](0,0)}
\end{pspicture}
\caption{Représentation graphique de $\Tps_{\tprec}$ comme un axe temporel}\label{fig:I<}
\end{center}
\end{figure}



Maintenant, revenons au découpage temporel du modèle.  L'idée est
que la dénotation de n'importe quel prédicat, par exemple \prd{dormir}, peut
être différente selon les instants.  Jusqu'à présent, dans un modèle
extensionnel
\(\Modele = \tuple{\Unv{A},\FI}\), la dénotation de \prd{dormir} était
\(\FI(\prd{dormir})\).  
%
Dorénavant, nous allons avoir autant de dénotations pour le prédicat qu'il y
a d'instants, c'est-à-dire que \prd{dormir} aura une dénotation
particulière  pour chaque instant de {\Tps}.  
Cela se formalise  en
modifiant légèrement la fonction d'interprétation {\FI}.\is{fonction!\elid\ d'interprétation}  Nous allons
considérer que {\FI} est à présent une fonction à \emph{deux} arguments : le premier
argument est un instant et le second une constante non logique de {\LO}. % symbole de prédicat. 
Par
exemple, si $i$ est un instant, $\FI(i,\prd{dormir})$%
\footnote{On pourrait aussi écrire
$\FI_i(\prd{dormir})$, mais la notation $\FI(i,\prd{dormir})$ a
l'avantage de bien expliciter la nouvelle arité de \FI.}
 est la dénotation
de \prd{dormir} à l'instant $i$, \ie\ l'ensemble de tous les dormeurs de
{\Modele} à l'instant $i$.


En faisant cela, nous avons d'une certaine manière temporalisé notre
modèle : nous y avons injecté  du temps (\ie\ $\Tps_{\tprec}$). Et ainsi nous l'avons rendu intensionnel.  
%Par
%conséquent, un modèle (temporel) {\Modele} a maintenant trois
%composants :  un domaine d'individus (\Unv{A}), un ensemble d'instants ordonné
%($\Tps_{\tprec}$) et une fonction d'interprétation (\FI) à deux arguments.
%Nous écrirons donc : \(\Modele = \tuple{\Unv{A},\Tps_{\tprec},\FI}\).

\begin{defi}[Modèle intensionnel (1)]\is{modele@modèle!\elid\ intensionnel}
Un modèle intensionnel temporel $\Modele$ est constitué d'un ensemble d'individus \Unv A, 
d'un ensemble d'instants ordonnés $\tuple{\Tps,\tprec}$ et d'une fonction d'interprétation $\FI$ à deux arguments qui pour chaque instant de $\Tps$ et pour chaque constante de $\CON$ nous donne la dénotation de cette constante à cet instant.
\\
On notera \(\Modele = \tuple{\Unv{A},\Tps_{\tprec},\FI}\).
\end{defi}


Nous cherchions à multiplier les modèles. En insérant $\Tps_{\tprec}$ dans {\Modele} nous obtenons une sorte de super-modèle qui convient à nos besoins, car il équivaut à un \emph{ensemble} de modèles extensionnels. C'est comme si nous avions un $\Modele_i=\tuple{\Unv A,\FI_i}$ pour chaque $i$ de {\Tps}, et que nous avions «multiplié la notion de modèle par $\Tps$~». Mais techniquement nous l'avons multipliée «de l'intérieur». Une structure \tuple{\Unv A,\Tps_{\tprec}, \FI} (\ie\ donc un modèle intensionnel) n'est pas exactement un ensemble de modèles extensionnels, mais ce qu'il convient d'appeler une \emph{famille} de modèles extensionnels dans laquelle {\Tps} joue le rôle de ce que l'on nomme, en mathématiques, un ensemble d'\kwi{indices}{indice}.\label{cf:indices} 
Les indices fonctionnent comme des étiquettes qui identifient les différents états du modèle. Ils sont cruciaux en sémantique intensionnelle et nous en ferons souvent usage.

%\paragraph
{Illustrons tout cela avec un exemple simple.}  Soit \(\Modele =
\tuple{\Unv{A},\Tps_{\tprec},\FI}\) (c'est un modèle-jouet) avec 
quatre individus dans le domaine :
\(\Unv{A} = \set{\Obj{Alice}; \Obj{Bruno}; \Obj{Charles};\Obj{Dina}}\),
et quatre instants : \(\Tps = \set{i_1;i_2;i_3;i_4}\).  
Comme {\Tps} est ordonné par $\tprec$, il faut que l'ordre nous soit donné ; le voici : $i_1\tprec i_2 \tprec i_3\tprec i_4$.
Et regardons
toujours le prédicat \prd{dormir}.  Voici à quoi peut ressembler son
interprétation dans {\Modele} :

\ex. \label{x:ModeleI}
\(
\FI(i_1,\prd{dormir}) = \set{\Obj{Bruno};\Obj{Dina}}
\)
\\
\(
\FI(i_2,\prd{dormir}) = \set{\Obj{Alice};\Obj{Dina};\Obj{Charles}}
\)
\\
\(
\FI(i_3,\prd{dormir}) = \set{\Obj{Charles}}
\)
\\
\(
\FI(i_4,\prd{dormir}) = \Evide
\)



Cela se commente assez facilement.  Par exemple, à $i_2$, les individus
qui dorment sont \Obj{Alice}, \Obj{Dina} et \Obj{Charles}.  \Obj{Dina} a dormi de $i_1$ à $i_2$. Et à
$i_4$, plus personne ne dort.
Ce qui vaut pour \prd{dormir} vaut évidemment pour n'importe quel prédicat.
Ce qui est important en ajoutant {\Tps} dans le modèle, c'est que
la fonction d'interprétation {\FI} est modifiée : les valeurs que {\FI}
attribue aux constantes non logiques \emph{dépendent} d'un instant
donné.   Avec les prédicats, on voit bien que ce mécanisme reflète le
cours des choses, leur évolution, bref le temps qui passe.  
Les modèles extensionnels que nous manipulions auparavant étaient présentés comme des images fixes, de grandes photographies  du monde ; avec les modèles intensionnels temporels, nous sommes passés au cinéma : ils nous livrent le grand film de l'histoire du monde.


Faisons, pour terminer, une remarque sur les individus et les constantes d'individus. Puisque {\FI} dépend du temps et que c'est également {\FI} qui interprète les constantes d'individus, on peut, théoriquement, très bien envisager une situation où pour deux instants $i$ et $i'$,  \(\FI(i,\cns a)\neq\FI(i',\cns a)\), autrement dit où la dénotation d'une constante changerait avec le temps. 
Par rapport aux notions que nous manipulons depuis le chapitre~\ref{LCP}, cela modéliserait l'idée qu'entre les instants $i$ et $i'$, la personne que nous appelons \cns a dans {\LO} (ou \sicut{Alice} en français) est devenue... un autre individu. Même si cela semble assez étrange, il pourrait y avoir une manière de justifier un tel cas de figure. Les individus du domaine \Unv A (y compris les objets inanimés) peuvent toujours changer et évoluer au cours du temps, ne serait-ce qu'en ce qui concerne leur taille, leur forme, leur apparence, etc. Poser \(\FI(i,\cns a)\neq\FI(i',\cns a)\) pourrait alors permettre de rendre compte du fait que sur un intervalle de temps donné, Alice (\ie\ \cns a) a tellement changé que l'on juge convenable de la représenter par deux individus distincts (et donc successifs) dans le domaine \Unv A.  Quelque chose qui pourrait aller dans ce sens est la possibilité d'énoncer en français des phrases non absurdes comme :

\ex. Alice a beaucoup changé, elle n'est plus la même.


Cette manière de procéder pour modéliser des \emph{changements} n'est pas si 
saugrenue qu'il n'y paraît (et d'ailleurs nous la revisiterons au chapitre~\ref{GN++}, vol.~2), mais nous n'allons pas l'adopter ici pour les raisons suivantes. 
Nous allons considérer que dans un modèle intensionnel, les éléments du domaine \Unv A servent à représenter les individus dans leur \emph{continuité} à travers le temps et les diverses évolutions qui se produisent.  Et lorsqu'un individu nous apparaît comme ayant changé sur une certaine période, ce qui s'est,  en fait, modifié avec le temps ce sont les \emph{propriétés} que possède cet individu --~autrement dit la dénotation des prédicats. Par exemple, si à $i$ Alice est une enfant, et que $i'$ se situe suffisamment d'années plus tard pour qu'à ce moment Alice soit maintenant une adulte, le modèle nous donnera : 
\(\Obj{Alice}\in \FI(i,\prd{enfant})\),
\(\Obj{Alice}\not\in \FI(i,\prd{adulte})\)
et
\(\Obj{Alice}\not\in \FI(i',\prd{enfant})\),
\(\Obj{Alice}\in \FI(i',\prd{adulte})\).  
Mais \sicut{Alice} et \cns a dénoteront toujours l'individu \Obj{Alice}.
 
Par conséquent, nous allons donc contraindre tout modèle intensionnel temporel à l'aide d'un postulat qui dit que toute constante d'individu dénote toujours le même individu, quel que soit l'instant où l'on se place.  
En faisant cela, nous traitons ces constantes comme ce que l'on appelle, en sémantique et en philosophie du langage, des \kwi{désignateurs rigides}{designateur
rigide@désignateur rigide}\footnote{Le terme (et le concept) de \emph{désignateur rigide} a été introduit par \citet{Kripke:72}\Andexn{Kripke, S.} essentiellement en relation avec le traitement des modalités. Dans ce contexte, il prend une valeur et une interprétation philosophiques un peu différente ; mais \alien{mutatis mutandis}, appliqué à la temporalité, sa définition reste la même.}\label{drigide1}.

\begin{postu}[Désignateurs rigides (1)]\label{postu:DR1}
Soit un modèle $\Modele=\tuple{\Unv A,\Tps_{\tprec},\FI}$. Pour toute constante $\vrb\alpha\in\CON_0$ et pour toute paire d'instants $i,i'\in \Tps$, 
\(\FI(i,\vrb\alpha)=\FI(i',\vrb\alpha)\).
\\
On dit alors que les constantes d'individus sont des \kwo{désignateurs rigides}.
\end{postu}


Mais notons que notre définition d'un modèle intensionnel nous permet tout de même de tolérer, à profit, des exceptions à ce postulat.
Souvenons-nous de ce que nous avons vu en \S\ref{ss:re/dicto}.  
Si nous posons que la constante \cns m traduit le {\GN} français \sicut{Miss France}, alors il devient intéressant d'avoir, par exemple, \(\FI(i,\cns m)=\Obj{Gourdella}\) et \(\FI(i',\cns m)=\Obj{Crucheline}\). Techniquement, pour intégrer de telles exceptions, il suffit de distinguer un sous-ensemble de $\CON_0$ contenant les constantes «non rigides» de {\LO} et de dire que le postulat ne s'applique qu'à $\CON_0$ privé de ce sous-ensemble. 

Enfin, le postulat~\ref{postu:DR1} a une conséquence non triviale sur notre modélisation temporelle du monde. Si chaque constante a toujours la même dénotation quel que soit l'instant choisi, cela semble impliquer que tout individu du domaine (du moins ceux nommées par des constantes) «existe» éternellement dans le modèle.  De prime abord, c'est plutôt contre-intuitif : rares sont les individus qui existent de toute éternité ; aussi bien les choses que les êtres animés apparaissent ou naissent un jour puis meurent et disparaissent après avoir fait leur temps.
Doit-on alors s'inquiéter de ce que $\Modele$ nous donnera toujours $\FI(i,\cns a) = \Obj{Alice}$, même pour $i$ situé en $-$400 avant J.C. ou au \textsc{xxv}\ieme siècle ? Eh bien pas vraiment. 
Car nous aurons intérêt à faire la différence entre la \emph{présence} d'un individu dans le modèle et l'existence concrète d'un individu dans le monde. 
Le fait qu'un individu est présent dans le modèle (\ie\ qu'il est un élément de \Unv A) est lié à la possibilité d'y faire référence dans des énoncés ; et la référence est en quelque sorte intemporelle : la langue nous permet toujours de référer à un individu à partir de n'importe quel point de repère temporel, y compris d'un point de repère où cet individu n'existe pas (ou n'existe plus ou pas encore). Par exemple, je pense que l'on est en droit de juger que les phrases suivantes sont vraies :

\ex. \label{x:MozartPresent}
\a. Mozart est l'auteur de \emph{La Flûte Enchantée}.
\b. Mozart est un compositeur autrichien.

Cela montre que l'on peut non seulement (et bien évidemment) faire référence à Mozart aujourd'hui (plus de deux siècles après sa mort), mais aussi que l'on peut dire certaines choses (mais pas toutes !) à son sujet \emph{au présent}\footnote{Et il serait erroné de voir dans \ref{x:MozartPresent} un cas de présent de narration à interpréter comme une variante stylistique d'un temps du passé.}.  En pratique, cela veut dire que pour un instant $i_1$ qui se situe à notre époque, et si \cnsi m0 est la traduction de \sicut{Mozart}, alors $\FI(i_1,\cnsi m0)$ est bien définie par le modèle et vaut \Obj{Mozart}, mais également que, par exemple,  $\FI(i_1,\prd{compositeur})$ est un ensemble qui contient \Obj{Mozart}. 

\newpage

\sloppy 

La notion d'existence dans le monde, quant à elle, est corrélée, là encore, à certaines propriétés, comme celles d'être vivant, d'être localisé spatialement quelque part, etc.  Et ces propriétés correspondent à des prédicats dont la dénotation change avec le temps. Et bien sûr \(\Obj{Mozart}\not\in\FI(i_1,\prd{vivant})\).  Par conséquent, on peut figurer (éternellement) dans le modèle (c'est la bonne nouvelle), sans pour autant exister matériellement dans le monde sur de très longues périodes.

\fussy


\is{temps|)}


\subsection{Opérateurs temporels}
%--------------------------------
\is{operateur@opérateur!\elid\ temporel|(}

Nous venons de temporaliser la notion de modèle ;
c'est-à-dire que nous pouvons (et même devons)  tenir compte
du temps qui passe dans notre représentation du monde.  Mais il nous faut
maintenant nous donner les 
moyens de parler du passé, du présent et du futur dans {\LO},
c'est-à-dire de refléter dans le système formel une certaine
contribution (approximative et simplifiée) des temps grammaticaux du
français.  
%Nous réalisons cela au moyen d'opérateurs temporels qui, bien entendu, ont une syntaxe et une interprétation.


\subsubsection{Syntaxe}
%''''''''''''''''''''''

Même si, dans une langue comme le français, ce sont les verbes (donc des prédicats) qui portent les marques de flexion\is{flexion}
temporelle,  nous allons considérer  que, dans {\LO}, le rôle sémantique des
temps concernent les formules.
%les phrases (simples, \ie\ propositions grammaticales), donc les
%formules.  
Nous dirons ainsi qu'une formule (et par extension une phrase) est au présent, au passé ou au futur.

Le temps est représenté dans {\LO} à l'aide d'opérateurs qui se placent
devant une formule (un peu comme la négation $\Xlo\neg$).  Ce sont des
opérateurs temporels, nous allons en utiliser principalement deux : {\Xlo\mP} pour le passé et
{\Xlo\mF} pour le futur%
\footnote{
Attention : ne pas confondre {\Xlo\mF} de futur et \FI, la fonction d'interprétation.}.
Nous les intégrons à la syntaxe de {\LO} en les faisant figurer dans l'ensemble des symboles logiques du vocabulaire et en ajoutant la règle suivante qui vient compléter la définition~\ref{SynP} du chapitre~\ref{LCP} (déjà augmentée par la règle (\RSyn\ref{SynPatoi}) au chapitre~\ref{ch:gn}).
%Donc on doit modifier (\ie\ augmenter) la syntaxe de {\LO}, en ajoutant
%les règles qui insèrent ces opérateurs.

\begin{defi}[Syntaxe de {\mP} et {\mF}]
\begin{enumerate}[resume*=RglSyn1] %[(\RSyn1)]{\setcounter{enumi}{\value{RglSynt}}}
\item Si $\Xlo\phi$ est une formule bien formée de {\LO}, alors $\Xlo\mP \phi$
et $\Xlo\mF \phi$ le sont aussi.
\label{SynPPF}
\setcounter{RglSynt}{\value{enumi}}
\end{enumerate}
\end{defi}


Notons au passage que {\Xlo\mP} et {\Xlo\mF}
permettent l'application de la 
définition~\ref{d:portéeOp} vue au chapitre~\ref{ch:gn}
(p.~\pageref{d:portéeOp}), sur la portée d'un opérateur.  Ainsi par
exemple dans $\Xlo\mP\phi$,  on pourra dire que la sous-formule $\vrb\phi$ est
la portée de cette occurrence de $\mP$.


$\Xlo\mP \phi$ est une formule dont on trouvera la dénotation en
allant chercher dans le passé ;  et $\Xlo\mF \phi$ en allant chercher dans
le futur. Ainsi une façon de prononcer explicitement $\Xlo\mP \phi$ et $\Xlo\mF \phi$ peut être, respectivement «il a été vrai que $\Xlo\phi$~» (ou «~\vrb\phi\ a été vraie») et «il sera vrai que $\Xlo\phi$~» (ou «~\vrb\phi\ sera vraie»). 
Une phrase au présent sera traduite simplement par $\vrb\phi$, sans
opérateur temporel.
Par exemple, nous pouvons maintenant produire les traductions sémantiques suivantes :

\ex.
\a. Alice a dormi.
\\\(\Xlo\mP\prd{dormir}(\cns{a})\)
\b. Alice dormira.
\\\(\Xlo\mF \prd{dormir}(\cns{a})\)
\b. Alice dort.
\\\(\Xlo\prd{dormir}(\cns{a})\)



Bien sûr, rien n'empêche la règle (\RSyn\ref{SynPPF}) de s'appliquer récursivement, sur elle-même ou en interaction avec d'autres règles de la syntaxe de {\LO}. Nous pouvons donc aussi écrire des formules comme : \(\Xlo\mP\mP\mP\prd{dormir}(\cns{a})\), \(\Xlo\mF\mP\prd{dormir}(\cns{a})\), \(\Xlo\mP\mF\mP\mF\prd{dormir}(\cns{a})\), \(\Xlo\mF\neg\prd{dormir}(\cns{a})\), \(\Xlo\mP\neg\mF\prd{dormir}(\cns{a})\), etc.
De même, dans une formule complexe, qui comporte plusieurs sous-formules, les opérateurs temporels peuvent se placer à divers endroits, comme le montrent les exemples~\Next.



\ex.
\a. \(\Xlo\mP\exists x [\prd{enfant}(x) \wedge \prd{dormir}(x)]\)
\b. \(\Xlo\exists x \mP[\prd{enfant}(x) \wedge \prd{dormir}(x)]\)
\b. \(\Xlo\exists x [\prd{enfant}(x) \wedge \mP\prd{dormir}(x)]\)


Pour avoir une idée suffisamment claire de ce que signifient toutes ces formules, nous devons immédiatement définir l'interprétation de 
{\Xlo\mP} et {\Xlo\mF} en complétant la sémantique de~{\LO}.

\subsubsection{Sémantique}
%''''''''''''''''''''''''''

\is{passe@passé}\is{futur}\is{present@présent}

On le sait, le passé et le futur sont des notions relatives : dans notre modélisation avec {\Tps}, le passé sera l'ensemble des $i$ qui précèdent l'instant présent, et le futur l'ensemble de ceux qui le suivent. 
Mais peut-on considérer que {\Tps} est en mesure d'identifier un instant particulier qui correspondrait au présent ? Non, car en tant qu'instant élément de {\Tps}, le présent est insaisissable, il avance toujours inexorablement sur la ligne temporelle. On sait aussi que, linguistiquement, le présent coïncide, d'une manière ou d'une autre, avec le moment de l'énonciation.  Et ce moment ne fait pas partie du modèle, mais du contexte ; et jusqu'à présent nous avons pris soin de séparer le contexte du modèle\footnote{À juste titre. Nous y reviendrons dans le chapitre~\ref{Ch:contexte} (vol.~2).}.  Car le modèle sert à représenter idéalement et objectivement le cours de l'histoire du monde, et ce cours reste le même quelle que soit la position du locuteur ou de l'observateur.  
Mais en fait, tout cela n'est pas grave : si aucun instant de {\Tps} n'a de raison de recevoir le privilège d'être \emph{le} présent, cela veut dire que tous le sont potentiellement. En effet, tous les $i$ de $\Tps$ ont été ou seront leur \emph{propre présent}.  Finalement, en tant que perspective temporelle, le présent est lui aussi relatif : pour tout $i$ de $\Tps$, le présent de $i$, c'est $i$.
Et cela convient tout à fait à l'objectif théorique de notre système sémantique qui est de décrire, de façon générale, les conditions de vérité de toute phrase, quel que soit le moment de son énonciation.


Cette idée converge avec le principe fondamental de l'interprétation en sémantique intensionnelle qui est qu'à présent, la dénotation d'une expression se définit et se calcule non seulement par rapport à un modèle et une fonction d'assignation, mais aussi \emph{par rapport à un instant $i$ donné}. Et nos règles sémantiques vaudront, comme il se doit, pour n'importe quel instant de $\Tps$. 
Se faisant, nous multiplions bien, comme voulu, les dénotations potentielles de toute expression.
Cela a une implication immédiate sur notre manière de noter les dénotations :


\begin{nota}[Dénotation en sémantique intensionnelle (1)]
Soit $\vrb\alpha$ une expression de {\LO}.  Et soit $\Modele=\tuple{\Unv A,\Tps_{\tprec},\FI}$ un modèle intensionnel, $i$ un instant de $\Tps$ et $g$ une fonction d'assignation.
\(\denote{\vrb\alpha}^{\Modele,i,g}\) est la dénotation de
$\vrb\alpha$ dans {\Modele}, à l'instant $i$ et relativement à  $g$.
\end{nota}

En relativisant la dénotation à un instant $i$ et en écrivant \(\denote{\vrb\alpha}^{\Modele,i,g}\), nous faisons jouer à $i$ le rôle de ce que l'on appelle un \kwo{indice intensionnel}\is{indice!\elid\ intensionnel}, ou plus simplement un indice (cf.\ \alien{supra} p.~\pageref{cf:indices}).  Un indice est, par définition, un paramètre nécessaire au calcul sémantique, il sert de point de référence, en identifiant l'état du monde concerné par l'interprétation. Dans les calculs de \(\denote{\vrb\alpha}^{\Modele,i,g}\), nous appellerons généralement $i$ l'instant d'évaluation.

Habituellement, si \vrb\phi\ est une formule qui traduit une phrase complète, quand nous calculerons \(\denote{\vrb\phi}^{\Modele,i,g}\), nous considérerons que, par défaut, $i$ correspond au moment de l'énonciation. C'est ce qu'il y a de plus naturel, mais il faut savoir que cette hypothèse n'est pas nécessaire et qu'elle se fait extérieurement au système sémantique (c'est une sorte de complément pragmatique aux règles d'interprétation de \LO). 

L'ajout de $i$ ne change rien aux règles sémantiques de {\LO} que nous avons vues jusqu'ici, si ce n'est que maintenant elles se formulent avec \(\denote{\cdot}^{\Modele,i,g}\). Par exemple, les conditions de vérité de \sicut{Alice dort} s'explicitent en :
\(\denote{\prd{dormir}(\cns a)}^{\Modele,i,g}=1\) ssi
\(\denote{\cns{a}}^{\Modele,i,g} \in
\denote{\prd{dormir}}^{\Modele,i,g}\), ssi \(\FI(i,\cns{a}) \in
\FI(i,\prd{dormir})\), etc.  Les règles (\RSem\ref{RIprd2})--(\RSem\ref{RIatoi}) traînent un indice $i$ qui reste toujours le même. Mais bien sûr cela change avec (\RSem\ref{RSemTps}) qui interprète {\mP} et {\mF} :


\begin{defi}[Interprétation de {\mP} et {\mF}]\label{Def:SemPF}
\begin{enumerate}[sem,resume=RglSem2] %[(\RSem1)]{\setcounter{enumi}{\value{RglSem}}}
\item \label{RSemTps}
\begin{enumerate}
\item \(\denote{\Xlo\mP\phi}^{\Modele,i,g}=1\), ssi il existe un instant
$i'$ de {\Tps} tel que $i' \tprec i$ et \(\denote{\Xlo\phi}^{\Modele,i',g}=1\).
\item \(\denote{\Xlo\mF\phi}^{\Modele,i,g}=1\), ssi il existe un instant
$i'$ de {\Tps} tel que $i \tprec i'$ et \(\denote{\Xlo\phi}^{\Modele,i',g}=1\).
\end{enumerate}
\setcounter{RglSem}{\value{enumi}}
\end{enumerate}
\end{defi}

La règle (a) dit que
$\Xlo\mP\phi$ %(<<~passé de $\phi$~>>) 
est vraie à l'instant $i$ ssi il y a
un instant antérieur à $i$ 
%($i'$) 
où $\Xlo\phi$ est vraie ; donc ssi $\Xlo\phi$
est vraie dans le passé de $i$.
De même, la  règle (b) dit que $\Xlo\mF\phi$ %(<<~futur de $\phi$~>>)
est vrai à l'instant $i$ ssi il y a un instant postérieur à $i$ %($i'$)
où $\Xlo\phi$ est vraie ; donc ssi $\Xlo\phi$ est vraie dans le futur 
%(ou l'avenir) 
de $i$.

Sémantiquement, les opérateurs {\mP} et {\mF} font donc simplement du décalage temporel, mais avec un aller-retour. On peut les voir comme de petites machines à voyager (mentalement !) dans le temps. Par exemple, pour vérifier que $\Xlo\mP\phi$ (\ie\ «~\vrb\phi\ au passé») est vraie à l'instant $i$, on se transporte dans un instant du passé de $i$ ; là on constate que \vrb\phi\ (donc «~\vrb\phi\ au présent») est vraie --~cela on sait le faire depuis le chapitre~\ref{LCP} ; puis on revient à $i$ pour conclure que «~\vrb\phi\ au passé» est vraie.

Illustrons cela en reprenant le modèle $\Modele$ vu précédemment en \ref{x:ModeleI}, répété ici, avec $i_1\tprec i_2 \tprec i_3\tprec i_4$ :
%\(\Modele = \tuple{\Unv{A},\Tps_{\tprec},\FI}\) avec
%\(\Unv{A} = \set{\Obj{Alice}; \Obj{Bruno}; \Obj{Charles};\Obj{Dina}}\),
% \(\Tps = \set{i_1;i_2;i_3;i_4}\) et
% \(\mathord{\tprec} =
% \set{\tuple{i_1,i_2};\tuple{i_1,i_3};\tuple{i_1,i_4};\tuple{i_2,i_3};\tuple{i_2,i_4};\tuple{i_3,i_4}}\)%
%% \footnote{Cet ensemble est ce que l'on appelle le \kwo{graphe} de la
%%   relation d'ordre {\tprec}.  C'est une manière rigoureuse, bien que
%%   laborieuse, de 
%%   définir une relation.  Pour cet exemple, l'ordre donné
%%   correspond simplement à : \(i_1 \tprec i_2 \tprec i_3 \tprec i_4\).}.

\ex.[\ref{x:ModeleI}]
\(
\FI(i_1,\prd{dormir}) = \set{\Obj{Bruno};\Obj{Dina}}
\)
\\
\(
\FI(i_2,\prd{dormir}) = \set{\Obj{Alice};\Obj{Dina};\Obj{Charles}}
\)
\\
\(
\FI(i_3,\prd{dormir}) = \set{\Obj{Charles}}
\)
\\
\(
\FI(i_4,\prd{dormir}) = \Evide
\)


\sloppy

Si nous voulons évaluer \(\Xlo\mP\prd{dormir}(\cns{a})\) par rapport à $i_3$ par exemple, la règle (\RSem\ref{RSemTps}) nous dit que \(\denote{\Xlo\mP
\prd{dormir}(\cns{a})}^{\Modele,i_3,g}=1\), ssi il existe un instant $i$ tel que
$i \tprec i_3$ et
\(\denote{\Xlo\prd{dormir}(\cns{a})}^{\Modele,i,g}=1\).
L'instant $i_2$ fait l'affaire, puisque nous voyons que \(\denote{\Xlo\prd{dormir}(\cns{a})}^{\Modele,i_2,g}=1\). Donc \(\denote{\Xlo\mP
\prd{dormir}(\cns{a})}^{\Modele,i_3,g}=1\) : il y a bien dans le passé de $i_3$ un instant où Alice dort, c'est $i_2$.
Pour les mêmes raisons, \(\denote{\Xlo\mP\prd{dormir}(\cns{a})}^{\Modele,i_4,g}=1\), car $i_2 \tprec i_4$.
En revanche, \(\denote{\Xlo\mP\prd{dormir}(\cns{a})}^{\Modele,i_2,g}=0\), car
il n'y a pas ici d'instant antérieur à $i_2$ qui fasse l'affaire (le
seul instant plus ancien que $i_2$, c'est $i_1$, et Alice ne dort pas à $i_1$).

\fussy

Pour \(\Xlo\mF\prd{dormir}(\cns{a})\), la formule n'est vraie que pour
$i_1$ (grâce à $i_2$ qui est dans le futur de $i_1$). Et \(\Xlo\mF\prd{dormir}(\cns{c})\) est vraie à $i_1$ et à $i_2$, mais pas à $i_3$.


%% exo

\smallskip

% -*- coding: utf-8 -*-
\begin{exo}\label{exo:PF}
Toujours à partir du modèle $\Modele$ donné en \ref{x:ModeleI}, \pagesolution{crg:PF}
calculez :\addtolength{\multicolsep}{-9pt}
\begin{multicols}{2}
\begin{enumerate}
\item \(\denote{\Xlo\mP\prd{dormir}(\cns d) \wedge \mF\prd{dormir}(\cns d)}^{\Modele,i_3,g}\)
\item \(\denote{\Xlo\mP[\prd{dormir}(\cns b) \wedge \prd{dormir}(\cns c)]}^{\Modele,i_4,g}\)
\item \(\denote{\Xlo\mP\prd{dormir}(\cns b) \wedge \mP\prd{dormir}(\cns c)}^{\Modele,i_4,g}\)
\item \(\denote{\Xlo\prd{dormir}(\cns c) \implq \mF\prd{dormir}(\cns c)}^{\Modele,i_2,g}\)
\item \(\denote{\Xlo\mF\neg\exists x\,\prd{dormir}(x)}^{\Modele,i_2,g}\)
\item \(\denote{\Xlo\neg\exists x\,\mF\prd{dormir}(x)}^{\Modele,i_2,g}\)
\end{enumerate}
\end{multicols}

\smallskip

\noindent
Quelle est la meilleure manière de traduire dans {\LO}  \sicut{tout le monde a dormi} ?

\begin{solu}(p.~\pageref{exo:PF})\label{crg:PF} 

Nous appliquons les règles (\RSem\ref{RSemTps}) de la définition \ref{Def:SemPF} p.~\pageref{RSemTps} qui, en substance, disent que  \(\denote{\Xlo\mP\phi}^{\Modele,i,g}=1\) ssi il y a un instant $i'$ avant $i$ auquel \vrb\phi\ est vraie et \(\denote{\Xlo\mF\phi}^{\Modele,i,g}=1\) ssi il y a un instant $i'$ après $i$ auquel \vrb\phi\ est vraie.

\begin{enumerate}
\item \(\denote{\Xlo\mP\prd{dormir}(\cns d) \wedge \mF\prd{dormir}(\cns d)}^{\Modele,i_3,g}=0\) parce que $\Xlo\mF\prd{dormir}(\cns d)$ est faux à $i_3$ (personne ne dort à $i_4$).

\item \(\denote{\Xlo\mP[\prd{dormir}(\cns b) \wedge \prd{dormir}(\cns c)]}^{\Modele,i_4,g}=0\) parce qu'il n'y a pas d'instants avant $i_4$ où Bruno et Charles dorment en même temps.

\item \(\denote{\Xlo\mP\prd{dormir}(\cns b) \wedge \mP\prd{dormir}(\cns c)}^{\Modele,i_4,g} =1\)
parce que $\Xlo\prd{dormir}(\cns b)$ est vraie à $i_1$ (avant $i_4$) et $\Xlo\mP\prd{dormir}(\cns c)$ est vraie à $i_2$ (avant $i_4$).
 
\item \(\denote{\Xlo\prd{dormir}(\cns c) \implq \mF\prd{dormir}(\cns c)}^{\Modele,i_2,g}=1\) parce que Charles dort à $i_2$ et aussi à $i_3$ qui est après $i_2$.

\item \(\denote{\Xlo\mF\neg\exists x\,\prd{dormir}(x)}^{\Modele,i_2,g}=1\) parce que personne ne dort à $i_4$.

\item \(\denote{\Xlo\neg\exists x\,\mF\prd{dormir}(x)}^{\Modele,i_2,g}=0\) parce que Charles dort à $i_3$.

\end{enumerate}

\sloppy

Pour traduire \sicut{tout le monde a dormi}, nous avons le choix entre \(\Xlo\mP\forall x\,\prd{dormir}(x)\) et \(\Xlo\forall x\,\mP\prd{dormir}(x)\).  La première formule dit qu'il existe un moment dans le passé où tout le monde dort ; autrement dit, tout le monde a dormi au même moment, et il n'est pas certain que la phrase du français véhicule cette condition.  La seconde formule dit que pour chaque individu, il y a un moment dans le passé durant lequel il dort ; c'est une traduction plus générale que la première et qui convient probablement mieux aux conditions de vérité de la phrase.

\fussy

\end{solu}
\end{exo}





\subsubsection{Combinaisons d'opérateurs}
%''''''''''''''''''''''''''''''''''''''''

Une bonne manière d'apprivoiser l'interprétation de \mP\ et \mF\ est d'examiner le sens de formules qui combinent plusieurs opérateurs temporels, puisque, comme on l'a vu, (\RSyn\ref{SynPPF}) nous permet d'empiler 
 les $\mP$ et $\mF$ devant une formule
$\vrb\phi$.

Commençons avec $\Xlo\mP\mP\phi$. La sémantique nous dit que 
% 
\(\denote{\Xlo\mP\mP\phi}^{\Modele,i,g}=1\) ssi $\Xlo\mP\phi$ est vraie
à un instant $i'$ avant $i$. Et $\Xlo\mP\phi$ est vraie à $i'$ ssi $\Xlo\phi$ est vraie à un instant $i''$ avant $i'$.  On fait donc deux bonds successifs dans le passé comme l'illustre le schéma en  \ref{f:PP}.
\ex. 
\scalebox{.9}{\begin{pspicture}[shift=*](8,1)
\psline{->}(0,.5)(7,.5)\rput[B](7,.7){\Unv{I}}%
\dotnode[style=instant](1.5,.5){i2}\rput[B](1.5,0){$i''$}%
\dotnode[style=instant](3,.5){i1}\rput[B](3,0){$i'$}%
\dotnode[style=instant](4.5,.5){i0}\rput[B](4.5,0){$i$}%
\psset{nodesep=1.5pt}%
\ncarc[arcangle=-65,linecolor=darkgray]{->}{i0}{i1}%
\ncarc[arcangle=-65,linecolor=darkgray]{->}{i1}{i2}%
\end{pspicture}%
}\label{f:PP}

Ainsi, $\mP\mP$ est en quelque sorte un passé de passé, ou un passé dans le passé.  
Pour cette raison, on a pu considérer que ce «double \mP~» correspondait à la contribution temporelle de temps composés comme, en français, le plus-que-parfait \ref{x:PPa}\is{plus-que-parfait}, ou  le passé antérieur \ref{x:PPb}, voire 
un passé 
surcomposé comme \ref{x:PPc}. 


\ex.
\a. Alice avait dormi.\label{x:PPa}
\b. Alice eut dormi.\label{x:PPb}
\c. Alice a eu dormi.\label{x:PPc}


%\input{fig/PP.pstex_t}



Pour $\Xlo\mF\mP\phi$, il faut faire attention : on interprète toujours les formules en les décomposant rigoureusement selon la règle (\RSyn\ref{SynPPF}) ; autrement dit, on commence toujours par interpréter l'opérateur le plus à gauche. 
%est un passé dans le futur.  Une telle formule est vraie
Donc $\Xlo\mF\mP\phi$ est vraie à l'instant $i$ ssi $\Xlo\mP\phi$ est 
vraie à un instant $i'$ postérieur à $i$.  Ceci est vrai ssi $\Xlo\phi$
est vraie à un instant $i''$ 
antérieur à $i'$.  Cette fois-ci, on fait d'abord un bond dans le futur, puis un bond qui recule dans le passé. Notons que ce deuxième bond, en arrière, peut nous ramener aussi bien avant qu'après l'instant de départ $i$, comme le montre le schéma \ref{f:FP}.

\ex.
\scalebox{.9}{\begin{pspicture*}[shift=-.5](2,0)(7.5,1.3)
\psline{->}(0,.5)(7,.5)\rput[B](7,.7){\Unv{I}}%
\dotnode[style=instant](4.5,.5){i2}\rput[B](4.5,0){$i''$}%
\dotnode[style=instant](6,.5){i1}\rput[B](6,0){$i'$}%
\dotnode[style=instant](3,.5){i0}\rput[B](3,0){$i$}%
\psset{nodesep=1.5pt}%
\ncarc[arcangle=65,linecolor=darkgray]{->}{i0}{i1}%
\ncarc[arcangle=-45,linecolor=darkgray]{->}{i1}{i2}%
\end{pspicture*}%
}
ou\quad
\scalebox{.9}{\begin{pspicture}[shift=-.5](1,0)(7.5,1.3)
\psline{->}(1,.5)(7,.5)\rput[B](7,.7){\Unv{I}}%
\dotnode[style=instant](2,.5){i2}\rput[B](2,0){$i''$}%
\dotnode[style=instant](6,.5){i1}\rput[B](6,0){$i'$}%
\dotnode[style=instant](3,.5){i0}\rput[B](3,0){$i$}%
\psset{nodesep=1.5pt}%
\ncarc[arcangle=35,linecolor=darkgray]{->}{i0}{i1}%
\ncarc[arcangle=-55,linecolor=darkgray]{->}{i1}{i2}%
\end{pspicture}%
}\label{f:FP}

$\mF\mP$ est donc un passé dans le futur (un passé vu du futur). 
Cela semble correspondre à l'emploi du futur antérieur en français :

\ex. 
 Alice aura dormi.


Inversement, on comprend rapidement que $\Xlo\mP\mF\phi$ est un futur dans le passé. La formule est  vraie à $i$ ssi par
rapport à un 
instant du passé de $i$, il y a un instant futur où $\Xlo\phi$ est vraie.  
On fait d'abord un bond dans le passé, et de là on se dirige vers le futur pour vérifier que $\Xlo\phi$ est vraie \ref{f:PF}.

\ex.
\scalebox{.9}{\begin{pspicture}[shift=-.5](1,0)(7,1.3)
\psline{->}(1,.5)(6.5,.5)\rput[B](6.5,.7){\Unv{I}}%
\dotnode[style=instant](3,.5){i2}\rput[B](3,0){$i''$}%
\dotnode[style=instant](1.5,.5){i1}\rput[B](1.5,0){$i'$}%
\dotnode[style=instant](4.5,.5){i0}\rput[B](4.5,0){$i$}%
\psset{nodesep=1.5pt}%
\ncarc[arcangle=-65,linecolor=darkgray]{->}{i0}{i1}%
\ncarc[arcangle=45,linecolor=darkgray]{->}{i1}{i2}%
\end{pspicture}%
}
ou \quad
\scalebox{.9}{\begin{pspicture*}[shift=-.5](1.4,0)(7.5,1.3)
\psline{->}(1.1,.5)(7,.5)\rput[B](7,.7){\Unv{I}}%
\dotnode[style=instant](5.5,.5){i2}\rput[B](5.5,0){$i''$}%
\dotnode[style=instant](2.2,.5){i1}\rput[B](2.2,0){$i'$}%
\dotnode[style=instant](4.5,.5){i0}\rput[B](4.5,0){$i$}%
\psset{nodesep=1.5pt}%
\ncarc[arcangle=-35,linecolor=darkgray]{->}{i0}{i1}%
\ncarc[arcangle=55,linecolor=darkgray]{->}{i1}{i2}%
\end{pspicture*}%
}
\label{f:PF}

En français, on peut exprimer l'idée d'un futur du passé à l'aide du futur périphrastique en \sicut{aller} conjugué au passé \ref{x:PFa}, ainsi qu'avec certains emplois du conditionnel, comme en \ref{x:PFb}.

\ex.
\a. Alice allait dormir. \label{x:PFa}
\b. Alice ne se doutait pas encore qu'\emph{elle dormirait lorsque le Père Noël passerait}. \label{x:PFb}



Enfin, le double futur 
$\Xlo\mF\mF\phi$ nous fait faire deux bonds successifs dans l'avenir \ref{f:FF}. $\Xlo\mF\mF\phi$ revient à \sicut{il sera vrai qu'il sera vrai que $\Xlo\phi$}. 

\ex.
\scalebox{.9}{\begin{pspicture}[shift=*](8,1.2)
\psline{->}(0,.5)(7,.5)\rput[B](7,.7){\Unv{I}}%
\dotnode[style=instant](4.5,.5){i1}\rput[B](4.5,0){$i'$}%
\dotnode[style=instant](6,.5){i2}\rput[B](6,0){$i''$}%
\dotnode[style=instant](3,.5){i0}\rput[B](3,0){$i$}%
\psset{nodesep=1.5pt}%
\ncarc[arcangle=65,linecolor=darkgray]{->}{i0}{i1}%
\ncarc[arcangle=65,linecolor=darkgray]{->}{i1}{i2}%
\end{pspicture}%
}\label{f:FF}

Le mieux que nous pouvons trouver pour rendre compte, plus ou moins, de cette idée en français serait d'utiliser un futur périphrastique doublé d'une conjugaison au futur :

\ex.  Alice ira dormir.


Nous pourrions également nous «amuser» à développer les conditions de vérités de formules comme  $\Xlo\mP\mF\mF\phi$, $\Xlo\mP\mP\mP\phi$,
$\Xlo\mF\mP\mF\phi$, $\Xlo\mF\mP\mF\mF\phi$, etc.  
Mais cela deviendrait vite artificiel, d'autant plus que non seulement 
il n'existe pas de formulations simples en français qui reflètent le sens de ces combinaisons, mais aussi, je m'empresse de le dire ici, que les analogies présentées ci-dessus entre les temps verbaux\is{temps!\elid\ verbal} et les combinaisons de $\mP$ et $\mF$ ne tiennent pas vraiment la route sémantiquement. La contribution des temps verbaux de la langue est bien plus complexe qu'une simple opération de décalage temporel, nous aurons plusieurs fois l'occasion de revenir sur ce point (notamment au chapitre~\ref{Ch:temps2}, vol.~2).

\label{negPF}
En revanche, il est intéressant d'examiner les interactions des opérateurs temporels avec la négation. 
Par exemple, $\Xlo\mP\neg\phi$ est vraie à $i$, ssi il existe un instant du passé de $i$ où $\Xlo\phi$ est fausse. 
Autrement dit, $\Xlo\mP\neg\phi$ signifie que «~$\Xlo\phi$ a été fausse».
De même,  $\Xlo\mF\neg\phi$ signifie que «~$\Xlo\phi$ sera fausse» (à un moment dans le futur de l'instant d'évaluation).  
À ne surtout pas confondre avec $\Xlo\neg\mP\phi$ et $\Xlo\neg\mF\phi$. 
$\Xlo\neg\mP\phi$ est vraie à l'instant $i$, ssi \emph{il n'existe pas} d'instant antérieur à $i$ où $\Xlo\phi$ est vraie. Cela signifie donc que $\Xlo\phi$ n'a jamais été vraie dans le passé. Et $\Xlo\neg\mF\phi$ que $\Xlo\phi$ ne sera jamais vraie dans le futur.
Partant, quelle est l'interprétation de $\Xlo\neg\mP\neg\phi$ et de $\Xlo\neg\mF\neg\phi$ ?

Avant de répondre à cette question, prenons le temps de faire un petit détour en revenant sur les règles (\RSem\ref{RSemTps}).  Leur formulation nous montre bien qu'il s'agit en fait de règles de quantification, et plus précisément des règles de quantification existentielle sur les instants. Rien ne nous empêche alors d'envisager aussi de la quantification universelle sur les instants. C'est d'ailleurs ce que fait la logique temporelle de \citet{Prior:67}, en introduisant deux autres opérateurs temporels {\mH} et {\mG} qui sont respectivement les duaux universels de {\mP} et {\mF}. Leurs interprétations sont donc les suivantes : \(\denote{\Xlo\mH\phi}^{\Modele,i,g}=1\) ssi pour tout instant $i'$ tel que $i'\tprec i$, \(\denote{\Xlo\phi}^{\Modele,i',g}=1\), et 
\(\denote{\Xlo\mG\phi}^{\Modele,i,g}=1\) ssi pour tout instant $i'$ tel que $i\tprec i'$, \(\denote{\Xlo\phi}^{\Modele,i',g}=1\).
Ce sont des opérateurs très forts ; par exemple $\Xlo\mG\phi$ signifie que $\Xlo\phi$ sera toujours et définitivement vraie dans le futur de l'instant d'évaluation, et la formule $\Xlo[\phi \wedge \mG\phi \wedge \mH\phi]$ nous permet de dire que $\Xlo\phi$ est une vérité éternelle.  Pour cette raison, $\mH$ et $\mG$ ne sont pas d'une grande utilité pratique pour l'analyse sémantique de phrases de la langue. Mais le fait est que $\Xlo\mH\phi$ et $\Xlo\mG\phi$ sont respectivement équivalentes à $\Xlo\neg\mP\neg\phi$ et $\Xlo\neg\mF\neg\phi$.


\smallskip

% -*- coding: utf-8 -*-
\begin{exo}\label{exo:PPP}
En supposant un modèle suffisamment réaliste \pagesolution{crg:PPP}
(c'est-à-dire avec {\Tps}
comprenant un très grand nombre d'instants), comparez les conditions
de vérité de $\Xlo\mP\phi$ et $\Xlo\mP\mP\mP\phi$ (en particulier en vous intéressant à ce qui distingue sémantiquement ces deux formules). 
\begin{solu}(p.~\pageref{exo:PPP})\label{crg:PPP} 

Commençons par poser les conditions
de vérité de $\Xlo\mP\phi$ et $\Xlo\mP\mP\mP\phi$ en appliquant la règle (\RSem\ref{RSemTps}) (déf. \ref{Def:SemPF} p.~\pageref{RSemTps}).
\begin{itemize} 
\item \(\denote{\Xlo\mP\phi}^{\Modele,i,g}=1\) ssi il existe un instant $i'$ tel que $i'\tprec i$ et \(\denote{\Xlo\phi}^{\Modele,i',g}=1\).

\item \(\denote{\Xlo\mP\mP\mP\phi}^{\Modele,i,g}=1\) ssi il existe trois instants $i'$, $i''$ et $i'''$ tels que $i'''\tprec i'' \tprec i' \tprec i$ et \(\denote{\Xlo\phi}^{\Modele,i''',g}=1\). En effet, $\Xlo\mP\mP\mP$ nous fait faire trois bonds dans le passé, ce qui peut donc se résumer par cette formulation de conditions de vérité.
\end{itemize}

À partir de là, on constate rapidement que \(\xlo{\mP\mP\mP\phi}\satisf\xlo{\mP\phi}\).  Si $\Xlo\mP\mP\mP$ est vraie à $i$, alors $\Xlo\phi$ est vraie à $i'''$, or $i'''$ est un instant du  passé de $i$. Donc il existe bien un instant antérieur à $i$ où $\Xlo\phi$ est vraie, ce qui fait que $\Xlo\mP\phi$ est vraie à $i$. Autrement dit : si l'on fait trois bonds dans le passé pour aller vérifier $\Xlo\phi$, on peut tout aussi bien le faire avec un seul grand bond -- qui couvre les trois précédents. 

En revanche a-t-on \(\xlo{\mP\phi}\satisf\xlo{\mP\mP\mP\phi}\) ? Pour s'en assurer, cherchons un contre-exemple. Il s'agirait d'un cas (et même d'un instant $i$) où $\Xlo\mP\phi$ est vraie et $\Xlo\mP\mP\mP\phi$ est fausse. Pour cela, il faut que {\Tps} comporte un instant $i'$ tel que 1) $i'\tprec i$, 2) $\Xlo\phi$ est vraie à $i'$, 3) il existe \emph{au maximum} un instant intercalé entre $i'$ et $i$  (mais pas deux !), et 4) $\Xlo\phi$ n'est vraie à aucun instant antérieur à $i'$.  C'est donc un cas de figure très particulier où $\Xlo\phi$ n'est vraie que dans un passé très proche de $i$.  

Autrement dit, dans tous les cas où $\Xlo\mP\mP\mP\phi$ est vraie, $\Xlo\mP\phi$ est vraie, et dans \emph{presque} tous les cas où $\Xlo\mP\phi$ est vraie, $\Xlo\mP\mP\mP\phi$ est vraie aussi. «Presque» tous les cas ne suffit pas pour conclure à une équivalence logique, mais on n'en est pas loin.  D'autant plus que si on ajoute l'hypothèse que {\Tps} contient une infinité d'instants et surtout que $\tprec$ est un ordre dense, cela implique, par définition de la densité, que pour toute paire d'instants de {\Tps}, il en existe toujours un troisième (et donc une infinité) situé entre les deux. Et dans ce cas la condition 3 ci-dessus ne peut pas être vérifiée ; il n'y a pas de contre-exemple à \(\xlo{\mP\phi}\satisf\xlo{\mP\mP\mP\phi}\) et les deux formules sont alors sémantiquement équivalentes. Si on ne pose pas cette hypothèse, elles ne sont pas équivalentes, mais elles sont sémantiquement très proches.
\end{solu}
\end{exo}


\medskip

Terminons par une petite observation qui se rapporte à ce que nous avons vu en \S\ref{ss:ModNonExt}.
De par leur sémantique, $\mP$ et $\mF$ sont aussi des candidats pour traduire la contribution d'adjectifs intensionnels comme \sicut{ancien} et \sicut{futur}.  
Nous pouvons, par exemple, proposer les traductions suivantes :

\ex.
\a. Jean est un futur Prix Nobel.\\
\(\Xlo\mF\prd{prix-nobel}(\cns j)\)
\b. Tous les patients sont d'anciens alcooliques.\\
\(\Xlo\forall x [\prd{patient}(x)\implq \mP\prd{alcoolique}(x)]\)

\Last[a] a ainsi les mêmes conditions de vérité que \sicut{Jean sera Prix Nobel}, ce qui semble assez vraisemblable. On peut néanmoins tenter de perfectionner un peu ces traductions : par exemple, pour dire que \vrb x est un ancien alcoolique, il est peut-être plus précis d'écrire \(\Xlo[\mP\prd{alcoolique}(x)\wedge\neg\prd{alcoolique}(x)]\) (ajoutant ainsi qu'\vrb x \emph{n'est plus} alcoolique\footnote{Pour vérifier que cette deuxième condition relève bien des conditions de vérité (et pas d'une implicature par exemple), on peut examiner l'effet de la négation. Or \sicut{Pierre n'est pas un ancien alcoolique} semble bien signifier soit qu'il n'a jamais bu, soit qu'il boit encore, ce qui est effectivement la négation de la conjonction \(\Xlo[\mP\prd{alcoolique}(\cns p)\wedge\neg\prd{alcoolique}(\cns p)]\).}).  Et ces adjectifs nous aident également à nous faire une idée de la portée des opérateurs temporels introduits par les temps verbaux.\is{temps!\elid\ verbal} Par exemple la traduction la plus plausible de {\Next} est certainement \Next[a] plutôt que \Next[b].

\ex.
Marie a épousé un ancien footballeur. 
\a. \(\Xlo\mP\exists x [\prd{épouser}(\cns m,x)\wedge[\mP\prd{footballeur}(x)\wedge \neg\prd{footballeur}(x)]]\)
\b. \(\Xlo\exists x [\mP\prd{épouser}(\cns m,x)\wedge[\mP\prd{footballeur}(x)\wedge \neg\prd{footballeur}(x)]]\)

\Last[a] signifie que le mari de Marie était déjà ancien footballeur lorsqu'elle l'a épousé, alors que  \Last[b] dit simplement qu'il n'est plus footballeur aujourd'hui sans exclure qu'il ait cessé de l'être \emph{après} son mariage.  
%Or il semble que \Last\ s'interprète selon \Last[a] et non selon \Last[b]. 
Cela tendrait à montrer que la contribution temporelle des temps verbaux porte sur toute la proposition (ou au moins sur le groupe verbal) et pas seulement sur le prédicat qui traduit le verbe.


\is{operateur@opérateur!\elid\ temporel|)}



\subsection{Problèmes}
%---------------------
\label{ss:PbTmps}

Dans cette dernière section, nous allons passer en revue quelques problèmes que peut (éventuellement) poser, pour l'analyse sémantique, cette formalisation de la temporalité en termes de {\mP} et {\mF}. 


Une première critique que l'on peut adresser à cette sémantique temporelle concerne la manière dont $\Tps_{\tprec}$ modélise le cours du temps et en particulier le futur.\is{futur} $\Tps$ a une structure linéaire et $\Modele$ est défini de telle sorte que pour tout instant $i$ du futur, l'état du monde y est parfaitement et définitivement déterminé. Or peut-on vraiment se permettre de présumer que le modèle connaît l'avenir ? 

À cette question, plusieurs éléments de réponses peuvent être apportés, et je pense qu'au bout du compte, cette critique ne pose pas vraiment de problème compromettant pour le système sémantique. Il faut bien avoir en tête qu'un modèle est avant tout un outil théorique qui nous sert à formaliser des conditions de vérité.  Ce qui est donc primordial pour nous, c'est de savoir si nos règles d'interprétation (notamment pour \mF) sont suffisamment conformes à ce que la langue nous permet de faire avec sa sémantique.  Un modèle, par définition, contient tout ce à quoi on peut faire référence, et évidemment la langue ne nous interdit nullement de faire référence à l'avenir --~même si celui-ci n'existe pas encore et reste très incertain. Lorsqu'un locuteur affirme une phrase comme \sicut{Alice te téléphonera après-demain dans l'après-midi}, il accomplit une assertion d'un type particulier : c'est une sorte de prédiction, et se faisant, il s'engage de manière un peu audacieuse en posant une description précise (bien que partielle) de l'état du monde à un instant à venir. Il se trompe peut-être, sur le moment nous ne pouvons pas le vérifier, l'avenir nous le dira. Mais cela ne remet pas fondamentalement en cause les conditions de vérité de $\Xlo\mF\phi$ ; elles semblent un peu irréalistes parce qu'elles nous demandent de nous projeter dans l'avenir pour vérifier que $\Xlo\phi$ est vraie, et nous ne pouvons par faire un tel \emph{saut} temporel. Mais ce n'est pas si grave. En pratique, pour évaluer une phrase au passé, il nous faut de la mémoire ou des archives ; pour évaluer une phrase au futur, il nous faut... de la patience. Autrement dit, l'évaluation effective de \(\denote{\Xlo\mF\phi}^{\Modele,i,g}\) ne peut se faire qu'a posteriori (par rapport à $i$), lorsque l'instant où $\Xlo\phi$ doit être vérifiée sera devenu le présent (ou le passé). À ce moment là, l'état du monde sera univoquement défini, comme ce que propose \Modele, et le calcul de \(\denote{\Xlo\mF\phi}^{\Modele,i,g}\) se fera bien conformément à la règle (\RSem\ref{RSemTps}b). 

Et n'oublions pas qu'ordinairement, dans la pratique langagière, nous utilisons rarement les conditions de vérité pour vérifier qu'une formule est vraie dans un modèle ; nous faisons plutôt l'inverse, c'est-à-dire admettre, ou poser, qu'une formule est vraie pour ensuite apprendre des informations sur le monde (et donc sur le modèle).  Prise de cette manière, (\RSem\ref{RSemTps}b) nous permet simplement d'envisager (voire d'hypothéquer) l'avenir en «découvrant» ce que le modèle prédestine. 

Enfin, il faut reconnaître que dès que nous voulons intégrer formellement la dimension temporelle dans notre modèle, nous sommes bien obligés de définir (au moins théoriquement) un état du monde pour tout instant $i$, puisque, comme nous l'avons vu, tous les $i$ de $\Tps$ sont potentiellement un présent, et donc aussi un passé et un futur (par rapport à d'autres instants). $\Tps$ présente une vision absolue du temps\footnote{Et en quelque sorte cette vision est newtonienne.} et dans $\Tps$ tous les instants sont logés à la même enseigne.


Cependant, il ne faut pas négliger pour autant cette propriété particulière que nous venons de voir, à savoir qu'une phrase au futur ne peut pas, en général, être évaluée au moment de son énonciation. 
Cela montre que le futur grammatical n'est pas un simple symétrique du passé.
On explicite habituellement cette particularité en considérant que le futur est plus qu'un temps et qu'il comporte également une composante \emph{modale} dans sa sémantique.  Et à cet égard, nous verrons, à partir de la section suivante, des éléments de formalisation supplémentaires qui nous permettront de restituer cette idée d'un futur incertain et inconnu sans aller jusqu'à poser qu'il n'existe pas ou qu'il n'est pas défini dans le modèle.

%En bref, le modèle connaît l'avenir -- qui par définition, évidemment, n'existe pas encore. 

\medskip\largerpage[-1]

Le deuxième problème qui se pose à notre sémantique temporelle est un peu plus évident. Il concerne l'expressivité de {\LO} par rapport aux langues naturelles.
Nous savons bien que les temps verbaux\is{temps!\elid\ verbal} dans les langues ont des
valeurs sémantiques beaucoup plus riches et fines que ce que nous
donnent les opérateurs {\mP}, {\mF} et leurs  combinaisons.  
Par exemple, {\mP} ne peut pas rendre compte de la différence sémantique
entre \Next[a] et \Next[b], qui sont toutes deux des phrases au passé :

\ex.
\a. Alice a dormi.
\b. Alice dormait.


Nous savons que ce qui distingue sémantiquement ces deux phrases n'est pas
%tellement 
une question de temporalité\is{temporalité} mais une question
d'\emph{aspect}.\is{aspect}  
Le passé composé et l'imparfait ont des valeurs
aspectuelles différentes, et l'aspect n'est pas ce qui nous dit
quand a lieu tel ou tel événement, mais comment les événements
se déroulent dans le flux temporel, ou plus exactement comment ces
déroulements sont perçus et/ou présentés dans les énoncés. Les
opérateurs {\mP} et {\mF} ne font que du décalage sur {\Tps}, ils
n'ont rien à dire sur les  déroulements et leurs conceptualisations. 

C'est d'ailleurs pour cela que les exemples données dans la section
précédentes sont en fait un peu tendancieux : par exemple, la séquence
{\mP\mP}, en fait, ne reflète pas très bien, en termes de conditions
de vérité, la contribution sémantique du plus-que-parfait ou du passé
antérieur.  D'abord à cause des propriétés aspectuelles de ces temps
verbaux, mais aussi parce que $\Xlo\mP\mP\phi$ a (presque) exactement les
mêmes conditions de vérité que $\Xlo\mP\phi$ (c'était le sens de l'exercice
\ref{exo:PPP}).

Tout cela montre que le système sémantique %(\ie\ le langage \LO) 
est
\emph{insuffisant}. %En ce sens c
Cela n'implique pas forcément que le
système est mauvais : il ne fait pas tout ce que l'on attendrait de lui,
mais cela ne veut pas forcément dire qu'il fait mal ce qu'il fait.  
Et d'ailleurs, j'avais pris soin précédemment (cf. p~\pageref{TpsvsAsp}) d'annoncer que la temporalisation de {\LO} ne concernerait, justement, que la temporalité, et pas l'aspect. 
Nous pourrions encore envisager de réparer cette insuffisance en 
complétant le système, en lui ajoutant de l'expressivité, sans pour autant
supprimer ce qui a été fait dans cette section. 
%Mais nous ne le ferons pas ; 
%d'abord parce que ce serait techniquement un peu trop lourd, et nous verrons
%dans le chapitre~\ref{Ch:temps2} une manière plus efficace et plus commode  

\medskip

Malheureusement, il nous faut lui donner le coup de grâce et
reconnaître qu'il fait de mauvaises prédictions (\ie\ de mauvaises
analyses) : en plus d'omettre de dire des choses que l'on aimerait qu'il
dise, le système dit aussi des choses qu'il ne devrait pas dire.
Une des critiques les plus décisives adressées au système $\mP$-$\mF$
 réside dans ce fameux exemple de
B. Partee\Andex{Partee, B.}\footnote{\citet{Partee:73JoP}. 
%; cité aussi par   \citet{ChierchiaMcCG:90}. 
L'exemple original, en anglais, est :
  \sicut{John didn't turn off the stove}.} :

\ex.  \label{x:gaz1}
Jean n'a pas coupé le gaz.


En fait, le problème qui se pose apparaît de manière systématique
avec des phrases négatives au passé ou au futur.
Essayons de traduire \ref{x:gaz1} dans {\LO} à l'aide de la
sémantique temporelle intensionnelle.
Supposons, pour simplifier, que \sicut{le gaz} se traduit par une
constante, \cns{g}. A priori, deux possibilités s'offrent alors à nous :
\ref{l:gaz1} ou \ref{l:gaz2}.

\ex.  
\a. \label{l:gaz1}
\(\Xlo\neg\mP\prd{couper}(\cns{j},\cns{g})\)
\b. 
\label{l:gaz2}
\(\Xlo\mP\neg\prd{couper}(\cns{j},\cns{g})\)


Laquelle de ces deux formules traduit correctement \ref{x:gaz1} ?  En fait, aucune des deux. Mais il nous faut le
démontrer, en explicitant leurs conditions de vérité.
Pour cela revenons à ce que nous avons vu plus haut (p.~\pageref{negPF}) sur la différence entre $\Xlo\neg\mP\phi$ et $\Xlo\mP\neg\phi$. 
\ref{l:gaz1} est vraie à un instant $i$ ssi \emph{il n'existe pas} d'instant $i'$ antérieur à $i$  où $\Xlo\prd{couper}(\cns j,\cns g)$ est vraie. 
Autrement dit \ref{l:gaz1} signifie que Jean n'a \emph{jamais} coupé le gaz de sa vie. 
Et ce n'est pas le sens de \ref{x:gaz1}, qui n'est pas aussi fort. Imaginons le dialogue suivant, dans le contexte d'un couple, Marie et Jean, sur la route des vacances :

\ex. 
\a.[Marie : ] --- Oh ! tu n'as pas coupé le gaz, andouille !
\b.[Jean : ] --- Si, je l'ai coupé... un jour, en 1996, c'était un mardi matin,
je m'en rappelle très bien...

Jean est d'une mauvaise foi éhontée, et c'est parce qu'il interprète la phrase de Marie comme si elle signifiait \ref{l:gaz1}.


Les conditions de vérité \ref{l:gaz2} \emph{semblent} un peu meilleures. 
Mais en réalité, elles sont encore pires que celles de \ref{l:gaz1}. 
\ref{l:gaz2} est vraie à un instant $i$ ssi il existe un instant $i'$ antérieur à $i$  où $\Xlo\prd{couper}(\cns j,\cns g)$ est fausse.
Quel est le problème ici ?  C'est que \ref{l:gaz2} est trivialement
vraie : il suffit qu'il existe un instant  quelconque du passé où Jean
ne coupe pas le gaz pour que la formule soit vraie.  Et un tel instant
existe, car Jean ne passe pas sa vie à couper le gaz. 
Y compris lorsque Jean a effectivement coupé le gaz : il existe des instants, un peu avant et un peu après, où il n'est pas en train de le faire. 
Dans un tel cas de figure, \ref{x:gaz1} est fausse (il a bien coupé le gaz), mais \ref{l:gaz2} est vraie (car il existe un, et même plusieurs, instants du passé où il ne le fait pas).
Revenons à notre couple, Jean et Marie, mais cette fois avec le dialogue suivant :

\ex. 
\a.[Marie : ] --- Oh ! tu n'as pas coupé le gaz, andouille !
\b.[Jean : ] --- Si si, je l'ai coupé  avant de partir.
\b.[Marie : ] --- Ouais... mais juste avant de le couper, tu ne l'as pas coupé ! Donc j'ai raison !

Là c'est Marie qui est d'une odieuse mauvaise foi : elle interprète (ou réinterprète) sa première phrase comme si elle signifiait \ref{l:gaz2}.

Nous devons donc conclure que notre langage {\LO} n'est pas en mesure de traduire correctement le sens de \ref{x:gaz1} --~car il n'y a pas d'autres possibilités pour placer la négation dans une formule au passé.



Ce que veut dire le locuteur en prononçant \ref{x:gaz1} (dans des
circonstances normales et probes) c'est la chose suivante : 
«je pense à \emph{une certaine période} du passé, et
durant cette période particulière, il n'est pas vrai que Jean coupe le gaz».  
Cela devrait nous rappeler quelque chose... Nous avons vu que $\mP$ et $\mF$ sont en fait des opérateurs de quantification existentielle, un peu comme $\Xlo\exists$, mais qui quantifient sur {\Tps} et non sur {\Unv A}. Et au chapitre~\ref{ch:gn}, \S\ref{ss:RestrDQuant}, nous avions vu qu'en général, dans la langue, nous quantifions rarement sur le domaine complet, mais seulement sur un sous-domaine, restreint contextuellement (ce que nous implémentions au moyen d'un pseudo-prédicat $\Xlo C$).  C'est la même chose qui se passe avec la temporalité et la quantification sur les instants. Et dans la glose ci-dessus, cette \emph{certaine période} joue précisément le rôle de la restriction contextuelle sur le domaine de quantification \Tps. 
C'est aussi le rôle joué par des compléments circonstanciels de temps comme \sicut{ce matin}, \sicut{tout à l'heure}, \sicut{avant de partir}, \sicut{le 7 décembre}, etc.  Or  {\LO} ne nous permet pas plus d'encoder des restrictions implicites pour $\mP$ et $\mF$ que de traduire des circonstanciels de temps.
Il s'agit là d'un autre symptôme de l'insuffisance de notre système.
Comme signalé précédemment, il est encore possible de régler ces problèmes en améliorant  la sémantique temporelle fondée sur $\mP$ et $\mF$ (en augmentant {\LO} et en complexifiant les règles d'interprétation).
Mais nous ne le ferons pas ici ; 
d'abord parce que ce serait techniquement un peu trop lourd, et nous verrons
dans le chapitre~\ref{Ch:temps2} (vol.~2) une manière plus efficace, plus commode  et plus standard de traiter la contribution des temps verbaux. 

Cela signifie, pour conclure, que nous allons finalement abandonner l'usage de $\mP$ et $\mF$ dans notre langage formel ; nous allons les conserver, ainsi que $\Tps$, le temps des sections qui suivent, mais nous nous en séparerons à la fin de ce chapitre pour mieux réintroduire la temporalité dans le chapitre~\ref{Ch:temps2}.  Néanmoins ce que nous venons de voir n'aura pas servi à rien. 
D'abord cela nous a donné l'occasion de nous pencher sur l'idée de temporalité insérée dans un modèle, et nous serons amenés à réutiliser $\Tps$ par la suite (en l'étoffant un peu). Et surtout cela nous aura permis de nous familiariser assez simplement avec le principe de base de la formalisation de l'intensionnalité. 
Les sections suivantes poursuivent cette formalisation et nous allons voir que fondamentalement le mécanisme est à peu près le même.






\section{Modalités et mondes possibles}
%======================================
\label{s:mondes}


Récapitulons quelques aspects essentiels de la théorie sémantique
exposée jusqu'ici.  Il doit maintenant être clair que la
dénotation d'une expression interprétable dépend d'une certaine
configuration du monde, ce que nous avons formalisé à l'aide de
l'outil des modèles.  Nous avons vu également, dans la sémantique
temporelle de la section précédente, que la dénotation d'une
expression dépend aussi d'un autre paramètre, que nous avons appelé un \emph{indice}, et qui peut varier au
sein du modèle.  Ce paramètre est le point de référence temporel,
c'est-à-dire l'instant auquel il faut se reporter pour
envisager la dénotation d'une expression.   La conséquence pour le
système sémantique est importante : à présent, pour un modèle {\Modele}
donné, une même expression peut avoir différentes valeurs sémantiques,
il suffit de changer l'instant d'évaluation.   Nous sommes passés à une
sémantique intensionnelle.
Car, rappelons-le, l'intensionnalité est cette propriété du système
interprétatif  qui permet d'envisager un (grand) éventail de
dénotations pour chaque expressions du langage.


Dans cette section (et la suivante) nous allons systématiser encore
davantage cette idée de variabilité des valeurs sémantiques d'une
expression donnée dans un modèle donné.  Et dans un premier temps
(\S\ref{s:savoir&ignorer}) nous allons voir comment cela se
justifie en regard de l'usage que l'on peut (et doit) faire de l'outil
de modèle lorsqu'il s'agit de le mettre en rapport avec la notion de
connaissance.  


\subsection{Savoir et ignorer}
%-----------------------------
\label{s:savoir&ignorer}

%Reprenons. 
\is{modele@modèle} 
Un modèle est une description mathématique
du monde : il encode l'ensemble des \emph{informations} que l'on a, au
moins potentiellement, sur le monde.  Et pour être plus précis, ce
«on» dont il est ici question et qui possède ces informations est
un locuteur ou un allocutaire donné\footnote{Il s'agira du locuteur
  ou de l'allocutaire selon que c'est l'angle de la
  production ou celui de la compréhension qui est adopté.  Disons de
  manière générale qu'il s'agit du sujet parlant (et «comprenant») dans la
  peau duquel nous nous glissons lorsque nous effectuons nos calculs sémantiques.}.  Ainsi
un modèle peut être vu comme un outil 
formel qui a vocation de réaliser de la \emph{représentation de
connaissances} (en l'occurrence des connaissances d'un locuteur
donné).  Car rappelons que comprendre une phrase c'est, entre autres,
être capable de confronter son contenu (c'est-à-dire son sens) avec
les informations fournies dans un modèle.


Ajoutons qu'un modèle réaliste%
\footnote{Par «réaliste» j'entends ici : descriptivement en phase
avec la réalité du monde qu'habite (ou qu'évoque) le locuteur.  En
revanche, il ne faudrait surtout pas penser que ce que j'appelle un
modèle réaliste est une réplique vraisemblable et fidèle des
ressources et processus mentaux et cognitifs des locuteurs humains. Un
tel «réalisme» n'est pas l'objectif de la présente étude.}  
%%
vise à décrire le monde \emph{en entier}.  Cela signifie que non
seulement un tel modèle est supposé fournir un domaine de
quantification (\Unv{A}) contenant tous les individus (tous les
objets) auxquels un locuteur est susceptible de faire référence dans
ses phrases, mais aussi que sa fonction d'interprétation (\FI) est
définie pour tous les prédicats du langage du locuteur, c'est-à-dire
au moins tous les noms, verbes, adjectifs de sa langue. 
Multiplions tout cela par l'axe temporel (\Tps) introduit
\alien{supra}, et nous imaginons tout de suite les formidables
proportions qu'atteint ce genre de structure dans son développement.
Autrement dit, un modèle (réaliste) est par définition %(et par nature) 
une \emph{immense}  base de représentation de connaissances
%, ou, si l'on veut, une représentation d'une immense quantité de connaissances
%sur le monde ; 
; elle est immense car exhaustive : un modèle nous dit \emph{tout}
sur le monde.


Or nous, les locuteurs, ne savons pas tout !  Et c'est heureux, car
sinon nous n'aurions rien à nous dire, nous n'aurions aucune
information à nous échanger, à transmettre ou à recevoir, puisque nous
les posséderions déjà toutes.  L'omniscience est d'un ennui insondable.
Puisque nous nous soucions de «réalisme» ici, et qu'un modèle peut servir à représenter des connaissance, il devient assez légitime de se poser la question de  comment
rendre compte, en termes de modèle, du fait qu'un locuteur
donné ne sait pas tout.  On sera peut-être tenté de répondre que si
ce locuteur ne sait pas tout, c'est que sa base personnelle de connaissances du
monde (donc son modèle) est incomplète.  C'est là une idée
tout à fait raisonnable et une piste qui mérite d'être poursuivie.
Cependant je vais d'abord montrer qu'il est difficile de définir
proprement ce que serait un modèle incomplet, ou pour dire les choses
plus précisément, qu'il est difficile de formaliser l'incomplétude de
connaissance au moyen d'un modèle incomplet.  Mais nous verrons
ensuite comment contourner ce problème.

Reprenons notre exemple de modèle-jouet précédent : 


\ex.
\(\Modele =
\tuple{\Unv{A},\Tps_{\tprec},\FI}\) avec 
\(\Unv{A} = \set{\Obj{Alice}; \Obj{Bruno}; \Obj{Charles};\Obj{Dina}}\)
et 
 \(\Tps = \set{i_1;i_2;i_3;i_4}\)

Faisons une première remarque.  Ce modèle {\Modele} est
\emph{partiel}, car il n'ambitionne pas de décrire le monde entier (ce
n'est qu'un modèle-jouet) : il se contente de décrire un micro-univers
contenant quatre entités.  Cela ne veut pas dire pour autant que
{\Modele} serait incomplet (dans le sens qui nous occupe ici).
L'idée de modèle incomplet que nous avons soulevée vise à traduire
formellement l'ignorance du locuteur sur certains faits du monde. Et
l'ignorance (comme le savoir) a en fait surtout à voir avec la
fonction d'interprétation~\FI.

%Regardons de plus près.  
Supposons que nous (locuteurs à qui
«appartient» le modèle {\Modele}) savons parfaitement que
l'individu \Obj{Alice} dort à $i_1$.  Cela signifie que nous sommes
sûrs d'au moins une chose,  c'est que: 
\ex.  
\(\Obj{Alice} \in \FI(i_1,\prd{dormir})\)


Continuons en supposant aussi que nous \emph{ne savons pas} si les autres
individus de {\Unv{A}} dorment ou pas à $i_1$.  À présent, sous ces
hypothèses, comment pouvons-nous définir $\FI$, pour les
arguments qui nous occupent, $i_1$ et \prd{dormir} ?  Nous \emph{ne
pouvons pas} donner :
\ex.  \label{x:dormiri1}
\(\FI(i_1,\prd{dormir}) = \set{\Obj{Alice}}\)

%Pourquoi ?  
Parce que \ref{x:dormiri1} dit précisément qu'\Obj{Alice} est
\emph{le seul} individu de {\Unv{A}} qui dort à $i_1$.  Et donc avec
\ref{x:dormiri1} \emph{nous savons} que \Obj{Bruno}, \Obj{Charles}
et \Obj{Dina} ne dorment pas.  Mais cela n'est pas notre hypothèse de
départ qui était que pour ceux-là nous \emph{ne savons pas} s'ils dorment ou
pas.  Comment procéder alors ?

En fait ce qui correspond à l'ignorance, ou l'incertitude, c'est une
\emph{alternative} de valeurs pour \(\FI(i_1,\prd{dormir})\).  Autrement
dit, ne pas savoir si \Obj{Bruno}, \Obj{Charles}
et \Obj{Dina} dorment (tout en sachant que c'est le cas pour
\Obj{Alice}), c'est admettre que l'on peut choisir parmi :

\ex.  \label{F.alt}
\(\FI(i_1,\prd{dormir}) = \set{\Obj{Alice}}\)
\\\emph{ou bien}\\ 
\(\FI(i_1,\prd{dormir}) = \set{\Obj{Alice};\Obj{Bruno}}\)
\\\emph{ou bien} \\
\(\FI(i_1,\prd{dormir}) = \set{\Obj{Alice};\Obj{Charles}}\)
\\\emph{ou bien}\\ 
\(\FI(i_1,\prd{dormir}) = \set{\Obj{Alice};\Obj{Dina}}\)
\\\emph{ou bien}\\ 
\(\FI(i_1,\prd{dormir}) = \set{\Obj{Alice};\Obj{Bruno};\Obj{Charles}}\)
\\\emph{ou bien}\\
\(\FI(i_1,\prd{dormir}) = \set{\Obj{Alice};\Obj{Bruno};\Obj{Dina}}\)
\\\emph{ou bien}\\
\(\FI(i_1,\prd{dormir}) = \set{\Obj{Alice};\Obj{Charles};\Obj{Dina}}\)
\\\emph{ou bien}\\
\(\FI(i_1,\prd{dormir}) =
\set{\Obj{Alice};\Obj{Bruno};\Obj{Charles};\Obj{Dina}}\) 


Seulement, dans {\Modele}, $\FI$ est une fonction.  Et donc elle ne peut
(et ne doit) donner qu'une seule valeur pour le couple d'arguments $(i_1,\prd{dormir})$ ; en
tant que fonction, $\FI$ par elle-même ne peut pas présenter un choix de
dénotations.   D'ailleurs c'est plutôt la liste \ref{F.alt} qui présente
un choix de valeurs pour $\FI$.  En fait \ref{F.alt} expose 8 fonctions
d'interprétations\is{fonction!\elid\ d'interprétation} \emph{différentes} (il se trouve juste que ces 8 ont
été appelées $\FI$, pour bien insister sur l'alternative).  Et comme une
fonction d'interprétation est définitoire d'un modèle, \ref{F.alt} nous
propose donc 8 \emph{modèles différents}.  Par conséquent, ne pas
savoir si une phrase P est vraie, %ou être incertain de la vérité de P,
cela revient à avoir le choix parmi de {multiples modèles}, \ie\ de multiples états du monde.

Nous retrouvons évidemment le principe de l'intensionnalité, très similaire à ce que nous avons vu dans la section précédente. 
Mais cette fois, les différents états du monde dont nous avons besoin ne visent pas à décrire les étapes successives de «l'Histoire», mais à décrire 
comment peut ou pourrait être le monde. 
Formellement nous allons procéder de la même façon, en multipliant le modèle de l'intérieur : pour chaque instant de $\Tps$, nous allons envisager un multitude (et peut-être même une infinité) de réalités alternatives, c'est-à-dire d'états \emph{possibles} du monde. 
Et c'est pour cette raison que ces alternatives sont appelées simplement des \kwo{mondes possibles}\is{monde!\elid\ possible}. Un monde possible est donc une certaine variante de la réalité, parmi beaucoup d'autres. 

Nous aborderons précisément l'implémentation de cette notion dans notre système sémantique au cours des pages qui suivent, mais nous pouvons dès à présent voir que, comme son nom l'indique, elle est étroitement liée à celle de possibilité et donc de modalité.  
En effet, si selon les informations données en \ref{F.alt}, nous ne savons pas si Bruno dort à $i_1$, ou si
simplement nous n'en sommes pas sûr, cela revient bien à admettre
qu'il est \emph{possible}\is{possible} qu'à $i_1$ Bruno dorme (et, cela va de pair,
qu'il est possible qu'il ne dorme pas). Autrement dit, être incertain
au sujet d'une chose consiste à envisager un certain nombre de
«possibles», et un monde possible correspond en fait à ce que nous appelons, dans le langage ordinaire, une possibilité.\is{possibilité} 

Et cela laisse aussi deviner que la connaissance (ainsi que l'ignorance) est une forme de modalité ; nous reviendrons sur ce point plus précisément en \S\ref{modalflavours1} et au chapitre~\ref{Ch:modalites} (vol.~2).  
Mais nous pouvons déjà nous faire une idée de la manière de rendre compte des connaissances partielles d'un locuteur. Le principe est que tout locuteur ne retient qu'un sous-ensemble de mondes possibles et qu'il rejette les autres : il retient les mondes qui décrivent des états de choses qui sont conformes avec tout ce qu'il sait mais qui diffèrent entre eux sur ce qu'il ne sait pas ; et il rejette tous les mondes qui décrivent des états de choses qu'il sait être faux. C'est ce qui est illustré en \ref{F.alt}, où tout ce que nous savons, ou ce dont nous pensons être sûrs, est tout ce qu'il y a
\emph{en commun} à l'ensemble des alternatives.  Avec \ref{F.alt}, nous sommes
sûrs qu'Alice fait partie des dormeurs.



Donc, pour résumer, les locuteurs étant ce qu'ils sont (ignorants,
incertains ou oublieux), une description sémantique adéquate de leurs
énoncés doit envisager, d'une manière ou d'une autre, une multiplicité
de mondes --~et si possible en conservant la notion d'un
«super-modèle» englobant ces variantes.  Ajoutons, et cela va dans
le même sens, que ce n'est pas forcément par ignorance qu'un locuteur
peut se retrouver à choisir parmi différents mondes possibles : il peut très
bien choisir de se positionner délibérément vis-à-vis d'un monde 
alternatif qu'il sait être distinct de sa vision (et de ses
connaissances) personnelle(s) du monde réel.  C'est ce qui se produit
par exemple lorsqu'un locuteur fait preuve d'imagination, et c'est
quelque chose de très courant dans la pratique langagière.  

\largerpage

Le reste de cette section va donc se consacrer à la présentation de la
mise en place formelle et théorique de cette idée de démultiplication du
modèle au moyen de mondes possibles.  Et pour appréhender cette
notion, nous allons commencer par 
examiner le phénomène des \emph{modalités} en français, car elles
constituent un procédé sémantique assez simple de manipuler les mondes.



\subsection{Possible et nécessaire}
%----------------------------------
\label{ss:possnec}\is{possible}\is{necessaire@nécessaire}\is{modalite@modalité|sqq}

Je ne vais pas immédiatement donner une définition générale de ce que l'on appelle la ou les modalité(s) en sémantique ; ce sera plus simple lorsque nous aurons un peu avancé dans l'observation du phénomène. 
Contentons-nous pour le moment de considérer que la modalité a trait, entre autres, aux notions de  \emph{possible} et de \emph{nécessaire}.
De plus, la présentation qui suit sera extrêmement sommaire (et peut-être un peu approximative), car les modalités représentent en elles-mêmes un continent de la sémantique et sont
pléthoriques dans le langage, à la fois par leurs fréquences d'emploi
et par leurs variétés de réalisations et de déclinaisons sémantiques.
Dans cette section, je veux surtout mettre l'accent sur quelques particularités de l'interprétation de phrases modales. 

Le possible et le nécessaire sont ce que l'on appelle des \kwo{forces modales}\is{force!\elid\ modale} (il en existe quelques autres, mais ce sont les deux principales, avec bien sûr l'impossible, qui est la négation du possible). 
En français, ils sont exprimés par des tournures comme \sicut{il est possible que}, \sicut{il est nécessaire que}, des adverbes comme \sicut{éventuellement}, \sicut{peut-être}, \sicut{nécessairement}, \sicut{forcément}, et les verbes\footnote{Ces verbes sont souvent appelés \emph{auxiliaires} modaux dans les grammaires, même si, d'un point de vue strictement syntaxique, ce ne sont pas vraiment des auxiliaires en français.} \sicut{pouvoir} et \sicut{devoir}. 
Il y a bien sûr des nuances importantes entre ces différentes expressions, et certaines sont même particulièrement polysémiques. Mais fions-nous pour le moment à une idée intuitive de ce qu'elles peuvent signifier.


Pour ce faire, faisons un peu de fiction.
Imaginez-vous au casino, à une table de
roulette.  Vous venez de miser sur le numéro 23, et quelqu'un vous
murmure alors à l'oreille :

\ex.  \label{x:23.1}
Le 23 peut sortir (au prochain coup).


Laissons \sicut{au prochain coup} entre parenthèses, pour nous
concentrer sur la partie de la phrase que nous pouvons simplement gloser par
\sicut{il est possible que le 23 sorte}.  Selon au moins une interprétation, \ref{x:23.1} apparaît ni
plus ni moins comme une remarque absurde, parce que la phrase est
\emph{trivialement} vraie.  Si le jeu est, comme il se doit, de pur
hasard, et si le croupier ne triche pas, il n'y a aucune raison
d'exclure la sortie du numéro 23 au prochain coup.
Donc, oui, \emph{évidemment} que le 23 peut sortir !\footnote{Bien sûr, il y a d'autres manières de réagir face à \ref{x:23.1}, en accordant un peu plus de crédit au locuteur et donc en écartant l'interprétation où la phrase est trivialement vraie. Mais dans ce cas notre réaction sera probablement d'être interloqué ; nous nous demanderons pourquoi il nous dit cela ? que veut-il précisément dire par là ? sait-il des choses que nous ignorons ? mais lesquelles ?... Allons plus loin en imaginant le dialogue suivant :\ExNBP
\ex.
-- Le 23 peut sortir.\\
-- Ah bon ? Et pas les autres numéros ?\\
-- Si si, les autres numéros aussi peuvent sortir.\\
-- ...

On voit bien que \ref{x:23.1} n'est pertinente que si le locuteur sous-entend que d'autres numéros ne peuvent pas sortir, et c'est surtout ce sous-entendu qui est informatif en contexte.%  
%Autrement dit nous aurons du mal à interpréter précisément \ref{x:23.1}. 
}

De même \ref{x:23.2}, la négation de \ref{x:23.1}, que nous gloserons en \sicut{il est impossible que le 23 sorte}, est tout aussi incongrue  car elle est,
elle, trivialement fausse, pour les mêmes raisons que précédemment.

\ex.  \label{x:23.2}
Le 23 ne peut pas sortir (au prochain coup).


La phrase \ref{x:23.3} nous paraîtra également immédiatement fausse, mais là pour une autre raison : les
numéros de la roulette des casinos vont de 0 à 36.

\ex.  \label{x:23.3}
Le 38 peut sortir (au prochain coup).


Et nous ferons le même genre d'observations avec des phrases exprimant la
nécessité comme :

\ex.  
\a.
Nécessairement le 38 ne sortira pas (au prochain coup).\label{x:23.4}
\b.  
Nécessairement le 23  sortira (au prochain coup).\label{x:23.5}

Nous constatons immédiatement que \ref{x:23.4}  est vraie (sachant que \ref{x:23.4} équivaut à \sicut{il est impossible que le 38 sorte}) et que
\ref{x:23.5} est fausse. 

Si nous arrivons à expliquer pourquoi ces exemples
\ref{x:23.1}--\ref{x:23.5} ont ces valeurs de vérité, nous serons
alors sur la bonne voie pour expliciter leurs sens.  
% *** Etoffer ^ ? ***
Il est important de remarquer que ces phrases ont une caractéristique
sémantique assez particulière, qui apparaît clairement si nous
comparons \ref{x:23.1} avec \ref{x:23.6a} et \ref{x:23.6b} :

\ex.   \label{x:23.6}
\a. Le 23 est sorti (au dernier coup).\label{x:23.6a}
\b. Le 23 va sortir (au prochain coup).\label{x:23.6b}


\label{ss:possnec-wdep}
Si nous nous interrogeons sur les valeurs de vérité de \ref{x:23.6a} et
\ref{x:23.6b} comme nous l'avons fait pour
\ref{x:23.1}--\ref{x:23.5}, nous retrouvons le principe
fondamental de la théorie, qui dit que la dénotation d'une expression
dépend des circonstances précises, c'est-à-dire du modèle. Or dans le petit exercice d'imagination que nous sommes en train de faire, nous ne nous sommes pas donné un modèle très précis, nous n'avons pas fait d'hypothèse sur les numéros qui réellement sont tombés ou vont tomber dans la situation envisagée. Et c'est
 pour cela que nous ne savons pas nous prononcer sur les
valeurs de vérité de \ref{x:23.6a} et \ref{x:23.6b}.  Alors qu'avec
\ref{x:23.1}--\ref{x:23.5} nous savions répondre sans hésiter.  
Certes, nous avons vu en \S\ref{ss:PbTmps} que nous ne sommes généralement pas capable de trouver «en direct» la valeur de vérité d'une phrase au futur comme \ref{x:23.6b} ; pour autant il y a malgré tout une différence notable entre \ref{x:23.6b} et \ref{x:23.5} : si nous attendons un peu jusqu'au prochain tirage et s'il se trouve que c'est le 23 qui sort, nous pourrons alors dire rétroactivement que \ref{x:23.6b} était vraie ; en revanche, même dans ces circonstances, \ref{x:23.5} demeurera fausse, car il n'était pas \emph{nécessaire} que le 23 tombe. 


Est-ce à dire que la dénotation de ces phrases
\ref{x:23.1}--\ref{x:23.5} est complètement indépendante du modèle ?
En fait non.  Par exemple, si nous imaginons une scène au casino où le
croupier triche efficacement et a décidé de faire sortir le 16 au
prochain coup ; dans cette circonstance \ref{x:23.2} sera vraie.  Cette
phrase n'est donc pas si trivialement fausse que nous l'avions jugée
initialement.  De même, si nous imaginons un modèle où le casino en
question possède des roulettes à 42 numéros, \ref{x:23.4} devient
fausse.  Par conséquent, comme pour la plupart des phrases
déclaratives, les phrases modales \ref{x:23.1}--\ref{x:23.5} ont une
dénotation qui dépend du modèle --~mais visiblement pas au même point
que les phrases \ref{x:23.6a}--\ref{x:23.6b}.  Et l'analyse sémantique
doit pouvoir expliquer cette différence de dépendance entre les
phrases modales et les autres.

Cette analyse peut, dans ses grandes lignes, s'esquisser %assez simplement 
en utilisant la notion de mondes possibles vue précédemment en
\S\ref{s:savoir&ignorer}. Pour que \ref{x:23.1}, qui exprime une possibilité, soit vraie dans un monde donné,  il suffit qu'il existe un monde possible dans lequel le 23 sort au prochain coup. 
En effet, nous avions assimilé les mondes possibles à ce que l'on appelle couramment des possibilités, et intuitivement \ref{x:23.1} revient bien à dire \sicut{il y a une possibilité que le 23 sorte}. Le possible\is{possible} est donc une quantification existentielle sur les mondes possibles, c'est-à-dire les états alternatifs et concevables de la réalité. Et, on s'en doute, le nécessaire\is{necessaire@nécessaire} est analysé comme le dual du possible, c'est-à-dire une quantification universelle sur les mondes possibles : \ref{x:23.5} est vraie dans un monde donné, ssi le 23 sort au prochain coup dans tous les mondes possibles. 
Cette analyse nous permet de comprendre pourquoi les phrases modales dépendent aussi faiblement de l'état du monde dans lequel se trouve le locuteur.
Pour prouver que \ref{x:23.1} est vraie dans ce monde, nul besoin de vérifier que le 23 va réellement sortir au prochain coup, puisqu'il suffit qu'il existe un monde possible où cela se produit. Par monde possible qui existe, il faut comprendre un état du monde envisageable\footnote{Bien entendu, une part importante de l'étude sémantique des modalités consiste à expliciter précisément ce que signifie \emph{envisageable} ici. Nous aurons plusieurs fois l'occasion de revenir sur ce point.}. Et comme rien ne nous empêche d'envisager un cas de figure où le 23 sort, nous savons immédiatement que \ref{x:23.1} est vraie.  Inversement, parmi tous les mondes possibles envisageables, il n'y a pas de raison d'exclure ceux où un autre numéro que le 23 sort, ce qui fait que \ref{x:23.5} est évidemment fausse. 

D'un autre côté, cette analyse permet également de considérer que la dénotation des phrases modales dépend tout de même de l'état courant du monde ou du contexte. Car selon les circonstances, les locuteurs seront amenés à retenir différents mondes possibles comme envisageables. Par exemple, si au casino, nous nous trouvons dans une situation où nous savons avec certitude que le croupier triche et va faire sortir un numéro pair, alors nous ne tiendrons pas pour envisageables les mondes possibles où le 23 sort (ainsi que les autres numéros impairs ; et par la même occasion, nous en profiterons pour ne pas miser  sur le 23).

Bien sûr, la sémantique du possible et du nécessaire ne se réduit pas simplement à cette analyse, mais nous avons là déjà cerné quelques points importants de leur interprétation, et en particulier le recours aux mondes possibles.  Et c'est ainsi que l'on peut donner une définition sémantique suffisamment générale des modalités : une modalité est ce qui, durant l'interprétation d'une phrase par rapport à un état du monde donné, nous invite à consulter d'autres états du mondes\footnote{À ce titre, nos opérateurs temporels $\mP$ et $\mF$ sont aussi, à leur manière, des opérateurs modaux.}. Les modalités sont donc fondamentalement intensionnelles, et l'intensionnalité est fondamentalement liée à la notion de mondes possibles.


\subsection{Formalisation de la modalité}
%----------------------------------------

Nous allons à présent voir comment formaliser  ce type de
conditions de vérité dans notre système sémantique, d'abord en
enrichissant le modèle, puis en perfectionnant adéquatement le langage
{\LO}.  



\subsubsection{Mondes possibles}
%'''''''''''''''''''''''''''''''
\is{monde!\elid\ possible|(}
\label{sss:mondepossible}

%Commençons par 
Notre est objectif est, une fois encore, de multiplier le modèle en y introduisant une multitude de réalités alternatives, c'est-à-dire des variantes de l'état du monde.
Nous allons procéder exactement comme avec $\Tps$, en multipliant le modèle «de l'intérieur». 
À cet effet, nous nous donnons un ensemble $\Unv W$\is{W@\Unv W} non vide (et éventuellement infini) de symboles que nous noterons $\w_0$, $\w_1$, $\w_2$, etc.\footnote{Comme d'habitude, les indices numériques n'ont en soi aucune signification propre, ils servent simplement à distinguer les symboles entre eux.} C'est ce que nous appellerons les \kwo{mondes possibles}\is{monde!\elid\ possible}, et \Unv W est donc l'ensemble de tous les mondes possibles. Et pour désigner des mondes possibles quelconques, nous écrirons $w$, $w'$, $w''$...

\begin{nota}[Mondes possibles]
Nous notons \Unv W l'ensemble de tous les mondes possibles.\\
Les mondes possibles eux-mêmes seront représentés par les symboles $\w_0$, $\w_1$, $\w_2$...
\end{nota}


C'est essentiellement par commodité que nous nommons \emph{mondes possibles} ces symboles, car formellement ils ne sont que cela : des symboles.  Dans le système, leur rôle est d'étiqueter, d'indexer, ou simplement de nommer des états possibles du monde (à un moment donné), et ce sont ces états possibles du monde qui sont véritablement les mondes possibles.  Comme nous l'avons vu précédemment avec la temporalité, ce qui détermine un état particulier du monde, c'est l'ensemble des valeurs que nous donne la fonction d'interprétation $\FI$.\is{fonction!\elid\ d'interprétation} 
Techniquement, le principe est donc toujours le même :
%, celui que nous avons vu en \S\ref{s:ModeleTemp}, qui consiste à
nous allons ajouter un paramètre $w$ à la fonction.
Ainsi 
$\FI$ devient une fonction à trois arguments, à savoir : la constante à
interpréter, l'instant et le monde possible par rapport auxquels on
l'interprète.  Par convention, nous choisirons ici l'ordre de notation
suivant : le premier argument de {\FI} est le monde possible, le
deuxième l'instant et le troisième la constante interprétée. 
$\FI(w,i,\prd{dormir})$ est donc la dénotation de \prd{dormir} dans le
monde $w$, à l'instant $i$.  Selon le $w$ choisi, cette dénotation, \ie\ l'ensemble de tous les dormeurs, pourra varier puisque les différents $w$ renvoient à différentes réalités. 
\Unv W est donc maintenant un composant de notre modèle intensionnel.

\begin{defi}[Modèle intensionnel (2)]\is{modele@modèle!\elid\ intensionnel}
Un modèle intensionnel $\Modele$ est constitué d'un ensemble d'individus \Unv A, d'un ensemble de mondes possibles \Unv W,
d'un ensemble d'instants ordonnés $\tuple{\Tps,\tprec}$ et d'une fonction d'interprétation $\FI$ à trois arguments qui pour chaque monde de \Unv W,  chaque instant de $\Tps$ et  chaque constante de $\CON$ nous donne la dénotation de cette constante dans ce monde, à cet instant.
\\
On notera \(\Modele = \tuple{\Unv{A},\Unv W,\Tps_{\tprec},\FI}\).
\end{defi}




Pour illustrer cela voici un exemple partiel de l'interprétation de
\prd{dormir} dans notre modèle-jouet \(\Modele  =
\tuple{\Unv{A},\Unv{W},\Tps_{\tprec},\FI}\), avec
$\Unv{W}=\set{\w_1;\w_2;\w_3;\w_4}$ : 

\ex.  \label{x:M.w}
\(\FI(\w_1,i_1,\prd{dormir})=\set{\Obj{Bruno}}\)\\
\(\FI(\w_2,i_1,\prd{dormir})=\set{\Obj{Alice};\Obj{Bruno}}\)\\
\(\FI(\w_3,i_1,\prd{dormir})=\set{\Obj{Bruno};\Obj{Dina}}\)\\
\(\FI(\w_4,i_1,\prd{dormir})=\set{\Obj{Bruno};\Obj{Dina}}\)\\
\(\FI(\w_1,i_2,\prd{dormir})=\set{\Obj{Charles}}\)\\
etc.


Ici dans le monde $\w_1$ à l'instant $i_1$ seul \Obj{Bruno} dort, et dans le monde $\w_2$ \Obj{Bruno} et \Obj{Alice} dorment, etc. Une chose est \emph{sûre} dans ce mini-modèle \Modele, c'est qu'à l'instant $i_1$, \Obj{Bruno} dort. Car il est dans la dénotation de \prd{dormir} dans les quatre mondes. 
On remarque aussi que $\w_3$ et $\w_4$ sont deux mondes dans lesquels les mêmes individus,
\Obj{Bruno} et \Obj{Dina}, dorment à l'instant $i_1$.  Mais cela ne
veut pas dire que $\w_3$ et $\w_4$ étiquettent des états identiques de
la réalité.  Car dans $\w_3$ et $\w_4$, à l'instant $i_1$, il se passe
par ailleurs beaucoup d'autres choses qui peuvent être différentes.
Par exemple on peut imaginer qu'à l'instant $i_1$, Alice
est en train de lire un illustré  dans $\w_3$, mais pas dans $\w_4$ (dans ce cas, nous pourrons avoir, par exemple, \(\FI(\w_3,i_1,\prd{lire})=\set{\tuple{\Obj{Alice},\Obj{LeCosmoschtroumpf}}}\) et 
\(\FI(\w_4,i_1,\prd{lire})=\eVide\)).

Cela nous amène à faire plusieurs remarques. D'abord, pour être tout à fait précis, ce qui étiquette un état particulier et distingué de la réalité ce n'est pas simplement un monde $w$, mais un couple \tuple{w,i}, qui fonctionne comme un couple de coordonnées\footnote{Il serait impropre de parler de coordonnées spatio-temporelles, puisque $w$ ne renvoie pas à une localisation spatiale, mais à une réalité toute entière. Il s'agirait plutôt donc de coordonnées {«mondano-temporelles»} --~terme un peu extravagant, que bien sûr nous n'utiliserons pas ici...} d'une configuration donnée et possible du monde.  Et d'ailleurs si nous fixons un $w$ et un $i$, $\FI$ nous donnera ce qui équivaut à un de nos anciens modèles extensionnels (une image figée d'un certain état du monde) ; de même si nous fixons $w$ et que nous regardons tout ce que donne $\FI$ pour tous les $i$ de \Tps, nous obtenons un modèle temporel de \S\ref{s:ModeleTemp} qui décrit toute l'histoire du monde $w$.  

\newpage

Ensuite, si nous laissons de côté les modèles-jouets, et que nous revenons à une vision plus «réaliste» de ce que doit représenter un modèle intensionnel, nous devons nous poser la question de savoir quels états possibles du monde sont éligibles pour figurer dans \Unv W. La réponse est très simple : tous. 
Tout état possible \emph{imaginable} de la réalité, aussi farfelu et éloigné de notre réalité quotidienne soit il, fait partie de \Unv W.
La raison en est que notre système  doit pouvoir interpréter toute phrase sémantiquement bien formée, et il n'y a aucune raison d'exclure celles qui racontent des choses extraordinaires ou fantastiques (c'est d'ailleurs ce qui nous permet de lire sans problème de la fiction). Interpréter une phrase implique en quelque sorte de supposer sa vérité (pour établir ses conditions de vérité), et supposer qu'une phrase est vraie revient imaginer un monde où ce qu'elle exprime se réalise. Par conséquent, dans \Unv W, il y a, \emph{entre autres}, des mondes où toute l'histoire de Harry Potter est vraie ; et bien sûr, tout un tas d'autres réalités plus ou moins plausibles ou plus ou moins fantaisistes. Introduire \Unv W dans le modèle c'est accepter de lâcher la bride à l'imagination\footnote{Cela peut sembler discorder un peu avec ce qui était mentionné en \S\ref{ss:possnec}, où je parlais d'états du monde \emph{envisageables}, ce qui est moins effréné que de s'intéresser à tous les états du monde imaginables. Mais ici nous définissons le \emph{domaine} de \emph{tous} les mondes possibles, \Unv W, qui en soi se doit d'être exhaustif, pour les raisons expliquées \alien{supra}. Bien sûr, par la suite, quand nous serons amenés à quantifier sur les mondes, comme toujours, nous aurons à restreindre contextuellement le domaine, et c'est ainsi que nous retrouverons la notion, plus retenue, de mondes envisageables. C'est ce qu'aborde \S\ref{sss:MKripke}.}. 
Bien sûr, parmi tous les mondes de \Unv W, il y en a un qui a un statut très particulier et privilégié, c'est \emph{le} monde réel.\is{monde!\elid\ réel} 
Il n'y en a qu'un, c'est celui dans lequel vous et moi nous vivons réellement. Nous aurons plusieurs fois l'occasion de nous y intéresser, mais pour le moment, adoptons simplement la convention de notation {\wo} pour le représenter\footnote{Pour représenter le monde réel, on rencontre fréquemment, chez beaucoup d'auteurs, la notation $\w_{@}$, voire $@$ --~coquetteries que nous n'utiliserons pas ici.}.

Par conséquent, on se doute bien que la quantité de mondes qui
composent \Unv{W} est tout à fait gigantesque, et peut-être même infinie. Car chaque monde
représente (ou étiquette) un état possible de la réalité, et il suffit
d'un détail, même infime, pour distinguer définitivement deux états de la réalité.  Techniquement, pour juger que deux mondes $w$ et $w'$ sont distincts, il suffit qu'il existe ne serait-ce qu'un prédicat unaire \vrb P\ de {\LO} et un individu \Obj x de \Unv A tels que \(\Obj x \in \FI(w,i,\vrb P)\) et \(\Obj x \not\in \FI(w',i,\vrb P)\).

%\largerpage[-1]

Il y a cependant deux limites que nous ne devrons pas nous autoriser à franchir pour définir \Unv W. La première est que tous les mondes possibles doivent respecter les lois  usuelles de la logique, sans quoi notre système sémantique cesserait de fonctionner formellement. Par exemple, nous ne pourrons pas imaginer un monde $w$ dans lequel une formule comme \(\Xlo[\phi\wedge\neg\phi]\) sera vraie. 
Le second garde-fou est que chaque mot de la langue (et partant chaque prédicat de {\LO}) doit avoir le même sens dans tous les mondes possibles\footnote{Et si un mot est polysémique, il doit conserver la même polysémie dans tous les mondes.}.  
La raison est que la notion de mondes possibles est avant tout, pour nous, un outil de la sémantique ;  nous devons donc d'abord poser la langue comme un système stable que nous séparons du modèle, pour pouvoir ensuite l'étudier.
Nous n'aurions aucun intérêt à prendre en compte des mondes hypothétiques dans lesquels la langue serait \pagebreak différente de ce qu'elle est en réalité (parce que les mots changeraient de sens), puisque c'est --~et ne le perdons pas de vue~-- la langue en soi qui est l'objet de notre étude%
\footnote{Évidemment, cette hypothèse de travail pourrait sembler par trop simpliste.  Dans la réalité, nous savons bien qu'il peut arriver que des locuteurs ne soient pas d'accord sur le sens de certains mots.  Nous dirons alors ici qu'ils ne parlent pas exactement la même langue --~il existe d'ailleurs le terme d'\emph{idiolecte}\is{idiolecte} pour renvoyer à ce fait linguistique.  Cette variabilité de système entre locuteurs ne doit pas être irrévocablement négligée dans une étude sémantique, mais l'hypothèse que je fais ici est que ce phénomène ne relève probablement pas de l'intensionnalité: des locuteurs qui ne s'accordent pas sur le sens de certains mots ne se placent pas des mondes différents, mais dans des contextes discordants.}. 

\largerpage

Enfin, dans le modèle, le domaine des individus \Unv A est introduit indépendamment de \Unv W, et est donc en quelque sorte transversal à l'ensemble des mondes possibles. Cela implique que \Unv A contient tous les individus qui existent, ou ont existé ou existeront, réellement mais aussi tous les individus fictifs comme Dracula, James Bond, R2D2, Lucky Luke, Excalibur... %Homer Simpson
Donc globalement, et dans l'absolu, les individus de \Unv A sont avant tout des «individus possibles»,\is{individu!\elid\ possible} et ce n'est que \emph{par rapport} à tel ou tel monde de \Unv W qu'ils acquièrent le statut d'individus réels. À cet effet, un prédicat \prd{exister}\is{existence!prédicat d'\elid}\label{prd:exister} 
ou \prd{réel} (un peu comparable au prédicat \prd{vivant} que nous avons vu avec la temporalité)
peut nous être particulièrement utile : $\FI(\w,i,\prd{exister})$ nous donne l'ensemble des individus et objets qui peuplent réellement le monde $\w$ (à l'instant $i$).  

La transversalité de \Unv A nous permet également d'intégrer facilement l'hypothèse des \kwi{désignateurs rigides}{designateur rigide@désignateur rigide} de \citet{Kripke:72}.  Cette hypothèse repose sur l'idée qu'un nom propre\is{nom!\elid\ propre}, contrairement à une description définie, a cette propriété de toujours référer directement à la même entité, sans médiation sémantique.
Ici «toujours» signifie «dans tous les mondes possibles».  Par exemple, le nom \sicut{Mozart} désigne le même individu dans tous les mondes ; cela implique qu'il n'y a qu'un seul \Obj{Mozart} dans {\Unv A}, mais cela n'implique pas nécessairement que \Obj{Mozart} est toujours semblable et identique : nous pouvons très bien imaginer des mondes où Mozart satisfait des propriétés très différentes de celles qu'il a eues dans \wo. Par exemple des mondes où Mozart serait mort à 18 ans, ou bien à 75 ans ; des mondes où il serait né cent ans plus tard ; des mondes où il aurait été cordonnier, etc.  Cela nous permet de résoudre un étrange paradoxe qui est que, lorsque nous interprétons une proposition comme \sicut{si Mozart était né dans une famille de cordonniers...}, nous imaginons un individu fort dissemblable du Mozart historique, tout en sachant que nous ne parlons pas de n'importe qui, mais bien de Mozart. 

L'hypothèse des désignateurs rigides est donc un postulat épistémologique ; elle permet de capter une certaine notion d'\emph{identité}, celle qui correspond à une sorte de permanence et d'unicité référentielles des noms à travers les mondes. 
Formellement, dans notre système, ce sont les constantes d'individu qui vont jouer le rôle des désignateurs rigides, en respectant la condition suivante. %:
%%si $\Xlo c$ est une constante d'individu, alors quels que soient les mondes $w$ et $w'$ de \Unv W et les instants $i$ et $i'$ de \Tps, \(\FI(w,i,\xlo c)=\FI(w',i',\xlo c)\).

\begin{postu}[Désignateurs rigides (2)]\label{postu:DR2}
Soit un modèle $\Modele=\tuple{\Unv A,\Unv W,\Tps_{\tprec},\FI}$. Pour toute constante $\vrb\alpha\in\CON_0$, quels que soient les mondes $w$ et $w'$ de \Unv W et les instants $i$ et $i'$ de \Tps, \(\FI(w,i,\xlo\alpha)=\FI(w',i',\xlo\alpha)\).
\\
Les constantes d'individus de {\LO} sont des \kwo{désignateurs rigides}.
\end{postu}


Comme avec la temporalité, l'hypothèse des désignateurs rigides enrichie du prédicat \prd{exister}\is{existence!prédicat d'\elid} dans {\LO} nous permet d'éviter le piège d'autres paradoxes, qui ont pu tracasser les philosophes, comme ce qu'illustrent les exemples \Next. 

\ex.
\a. Dracula n'existe pas.
\b. Dracula est un vampire, mais il n'existe pas car les vampires n'existent pas.

Il suffit pour cela de ne pas traduire le verbe \sicut{exister} au moyen de $\Xlo\exists$, mais par le prédicat \prd{exister}.  En effet si nous traduisions \Last[a] par \(\Xlo\neg\exists x [x=\cns d]\), nous aurions une contradiction, puisque si la constante \cns d est définie, c'est qu'il existe un individu de \Unv A qu'elle dénote, et donc il y a bien dans le domaine une valeur pour \vrb x telle que \(\Xlo[x=\cns d]\) est vraie. En revanche, si nous traduisons \Last[a] par \(\Xlo\neg\prd{exister}(\cns d)\), le problème disparaît car il suffit que \Obj{Dracula} ne soit pas dans la dénotation de \prd{exister} dans le monde où l'on se place pour que la formule soit vraie.  De même \Last[b] n'est pas incohérente si l'on considère que la dénotation de \prd{vampire} n'est pas vide mais ne contient que des vampires fictifs et si l'on traduit la dernière phrase  par \(\Xlo\forall x [\prd{vampire}(x)\implq \neg\prd{exister}(x)]\) (et non par \(\Xlo\neg\exists x\, \prd{vampire}(x)\)). 

\medskip

Une dernière remarque qui peut être suscitée par l'ajout des mondes possibles dans le modèle concerne leur statut ontologique.  De nombreuses questions ont occupé (et occupent encore) les philosophes et les théoriciens : quelle est la nature exacte des mondes possibles ? Sont-ils des objets de pensée %(épistémiques) 
ou des composants de la réalité ? Quelle est leur légitimité métaphysique ? Sont-ils cognitivement pertinents ? %Sont-ils sémantiquement adéquats ? 
etc.
Ce sont des débats qui ont leur importance (ne serait-ce que d'un point de vue épistémologique), mais qui, me semble-t-il, sont en dehors de la portée du présent ouvrage\footnote{Pour un aperçu synthétique mais instructif de ces débats, on peut se reporter, entre autres, au chapitre 3 (\S3.5) de \citet{Gamut:2}.}.  Nous nous contenterons de considérer les mondes possibles comme étant, au minimum, des objets théoriques, c'est-à-dire des éléments nécessaires de la théorie (et donc du système formel). Le fait que \Unv W est immense et certainement infini, ne devrait {a priori} être vu ni comme un obstacle théorique ni comme un égarement empirique. Nous pouvons très bien raisonner sur des objets innombrables, aussi facilement que nous pouvons calculer et raisonner sur les nombres réels sans avoir à stocker dans notre cerveau «l'intégralité» de l'ensemble $\mathbb R$.
L'essentiel est que les mondes possibles n'empêchent pas notre système de faire des prédictions fiables à partir des analyses que nous y formulons.

\medskip

Pour enchaîner avec les sections qui suivent, terminons en mentionnant que, bien sûr, l'ajout des mondes possibles comme paramètres pour
l'interprétation se répercute au niveau de toutes les expressions
interprétables du langage.  Concrètement, cela implique que la
dénotation de toute expression doit s'exprimer relativement à un
modèle, une fonction d'assignation, un monde possible et un instant.
Ce sont maintenant autant de paramètres à accoler à notre notation
\denote{\,} pour les valeurs sémantiques.  Les mondes possibles sont, comme les instants, des \kwo{indices intensionnels}.\is{indice!\elid\ intensionnel}


\begin{nota}[Dénotation dans un modèle intensionnel]\label{nota:4.3}
Soit $\Xlo\alpha$ une expression de {\LO}, {\Modele} un modèle
intensionnel, % tel que \(\Modele =
%\tuple{\Unv{A},\Unv{W},\Tps_{\tprec},\FI}\), 
et $g$ une fonction
d'assignation de valeurs aux variables de {\LO}.
\\
\(\denote{\Xlo\alpha}^{\Modele,w,i,g}\) représente la dénotation de
$\Xlo\alpha$ par rapport au modèle {\Modele}, dans le monde $w$, à
l'instant $i$ et relativement à l'assignation $g$.
\end{nota}


\is{monde!\elid\ possible|)}

\subsubsection{Syntaxe de la modalité}
%'''''''''''''''''''''''''''''''''''''

Maintenant que nous avons décuplé les modèles en y faisant figurer
tous les états \emph{possibles} du monde, nous allons pouvoir rendre
compte de l'analyse des modalités au sein de notre système formel.  En
pratique, cela veut dire que nous allons pouvoir exprimer (\ie\
représenter) puis interpréter des énoncés modaux dans le langage
{\LO}.   Pour ce faire, nous devons d'abord augmenter le vocabulaire et
la syntaxe de {\LO} pour y traduire la contribution de tournures comme
\sicut{il est possible que}, \sicut{il est nécessaire que}, etc.

Comme pour la temporalité avec {\Xlo\mP} et {\Xlo\mF}, nous utilisons deux
opérateurs, dits modaux\is{operateur@opérateur!\elid\ modal} (et empruntés à la logique modale traditionnelle), qui s'appliquent à des formules.  Ce sont les opérateurs 
{\Xlo\doit}, qui exprime le nécessaire,\is{necessaire@nécessaire} et 
{\Xlo\peut}, qui exprime le possible.\is{possible}  Nous les
introduisons dans {\LO} à l'aide de la règle syntaxique suivante :




\begin{defi}[Syntaxe de \peut\ et \doit]
\begin{enumerate}[resume*=RglSyn1] %[(\RSyn1)]{\setcounter{enumi}{\value{RglSynt}}}
\item Si $\Xlo\phi$ est une formule bien formée de {\LO}, alors $\Xlo\peut\phi$
et $\Xlo\doit\phi$ le sont aussi.
\setcounter{RglSynt}{\value{enumi}}
\end{enumerate}
\end{defi}

Usuellement, $\Xlo\peut\phi$ se lit  «il est possible que $\Xlo\phi$~» ou
simplement «~$\Xlo\phi$ est possible», et  $\Xlo\doit\phi$ se lit «il est
nécessaire que $\Xlo\phi$~» ou «nécessairement $\Xlo\phi$~» ou «~$\Xlo\phi$
est nécessaire»%
\footnote{Astuce mnémotechnique pour ne pas confondre les deux
  symboles : le losange {\Xlo\peut} est un carré «en équilibre» sur
  un de ses angles, et c'est un équilibre fragile : on se dit qu'il va
  \emph{peut-être} tomber à gauche, \emph{peut-être} tomber à droite ;
  d'où l'idée de \emph{possible}.  Le carré {\Xlo\doit}, lui, est stable,
  solidement posé sur sa base : on est \emph{sûr} qu'il ne basculera pas ;
  d'où l'idée de \emph{nécessité}.}.
En ce qui concerne les traductions de phrases de la langue dans {\LO}, $\Xlo\peut$ et $\Xlo\doit$ pourront, \emph{pour l'instant}, être utilisés pour traduire diverses expressions du possible et du nécessaire. Par exemple, \(\Xlo\peut\prd{dormir}(\cns a)\) sera la traduction commune de \Next[a--d] et 
\(\Xlo\doit\prd{dormir}(\cns a)\) de \NNext[a--d].

\ex. 
\a. Il est possible qu'Alice dorme.
\b. Il se peut qu'Alice dorme.
\b. Alice peut dormir.
\b. Alice dort peut-être.

\ex. 
\a. Il est nécessaire qu'Alice dorme.
\b. Alice dort nécessairement.
\b. Alice doit dormir.
\b. Forcément Alice dort.


En vérité, toutes ces formulations ne sont pas entièrement équivalentes sémantiquement, et nous verrons plus tard qu'il existe une importante variété de valeurs et de nuances des modalités dans la langue, ce que $\Xlo\peut$ et $\Xlo\doit$ semblent masquer.  Mais nous verrons aussi qu'il est possible de rendre compte, en partie,  de cette variété sémantique tout en conservant cette syntaxe.
%Nous verrons plus tard qu'il existe diverses variétés de modalité et
%que les formulations en français que nous pouvons donner pour
%traduire $\peut$ et $\doit$  ne sont pas toutes équivalentes.

Il n'y a pas grand chose de plus à ajouter sur la syntaxe de la modalité dans {\LO}, si ce n'est, encore une fois, que le langage formel nous permet bien sûr d'empiler les opérateurs dans des combinaisons plus ou moins audacieuses telles que 
$\Xlo\doit\doit\phi$, $\Xlo\peut\doit\peut\peut\phi$,
$\Xlo[\doit\phi\implq\doit\psi]$,
$\Xlo\peut\neg\peut\neg\phi$, $\Xlo\doit\mP\peut\phi$, 
etc.  
Certaines deviennent vite très artificielles dans une perspective d'étude linguistique de la langue, d'autres s'éclairent facilement avec la sémantique des opérateurs modaux.



\subsubsection{Sémantique de la modalité}
%''''''''''''''''''''''''''''''''''''''''
\label{modalflavours1}

Rappelons d'abord que maintenant nous écrivons \(\denote{\Xlo\phi}^{\Modele,w,i,g}=1\) pour
dire que $\vrb\phi$ est vraie dans le monde $w$ (et à l'instant $i$, et
relativement à l'assignation $g$) : l'interprétation d'une expression dépend, entre autres, d'un monde possible.  Par défaut, lorsque l'on interprète une phrase entière, l'indice de monde possible (\ie\ le monde dit d'évaluation) \emph{pourra} être assimilé au monde réel, mais il faut toujours garder à l'esprit que cet indice peut en théorie être n'importe quel monde de \Unv W, et nos règles sémantiques doivent donc valoir pour tout monde $w$.

Comme nous l'avons vu, le possible et le nécessaire correspondent à des quantifications respectivement existentielles et universelles sur les mondes possibles. Voici donc un premier jet des règles d'interprétation de $\Xlo\peut$ et $\Xlo\doit$, en se plaçant dans un modèle intensionnel \(\Modele =
\tuple{\Unv{A},\Unv{W},\Tps_{\tprec},\FI}\).


\begin{defi}[Interprétation de $\peut$ et $\doit$ (1\iere version)]\label{d:semMod1}
\begin{enumerate}[sem,resume=RglSem2] %[(\RSem1)]{\setcounter{enumi}{\value{RglSem}}}
\item \label{RSemMod}
\begin{enumerate}
\item \(\denote{\Xlo\peut\phi}^{\Modele,w,i,g}=1\), ssi il existe un monde
$w'$ de {\Unv{W}} 
tel que\\ \(\denote{\Xlo\phi}^{\Modele,w',i,g}=1\).
\item \(\denote{\Xlo\doit\phi}^{\Modele,w,i,g}=1\), ssi pour tout monde
  $w'$ de \Unv{W},  \(\denote{\Xlo\phi}^{\Modele,w',i,g}=1\).
\end{enumerate}
%\setcounter{RglSem}{\value{enumi}}
\end{enumerate}
\end{defi}


Il est fondamental de ne pas confondre le monde d'évaluation, qui est le monde de «départ» du calcul et qui est
noté $w$ dans les règles ci-dessus, avec le ou les
mondes dans lesquels on se rend au cours du calcul pour évaluer
la sous-formule $\Xlo\phi$ --~mondes nommés $w'$ dans les règles
(\RSem\ref{RSemMod}).  
Le monde d'évaluation nous donne véritablement le point de vue par rapport auquel la modalité s'interprète.
Par exemple \(\denote{\Xlo\peut\phi}^{\Modele,w,i,g}=1\) dit que \(\Xlo\peut\phi\) est vraie dans $w$, ce qui revient à dire que \vrb\phi\ est possible dans $w$ ; et ce serait une erreur de dire, en lisant (\RSem\ref{RSemMod}a), que \vrb\phi\ est possible dans $w'$\footnote{Pour l'instant l'erreur n'est pas très grave puisque d'après (\RSem\ref{RSemMod}a), si \vrb\phi\ est vraie dans $w'$, alors $\Xlo\peut\phi$ l'est aussi : en effet, pour que $\Xlo\peut\phi$ soit vraie dans $w'$ il suffit qu'il existe un monde où \vrb\phi\ est vraie, et par hypothèse ce monde existe, c'est $w'$ lui-même. Mais nous verrons que cette implication ne tient plus forcément quand nous aurons amendé les règles d'interprétation.}. 
La vérité de \(\Xlo\peut\phi\) dans un monde n'est pas directement corrélée à la vérité de \vrb\phi\ dans ce même monde, mais à celle de \vrb\phi\ dans un monde qui peut être différent. On peut ainsi très bien avoir \(\denote{\Xlo\phi}^{\Modele,w,i,g}=0\) et \(\denote{\Xlo\peut\phi}^{\Modele,w,i,g}=1\) (\ie\ dans $w$, \vrb\phi\ est fausse mais reste possible).

Notons que les règles (\RSem\ref{RSemMod}) s'insèrent dans une liste de règles sémantiques qui sont très similaires à celles de la définition~\ref{RI2} du chapitre~\ref{ch:gn}  (p.~\pageref{RI2}), mais pas complètement identiques puisqu'à présent les valeurs sémantiques se calculent relativement aux indices $w$ et $i$. Cependant, au delà de l'ajout de ces paramètres, les règles restent inchangées, en transmettant toujours les mêmes indices $w$ et $i$ (nous ferons un récapitulatif de toutes les règles plus tard dans le chapitre).
Par exemple, la règle (\RSem\ref{RIprd2}a) doit se reformuler en : $\denote{\Xlo P(\alpha)}^{\Modele,w,i,g}=1$ ssi \(\denote{\Xlo\alpha}^{\Modele,w,i,g} \in
\denote{\Xlo P}^{\Modele,w,i,g}\). Seules les règles (\RSem\ref{RSemMod}) changent l'indice $w$ au cours du calcul, de même que seules les règles (\RSem\ref{RSemTps}) changeaient l'indice~$i$.

%Regardons un exemple avec l'exercice suivant.
En guise d'application, prenons le temps de regarder les deux exercices suivants (des éléments de correction sont donnés juste après).

\medskip

% -*- coding: utf-8 -*-
\begin{exo}\label{exo4:mod1}
En reprenant le modèle-jouet {\Modele} esquissé en \ref{x:M.w} et rappelé ici :
\pagesolution{crg:mod1}
\begin{quote}
\(\FI(\w_1,i_1,\prd{dormir})=\set{\Obj{Bruno}}\)\\
\(\FI(\w_2,i_1,\prd{dormir})=\set{\Obj{Alice};\Obj{Bruno}}\)\\
\(\FI(\w_3,i_1,\prd{dormir})=\set{\Obj{Bruno};\Obj{Dina}}\)\\
\(\FI(\w_4,i_1,\prd{dormir})=\set{\Obj{Bruno};\Obj{Dina}}\)\\
\(\FI(\w_1,i_2,\prd{dormir})=\set{\Obj{Charles}}\)
\end{quote}
calculez les valeurs suivantes :
\addtolength{\multicolsep}{-8pt}\addtolength{\columnsep}{-24pt}
\begin{multicols}{2}
\begin{enumerate}
\item \(\denote{\Xlo\peut\prd{dormir}(\cns a)}^{\Modele,\w_1,i_1,g}\)
\item \(\denote{\Xlo\peut\prd{dormir}(\cns a)}^{\Modele,\w_2,i_1,g}\)
\item \(\denote{\Xlo\doit\prd{dormir}(\cns a)}^{\Modele,\w_1,i_1,g}\)
\item \(\denote{\Xlo\doit\prd{dormir}(\cns b)}^{\Modele,\w_1,i_1,g}\)
\item \(\denote{\Xlo\peut\prd{dormir}(\cns c)}^{\Modele,\w_3,i_1,g}\)
\item \(\denote{\Xlo\neg\doit\prd{dormir}(\cns d)}^{\Modele,\w_4,i_1,g}\)
\item \(\denote{\Xlo\peut\prd{dormir}(\cns a)\wedge\peut\prd{dormir}(\cns d)}^{\Modele,\w_4,i_1,g}\)
\item \(\denote{\Xlo\peut[\prd{dormir}(\cns a)\wedge\prd{dormir}(\cns d)]}^{\Modele,\w_4,i_1,g}\)
\end{enumerate}
\end{multicols}
\begin{solu} (p.~\pageref{exo4:mod1})\label{crg:mod1}

Les dénotations sont calculées en utilisant les règles d'interprétation (\RSem\ref{RSemMod}) de la définition \ref{d:semMod1}, p.~\pageref{d:semMod1}.
\sloppy\begin{enumerate}
\item \(\denote{\Xlo\peut\prd{dormir}(\cns a)}^{\Modele,\w_1,i_1,g}= 1\) car Alice dort dans $\w_2$.
\item \(\denote{\Xlo\peut\prd{dormir}(\cns a)}^{\Modele,\w_2,i_1,g}=1\) pour la même raison.
\item \(\denote{\Xlo\doit\prd{dormir}(\cns a)}^{\Modele,\w_1,i_1,g}=0\) car il y a des mondes où Alice ne dort pas à $i_1$, par exemple~$\w_1$.
\item \(\denote{\Xlo\doit\prd{dormir}(\cns b)}^{\Modele,\w_1,i_1,g}=1\) car Bruno dort dort dans les quatre mondes à $i_1$.
\item \(\denote{\Xlo\peut\prd{dormir}(\cns c)}^{\Modele,\w_3,i_1,g}=0\) car Charles ne dort dans aucun monde à $i_1$.
\item \(\denote{\Xlo\neg\doit\prd{dormir}(\cns d)}^{\Modele,\w_4,i_1,g}=1\) car Dina ne dort pas dans tous les mondes.
\item \(\denote{\Xlo\peut\prd{dormir}(\cns a)\wedge\peut\prd{dormir}(\cns d)}^{\Modele,\w_4,i_1,g}=1\) car Alice dort dans $\w_2$ et Dina dort dans~$\w_3$.
\item \(\denote{\Xlo\peut[\prd{dormir}(\cns a)\wedge\prd{dormir}(\cns d)]}^{\Modele,\w_4,i_1,g}=0\) car il n'y a pas de monde dans lequel à la fois Alice et Dina dorment.
\end{enumerate}
\fussy
\end{solu}
\end{exo}


\smallskip

% -*- coding: utf-8 -*-
\begin{exo}\label{exo:4equiv}
Chaque formule de la liste suivante est équivalente à une autre formule
\pagesolution{crg:4equiv}
de la liste.  Indiquez, en le démontrant, quelles sont ces paires
d'équivalences.  
\addtolength{\multicolsep}{-10pt}
\begin{multicols}{4}
\begin{exolist}
\item $\Xlo\peut\phi$
\item $\Xlo\doit\phi$
\item $\Xlo\peut\neg\phi$
\item $\Xlo\doit\neg\phi$
\item $\Xlo\neg\peut\phi$
\item $\Xlo\neg\doit\phi$
\item $\Xlo\neg\peut\neg\phi$
\item $\Xlo\neg\doit\neg\phi$
\end{exolist}
\end{multicols}

\smallskip

\noindent Puis corroborez  ces résultats en donnant des traductions en français de chacune des formules.

\begin{solu} (p.~\pageref{exo:4equiv})\label{crg:4equiv}

Les équivalences s'infèrent à partir des règles d'interprétation des modalités  (\RSem\ref{RSemMod}) p.~\pageref{RSemMod} et de celle de la négation (\RSem\ref{RIneg}), \S\ref{s:reglsem} p.~\pageref{RIneg}.
Ce sont les suivantes :

\begin{exolist}
\item $\Xlo\peut\neg\phi$ et $\Xlo\neg\doit\phi$ : la première dit qu'il y a un monde où \vrb\phi\ est fausse, et la seconde qu'il est faux que \vrb\phi\ est vraie dans tous les mondes.

$\Xlo\peut\neg\phi$ {\rtrad} \sicut{il est possible que non {\vrb\phi}}
et
$\Xlo\neg\doit\phi$ {\rtrad} \sicut{il n'est pas nécessaire que {\vrb\phi}}.

\item $\Xlo\doit\neg\phi$ et $\Xlo\neg\peut\phi$ : la première dit que \vrb\phi\ est fausse dans tous les mondes, et la seconde qu'il n'y a pas de monde où \vrb\phi\ est vraie.

$\Xlo\doit\neg\phi$ {\rtrad} \sicut{il est nécessaire que non {\vrb\phi}}
et $\Xlo\neg\peut\phi$ {\rtrad} \sicut{il est impossible que {\vrb\phi}}.

\item $\Xlo\neg\peut\neg\phi$ et $\Xlo\doit\phi$ : la première dit qu'il n'y a pas de monde où \vrb\phi\ est fausse, et la seconde que \vrb\phi\ est vraie dans tous les mondes.

$\Xlo\neg\peut\neg\phi$ {\rtrad} \sicut{il est impossible que non {\vrb\phi}}
et $\Xlo\doit\phi$ {\rtrad} \sicut{il est nécessaire que {\vrb\phi}}.

\item $\Xlo\neg\doit\neg\phi$ et $\Xlo\peut\phi$ : la première dit que \vrb\phi\ n'est pas fausse dans tous les mondes et la seconde qu'il y a un monde où \vrb\phi\ est vraie.

$\Xlo\neg\doit\neg\phi$ {\rtrad} \sicut{il n'est pas nécessaire que non {\vrb\phi}}
et $\Xlo\peut$ {\rtrad} \sicut{il est possible que {\vrb\phi}}.

\end{exolist}


\end{solu}
\end{exo}


\smallskip


Dans l'exercice~\ref{exo4:mod1}, on constate assez facilement que 
\(\denote{\Xlo\peut\prd{dormir}(\cns a)}^{\Modele,\w_1,i_1,g}=1\), 
car il existe bien un monde dans \Modele\ où $\Xlo\prd{dormir}(\cns a)$ est vraie à l'instant $i_1$, c'est $\w_2$ (en effet \(\denote{\Xlo\prd{dormir}(\cns a)}^{\Modele,\w_1,i_1,g}=1\)).
Et \(\denote{\Xlo\peut\prd{dormir}(\cns a)}^{\Modele,\w_2,i_1,g}=1\) pour exactement les mêmes raisons : $\Xlo\peut\phi$ est vraie dans $w$ (à l'instant $i$) s'il existe un monde où $\Xlo\phi$ est vraie (à l'instant $i$), et ce monde peut très bien être $w$ lui-même.  
Ces résultats font apparaître une conséquence notable de la sémantique de (\RSem\ref{RSemMod}), et qui est que la dénotation d'une formule modale ne dépend pas du monde dans laquelle nous l'évaluons : si \(\Xlo\peut\prd{dormir}(\cns a)\) est vraie dans $\w_1$, par définition, elle est vraie dans tous les autres mondes aussi.
Cela nous rappelle une observation que nous avions faite en \S\ref{ss:possnec} (p.~\pageref{ss:possnec-wdep}),
mais une observation que nous avions nuancée : nous avions remarqué que la dénotation des phrases modales dépend \emph{moins} du monde d'évaluation que les phrases ordinaires, mais sans aller jusqu'à dire qu'elle en est complètement indépendante. Même si elles captent un élément essentiel de l'interprétation des modalités, nos règles (\RSem\ref{RSemMod}) ne sont donc pas absolument adéquates ; nous aimerions, lorsque nous calculons \(\denote{\Xlo\peut\phi}^{\Modele,w,i,g}\) ou \(\denote{\Xlo\doit\phi}^{\Modele,w,i,g}\), que le résultat dépende, au moins un peu, de $w$.
Par rapport au modèle jouet de l'exercice, ce n'est pas dramatique, mais dès que nous revenons à un modèle intensionnel «réaliste», cela devient plus problématique.

En effet, (\RSem\ref{RSemMod}) nous dit que $\Xlo\doit\phi$ signifie que $\Xlo\phi$ est vraie dans tous les mondes de \Unv W. Et comme \Unv W contient tous les mondes imaginables, les seules formules $\Xlo\phi$ dont on pourra dire qu'elles sont nécessaires, sont les \emph{tautologies}\is{tautologie}, c'est-à-dire les formules qui sont vraies en vertu de leur structure logique.  De même, presque toute formule de {\LO} est trivialement une formule possible : il suffit que $\Xlo\phi$ soit contingente\is{formule!\elid\ contingente} pour que $\Xlo\peut\phi$ soit vraie (car si $\Xlo\phi$ est contingente, par définition, on trouvera toujours au moins un monde $w$ de \Unv W où $\Xlo\phi$ est vraie). 
Or ce n'est
pas du tout ce qui se passe en langue naturelle : on peut parfaitement
imaginer une situation (\ie\ un monde possible) où \Next[a] est fausse ;
et inversement \Next[b] peut parfaitement être vraie dans un monde
donné, alors que la phrase racine (ici \sicut{Max est le coupable}) n'est nullement une contradiction ni une tautologie.

\ex.
\a. Il se peut que Max soit le coupable.
\b. C'est forcément Max le coupable.


Évidemment le problème qui se pose ici est toujours le même ; nous l'avions déjà rencontré avec la sémantique de la temporalité. Les règles d'interprétation quantifient sur tout le domaine, ici \Unv W, alors qu'il serait plus raisonnable de restreindre cette quantification à un sous-ensemble pertinent de \Unv W.
Par exemple, si $V$ est un ensemble de mondes strictement inclus dans \Unv W, et si nous disons que $\Xlo\doit\phi$ est vraie ssi $\Xlo\phi$ est vraie dans tous les mondes de $V$, alors $\Xlo\phi$ n'a plus besoin d'être une tautologie, car elle peut par ailleurs être fausse dans des mondes qui ne sont pas dans $V$.
Nous allons voir ci-dessous une façon assez classique de compléter les règles 
(\RSem\ref{RSemMod}) et qui intègre ce type de restriction. 
Pour autant, la définition~\ref{d:semMod1} n'est pas entièrement inappropriée dans un système formel, elle correspond à un type particulier de modalité, qui intéresse plus souvent les philosophes et les logiciens, et que l'on nomme la \kwo{modalité aléthique}\is{modalite@modalité!\elid\ aléthique}\footnote{\sicut{Aléthique} vient du grec \alien{alêthês} qui signifie \sicut{vrai}. La modalité aléthique est parfois aussi appelée \emph{modalité logique}.}.
La modalité aléthique est précisément celle par laquelle \sicut{nécessaire} est synonyme de \sicut{tautologique}, \sicut{possible} synonyme de \sicut{contingent} et \sicut{impossible} synonyme de \sicut{contradictoire}. On peut rencontrer ce type de modalité dans la langue naturelle, par exemple dans des énoncés de logique, comme la formulation d'un syllogisme : \sicut{si tous les homards sont gauchers et si Alfred est un homard, alors \emph{nécessairement} Alfred est gaucher}. 
Mais elle est loin d'être la plus usuelle, et pour une plus grande adéquation avec les usages linguistiques, notre système doit être amendé.





\subsubsection{Modèle de Kripke} 
%'''''''''''''''''''''''''''''''
\is{modele@modèle!\elid\ de Kripke}
%: \(\Modele = \tuple{\Unv{A},\Unv{W},\RK,\Tps_{\tprec},\FI}\)
\label{sss:MKripke}

La façon la plus simple de s'affranchir du problème emprunte à la logique modale classique, et notamment aux travaux de 
\citet{Kripke:63af}\Andex{Kripke, S.} et
\citet{Hintikka:61}\Andex{Hintikka, J.}.
Elle consiste à ajouter de la structure dans \Unv W, afin d'organiser les mondes possibles entre eux. Cette organisation se présente sous la forme d'une sorte de \emph{réseau} de mondes, et techniquement elle est modélisée au moyen d'une \emph{relation binaire} qui relie certains mondes avec certains autres mondes\footnote{À l'instar de la dénotation d'un prédicat binaire qui relie certains individus de \Unv A avec certains autres individus.}. 
Cette relation est habituellement notée $\RK$ et elle se nomme la 
\kwo{relation d'accessibilité}\is{relation!\elid\ d'accessibilité} du modèle.  
Et c'est précisément ce qu'elle représente : l'idée qu'à partir d'un  monde donné $w$, on peut accéder à certains mondes mais pas à d'autres. 
Ainsi la notation $w\RK w'$ signifie que le monde $w$ a accès au monde $w'$, ou que $w'$ est accessible depuis $w$. 
La relation $\RK$ trace donc des «chemins» qui nous permettent de passer d'un monde à un autre.

$\RK$ nous est donnée par le modèle, et dorénavant un modèle intensionnel sera déterminé par un quintuplet \(\Modele = \tuple{\Unv{A},\Unv{W},\RK,\Tps_{\tprec},\FI}\).  C'est ce que l'on appelle couramment un \kwo{modèle de Kripke}\is{modele@modèle!\elid\ de Kripke}.
La composante \tuple{\Unv{W},\RK} du modèle, c'est-à-dire l'ensemble de mondes possibles organisé par $\RK$, s'appelle une \kwo{structure modale}\is{structure!\elid\ modale} (ou parfois un \emph{cadre modal}, par traduction de l'anglais \alien{modal frame}).

Illustrons immédiatement cela à l'aide d'un modèle-jouet avec \Unv W\ contenant quatre monde $\w_1$, $\w_2$, $\w_3$ et $\w_4$.  Et supposons que $\RK$ soit définie comme suit : 
$\w_1 \RK \w_2$, $\w_2 \RK \w_3$, $\w_2 \RK \w_4$, %$\w_3 \RK \w_1$, 
$\w_3 \RK \w_4$, $\w_3\RK \w_3$, $\w_4 \RK \w_2$.
Il est possible de représenter graphiquement cette structure dans un plan où les mondes sont figurés par des points et la relation $\RK$ par une série de flèches, comme dans la figure~\ref{Fig:MFrame}.

\begin{figure}[h!]
\begin{center}%\Unv{M}:
\begin{pspicture}(5,3)
\cnode*(1,2.5){2pt}{w1}\rput(0.6,2.5){$\w_1$}
\cnode*(4,2.5){2pt}{w2}\rput(4.4,2.5){$\w_2$}
\cnode*(1,0.5){2pt}{w3}\rput(1,0.2){$\w_3$}
\cnode*(4,0.5){2pt}{w4}\rput(4,0.2){$\w_4$}
\psset{nodesep=3pt}
%\ncarc{->}{w1}{w2}
\ncarc[linecolor=darkgray]{->}{w1}{w2}
\ncarc[linecolor=darkgray]{->}{w2}{w3}
%\ncarc[linecolor=darkgray]{->}{w3}{w1}
\ncarc[linecolor=darkgray]{->}{w3}{w4}
\ncarc[linecolor=darkgray,arcangle=10]{->}{w4}{w2}
\ncarc[linecolor=darkgray,arcangle=10]{->}{w2}{w4}
\nccurve[angleA=115,angleB=185,ncurv=4,linecolor=darkgray]{->}{w3}{w3}
\end{pspicture}
\caption{Représentation graphique d'une structure modale}\label{Fig:MFrame}
\end{center}
\end{figure}


Ce genre de figure peut être utile pour faire des calculs en logique modale, 
mais ici elle ne nous servira qu'à observer que $\RK$, d'abord, n'est pas nécessairement symétrique. Par exemple dans ce modèle, à partir de $\w_1$ on accède à $\w_2$, mais de $\w_2$ on n'accède pas %directement 
à $\w_1$. Autrement dit, $w\RK w'$ n'équivaut pas à $w'\RK w$. 
De même, $\w_3$ accède à lui-même, mais ce n'est pas le cas des autres mondes. 
Cela peut paraître un peu contre-intuitif, mais en principe, une structure modale n'oblige pas $\RK$ à être réflexive (\ie\ à imposer «l'auto-accès» de chaque monde).

Évidemment tout cela semble un peu abstrait ; et ça l'est véritablement. 
Car la question qui se pose ici est : quelle est au juste la signification de cette notion d'accessibilité entre les mondes encodée par $\RK$ ?\is{relation!\elid\ d'accessibilité}
En fait, la réponse dépend fondamentalement de la façon d'interpréter les modalités. Or comme on s'en doute, il y a plusieurs façons de les interpréter, et donc plusieurs façons de définir $\RK$ et de lui assigner une signification précise. Nous reviendrons sur ce point dans les lignes qui suivent ainsi qu'en \S\ref{s:Ty2}.  Mais pour ne pas trop rester dans le vague ici, contentons-nous pour l'instant de considérer que $\RK$ peut, par exemple, servir à modéliser une idée de proximité, ou plus exactement de ressemblance entre les mondes. 
En effet, parmi toutes les possibilités que propose \Unv W, certains mondes peuvent être vus comme assez similaires, parce qu'ils ne diffèrent entre eux que par quelques faits ou quelques détails peu spectaculaires. Ainsi $w\RK w'$ se comprendra comme «~$w$ ressemble suffisamment à $w'$~».  Il est à noter que dans ce cas, et contrairement à l'exemple précédent (qui présentait une structure modale \emph{quelconque}), $\RK$ sera au moins réflexive et symétrique : $w$ ressemble évidemment à lui-même, et si $w$ ressemble à $w'$, alors $w'$ ressemble à $w$.
 



Revenons maintenant aux modalités. Leur interprétation tient compte de $\RK$ comme le montre la version corrigée de (\RSem\ref{RSemMod}) :

\begin{defi}[Interprétation de $\peut$ et $\doit$ (2\ieme version)]\label{d:semMod2}
\begin{enumerate}[sem] %[(\RSem1)]
{\setcounter{enumi}{\value{RglSem}}}
\item
\begin{enumerate}
\item \(\denote{\Xlo\peut\phi}^{\Modele,w,i,g}=1\), ssi il existe un monde
  $w'$ de {\Unv{W}} tel que $w \RK w'$ et \(\denote{\Xlo\phi}^{\Modele,w',i,g}=1\).
\item \(\denote{\Xlo\doit\phi}^{\Modele,w,i,g}=1\), ssi pour tout monde
  $w'$ de \Unv{W} tel que $w \RK w'$,  \(\denote{\Xlo\phi}^{\Modele,w',i,g}=1\).
\end{enumerate}
\setcounter{RglSem}{\value{enumi}}
\end{enumerate}
\end{defi}


Ces règles montrent que lorsque l'on interprète une formule modale par rapport à $w$, on ne quantifie que sur les mondes accessibles à $w$.
Nous avons bien ce que nous cherchions. D'une part, les règles ne quantifient pas sur tout \Unv W, mais seulement sur un sous-ensemble de mondes. Chaque monde $w$  s'associe en effet à  un ensemble de mondes particulier
 (\ie\ l'ensemble des mondes $w'$ tels que $w\RK w'$). Et donc, d'autre part, la dénotation d'une formule modale dépendra du monde d'évaluation puisque le domaine de quantification est directement dépendant de ce monde. 


\sloppy

Si nous reprenons l'extrait de  modèle de l'exercice \ref{exo4:mod1} complété par la structure de la
figure~\ref{Fig:MFrame}, en vertu des nouvelles règles (\RSem\ref{RSemMod}), nous constatons que \(\denote{\Xlo\peut\prd{dormir}(\cns a)}^{\Modele,\w_1,i_1,g}=1\) (comme auparavant) car $\w_1$ accède à $\w_2$ et Alice dort dans $\w_2$ ; mais à présent \(\denote{\Xlo\peut\prd{dormir}(\cns a)}^{\Modele,\w_2,i_1,g}=0\) car $\w_2$ accède à $\w_3$ et $\w_4$ et dans aucun de ces deux mondes Alice ne dort.  De même \(\Xlo\prd{dormir}(\cns d)\) n'est pas vraie dans tous les mondes du modèle, cependant \(\denote{\Xlo\doit\prd{dormir}(\cns d)}^{\Modele,\w_2,i_1,g}=1\) car \(\Xlo\prd{dormir}(\cns d)\) est vraie dans $\w_3$ et $\w_4$ et cela suffit.

\fussy

L'ajout d'une relation d'accessibilité entre les mondes améliore fondamentalement notre système intensionnel et particulièrement la sémantique des modalités. 
Mais dans une perspective d'analyse sémantique de la langue, ce n'est pas encore suffisant. Observons par exemple les phrases suivantes :

\ex.
\a. Alice peut jouer de la trompette à l'heure qu'il est.
\b. Les domestiques doivent parler anglais à la maîtresse de maison.

Nous pouvons facilement montrer que ces phrases sont ambiguës quant à l'interprétation de \sicut{pouvoir} et \sicut{devoir}. 

Ainsi \Last[a] peut signifier qu'Alice a le droit de jouer de la trompette (par exemple parce qu'elle a fini ses devoirs et qu'elle a maintenant l'autorisation de pratiquer son instrument). Dans ce cas, on dira que \sicut{pouvoir} exprime une \kwo{modalité déontique}\is{modalite@modalité!\elid\ deontique@\elid\ déontique}. La modalité déontique est liée à la notion de \emph{devoir}\footnote{\emph{Déontique} vient du grec \alien{deon} qui signifie \sicut{ce qu'il faut faire}.} : le possible déontique correspond à ce qui est \emph{permis}, le nécessaire à ce qui est \emph{obligatoire} et l'impossible à ce qui est \emph{interdit}.  
Ce sont effectivement des valeurs qu'ont les verbes \sicut{pouvoir} et \sicut{devoir} en français.  

Mais \Last[a] peut se comprendre autrement. Imaginons que ces derniers jours Alice avait une bronchite qui l'affaiblissait et l'essoufflait au point de l'empêcher de souffler dans sa trompette : elle ne pouvait pas jouer de la trompette. Mais maintenant elle est guérie et a récupéré, et le locuteur peut le signaler en énonçant \Last[a]. Nous appellerons cette lecture une \kwo{modalité dynamique}\is{modalite@modalité!\elid\ dynamique}\footnote{Ici \emph{dynamique} ne renvoie pas tant à l'idée de mouvement ; en l'occurrence le terme vient du grec \alien{dynamis} qui signifie, entre autres, la capacité, la faculté. Il se trouve que, dans la littérature sémantique, ce type de modalité a reçu diverses étiquettes, comme  \emph{circonstancielle}, \emph{volitionnelle} ou même \emph{habilitative} (en écho à l'anglais \alien{abilitative})... Selon les auteurs, leur couverture sémantique n'est pas toujours exactement la même (cf.\ par exemple \citet{Portner:09}\Andexn{Portner, P.} pour un inventaire scrupuleux).} ; il s'agit de la modalité par laquelle le possible renvoie à une idée de \emph{capacité} ou d'\emph{aptitude}, ce que le français exprime par \sicut{pouvoir} ou, de manière plus univoque, \sicut{être capable de}, \sicut{être en état de}, \sicut{être en mesure de}...\footnote{Notons qu'en français, comme dans beaucoup d'autres langues, il ne semble pas y avoir d'expression nette et simple du «nécessaire dynamique». On peut éventuellement envisager que cette valeur se réalise par certains emplois de périphrases comme \sicut{il est inévitable que}, \sicut{ne pas pouvoir s'empêcher de}, etc. Il n'est pas non plus exclu d'envisager qu'au moins dans certains cas, le futur puisse avoir également cette valeur.} 

Plusieurs autres interprétations de \Last[a] sont certainement disponibles, en affinant les nuances, mais je n'en citerai ici qu'une troisième. 
C'est celle via laquelle le locuteur manifeste qu'il ne sait pas si Alice est en train de jouer de la trompette actuellement, mais qu'il n'en exclut pas l'éventualité.  Dans ce cas, \Last[a] équivaut à peu près aux paraphrases \sicut{il se peut qu'Alice joue de la trompette} et \sicut{Alice joue peut-être de la trompette}.  Nous sommes alors face à une \kwo{modalité épistémique}\is{modalite@modalité!\elid\ epistemique@\elid\ épistémique}. Ce type de modalité se positionne par rapport un ensemble de connaissances factuelles\footnote{\emph{Épistémique} vient du grec \alien{epistêmê} qui signifie le savoir.} 
(du locuteur ou d'un groupe de locuteurs ou éventuellement d'autres personnes), et permet au locuteur d'indiquer son degré de croyance ou de confiance sur ce qu'il affirme ;
le nécessaire correspond alors à ce qui est \emph{su} ou \emph{certain}, le possible à ce qui est \emph{ignoré} ou \emph{incertain} (et l'impossible à ce que l'on sait être faux).
C'est cette modalité qui était sous-jacente dans la présentation faite en \S\ref{s:savoir&ignorer} lorsque nous parlions d'états du monde «envisageables».

De façon assez similaire, nous pouvons expliciter plusieurs lectures pour \Last[b]. Les plus saillantes sont la lecture déontique (un règlement oblige les domestiques à parler anglais) et la lecture épistémique (à partir de ce qu'il sait, le locuteur déduit que les domestiques parlent anglais).

Ces exemples suffisent (s'il y avait besoin) à montrer qu'il existe dans la langue une variété non négligeable de valeurs sémantiques des modalités, et que des expressions modales comme \sicut{pouvoir} et \sicut{devoir} sont à cet égard éminemment ambiguës, ou du moins polysémiques.
Nous avons donc un problème. Si les phrases {\Last} sont ambiguës, c'est qu'elles devraient recevoir chacune plusieurs  traductions sémantiques distinctes. 
Or dans {\LO} nous n'avons que deux symboles, $\Xlo\peut$ et $\Xlo\doit$, pour traduire respectivement  \sicut{pouvoir} et \sicut{devoir}. 
Notre système n'est pas assez précis.

Une solution qui nous vient naturellement pour régler ce problème serait de multiplier les opérateurs modaux. Nous nous donnerions un opérateur pour le possible épistémique, un pour le possible déontique, un pour le possible dynamique, etc. C'est techniquement faisable, mais nous devons voir ce que cela implique sur l'ensemble du système. Ce qui détermine la valeur sémantique d'une modalité, c'est précisément la relation d'accessibilité\is{relation!\elid\ d'accessibilité} qu'il faut utiliser pour interpréter cette modalité.  Cela nous donne l'occasion d'expliciter un peu les significations que peut prendre $\RK$ dans le modèle. 
Par exemple, si notre modalité est épistémique, alors $w\RK w'$ doit signifier quelque chose comme : tout ce qui est su (par exemple par le locuteur) dans $w$ est vrai dans $w'$\footnote{Mais la réciproque ne tient pas : tout ce qui est vrai dans $w'$ n'est pas forcément su (par le locuteur) dans $w$. Ce que la relation dit c'est, en fait, que $w'$ est \emph{compatible} avec tout ce qui est su dans $w$, ou encore que $w$ et $w'$ sont semblables au regard de ce qui est su dans $w$ et diffèrent au regard de ce qui y est ignoré.}.  Si la modalité est déontique, alors $w\RK w'$ signifie à peu près : toutes les lois ou tous les commandements qui sont actifs dans $w$ sont effectivement satisfaits (\ie\ obéis) dans $w'$. 
Pour la modalité dynamique, c'est un peu plus complexe : $w\RK w'$ signifie que
$w'$ est similaire à $w$ pour ce qui concerne les faits généraux (ou génériques), les propriétés intrinsèques et stables des choses et des individus, et les circonstances qui les entourent, mais $w$ et $w'$ peuvent différer quant aux événements particuliers et contingents qui s'y produisent.
Notons d'ailleurs que la modalité aléthique\is{modalite@modalité!\elid\ aléthique} n'est finalement qu'un cas particulier ; on peut considérer qu'elle se fonde sur une relation $\RK$ qui connecte \emph{tous} les mondes de \Unv W avec \emph{tous} les mondes de \Unv W.  Cela équivaut à une absence de relation d'accessibilité, puisqu'une telle relation n'instaure aucune organisation ou hiérarchie particulière parmi les mondes.

Par conséquent, multiplier les opérateurs modaux implique de multiplier les relations d'accessibilité\is{relation!\elid\ d'accessibilité} dans le modèle. 
Formellement, cela peut se réaliser en se donnant un ensemble de symboles (par exemple des nombres ou des lettres) qui renvoient chacun à une valeur modale particulière. Et pour chaque symbole $n$, nous aurons une relation d'accessibilité $R_n$ dans le modèle et une paire d'opérateurs modaux que nous pourrons noter 
$\Xlo\peut_n$ et $\Xlo\doit_n$ ou $\Xlo\peutn{n}$ et $\Xlo\doitn n$ dans \LO. 
Les règles d'interprétation des modalités seront alors paramétrées par ces étiquettes $n$, comme par exemple : 
\(\denote{\Xlo\peutn{n}\phi}^{\Modele,w,i,g}=1\), ssi il existe un monde
  $w'$ de {\Unv{W}} tel que $w \mathrel{R_n} w'$ et \(\denote{\Xlo\phi}^{\Modele,w',i,g}=1\).  Le modèle intensionnel devient donc 
\(\Modele = \tuple{\Unv{A},\Unv{W},\Unv R,\Tps_{\tprec},\FI}\), où \Unv R est\is{R@\Unv R} l'ensemble qui comprend toutes les relations $R_n$ dont nous avons besoin. 

C'est là une pratique courante en logique multimodale\is{multimodalité} et elle améliore indéniablement l'expressivité de {\LO}. 
Pour autant, si l'on approfondit l'étude sémantique des modalités dans la langue, on se rend compte que cette formalisation n'est pas encore parfaitement adéquate. Elle se heurte à quelques paradoxes logiques et pèche encore par quelques limites d'expressivité. Nous reviendrons sur ces problèmes dans le chapitre~\ref{Ch:modalites} (vol.~2). Elle masque aussi le fait que le choix exact de la relation d'accessibilité dépend très souvent du contexte.  Certes on peut admettre que des expressions comme \sicut{avoir le droit} ou \sicut{il faut que} sont assez univoquement déontiques\footnote{Encore que pour \sicut{falloir}, ce n'est pas si sûr...}, de même que \sicut{il se peut que}, \sicut{peut-être}, \sicut{probablement} sont intrinsèquement épistémiques. Mais cela ne suffit pas. 
Par exemple, les modalités épistémiques s'appuient sur un ensemble de connaissances du monde ; on considère couramment qu'il s'agit vraisemblablement des connaissances du locuteur ou d'un groupe comprenant le locuteur, mais le locuteur (et donc ses connaissances) est un paramètre qui est fixé par le contexte,\is{contexte} pas par le modèle.  Au mieux il faudrait que \Unv R contienne une relation d'accessibilité épistémique pour chaque locuteur possible à chaque instant de \Unv I, mais il nous manquerait encore un mécanisme qui permette de sélectionner celle qui est pertinente dans le contexte.  

L'interprétation de modalités déontiques peut aussi dépendre du contexte. Le locuteur qui énonce {\Next} se réfère implicitement à une certaine autorité qui décrète ce qui est permis, obligatoire, interdit.

\ex. Alice a le droit de jouer de la trompette.

Mais quelle est cette autorité ? Le locuteur lui-même ? Le médecin ? Les parents d'Alice ? Le règlement de copropriété ? Le droit pénal ?...  À chacune correspond une relation déontique différente, et {\Last} est, à cet égard, vague,  tout en ayant une valeur déontique : peut-être que le docteur a autorisé Alice à jouer de la trompette au vu de son état de santé, mais que ses parents le lui interdisent car il est tard et que cela dérangerait les voisins.  L'interprétation exacte de {\Last} dépend donc d'un certain état mental du locuteur et cela est un paramètre du contexte. 

Il existe plusieurs façons d'aménager notre langage {\LO} multimodal pour refléter cette dépendance contextuelle. 
Une solution simple et propre consisterait à ajouter $\RK$ parmi les indices d'évaluation, en écrivant ainsi \(\denote{\Xlo\phi}^{\Modele,w,\RK,g}\) pour l'interprétation de \vrb\phi.  Mais cette option a le défaut de nous obliger à interpréter tous les opérateurs modaux d'une formule avec la même relation $\RK$. Or nous n'avons aucune raison de faire cela, au contraire : lorsque plusieurs modalités s'empilent dans une phrase, comme par exemple \sicut{Alice doit pouvoir faire cet exercice}, elles ont normalement des valeurs différentes.
Nous avons donc plutôt intérêt à exploiter le système des multimodalités\is{multimodalité} ($\Xlo\peutn{n}$ et $\Xlo\doitn n$) tout en nous servant de l'assignation $g$ pour rendre compte de la sensibilité contextuelle de l'interprétation des modaux. 
Le principe est de traiter les symboles \vrb n un peu comme des variables : 
nous faisons en sorte que les assignations $g$ soient aussi définies pour l'ensemble de ces symboles, et à chacun d'eux les fonctions $g$ assignent une relation de $\Unv R$. Ainsi le «choix» de la valeur modale des opérateurs n'est pas directement déterminé par \vrb n, il se fait par l'intermédiaire de $g$ qui est une composante du contexte.  Précisons enfin que cette formalisation ne résout pas tous les problèmes (\cf la question du contenu précis de \Unv R), mais elle offre suffisamment d'expressivité et de cohérence formelle pour ce qui nous occupe présentement.





\subsubsection{Retour vers le futur}
%'''''''''''''''''''''''''''''''''''
\label{s:branchants}\is{futur|(}
\newpsstyle{wi}{dotstyle=o}

Pour conclure cette section sur les modalités et les mondes possibles ainsi que celle sur la temporalité, revenons un instant sur la critique que nous avions évoquée à propos de la modélisation du futur en \S\ref{ss:PbTmps} p.~\pageref{ss:PbTmps}.  Elle portait sur le caractère préétabli des états futurs du monde imposé par {\Tps}.  Les arguments que j'avais proposés pour y répondre valent toujours, mais maintenant que notre modèle comporte les mondes possibles, nous disposons d'un outillage qui permet de relativiser avantageusement le problème. 

Rappelons que ce qui identifie précisément un état possible de monde, c'est un indice double, c'est-à-dire un couple \tuple{w,i}. Supposons que nous nous placions dans le monde $\w_1$ à l'instant $i_1$ (donc dans l'état \tuple{\w_1,i_1}), alors il est raisonnable de considérer que l'ensemble de tous les indices \tuple{\w_1,i'}, avec $i'\tpreceq i_1$, constitue \emph{l'histoire passée} de $\w_1$ jusqu'à $i_1$. 
À l'inverse, l'avenir de \tuple{\w_1,i_1}, lui, n'est pas tout tracé, et il peut donc être judicieux d'utiliser plusieurs autres mondes possibles (en plus de $\w_1$) pour se donner de multiples alternatives des évolutions possibles et à venir du cours des choses à partir de \tuple{\w_1,i_1}.  Le futur de cet état du monde est donc un ensemble d'indices \tuple{w,i''} tels que $i_1\tprec i''$ et $w$ fait partie de l'ensemble des mondes qui peuvent être des continuations de $\w_1$ depuis $i_1$. En procédant de la sorte, nous modélisons le cours des choses sous la forme d'une \emph{structure branchante}\is{structure!\elid\ branchante} où l'histoire du monde (ou plus exactement \emph{des} mondes) se déploie en se ramifiant continuellement et de plus en plus, en diverses évolutions alternatives. 
Cette vision du «temps à venir»\is{temps} peut intuitivement se représenter sous la forme d'une arborescence, comme l'illustre la figure~\ref{futurs1} (ci-contre).

%noté {\psdot[style=wi](0,.5ex)} représente


\begin{figure}[h!]
\begin{center}
\scalebox{.9}{\newpsstyle{etc}{linestyle=dotted,dotsep=2pt}
\begin{pspicture}(10,6)
\psline[linecolor=gray](0,3)(4,3)
\psline[linecolor=gray](4,3)(5,3)
\psline[linecolor=gray](4,3)(5,2.2)
\psline[linecolor=gray](4,3)(5,4.1)
\psline[linecolor=gray](5,3)(6,3)
\psline[linecolor=gray](5,3)(6,3.4)
\psline[linecolor=gray](5,4.1)(6,4.1)
\psline[linecolor=gray](5,4.1)(6,4.9)
\psline[linecolor=gray](5,2.2)(6,2.4)
\psline[linecolor=gray](5,2.2)(6,2)
\psline[linecolor=gray](5,2.2)(6,1.5)
\psline[linecolor=gray](6,3)(7,3)
\psline[linecolor=gray,style=etc](7,3)(7.6,3)
\psline[linecolor=gray](6,3)(7,2.8)
\psline[linecolor=gray,style=etc](7,2.8)(7.6,2.8)
\psline[linecolor=gray](6,3.4)(7,3.5)
\psline[linecolor=gray,style=etc](7,3.5)(7.6,3.5)
\psline[linecolor=gray](6,4.1)(7,4)
\psline[linecolor=gray,style=etc](7,4)(7.6,4)
\psline[linecolor=gray](6,4.1)(7,4.3)
\psline[linecolor=gray,style=etc](7,4.3)(7.6,4.3)
\psline[linecolor=gray](6,4.9)(7,4.7)
\psline[linecolor=gray,style=etc](7,4.7)(7.6,4.7)
\psline[linecolor=gray](6,4.9)(7,5.4)
\psline[linecolor=gray,style=etc](7,5.4)(7.6,5.4)
\psline[linecolor=gray](6,2.4)(7,2.5)
\psline[linecolor=gray,style=etc](7,2.5)(7.6,2.5)
\psline[linecolor=gray](6,2)(7,2.2)
\psline[linecolor=gray,style=etc](7,2.2)(7.6,2.2)
\psline[linecolor=gray](6,2)(7,2)
\psline[linecolor=gray,style=etc](7,2)(7.6,2)
\psline[linecolor=gray](6,1.5)(7,1.7)
\psline[linecolor=gray,style=etc](7,1.7)(7.6,1.7)
\psline[linecolor=gray](6,1.5)(7,1.1)
\psline[linecolor=gray,style=etc](7,1.1)(7.6,1.1)
\psdot[style=wi](1,3)
\psdot[style=wi](2,3)
\psdot[style=wi](3,3)
\psdot[style=wi](4,3)
\psdot[style=wi](5,3)
\psdot[style=wi](5,2.2)
\psdot[style=wi](5,4.1)
\psdot[style=wi](6,3)
\psdot[style=wi](6,3.4)
\psdot[style=wi](6,4.1)
\psdot[style=wi](6,4.9)
\psdot[style=wi](6,2.4)
\psdot[style=wi](6,2)
\psdot[style=wi](6,1.5)
\psdot[style=wi](7,3)
\psdot[style=wi](7,2.8)
\psdot[style=wi](7,3.5)
\psdot[style=wi](7,4)
\psdot[style=wi](7,4.3)
\psdot[style=wi](7,4.7)
\psdot[style=wi](7,5.4)
\psdot[style=wi](7,2.5)
\psdot[style=wi](7,2.2)
\psdot[style=wi](7,2)
\psdot[style=wi](7,1.7)
\psdot[style=wi](7,1.1)
%\dotnode[style=instant](1,.5){i2}\rput[B](1.5,0){$i''$}%
\psdot[dotstyle=|](4,.5)\rput[B](4,0){$i_1$}%
\rput(-0.25,3){$\w_1$}
%\dotnode[style=instant](4.5,.5){i0}\rput[B](4.5,0){$i$}%
\psline{->}(0,.5)(9,.5)\rput[B](9,.7){\Unv{I}}%
\psline{->}(.2,0.3)(.2,5.5)\rput[B](-0.1,5.4){\Unv{W}}%
\end{pspicture}%
}%
\caption{Représentation des futurs branchants de \tuple{\w_1,i_1}}\label{futurs1}
\end{center}
\end{figure}

Dans ce genre de représentation graphique, chaque point du plan est un état \tuple{w,i} et on l'identifie en posant un repère où {\Tps} est l'axe des abscisses et \Unv W l'axe des ordonnées. Ainsi un état \tuple{w,i} a pour coordonnées... eh bien \tuple{w,i} tout simplement\footnote{Habituellement en mathématiques, on note les coordonnées dans l'ordre inverse : abscisse, ordonnées ; ici, pour faire simple, nous conservons notre notation \tuple{w,i}.}.  Tous les points qui sont sur une même ligne horizontale (parallèle à \Tps) sont des états d'un même monde $w$ à différents instants ; et tous les points sur une même ligne verticale sont toutes les variantes de l'état du monde à un même instant.
À partir de $i_1$ chaque branche tracée sur la figure représente une évolution possible du cours de l'histoire. Au fur et à mesure qu'elle avance dans le temps, l'histoire, à chaque intersection, emprunte une des branches qui se présentent et écarte les autres qui deviennent caduques. \pagebreak Ce faisant elle dessine progressivement le cours de l'histoire advenue, comme l'illustrent les figures~\ref{futurs2} (page suivante).

\begin{figure}[h!]
\begin{bigcenter}
\begin{tabular}{@{}c@{}}
\scalebox{.88}{\newpsstyle{etc}{linestyle=dotted,dotsep=2pt}
\begin{pspicture}(10,6)
\psline[linecolor=gray](0,3)(4,3)
%% \psline[linecolor=gray](4,3)(5,3)
%% \psline[linecolor=gray](4,3)(5,2.2)
\psline[linecolor=gray](4,3)(5,4.1)
%% \psline[linecolor=gray](5,3)(6,3)
%% \psline[linecolor=gray](5,3)(6,3.4)
\psline[linecolor=gray](5,4.1)(6,4.1)
\psline[linecolor=gray](5,4.1)(6,4.9)
%% \psline[linecolor=gray](5,2.2)(6,2.4)
%% \psline[linecolor=gray](5,2.2)(6,2)
%% \psline[linecolor=gray](5,2.2)(6,1.5)
%% \psline[linecolor=gray](6,3)(7,3)
%% \psline[linecolor=gray,style=etc](7,3)(7.6,3)
%% \psline[linecolor=gray](6,3)(7,2.8)
%% \psline[linecolor=gray,style=etc](7,2.8)(7.6,2.8)
%% \psline[linecolor=gray](6,3.4)(7,3.5)
%% \psline[linecolor=gray,style=etc](7,3.5)(7.6,3.5)
\psline[linecolor=gray](6,4.1)(7,4)
\psline[linecolor=gray,style=etc](7,4)(7.6,4)
\psline[linecolor=gray](6,4.1)(7,4.3)
\psline[linecolor=gray,style=etc](7,4.3)(7.6,4.3)
\psline[linecolor=gray](6,4.9)(7,4.7)
\psline[linecolor=gray,style=etc](7,4.7)(7.6,4.7)
\psline[linecolor=gray](6,4.9)(7,5.4)
\psline[linecolor=gray,style=etc](7,5.4)(7.6,5.4)
%% \psline[linecolor=gray](6,2.4)(7,2.5)
%% \psline[linecolor=gray,style=etc](7,2.5)(7.6,2.5)
%% \psline[linecolor=gray](6,2)(7,2.2)
%% \psline[linecolor=gray,style=etc](7,2.2)(7.6,2.2)
%% \psline[linecolor=gray](6,2)(7,2)
%% \psline[linecolor=gray,style=etc](7,2)(7.6,2)
%% \psline[linecolor=gray](6,1.5)(7,1.7)
%% \psline[linecolor=gray,style=etc](7,1.7)(7.6,1.7)
%% \psline[linecolor=gray](6,1.5)(7,1.1)
%% \psline[linecolor=gray,style=etc](7,1.1)(7.6,1.1)
\psdot[style=wi](1,3)
\psdot[style=wi](2,3)
\psdot[style=wi](3,3)
\psdot[style=wi](4,3)
%% \psdot[style=wi](5,3)
%% \psdot[style=wi](5,2.2)
\psdot[style=wi](5,4.1)
%% \psdot[style=wi](6,3)
%% \psdot[style=wi](6,3.4)
\psdot[style=wi](6,4.1)
\psdot[style=wi](6,4.9)
%% \psdot[style=wi](6,2.4)
%% \psdot[style=wi](6,2)
%% \psdot[style=wi](6,1.5)
%% \psdot[style=wi](7,3)
%% \psdot[style=wi](7,2.8)
%% \psdot[style=wi](7,3.5)
\psdot[style=wi](7,4)
\psdot[style=wi](7,4.3)
\psdot[style=wi](7,4.7)
\psdot[style=wi](7,5.4)
%% \psdot[style=wi](7,2.5)
%% \psdot[style=wi](7,2.2)
%% \psdot[style=wi](7,2)
%% \psdot[style=wi](7,1.7)
%% \psdot[style=wi](7,1.1)
%\dotnode[style=instant](1,.5){i2}\rput[B](1.5,0){$i''$}%
\psdot[dotstyle=|](4,.5)\rput[B](4,0){$i_1$}%
\psdot[dotstyle=|](5,.5)\rput[B](5,0){$i_2$}%
\rput(-0.25,3){$\w_1$}
%\dotnode[style=instant](4.5,.5){i0}\rput[B](4.5,0){$i$}%
\psline{->}(0,.5)(9,.5)\rput[B](9,.7){\Unv{I}}%
\psline{->}(.2,0.3)(.2,5.5)\rput[B](-0.1,5.4){\Unv{W}}%
\end{pspicture}%
}%
\\[1ex](a)\\
\scalebox{.88}{\newpsstyle{etc}{linestyle=dotted,dotsep=2pt}
\begin{pspicture}(10,6)
\psline[linecolor=gray](0,3)(4,3)
%% \psline[linecolor=gray](4,3)(5,3)
%% \psline[linecolor=gray](4,3)(5,2.2)
\psline[linecolor=gray](4,3)(5,4.1)
%% \psline[linecolor=gray](5,3)(6,3)
%% \psline[linecolor=gray](5,3)(6,3.4)
\psline[linecolor=gray](5,4.1)(6,4.1)
%% \psline[linecolor=gray](5,4.1)(6,4.9)
%% \psline[linecolor=gray](5,2.2)(6,2.4)
%% \psline[linecolor=gray](5,2.2)(6,2)
%% \psline[linecolor=gray](5,2.2)(6,1.5)
%% \psline[linecolor=gray](6,3)(7,3)
%% \psline[linecolor=gray,style=etc](7,3)(7.6,3)
%% \psline[linecolor=gray](6,3)(7,2.8)
%% \psline[linecolor=gray,style=etc](7,2.8)(7.6,2.8)
%% \psline[linecolor=gray](6,3.4)(7,3.5)
%% \psline[linecolor=gray,style=etc](7,3.5)(7.6,3.5)
\psline[linecolor=gray](6,4.1)(7,4)
\psline[linecolor=gray,style=etc](7,4)(7.6,4)
\psline[linecolor=gray](6,4.1)(7,4.3)
\psline[linecolor=gray,style=etc](7,4.3)(7.6,4.3)
%% \psline[linecolor=gray](6,4.9)(7,4.7)
%% \psline[linecolor=gray,style=etc](7,4.7)(7.6,4.7)
%% \psline[linecolor=gray](6,4.9)(7,5.4)
%% \psline[linecolor=gray,style=etc](7,5.4)(7.6,5.4)
%% \psline[linecolor=gray](6,2.4)(7,2.5)
%% \psline[linecolor=gray,style=etc](7,2.5)(7.6,2.5)
%% \psline[linecolor=gray](6,2)(7,2.2)
%% \psline[linecolor=gray,style=etc](7,2.2)(7.6,2.2)
%% \psline[linecolor=gray](6,2)(7,2)
%% \psline[linecolor=gray,style=etc](7,2)(7.6,2)
%% \psline[linecolor=gray](6,1.5)(7,1.7)
%% \psline[linecolor=gray,style=etc](7,1.7)(7.6,1.7)
%% \psline[linecolor=gray](6,1.5)(7,1.1)
%% \psline[linecolor=gray,style=etc](7,1.1)(7.6,1.1)
\psdot[style=wi](1,3)
\psdot[style=wi](2,3)
\psdot[style=wi](3,3)
\psdot[style=wi](4,3)
%% \psdot[style=wi](5,3)
%% \psdot[style=wi](5,2.2)
\psdot[style=wi](5,4.1)
%% \psdot[style=wi](6,3)
%% \psdot[style=wi](6,3.4)
\psdot[style=wi](6,4.1)
%% \psdot[style=wi](6,4.9)
%% \psdot[style=wi](6,2.4)
%% \psdot[style=wi](6,2)
%% \psdot[style=wi](6,1.5)
%% \psdot[style=wi](7,3)
%% \psdot[style=wi](7,2.8)
%% \psdot[style=wi](7,3.5)
\psdot[style=wi](7,4)
\psdot[style=wi](7,4.3)
%% \psdot[style=wi](7,4.7)
%% \psdot[style=wi](7,5.4)
%% \psdot[style=wi](7,2.5)
%% \psdot[style=wi](7,2.2)
%% \psdot[style=wi](7,2)
%% \psdot[style=wi](7,1.7)
%% \psdot[style=wi](7,1.1)
%\dotnode[style=instant](1,.5){i2}\rput[B](1.5,0){$i''$}%
\psdot[dotstyle=|](4,.5)\rput[B](4,0){$i_1$}%
\psdot[dotstyle=|](5,.5)\rput[B](5,0){$i_2$}%
\psdot[dotstyle=|](6,.5)\rput[B](6,0){$i_3$}%
\rput(-0.25,3){$\w_1$}
%\dotnode[style=instant](4.5,.5){i0}\rput[B](4.5,0){$i$}%
\psline{->}(0,.5)(9,.5)\rput[B](9,.7){\Unv{I}}%
\psline{->}(.2,0.3)(.2,5.5)\rput[B](-0.1,5.4){\Unv{W}}%
\end{pspicture}%
}%
\\[1ex](b)
%(a)&(b)
\end{tabular}
\caption{Structure branchante du point de vue de $i_2$, puis de $i_3$}\label{futurs2}
\end{bigcenter}
\end{figure}


Ces figures sont pratiques car elles illustrent très intuitivement le fonctionnement de la structure temporelle branchante. Mais elles ont le défaut de ne pas représenter très fidèlement sa modélisation, et à cet égard, elles peuvent parfois induire en erreur. C'est pourquoi je vais prendre ici le temps d'expliciter un peu la formalisation de ce type de structure.

Nous pouvons d'abord remarquer que dans la figure~\ref{futurs1}, la ramification prend sa source à \tuple{\w_1,i_1}.   Mais en réalité, dans le modèle, tout état  \tuple{w,i} (y compris ceux du «passé») est à l'origine de sa propre ramification de futurs possibles.
Ces embranchements passés ne sont pas représentés dans la figure parce qu'elle se place du point de vue d'état \tuple{\w_1,i_1}, mais en fait ils sont bien là dans le modèle et nous aurions tort de les oublier\footnote{Par exemple pour interpréter des phrases comme \sicut{Le Titanic aurait pu ne pas couler}.}.
Ensuite, en regardant l'évolution sur les trois figures, nous remarquons qu'à partir de $i_1$, l'histoire part un peu en zigzag (on change de monde) alors que dans le passé de $i_1$ elle est rectiligne (on est toujours sur $\w_1$). L'histoire passée ne devrait-elle pas être elle aussi en dents de scie ? Ce n'est pas un inoffensif artefact graphique,  cela reflète plutôt une trop grande simplification dans la figure.  En effet, si nous regardons la figure~\ref{futurs2}b, nous constatons que \tuple{\w_1,i_3} n'est finalement pas un état avéré du monde  $\w_1$ (car c'est semble-t-il bien l'histoire de $\w_1$ que la figure cherche à dessiner) ce qui, formulé ainsi, semble assez contradictoire. 

Comme nous nous en doutons, la structure branchante est assurée par une relation d'accessibilité.  Mais il s'agit d'une relation un peu particulière car elle doit aussi tenir compte de la temporalité, autrement dit elle dépend du temps.
Nous pourrions à cet effet envisager que la relation connecte directement des couples \tuple{w,i} plutôt que simplement des mondes ; mais techniquement, il est plus simple et plus satisfaisant de suivre la formalisation proposée par  \citet{Thomason:84}\Andex{Thomason, R.} dans son système nommé \emph{la sémantique $T\mathord{\times}W$}.\is{T W@$T\mathord{\times}W$} 
Ce système n'utilise pas, en fait, \emph{une} relation d'accessibilité, mais autant qu'il y a d'instants dans {\Tps}. Chaque relation est donc indexée par un instant $i$ et nous la noterons $\RK_i$\footnote{Thomason la note $\approx_i$ car il s'agit d'une relation d'équivalence, c'est-à-dire qu'elle est réflexive, symétrique et transitive.}. Poser \mbox{$w\RK_i w'$} signifie que les mondes $w$ et $w'$ partagent la même histoire passée jusqu'à l'instant $i$.
En pratique, cela veut dire que pour tout instant $i'$ tel que $i'\tpreceq i$,
les états du monde \tuple{w,i'} et \tuple{w',i'} sont identiques. Cela ne veut pas dire que nous avons le droit d'écrire $\tuple{w,i'}=\tuple{w',i'}$, car normalement les mondes $w$ et $w'$ restent distincts ($w\neq w'$), précisément parce qu'à des instants postérieurs à $i$, nous prévoyons que $w$ et $w'$ renvoient à des états du monde qui ne seront plus identiques. 
Et c'est ainsi que nous obtenons notre structure de futurs branchants (ou divergents).

\label{p.mhisto}

Notons que cette relation d'accessibilité engendre elle aussi une valeur particulière de modalité que l'on appelle la \kwo{modalité historique}\is{modalite@modalité!\elid\ historique} (ou parfois également \kwo{métaphysique})\is{modalite@modalité!\elid\ métaphysique}.  Avec cette valeur, le nécessaire correspond à ce qui est \emph{inévitable} (ou du moins historiquement inévitable, historiquement fixé). 
L'interprétation d'une formule modale dépend alors à la fois du monde et de l'instant d'évaluation. Par exemple, nous aurons \(\denote{\Xlo\doit\phi}^{\Modele,w,i,g}=1\), ssi pour tout monde
  $w'$ de \Unv{W} tel que $w \RK_i w'$,  \(\denote{\Xlo\phi}^{\Modele,w',i,g}=1\).

\sloppy
Pour que les $\RK_i$ jouent correctement leur rôle de relations d'accessibilité historiques, il faut contraindre notre  modèle à respecter les conditions suivantes. 
La première est que si  $w\RK_i w'$, alors pour tout $i'\tprec i$, $w\RK_{i'}w'$ ; c'est-à-dire que les connexions que la relation établit à l'instant $i$ sont aussi établies dans tout le passé de $i$. Cette condition est nécessaire pour garantir le parallélisme historique entre $w$ et $w'$, mais elle ne suffit pas encore pour dire qu'ils ont le \emph{même} passé. 
Pour cela, nous devons faire en sorte que \tuple{w,i} et \tuple{w',i} désignent des états du monde identiques, ce qui se réalise en contraignant la fonction d'interprétation {\FI} du modèle. {\FI} doit être telle que si $w\RK_i w'$, alors pour tout prédicat \vrb P de {\LO}, $\FI(w,i,\vrb P)=\FI(w',i,\vrb P)$\footnote{Cette égalité vaut aussi pour les constantes d'individu, mais nous l'avions déjà par l'hypothèse des désignateurs rigides.}.  Ainsi il se passe exactement les mêmes choses dans \tuple{w,i} et \tuple{w',i}. Et combinée à la condition précédente, cette égalité vaut aussi pour tous les instants antérieurs à $i$.
Les mondes $w$ et $w'$ partagent donc bien la même histoire jusqu'à $i$. 
La structure branchante que nous obtenons peut, plus rigoureusement, se représenter graphiquement comme dans la figure~\ref{futurs4}.

\fussy

\newpsstyle{rh}{linecolor=lightgray,linestyle=dashed,dash=2pt 1pt}


\begin{figure}[h!]
\begin{center}
\scalebox{.9}{\newpsstyle{etc}{linestyle=dotted,dotsep=2pt,linecolor=gray}
{\begin{pspicture}(10,5.5)(-0.5,0)\psset{yunit=.8cm,xunit=1.2cm}
\psline[linecolor=gray](0,1)(6.5,1)\psline[style=etc](6.5,1)(7,1)
\psline[linecolor=gray](0,2)(6.5,2)\psline[style=etc](6.5,2)(7,2)
\psline[linecolor=gray](0,3)(6.5,3)\psline[style=etc](6.5,3)(7,3)
\psline[linecolor=gray](0,4)(6.5,4)\psline[style=etc](6.5,4)(7,4)
\psline[linecolor=gray](0,5)(6.5,5)\psline[style=etc](6.5,5)(7,5)
\psline[linecolor=gray](0,6)(6.5,6)\psline[style=etc](6.5,6)(7,6)
\dotnode[style=wi](1,1){w11}\dotnode[style=wi](1,2){w12}
\dotnode[style=wi](1,3){w13}\dotnode[style=wi](1,4){w14}
\dotnode[style=wi](1,5){w15}\dotnode[style=wi](1,6){w16}
\ncline[style=rh]{<->}{w11}{w12}\ncline[style=rh]{<->}{w12}{w13}\ncline[style=rh]{<->}{w13}{w14}\ncline[style=rh]{<->}{w14}{w15}\ncline[style=rh]{<->}{w15}{w16}
\dotnode[style=wi](2,1){w21}\dotnode[style=wi](2,2){w22}
\dotnode[style=wi](2,3){w23}\dotnode[style=wi](2,4){w24}
\dotnode[style=wi](2,5){w25}\dotnode[style=wi](2,6){w26}
\ncline[style=rh]{<->}{w21}{w22}\ncline[style=rh]{<->}{w22}{w23}\ncline[style=rh]{<->}{w23}{w24}\ncline[style=rh]{<->}{w24}{w25}\ncline[style=rh]{<->}{w25}{w26}
\dotnode[style=wi](3,1){w31}\dotnode[style=wi](3,2){w32}
\dotnode[style=wi](3,3){w33}\dotnode[style=wi](3,4){w34}
\dotnode[style=wi](3,5){w35}\dotnode[style=wi](3,6){w36}
\ncline[style=rh]{<->}{w31}{w32}\ncline[style=rh]{<->}{w32}{w33}\ncline[style=rh]{<->}{w33}{w34}\ncline[style=rh]{<->}{w34}{w35}\ncline[style=rh]{<->}{w35}{w36}
\dotnode[style=wi](4,1){w41}\dotnode[style=wi](4,2){w42}
\dotnode[style=wi](4,3){w43}\dotnode[style=wi](4,4){w44}
\dotnode[style=wi](4,5){w45}\dotnode[style=wi](4,6){w46}
\ncline[style=rh]{<->}{w41}{w42}\ncline[style=rh]{<->}{w43}{w44}\ncline[style=rh]{<->}{w44}{w45}\ncline[style=rh]{<->}{w45}{w46}
\dotnode[style=wi](5,1){w51}\dotnode[style=wi](5,2){w52}
\dotnode[style=wi](5,3){w53}\dotnode[style=wi](5,4){w54}
\dotnode[style=wi](5,5){w55}\dotnode[style=wi](5,6){w56}
\ncline[style=rh]{<->}{w51}{w52}\ncline[style=rh]{<->}{w53}{w54}\ncline[style=rh]{<->}{w55}{w56}
\dotnode[style=wi](6,1){w61}\dotnode[style=wi](6,2){w62}
\dotnode[style=wi](6,3){w63}\dotnode[style=wi](6,4){w64}
\dotnode[style=wi](6,5){w65}\dotnode[style=wi](6,6){w66}
\ncline[style=rh]{<->}{w61}{w62}\ncline[style=rh]{<->}{w63}{w64}%\ncline[style=rh]{<->}{w65}{w66}
%\dotnode[style=instant](1,.5){i2}\rput[B](1.5,0){$i''$}%
\psdot[dotstyle=|](1,.5)\rput[B](1,0){$i_1$}%
\psdot[dotstyle=|](2,.5)\rput[B](2,0){$i_2$}%
\psdot[dotstyle=|](3,.5)\rput[B](3,0){$i_3$}%
\psdot[dotstyle=|](4,.5)\rput[B](4,0){$i_4$}%
\psdot[dotstyle=|](5,.5)\rput[B](5,0){$i_5$}%
\psdot[dotstyle=|](6,.5)\rput[B](6,0){$i_6$}%
\rput(-0.25,1){$\w_1$}\rput(-0.25,2){$\w_2$}\rput(-0.25,3){$\w_3$}\rput(-0.25,4){$\w_4$}\rput(-0.25,5){$\w_5$}\rput(-0.25,6){$\w_6$}
%\dotnode[style=instant](4.5,.5){i0}\rput[B](4.5,0){$i$}%
\psline{->}(0,.5)(8,.5)\rput[B](8,.7){\Unv{I}}%
\psline{->}(.2,0.3)(.2,6.5)\rput[B](-0.1,6.4){\Unv{W}}%
\end{pspicture}%
}


}
\caption{Structure branchante par une relation d'accessibilité historique}\label{futurs4}
\end{center}
\end{figure}

Cette figure montre que, formellement, chaque monde de \Unv W suit sa propre histoire, linéaire et unique, et ce jusqu'à la fin des temps. 
Les flèches verticales marquées \makebox[.82cm][l]{\psline[style=rh]{<->}(0,.6ex)(8mm,.6ex)} représentent les relations $\RK_i$, et donc deux états du monde reliés par ces flèches sont identiques.  Bien sûr il y a aussi dans \Unv W une multitude de mondes qui ne sont pas du tout reliés à ceux de la figure (ils ne sont juste pas représentés ici).
Supposons maintenant que nous nous considérons dans \tuple{\w_3,i_3}. Mais en fait, comme cet état est indiscernable (en soi et par son passé) des cinq autres états de $i_3$, nous pourrions tout aussi bien être dans \tuple{\w_1,i_3} ou \tuple{\w_5,i_3}, etc.  En réalité, nous ne savons pas exactement dans lequel des six états nous sommes. Lorsque le temps aura passé, et que nous nous retrouverons à $i_4$, nous en saurons un peu plus. Par exemple à ce moment-là, nous découvrirons que nous ne sommes pas dans $\w_1$ (ni donc dans $\w_2$) et donc que nous sommes dans un des quatre autres mondes (car ils ne sont pas identiques aux deux premiers).  Et ainsi de suite. À $i_5$ nous découvrirons peut-être que nous sommes dans $\w_5$ ou $\w_6$, et les alternatives $\w_3$ et $\w_4$ seront devenues caduques.

Cette formalisation de structure branchante repose donc fondamentalement sur une série d'équivalences entre états du monde qui induit, à un moment donné, une indétermination entre les mondes, et donc une relative ignorance des locuteurs sur le monde exact dans lequel ils se situent. Mais ce n'est pas grave (au contraire, cela confirme que nous ne connaissons pas l'avenir). Et à l'arrivée, ce que dessine la figure~\ref{futurs4} est, dans l'esprit, similaire à ce que suggère intuitivement la figure~\ref{futurs1}.  C'est juste que ce qui apparaît, dans cette dernière, comme une simple ligne historique (le passé de \tuple{\w_1,i_1}) est en réalité un faisceau de multiples états équivalents.
Notons à ce propos que l'ajout d'une structure branchante dans le modèle décuple encore formidablement les proportions de \Unv W. Par exemple, il existe une quantité inouïe de mondes possibles qui sont jusqu'à présent strictement identiques à notre monde réel et qui ne commenceront à en diverger qu'au \textsc{xxvii}\ieme siècle. 

La dernière question qui se pose évidemment c'est : comment utiliser ce type modèle pour rendre compte de la contribution sémantique des expressions linguistiques du futur ?
Nous ne la trancherons pas ici, mais nous pouvons voir quels éléments sont utiles pour y répondre. 
J'ai mentionné en \S\ref{ss:PbTmps} que le futur linguistique comportait une composante modale. Nous pourrions alors envisager d'introduire un nouvel opérateur, «temporo-modal», qui fait à la fois un décalage sur l'indice $i$ et de la quantification sur les $w$. 
Mais nous  pouvons aussi exploiter ce dont nous disposons déjà, en combinant $\Xlo\doit$ (à interpréter avec une relation d'accessibilité historique) et $\mF$.
La force modale associée au futur est très certainement de l'ordre du nécessaire, car les phrases au futur se présentent généralement comme des prédictions du locuteur. Mais il faudra bien avoir à l'esprit  
que $\Xlo\doit\mF\phi$ et $\Xlo\mF\doit\phi$ ne sont pas équivalentes, la première étant plus forte que la seconde. 
La traduction la plus satisfaisante du futur est probablement du côté de $\Xlo\doit\mF$, mais en perfectionnant l'interprétation du nécessaire, comme nous le verrons au chapitre~\ref{Ch:modalites}~(vol.~2).

\medskip

% -*- coding: utf-8 -*-
\begin{exo}\label{exo:mHisto}
On interprète $\Xlo\doit$ comme historique. 
\pagesolution{crg:mHisto}
Démontrez, en utilisant par exemple le modèle de la figure~\ref{futurs4}, que $\Xlo\doit\mF\phi$ et $\Xlo\mF\doit\phi$ ne sont pas équivalentes.
\\
Et démontrez que $\Xlo[\mP\phi\implq\doit\mP\phi]$ est vraie pour tout $w$ et tout $i$.
\begin{solu}(p.~\pageref{exo:mHisto})\label{crg:mHisto}

Selon la règle d'interprétation de $\mF$ (\RSem\ref{RSemTps}), p.~\pageref{RSemTps}, et celle de $\Xlo\doit$ (\RSem\ref{RSemMod}), p.~\pageref{d:semMod2}, appliquée aux modalités historiques (\S\ref{s:branchants}, p.~\pageref{p.mhisto}),  $\Xlo\doit\mF\phi$ dit que pour tout  monde actuellement identique au monde d'évaluation courant, il y a un instant du futur où \vrb\phi\ est vraie. $\Xlo\mF\doit\phi$ dit qu'il existe un instant du futur durant lequel, dans tous les mondes qui seront alors identiques au monde d'évaluation courant, \vrb\phi\ est vraie.  Reprenons la figure \ref{futurs4}, p.~\pageref{futurs4}, et supposons que $\vrb\phi$ est vraie dans les états \tuple{\w_3,i_5} et \tuple{\w_4,i_5} et que $\vrb\phi$ est fausse dans tous les états de $\w_5$ et $\w_6$.  Plaçons nous dans l'état \tuple{\w_4,i_4}.  De là, \(\denote{\Xlo\mF\doit\phi}^{\Modele,\w_4,i_4}=1\) car $i_5$ est postérieur à $i_4$ et dans tous les mondes en relation avec $\w_4$ à l'instant $i_5$ (c'est-à-dire $\w_3$ et $\w_4$), \vrb\phi\ est vraie. Mais \(\denote{\Xlo\doit\mF\phi}^{\Modele,\w_4,i_4}=0\) car tous les mondes reliés à $\w_4$ à l'instant $i_4$ sont $\w_3$, $\w_4$, $\w_5$ et $\w_6$, et dans $\w_5$ et $\w_6$ il n'y a pas d'instant postérieur à $i_4$ où \vrb\phi\ est vraie.

Soit \tuple{w,i} un état du monde quelconque.  Et supposons que $\Xlo\mP\phi$ est vraie dans \tuple{w,i}.  Donc il existe un instant $i'$ antérieur à $i$ tel que \vrb\phi\ est vraie dans \tuple{w,i'}.  Par définition des relations d'accessibilité historiques, tous les mondes $w'$ reliés à $w$ par $\RK_i$ sont aussi reliés à $w$ par $\RK_{i'}$ (parce que $i'\tprec i$), et tous les états \tuple{w',i'} sont identiques à \tuple{w,i'}.  Donc comme $\vrb\phi$ est vraie dans \tuple{w,i'}, nous savons que $\vrb\phi$ est vraie dans tous les états \tuple{w',i'} ; cela permet de conclure que $\Xlo\mP\phi$ est vraie dans tous les états \tuple{w',i} et donc que $\Xlo\doit\mP\phi$ est vraie dans \tuple{w,i}. Ainsi, si $\Xlo\mP\phi$ est vraie dans \tuple{w,i}, alors $\Xlo\doit\mP\phi$ est vraie dans \tuple{w,i}, c'est-à-dire que $\Xlo[\mP\phi\implq\doit\mP\phi]$ est vraie dans \tuple{w,i}.

\end{solu}
\end{exo}



\is{futur|)}




\section{Intension et extension} %{$\Intn$ et $\Extn$}
%============================
\label{s:intension}
\is{intension|(}

L'ajout de {\Tps} et  \Unv W dans le modèle rend notre système intensionnel, et il gagne ainsi en précision ontologique. 
Cela augmente significativement l'expressivité de {\LO}, en permettant d'aborder le traitement de la temporalité et des modalités. 
Mais cela a aussi une conséquence épistémologique très précieuse pour notre théorie, en ce que cela va nous permettre de formaliser précisément la notion de \emph{sens}.
%(Déf.~\ref{def:sens}, p.~\pageref{def:sens})
Certes le sens est déjà présent dans notre système formel, ce sont les conditions de vérité (ou plus généralement les conditions de dénotation). 
Techniquement, ces conditions sont des règles de calcul, articulées avec un \sicut{si et seulement si} ; les calculs qu'elles permettent s'exécutent dans le système, mais elles-mêmes, telles qu'elles sont formulées, se situent un peu à l'extérieur du système, ou du moins à l'interface entre le système et ses utilisateurs (\ie\ nous-mêmes).
Formaliser le sens consiste à en faire un objet complètement défini \emph{à l'intérieur du système}. Il devient alors techniquement plus autonome et plus facilement manipulable, et l'on peut ainsi l'étudier, l'exploiter, le mettre à l'épreuve avec des méthodes scientifiques suffisamment robustes.  Et par la même occasion, nous verrons que cette formalisation va nous permettre d'améliorer encore un peu l'expressivité de {\LO}. 


%ne sont que des éléments de formulation qui interviennent dans l'articulation des règles d'interprétation, elles ne sont pas entièrement 

\subsection{Sens et mondes}
%--------------------------------------
\is{sens|(}

Commençons par récapituler un point important : pour un modèle donné,
la valeur sémantique, c'est-à-dire la \kwa{dénotation}{denotation}, 
d'une expression dépend maintenant d'un monde possible et d'un instant
donnés. Et nous écrivons 
\(\denote{\Xlo\phi}^{\Modele,w,i,g}=1\)
 pour
dire que $\Xlo\phi$ est vraie dans le monde $w$ à l'instant $i$ (et
relativement au modèle {\Modele} et à l'assignation $g$).  Ces
paramètres $w$ et $i$ jouent le rôle de points de références sur
lesquels s'appuie l'évaluation sémantique de l'expression.  Ils
constituent ce que l'on appelle des \kwi{indices}{indice}. 
Techniquement un indice est ainsi défini
comme un couple de coordonnées \tuple{w,i}, qui identifie un état possible et particulier du monde.  


Passons maintenant au \kw{sens} d'une expression.  Et souvenons-nous
de la définition fregéenne que nous avions adoptée
(\S\ref{def:sens}, p.~\pageref{def:sens}) : le sens d'une expression
est ce qui nous donne sa dénotation.  À l'aide
de l'appareil formel que nous avons introduit dans ce chapitre, nous
pouvons la reformuler de la manière suivante (pour toute expression \vrb\alpha).

\begin{point}\label{pt:sensw}
{Connaître le
sens de $\Xlo\alpha$ c'est être capable de retrouver, pour n'importe quel monde
$w$ et n'importe quel instant $i$, quelle est la dénotation de \vrb\alpha\ dans $w$ à $i$.}
\end{point}
%

En effet, si vous connaissez le sens de \sicut{moustachu}, alors quel que soit le monde (et l'instant) dans lequel vous vous situez, vous saurez dire si un individu appartient ou non à la dénotation de \sicut{moustachu}.
De même, pour une phrase ou une formule \vrb\phi, comprendre $\vrb\phi$, c'est être à même de dire : «donnez-moi
un monde $w$ (quelconque) et un instant $i$ (quelconque), et je vous dirai si $\vrb\phi$ est vraie ou
fausse dans \tuple{w,i}~».  Car rappelons-nous qu'un indice \tuple{w,i} donné fournit
une description \emph{complète} d'un état de choses (grâce à la fonction d'interprétation {\FI} du modèle).
%Donc il suffit bien de connaître l'état de choses étiqueté par $w$ et
%le sens de $\phi$ pour juger de sa vérité dans $w$.

Le sens est donc une sorte de \emph{mécanisme} qui, à partir de la donnée d'un état du monde \tuple{w,i}, nous retourne la dénotation (dans \tuple{w,i}).
Formellement, un tel mécanisme correspond tout simplement à une \kwo{fonction}.
Une fonction qui à tout indice \tuple{w,i} associe la dénotation de l'expression par rapport à cet indice.
Un telle fonction s'appelle l'\kw{intension} d'une expression. \emph{Intension} sera maintenant pour nous synonyme de \emph{sens}, de même que \emph{extension} sera synonyme de \emph{dénotation}\footnote{Les notions d'intension et d'extension sont relativement anciennes en philosophie. \emph{Intension} est originellement synonyme de \emph{compréhension}. Par exemple, l'intension d'un concept est l'ensemble des traits caractéristiques qui le définissent en propre et l'opposent aux autres concepts.  L'extension d'un concept est en quelque sorte son «étendue» dans le monde, c'est-à-dire la classe des objets qui l'instancient. De même, la définition par extension d'un ensemble est l'énumération complète de son contenu ; sa définition par intension est la spécification des conditions nécessaires et suffisantes que doit satisfaire un objet pour en être un élément. L'assimilation du sens à l'intension de la logique modale remonte essentiellement à \citet{Carnap:47}\Andexn{Carnap, R.}, et la formalisation (sous forme de fonction) que nous adoptons ici reprend notamment celle de \citet{Kripke:63af}\Andexn{Kripke, S.}.}. 

\begin{defi}[Intension]\is{intension}\label{def:intension}
Soit \(\Modele=\tuple{\Unv{A},\Unv{W},\Tps_{\tprec},\FI}\) un modèle intensionnel, $g$ une fonction d'assignation
et $\vrb\alpha$ une expression interprétable de {\LO}.  
\\
L'\kw{intension}
de $\vrb\alpha$ est la fonction $\tuple{w,i} \longmapsto
\denote{\vrb\alpha}^{\Modele,w,i,g}$, c'est-à-dire la fonction qui à tout
monde $w$ de \Unv{W} et tout instant $i$ de {\Tps} associe l'extension de $\vrb\alpha$ dans \tuple{w,i}. 
\end{defi}


L'intension de \vrb\alpha, en tant que fonction, a pour ensemble de départ l'ensemble de tous les couples \tuple{w,i}, qui se note $\Unv W\times\Tps$.  L'ensemble d'arrivée de la fonction, lui, dépend de la nature de \vrb\alpha.  
Si \vrb\alpha\ est un prédicat unaire, alors son intension associe à chaque \tuple{w,i} un ensemble d'individus, \ie\ un sous-ensemble de \Unv A ; l'ensemble d'arrivée de la fonction est donc l'ensemble de tous les sous-ensembles de \Unv A, qui se note $\powerset(\Unv A)$.
Par exemple, l'intension du prédicat \prd{dormir} est cette fonction qui nous révèle l'ensemble des dormeurs pour chaque état du monde :

\ex.  
%Intension de \prd{dormir} :\\
\(%
\begin{array}[t]{c@{\;}c@{\;}l}
\Unv{W}\times\Tps &\Vers& \powerset(\Unv{A})\\
\tuple{w,i} &\longmapsto& \text{l'ensemble des dormeurs de \Unv{A} dans \tuple{w,i},}\\
&& \text{\ie\ } \denote{\prd{dormir}}^{\Modele,w,i,g}
  \end{array}\)\label{intdormir}


Ce qui peut sembler curieux dans \Last, c'est que cette fonction ne fait pas apparaître la «définition» du verbe ou prédicat \sicut{dormir} (comme on parle d'une définition de dictionnaire qui nous donne, informellement, le sens d'un mot).  
Nous retrouvons là la discussion de la fin du chapitre~\ref{LCP} (\S\ref{conclu:LCP}).
Rappelons que nous n'ouvrons pas le capot de la sémantique lexicale, et de fait  nous n'explicitons pas les détails de l'analyse sémantique du prédicat dans la fonction \Last. 
Mais non explicite ne veut pas dire absent ; le sens du prédicat est compris dans \Last, c'est l'ensemble des conditions qui font que la fonction sait toujours retrouver la dénotation correcte. 
Ce qui importe dans l'approche que nous suivons ici, c'est que l'objet formel défini en {\Last} a les mêmes propriétés que ce que nous concevons informellement comme le sens de \sicut{dormir}. 
Nous attendons du sens qu'il détermine la dénotation à partir de toute donnée de l'état du monde ; c'est ce que fait la fonction intension, et cela nous suffit\footnote{Notons d'ailleurs que, formellement, nous manipulions déjà des intensions depuis un petit moment : par l'intermédiaire de la fonction d'interprétation {\FI} du modèle. C'est une sorte de «super-fonction intension» qui pour chaque constante non logique du langage représente la fonction qui donne son extension pour tout indice \tuple{w,i}. {\FI} interprète le vocabulaire de base de {\LO} (\ie\ livre son sens), mais les intensions que nous définissons ici valent pour toute expression du langage.}.
Si par la suite, nous trouvons que l'intension a des propriétés qui contreviennent à ce que devrait être le sens, alors il sera temps de revenir sur cette hypothèse (en la supprimant, en l'amendant ou en la relativisant), mais en attendant, notre identification du sens à l'intension convient à notre programme de formalisation sémantique.

L'intension d'un prédicat unaire,  donc une fonction de $\Unv W \times \Tps$ vers $\powerset(\Unv A)$, s'appelle une \kwo{propriété}.\is{propriete@propriété} Il s'agit d'un terme technique de la théorie, mais il converge bien avec l'usage courant et informel que nous faisons du mot \emph{propriété}\footnote{En tout rigueur, il sera donc fautif d'appeler \emph{propriété} la dénotation d'un prédicat, même si c'est ce que nous avons fait occasionnellement dans les chapitres précédents. Cela dit, il faut avouer qu'il est parfois pratique --~pour alléger les explications~-- de regrouper sous une même dénomination le symbole (prédicat), la dénotation (ensemble) et le sens (fonction), en les appelant indistinctement des propriétés, dans une acception informelle.}. 


L'intension d'un prédicat binaire associe à tout indice un ensemble de couples d'individus. L'ensemble de tous les couples composés d'éléments de \Unv A se note $\Unv A \times \Unv A$ (ou $\Unv A^2$) et l'ensemble de tous les ensembles possibles de couples se note $\powerset(\Unv A \times \Unv A)$ (ou $\powerset(\Unv A^2)$).
Par exemple, l'intension du prédicat \prd{regarder} est donc une fonction de $\Unv W\times \Tps$ vers $\powerset(\Unv A\times \Unv A)$ :

\ex.  
%Intension de \prd{dormir} :\\
\(%
\begin{array}[t]{c@{\;}c@{\;}l}
\Unv{W}\times\Tps &\Vers& \powerset(\Unv{A}\times\Unv A)\\
\tuple{w,i} &\longmapsto& \text{l'ensemble de tous les couples } \tuple{\Obj x,\Obj y} \text{ tels que \Obj x regarde \Obj y}\\
&& \text{dans \tuple{w,i}, \ie\ } \denote{\prd{regarder}}^{\Modele,w,i,g}
  \end{array}\)\label{intregarder}

Une telle fonction, \ie\ donc le sens d'un prédicat binaire, s'appelle une \kwo{relation intensionnelle}\is{relation!\elid\ intensionnelle}, ou parfois aussi une \kwo{relation en intension}\is{relation!\elid\ en intension}.
Il en va de même pour tout autre prédicat $n$-aire, dont l'intension est une fonction de $\Unv W\times\Tps$ vers $\powerset(\Unv A^n)$ (et que l'on nomme également une relation intensionnelle).

L'intension d'un terme est une fonction beaucoup plus simple, qui va de $\Unv W \times\Tps$ vers \Unv A. 
En ce qui concerne les constantes d'individus (et les noms propres), tant que nous nous plaçons sous l'hypothèse des désignateurs rigides, leur intension n'est pas très spectaculaire\footnote{Mais rappelons que nous pouvons aussi nous autoriser à exclure quelques constantes de cette hypothèse, comme \cns m de Miss France par exemple.}. En effet l'intension de \cns a est simplement :

\ex.  
%Intension de \prd{dormir} :\\
\(%
\begin{array}[t]{c@{\;}c@{\;}l}
\Unv{W}\times\Tps &\Vers& \Unv{A}\\
\tuple{w,i} &\longmapsto& \Obj{Alice} %\denote{\prd{dormir}}^{\Modele,w,i,g}
  \end{array}\)\label{intAlice}

Elle renvoie toujours la même valeur, il s'agit d'une \emph{fonction constante}.
Une telle fonction constitue un
objet intensionnel un peu appauvri, mais qui
est, somme toute, assez conforme à l'intuition que nous avons sur
la sémantique des noms propres,\is{nom!\elid\ propre} à savoir qu'ils n'ont pas vraiment de
sens (du moins qu'ils n'ont pas de sens au même titre que les
substantifs par exemple\footnote{Et cela transparaît dans le fait qu'en pratique, contrairement aux autres {\GN}, nous n'avons pas besoin de \emph{comprendre} un nom propre, ce qui compte c'est simplement de savoir ce qu'il dénote.}).
Et il faudra bien prendre garde à ne pas confondre (ni assimiler) l'\emph{individu} \Obj{Alice} (l'extension de \cns a) et la \emph{fonction} {\Last} (l'intension de \cns a), car ce sont des objets formels de natures différentes. 
L'intension d'un terme s'appelle un \kw{concept d'individu}. 
Elle prend toute sa dimension lorsque le terme en question est une description définie, comme \(\Xlo\atoi x\,\prd{président}(x,\cns u)\) (en posant que \cns u dénote les États-Unis) :

\ex.  
\(%
\begin{array}[t]{c@{\;}c@{\;}l}
\Unv{W}\times\Tps &\Vers& \Unv{A}\\
\tuple{w,i} &\longmapsto& \text{l'unique individu qui est président des USA dans \tuple{w,i},}\\
&& \text{\ie\ } \denote{\Xlo\atoi x\,\prd{président}(x,\cns u)}^{\Modele,w,i,g}
  \end{array}\)\label{intIota}
 
\Last\ est \emph{le} concept \sicut{le-président-des-USA}, et nous voyons qu'en tant que fonction, il peut «s'incarner» en différents individus selon les indices $w$ et $i$ donnés en arguments. 

\sloppy
L'intension d'une description définie (\ie\ d'un \atoi-terme) nous donne également l'occasion d'examiner un cas de figure particulier, celui des présuppositions\is{presupposition@présupposition}. En effet, d'après l'interprétation que nous avons donnée à $\Xlo\atoi$, si nous prenons un état du monde \tuple{w,i} dans lequel il n'y a pas de président des USA (pour une raison quelconque, par exemple le pays serait devenu une monarchie...), alors  \(\denote{\Xlo\atoi x\,\prd{président}(x,\cns u)}^{\Modele,w,i,g}\) n'est pas défini. Et cela se répercute directement sur l'intension de l'expression. Ce n'est pas un problème, au contraire ; cela nous permet de gérer (en partie !) les présuppositions dans notre système. 
L'intension est une fonction sur $\Unv W\times\Tps$, et les fonctions ne sont pas toujours définie sur \emph{tout} leur ensemble de départ%
\footnote{Comme certaines fonctions numériques qui ne sont pas définies pour tous les nombres. Par exemple la fonction $x\mapsto\frac{1}{x}$ n'est pas définies pour $0$, et, sur les nombres réels, $x\mapsto\sqrt{x}$ n'est pas définie pour les nombres négatifs.}.
Ainsi, lorsque l'on présente une fonction donnée, il est très utile de préciser ce que l'on appelle son \emph{domaine de définition}\is{domaine!\elid\ de definition@\elid\ de définition (d'une fonction)},\label{H:DomDefF} qui est le sous-ensemble de l'ensemble de départ pour les éléments duquel la fonction donne une valeur (nous y reviendrons au chapitre suivant, \S\ref{sss:DomDef}).
Pour les intensions, leur domaine de définition est déterminé via les présuppositions. Le principe est le suivant : si \vrb\alpha\ est une expression qui présuppose \vrb\psi, alors l'intension de \vrb\alpha\ n'est définie que pour les indices \tuple{w,i} tels que \(\denote{\vrb\psi}^{\Modele,w,i,g}=1\).

\fussy

Enfin, l'intension d'une formule \vrb\phi\ est une fonction de $\Unv W\times \Tps$ vers \set{0;1}.  C'est la fonction qui pour chaque indice \tuple{w,i} renvoie la valeur «oui,
$\vrb\phi$ est vraie dans \tuple{w,i}~» ou {«non, $\vrb\phi$ est fausse dans \tuple{w,i}~»} : 

\ex.
\(\begin{array}[t]{l@{\;}c@{\;}l}
\Unv{W}\times\Tps &\Vers& \set{0;1}\\
\tuple{w,i} &\longmapsto& \left\{\begin{array}{@{\,}l}1 \text{ si $\vrb\phi$ est vraie dans \tuple{w,i}}\\
0 \text{ sinon}\end{array}\right.
  \end{array}\)

Nous retrouvons bien là le principe de la sémantique vériconditionnelle qui dit que connaître le sens d'une phrase, c'est être capable de juger de sa vérité en fonction des informations sur le monde.
L'intension d'une formule s'appelle une \kw{proposition}\footnote{Attention, il ne faudra pas confondre une proposition qui est le sens d'une phrase et ce qu'on appelle une proposition en syntaxe\is{proposition!\elid\ syntaxique} (cf.\ propositions subordonnées, principales, indépendantes...). Il se trouve qu'elles portent le même nom en français ; l'anglais les distingue par \alien{proposition} et \alien{clause}, respectivement.}.
Le tableau~\ref{T:Int} récapitule les différents types d'intensions pour chaque catégorie d'expressions de {\LO}.

%%

\begin{table}[h!]
\begin{bigcenter}
\caption{Extensions et intensions des différentes catégories de \LO}\label{T:Int}
\small%
%\rowcolors{3}{gray!25}{gray!10}
\begin{tabular}{ccc}\lsptoprule
\bfseries Catégories & \bfseries Extension & \bfseries Intension \\
\bfseries d'expressions & \bfseries (dénotation) & \bfseries (sens)\\\midrule
formule $\phi$ & une valeur de vérité & une fonction $\Unv{W}\times\Tps\longrightarrow\set{0;1}$\\
(phrase) & $0$ ou $1$ & %%ou un sous-ensemble de \Unv{W}\\
%%\rowcolor{gray!25}
%%&& 
= une \kw{proposition}\\[.8ex]
%\hline
terme, constante & un individu de \Unv{A} & une fonction
$\Unv{W}\times\Tps\longrightarrow\Unv{A}$ \\
(N propre, GN défini) & & = un \kw{concept d'individu} \\[.8ex]
%\hline
prédicat unaire & un ensemble d'individus & une fonction
$\Unv{W}\times\Tps\longrightarrow\powerset(\Unv{A})$\\
(N, Adj, GV) & ($\in \powerset(\Unv{A})$) & = une \kwo{propriété}\is{propriete@propriété}\\[.8ex]
%\hline
prédicat binaire & un ensemble de couples & une fonction
$\Unv{W}\times\Tps\longrightarrow\powerset(\Unv{A}\times \Unv{A})$\\
(V transitif,...) & d'individus ($\in \powerset(\Unv{A}\times\Unv{A})$)& = une \kwo{relation intensionnelle}\\
\lspbottomrule
\end{tabular}
\end{bigcenter}
\end{table}


Nous avons examiné l'intension des termes, mais qu'en est-il des variables ?
Nous pouvons leur assigner aussi une intension, en appliquant la définition~\ref{def:intension}.
Par exemple, pour la variable \vrb x, son intension est la fonction qui à tout indice \tuple{w,i} associe l'extension de $\vrb x$
dans \tuple{w,i} relativement à {\Modele} et $g$, c'est-à-dire
\(\denote{\vrb x}^{\Modele,w,i,g}\).  Et par définition, cette extension vaut
$g(\vrb x)$.  L'intension de \vrb x est donc :

\ex.  
\(%
\begin{array}[t]{c@{\;}c@{\;}l}
\Unv{W}\times\Tps &\Vers& \Unv{A}\\
\tuple{w,i} &\longmapsto& g(\vrb x) \text{, \ie\ } \denote{\vrb x}^{\Modele,w,i,g}
  \end{array}\)\label{intvar}

Il s'agit là encore d'une fonction constante puisque le résultat ne
dépend pas de l'argument \tuple{w,i}.  Cela ne devrait pas être déroutant car
les variables non plus n'ont pas vraiment de sens : elles se contentent
de pointer plus ou moins arbitrairement sur un objet du domaine et ce
pointage n'est pas déterminé par l'état possible du monde auquel le locuteur se réfère.  En revanche, l'intension de \vrb x dépend de l'assignation $g$, qui fait partie du contexte.\is{contexte}

\newpage

Cela nous amène à faire quelques observations importantes. D'une part :

\begin{point}\label{pt:sensindepi}
{L'intension,
c'est-à-dire le sens, d'une expression ne dépend pas de l'état du monde
dans lequel on se place pour évaluer cette expression.}
\end{point}

Car le sens
est bien ce qui est constant d'un état du monde à l'autre : par exemple le
sens du mot \sicut{dormir} (ou du prédicat \prd{dormir} dans {\LO}) ne
change pas selon qui dort ou ne dort pas dans telle ou telle situation.
%\sicut{Dormir} c'est \sicut{dormir} quel que soit le monde que l'on
%envisage.  
Et c'est bien ce qui est reflété par la formalisation des
intensions en tant que fonction sur les indices : puisque leur travail est de fournir une valeur pour \emph{tout} indice, elles font une généralisation, c'est-à-dire une 
\emph{abstraction} sur les \tuple{w,i}, elles ne dépendent pas des indices%
\footnote{Cette indépendance de la fonction vis-à-vis de ses arguments peut dérouter de prime abord lorsque l'on n'est pas très familier de la notion de fonction. Car, certes, la valeur précise que retourne la fonction-intension \emph{dépend} bien, à chaque fois, de l'indice particulier qu'on lui donne. C'est le propre d'une fonction : son travail est de mettre en rapport un argument (ici un indice) avec une valeur (\ie\ un résultat). Mais le mécanisme qui établit ce rapport vaut pour n'importe quel argument. C'est pourquoi une fonction, en soi et globalement, ne dépend pas de ses arguments, car elle fait le même «calcul» pour tous les arguments.}. 

D'autre part, d'après la
définition~\ref{def:intension} et ce que nous venons de voir au sujet des variables :

\begin{point}\label{pt:sensdepcontxt}
{L'intension d'une expression
\emph{dépend} d'un modèle  et d'une fonction d'assignation.}
\end{point}


Comme nous avons mis dans le modèle  tout ce qui est possible et imaginable, nous pouvons difficilement le faire varier\footnote{En fait, le modèle comprend la définition du sens des «mots» par l'intermédiaire de \FI. Ainsi dans notre approche, changer de modèle revient essentiellement à changer de langue.}. 
Au contraire, les assignations peuvent varier sans problème (elles sont un peu conçues pour cela), et par conséquent une expression donnée pourra ne pas avoir le même sens selon l'assignation choisie pour l'interpréter.
Supposons une fonction $g_1$ qui assigne à la variable $\vrb x$ la valeur
\Obj{Alice}. 
% L'intension d'une
%formule nous permet de reconnaître les mondes dans lesquels elle est vrai
Avec cette assignation, l'intension de la formule \ref{x:Int+Pronomb}, traduction de la
phrase \ref{x:Int+Pronoma}, est la fonction qui retourne $1$ pour tous les
états du monde dans lesquels \Obj{Alice} fait partie des dormeurs. Avec $g_1$, 
\ref{x:Int+Pronomb} est sémantiquement équivalente à \(\Xlo\prd{dormir}(\cns a)\).

\ex. \label{x:Int+Pronom}
\a. Elle dort.\label{x:Int+Pronoma}
\b. \(\Xlo\prd{dormir}(x)\) \label{x:Int+Pronomb}

Et si nous interprétons \ref{x:Int+Pronomb} avec $g_2$ telle que
$g_2(\vrb x)=\Obj{Dina}$, alors  la formule est équivalente à \(\Xlo\prd{dormir}(\cns d)\). 
%%est une autre fonction, celle qui retourne $1$ pour tous les mondes
%%$w$ dans lesquels c'est \Obj{Dina} qui fait partie des dormeurs.


Nous avons pris l'habitude de considérer que les assignations sont un composant du contexte, et de ce fait, le point~\ref{pt:sensdepcontxt} ne fait rien d'autre que nous rappeler ce qui est bien connu en sémantique, à savoir que le sens dépend du contexte.

Les points précédents et la définition~\ref{def:intension} nous permettent également d'inférer que
l'intension d'une expression est sa \emph{valeur sémantique indépendante} de $w$ et de $i$ (mais dépendante de \Modele\ de $g$).  Comme nous utilisons \denote{\cdot} pour représenter les valeurs sémantiques, nous avons une notation évidente pour l'intension d'une expression \vrb\alpha, c'est \(\denote{\vrb\alpha}^{\Modele,g}\).

\begin{nota}
Soit \(\Modele = \tuple{\Unv{A},\Unv{W}, \Tps_{\tprec},\FI}\) 
un modèle intensionnel, $g$ une fonction d'assignation
et $\vrb\alpha$ une expression interprétable de {\LO}.  L'intension de
$\vrb\alpha$ relativement à {\Modele} et $g$ se note
\(\denote{\vrb\alpha}^{\Modele,g}\).
\end{nota}



%\largerpage[2]

Cela ressemble à notre manière de noter la dénotation (\ie\ l'extension) de
$\vrb\alpha$ dans les chapitres~\ref{LCP} et \ref{ch:gn}%
\footnote{À ce propos, certains auteurs, comme \citet{DWP:81}, utilisent des notations différentes pour bien distinguer les deux ; ils écrivent \(\denote{\vrb\alpha}^{\Modele,g}_{\text{\textcent}}\) pour l'intension de \vrb\alpha.}.  
Mais nous ne ferons pas la confusion, car maintenant notre système est (et restera) intensionnel : les indices $w$ et $i$ sont des paramètres indispensables pour définir, et donc noter, l'extension.
Par conséquent,
\(\denote{\vrb\alpha}^{\Modele,g}\) peut apparaître comme une écriture
{«incomplète»} ; et c'est voulu.  
L'intension de \vrb\alpha\ est une fonction et \(\denote{\vrb\alpha}^{\Modele,g}\) est le \emph{nom} de cette fonction ; à ce titre, c'est aussi ce que l'on appelle un \emph{foncteur}\is{foncteur}, c'est-à-dire le symbole qui représente la fonction dans les notations formelles (comme {\FI} et $g$ sont des foncteurs dans les notations mathématiques $\FI(w,i,\cns a)$ et $g(\vrb x)$).
Nous pouvons donc donner à cette fonction (ou ce foncteur) \(\denote{\vrb\alpha}^{\Modele,g}\) des arguments attendus, c'est-à-dire des indices \tuple{w,i}, en les accolant entre parenthèses comme nous faisons habituellement\footnote{Notons à cet égard que généralement en mathématiques, les notations en chevrons \tuple{\,} et en parenthèses $(\,)$ ne sont que des variantes graphiques équivalentes pour représenter les listes et les $n$-uplets. Donc \tuple{w,i} est la même chose que $(w,i)$.}. Ainsi nous pouvons écrire  \(\denote{\vrb\alpha}^{\Modele,g}(w,i)\) ; c'est la valeur que retourne la fonction \(\denote{\vrb\alpha}^{\Modele,g}\) pour l'indice \tuple{w,i}, ce qui, par définition, est la dénotation de \vrb\alpha\ dans \tuple{w,i}, autrement dit \(\denote{\vrb\alpha}^{\Modele,w,i,g}\).
\is{sens|)}%
\is{intension|)}

\nopagebreak[2]

\begin{point}
Quels que soient \vrb\alpha, \Modele, $g$, $w$ et $i$, \(\denote{\vrb\alpha}^{\Modele,g}(w,i)=\denote{\vrb\alpha}^{\Modele,w,i,g}\).\\
Ces deux notations sont, par définition, équivalentes.
\end{point}


\newpage

\subsection{Simplifications et interlude graphique}
%--------------------------------------------------
\label{ss:s&ig}

Comme promis en \S\ref{ss:PbTmps}, nous allons à présent opérer une simplification du système, en abandonnant \emph{provisoirement} {$\Tps_{\tprec}$}.
Cette simplification n'a pour but que d'alléger nos notations et nos descriptions des intensions dans les formulations que nous serons amenés à manipuler lorsque le système \LO\ se perfectionnera encore dans les chapitres suivants.
Elle va un peu affaiblir notre langage {\LO} (\mP\ et \mF\ disparaissent) et diminuer la précision du modèle (les futurs branchants de \S\ref{s:branchants} ne sont plus implémentés), mais, que l'on se rassure, tout cela reviendra, sous une forme légèrement différente, dans le chapitre~\ref{Ch:temps2} (vol.~2).
Le gain en simplification est que seuls les mondes possibles vont jouer le rôle des indices intensionnels. Ainsi le domaine des intensions devient simplement \Unv W, et les dénotations sont plus sobrement notées \(\denote{\Xlo\alpha}^{\Modele,w,g}\).  Une conséquence est que, pour nous maintenant, les termes «états du monde»  et «mondes possibles» seront utilisés comme des synonymes. 
Et partout où, précédemment, nous écrivions \tuple{w,i} %(ou $(w,i)$) 
nous écrirons dorénavant simplement $w$.


Ensuite, nous avons vu précédemment que le sens d'une phrase, \ie\ une \emph{proposition},\is{proposition} est une fonction qui pour tout indice ($w$) nous renvoie la valeur vrai ($1$) ou faux ($0$). % -- précisément parce qu'elle sait distinguer les états du monde où la phrase est vraie de ceux où elle est fausse.
C'est donc une fonction qui fait du \emph{tri} parmi les indices, et ainsi elle «divise» l'ensemble de tous les indices ($\Unv W$) en deux sous-ensembles : 
d'une part, l'ensemble de tous les états du mondes où la phrase est vraie, et d'autre part celui qui contient tous les états où elle est fausse. 
%Par conséquent, à l'aide de cette fonction, nous pouvons très bien ranger dans un ensemble particulier tous les états du monde pour lesquels la phrase est vraie. La connaissance de cet ensemble est 
Or la connaissance du premier de ces sous-ensembles et la connaissance de la fonction intension sont complètement équivalentes, puisque les deux nous disent exactement dans quels états du monde la phrase est vraie.
Par conséquent, si nous le voulons, nous pouvons --~sans risque~-- adopter une vision alternative de ce qu'est l'intension d'une phrase : c'est l'ensemble de tous les indices où cette phrase est vraie.

\begin{point}\label{pt:prop°}
L'intension d'une formule \vrb\phi\ peut être assimilée à l'ensemble de tous les indices pour lesquels \vrb\phi\ est vraie.\is{intension}
%% \\
%% Formellement : \(\Ch{\denote{\vrb\phi}^{\Modele,g}}=\set{\tuple{w,i}\in \Unv W\times\Tps \tq \denote{\vrb\phi}^{\Modele,w,i,g}=1}\).
\end{point}

\largerpage

Cette assimilation d'une fonction qui se projette sur \set{0;1} avec un ensemble est très importante pour notre formalisme --~à tel point que je prendrai le temps d'y revenir plus longuement dans le prochain chapitre. 
En effet, ce sera pour nous particulièrement pratique, de temps en temps, de raisonner sur les propositions en les considérant comme des ensembles d'indices, parce que les ensembles sont souvent des objets plus simples à manipuler que des fonctions.
Mais il faudra bien garder à l'esprit qu'il ne s'agit que d'une assimilation ; techniquement un ensemble n'est pas la même chose qu'une fonction, et en toute rigueur, si nous voulons représenter les propositions en tant qu'ensembles de mondes, nous devons adopter une notation spécifique, \(\Ch{\denote{\vrb\phi}^{\Modele,g}}\) par exemple\footnote{Ce n'est pas une notation consacrée dans la littérature ; je l'introduis ici pour bien marquer la distinction de nature entre \({\denote{\vrb\phi}^{\Modele,g}}\) et \(\Ch{\denote{\vrb\phi}^{\Modele,g}}\), même si nous nous autoriserons à nommer les deux des \emph{propositions}. }.  Ainsi nous pouvons définir : \(\Ch{\denote{\vrb\phi}^{\Modele,g}}=\set{w \in \Unv W \tq \denote{\vrb\phi}^{\Modele,w,g}=1}\). 
Et de cette manière, nous disposons maintenant de trois variantes pour notifier qu'une formule \vrb\phi\ est vraie dans un monde $w$ : \(\denote{\vrb\phi}^{\Modele,w,g}=1\), \(\denote{\vrb\phi}^{\Modele,g}(w)=1\) et \(w\in\Ch{\denote{\vrb\phi}^{\Modele,g}}\).

La conception ensembliste des propositions nous permet d'en donner une visualisation graphique très simple et parfois très pratique. 
En effet, nous pouvons dessiner des propositions sur un plan où chaque point représente 
un monde possible  (ainsi ce plan figure tout \Unv{W}).
Une proposition, en tant qu'ensemble de mondes,  se dessine comme un
ensemble de points, \ie\ une surface du plan. C'est ce qu'illustre la figure~\ref{F:prop1} pour une formule \vrb\phi\ quelconque. 


%\newgray{Wgray}{.9}
\definecolor{Wgray}{gray}{.89}

\begin{figure}[h!]
\begin{center}
\begin{tabular}{ccc}
\pspicture*(0,0)(4,3)
{\rput(1.9,1.5){\scalebox{4}{\color{Wgray}$\Unv{W}$}}}%
{\psellipse[fillstyle=vlines,hatchcolor=lightgray](3.5,0.8)(1.6,2.5)}%
%\psellipse(1.8,1.5)(1.1,.4)% s
{\rput[bl](3.2,1.5){$\Xlo\phi$}}%
%\rput(1.5,1.5){$s$}%
%{\psellipse[fillstyle=vlines,hatchcolor=lightgray](1.5,0)(3,1.6)}%
%{\rput[bl](1.3,.2){$\psi$}}%
\psframe(4,3)
\endpspicture
&&
\pspicture*(0,0)(4,3)
%{\rput(1.9,1.5){\scalebox{4}{\color{Wgray}\Unv{W}}}}%
{\psframe[fillstyle=vlines,hatchcolor=lightgray,linestyle=none](0,0)(4,3)}%
{\psellipse[fillstyle=solid,fillcolor=white](3.5,0.8)(1.6,2.5)}%
%\psellipse(1.8,1.5)(1.1,.4)% s
{\rput[bl](3.2,1.5){$\Xlo\phi$}}%
{\rput[bl](.8,1.4){$\Xlo\neg\phi$}}%
%\rput(1.5,1.5){$s$}%
%{\psellipse[fillstyle=vlines,hatchcolor=lightgray](1.5,0)(3,1.6)}%
%{\rput[bl](1.3,.2){$\psi$}}%
\psframe(4,3)
\endpspicture
\\
\(\Ch{\denote{\Xlo\phi}^{\Modele,g}}\) : \psframe[fillstyle=vlines,hatchcolor=lightgray,linewidth=.6pt](0,-.1)(.6,.3)
&&
\(\Ch{\denote{\Xlo\neg\phi}^{\Modele,g}}\) : \psframe[fillstyle=vlines,hatchcolor=lightgray,linewidth=.6pt](0,-.1)(.6,.3)
\end{tabular}
\caption{Représentation graphique de $\Ch{\denote{\Xlo\phi}^{\Modele,g}}$ et $\Ch{\denote{\Xlo\neg\phi}^{\Modele,g}}$}\label{F:prop1}
\end{center}
\end{figure}


Ces représentations graphiques nous montrent qu'il est aisé de définir compositionnellement la version intensionnelle des connecteurs vérifonctionnels. 
Ainsi  l'intension de la négation d'une formule, c'est bien évidemment le \emph{complémentaire} par rapport à \Unv W de l'intension de cette formule. 
De même, comme le montrent les figures~\ref{F:prop2},  intensionnellement la conjonction correspond à \emph{l'intersection} et la disjonction à \emph{l'union}  de propositions.
Autrement dit :
\(\Ch{\denote{\Xlo\phi \wedge \psi}^{\Modele,g}}=\Ch{\denote{\Xlo\phi}^{\Modele,g}} \cap
\Ch{\denote{\Xlo\psi}^{\Modele,g}}\) 
et
\(\Ch{\denote{\Xlo\phi \vee \psi}^{\Modele,g}}=\Ch{\denote{\Xlo\phi}^{\Modele,g}} \cup
\Ch{\denote{\Xlo\psi}^{\Modele,g}}\).


\begin{figure}[h!]
\begin{center}
\begin{tabular}{ccc}
\pspicture*(0,0)(4,3)
{\psellipse[fillstyle=vlines,hatchcolor=lightgray](3.5,0.8)(1.6,2.5)}%
%\psellipse(1.8,1.5)(1.1,.4)% s
{\rput[bl](3.2,1.5){$\Xlo\phi$}}%
%\rput(1.5,1.5){$s$}%
{\psellipse[fillstyle=hlines,hatchcolor=lightgray](1.5,0)(3,1.6)}%
{\rput[bl](1.2,.5){$\Xlo\psi$}}%
{\rput[bl](2.3,.4){$\Xlo\phi \wedge \psi$}}%
\psframe(4,3)
\endpspicture
&
&
\pspicture*(0,0)(4,3)
{\psellipse[fillstyle=crosshatch,hatchcolor=lightgray](3.5,0.8)(1.6,2.5)}%
%\psellipse(1.8,1.5)(1.1,.4)% s
{\rput[bl](3.2,1.5){$\Xlo\phi$}}%
%\rput(1.5,1.5){$s$}%
{\psellipse[fillstyle=crosshatch,hatchcolor=lightgray](1.5,0)(3,1.6)}%
{\rput[bl](1.2,.5){$\Xlo\psi$}}%
{\rput[bl](2.3,.4){$\Xlo\phi \vee \psi$}}%
\psframe(4,3)
\endpspicture
\\
\(\Ch{\denote{\Xlo\phi \wedge \psi}^{\Modele,g}}\) : \psframe[fillstyle=crosshatch,hatchcolor=lightgray,linewidth=.6pt](0,-.1)(.6,.3)
&&
\(\Ch{\denote{\Xlo\phi \vee \psi}^{\Modele,g}}\) : \psframe[fillstyle=crosshatch,hatchcolor=lightgray,linewidth=.6pt](0,-.1)(.6,.3)
\end{tabular}
\caption{Représentation graphique de $\Ch{\denote{\Xlo\phi \wedge\psi}^{\Modele,g}}$ et $\Ch{\denote{\Xlo\phi\vee\psi}^{\Modele,g}}$}\label{F:prop2}
\end{center}
\end{figure}



La représentation graphique de \(\Ch{\denote{\Xlo\phi \implq \psi}^{\Modele,g}}\)
est moins intuitive (il faut se souvenir que $\Xlo\phi\implq\psi$ équivaut
à $\Xlo\neg\phi \vee\psi$), et il n'est pas souvent utile de la
dessiner. En revanche, la représentation graphique de la conséquence
logique (cf.~\ref{ss:vclw}), elle, est très simple et très utile :
\(\vrb\phi \satisf \vrb\psi\) ssi \(\Ch{\denote{\vrb\phi}^{\Modele,g}} \inclus
\Ch{\denote{\vrb\psi}^{\Modele,g}}\). 
Cela montre, si c'était encore nécessaire, que l'implication matérielle et la conséquence logique ne doivent pas être confondues. Graphiquement, avec l'implication on fabrique une nouvelle proposition à partir du dessin de deux propositions données, alors qu'avec la conséquence on dessine une relation, ou un rapport «géométrique», entre deux propositions.

\sloppy

Profitons-en également pour revenir un instant sur ce que nous avons vu précédemment sur les présuppositions.  Supposons qu'une formule \vrb\phi\ contienne un élément qui présuppose un contenu qui se représente par la formule \vrb\psi.  Nous ne connaissons qu'un symbole de \LO\ qui fait cela, c'est $\Xlo\atoi$ (mais rien n'empêche d'en définir d'autres si le besoin s'en fait sentir).  Par exemple, la formule \(\Xlo\prd{chauve}(\atoi x\,\prd{rdf}(x))\) présuppose qu'il existe un et un seul roi de France (\prd{rdf}), ce qui peut se représenter par la formule \(\Xlo\exists x [\prd{rdf}(x)\wedge \forall y [\prd{rdf}(y)\ssi y=x]]\). 
Si nous dessinons \(\Ch{\denote{\vrb\phi}^{\Modele,g}}\), \ie\ l'ensemble des mondes où \vrb\phi\ est vraie, nous respecterons forcément la condition \(\Ch{\denote{\vrb\phi}^{\Modele,g}} \inclus
\Ch{\denote{\vrb\psi}^{\Modele,g}}\) (l'ensemble des mondes où le roi de France est chauve est inclus dans l'ensemble des mondes où il existe un et un seul roi de France). Mais dans ce cas, \(\Ch{\denote{\Xlo\neg\phi}^{\Modele,g}}\) ne sera le complémentaire de \(\Ch{\denote{\vrb\phi}^{\Modele,g}}\) par rapport à tout \Unv W (comme dans la figure~\ref{F:prop1}), mais seulement par rapport à \(\Ch{\denote{\vrb\psi}^{\Modele,g}}\) ; autrement dit \(\Ch{\denote{\Xlo\neg\phi}^{\Modele,g}} = \Ch{\denote{\vrb\psi}^{\Modele,g}} - \Ch{\denote{\Xlo\phi}^{\Modele,g}}\) 
(l'ensemble des mondes de \(\Ch{\denote{\vrb\psi}^{\Modele,g}}\) privé de ceux de \(\Ch{\denote{\Xlo\phi}^{\Modele,g}}\)), et donc 
\(\Ch{\denote{\Xlo\neg\phi}^{\Modele,g}} \inclus
\Ch{\denote{\vrb\psi}^{\Modele,g}}\).

\fussy

\begin{figure}[h!]
\begin{center}
\begin{tabular}{ccc}
\pspicture*(0,0)(4,3)
{\psellipse[fillstyle=vlines,hatchcolor=lightgray](2.7,0)(1.6,1.7)}%
{\rput[bl](2.7,1){$\Xlo\phi$}}%
{\psellipse(1.5,0.4)(3.2,1.7)}%
{\rput[bl](.2,1.5){$\Xlo\psi$}}%
\psframe(4,3)
\endpspicture
&
&
\pspicture*(0,0)(4,3)
{\psellipse[fillstyle=hlines,hatchcolor=lightgray](1.5,0.4)(3.2,1.7)}%
{\rput[bl](.2,1.5){$\Xlo\psi$}}%
{\psellipse[fillstyle=solid](2.7,0)(1.6,1.7)}%
{\rput[bl](2.7,1){$\Xlo\phi$}}%
\psframe(4,3)
\endpspicture
\\
\(\Ch{\denote{\Xlo\phi}^{\Modele,g}}\) : \psframe[fillstyle=vlines,hatchcolor=lightgray,linewidth=.6pt](0,-.1)(.6,.3)
&&
\(\Ch{\denote{\Xlo\neg\phi}^{\Modele,g}}\) : \psframe[fillstyle=hlines,hatchcolor=lightgray,linewidth=.6pt](0,-.1)(.6,.3)
\end{tabular}
\caption{Représentation graphique \(\Xlo\phi\) et \(\Xlo\neg\phi\) présupposant \(\Xlo\psi\)}\label{F:prop3}
\end{center}
\end{figure}


La situation est légèrement différente (graphiquement) dans les cas où l'on analyse une phrase de la langue comme présupposant un certain contenu sans que cette présupposition soit directement prise en charge par un symbole de {\LO}.  
Mais le principe reste le même.  Au chapitre \ref{Ch:1}, nous avions vu qu'une phrase comme \ref{x:LaG}, par exemple, a une contribution sémantique qui peut se diviser en deux parts : un contenu présupposé \ref{x:LaGp} et un contenu «proféré» \ref{x:LaGa} qui correspond à ce que nous tenons habituellement pour les conditions de vérité, et donc le sens, de la phrase (puisque nous laissons généralement de côté les présuppositions dans nos traductions en {\LO}). 
%\fixme{***}

\ex. 
\a. Seul Lancelot est amoureux de  Guenièvre.\label{x:LaG}
\b. présupposé : \emph{Lancelot est amoureux de  Guenièvre.}\label{x:LaGp}
%  \a.[] 
(\(\Xlo \prd{amoureux}(\cns l,\cns g)\))
%  \z.
\b. proféré : %conditions de vérité : 
\emph{Tout individu amoureux de Guenièvre est Lancelot.}\label{x:LaGa}\\
(\(\Xlo\forall x [\prd{amoureux}(x,\cns g)\implq x=\cns l]\))


Résumons \ref{x:LaGp}  par la formule \vrb\psi\ et \ref{x:LaGa} par \vrb\phi.  Nous pouvons dessiner leurs intensions comme sur la figure \ref{F:prop4} --~\(\Ch{\denote{\vrb\psi}^{\Modele,g}}\) est l'ensemble qui recouvre la partie inférieure du plan et \(\Ch{\denote{\vrb\phi}^{\Modele,g}}\) l'ensemble de droite dont une partie de la frontière est tracée en pointillés.
Et là, même si \vrb\phi\ est ce que nous utiliserions pour traduire \ref{x:LaG}, la véritable intension de \ref{x:LaG}, en tenant compte de la présupposition, est en fait seulement la partie hachurée dans la figure. La partie supérieure de \(\Ch{\denote{\vrb\phi}^{\Modele,g}}\) (délimitée par des pointillés) contient les mondes où \emph{personne} n'aime Guenièvre, et ce sont bien des mondes où \vrb\phi\ \ref{x:LaGa} est (trivialement) vraie\footnote{Car rappelons que lorsque l'antécédent d'une implication est faux, l'implication est vraie. }. Donc ici encore l'intension réelle de la phrase sera un ensemble inclus dans l'intension de sa présupposition (\(\Ch{\denote{\text{\ref{x:LaG}}}^{\Modele,g}} \inclus
\Ch{\denote{\vrb\psi}^{\Modele,g}}\)) ; de même pour l'intension de la négation de \ref{x:LaG}) 
(\(\Ch{\denote{\neg\text{\ref{x:LaG}}}^{\Modele,g}} = 
\Ch{\denote{\vrb\psi}^{\Modele,g}} - \Ch{\denote{\vrb\phi}^{\Modele,g}}\)).

\begin{figure}[h!]
\begin{center}
\begin{tabular}{ccc}
\begin{pspicture*}(0,0)(4,3)
{\psellipse[linewidth=.6pt,linestyle=dashed](3.5,0.8)(1.6,2.5)}%
%\psellipse(1.8,1.5)(1.1,.4)% s
{\rput[bl](3.2,1.7){$\Xlo\phi$}}%
%\rput(1.5,1.5){$s$}%
%{\psellipse(2.2,0)(3,1.6)}%
{\rput[bl](1.2,.5){$\Xlo\psi$}}%
\begin{psclip}{\psellipse[linestyle=none](2.2,0)(3,1.6)}
  \psellipse[fillstyle=vlines,hatchcolor=lightgray](3.5,0.8)(1.6,2.5)
\end{psclip}%
\begin{psclip}{\psframe(4,3)}
\psellipse(2.2,0)(3,1.6)%
\end{psclip}
\end{pspicture*}
%\\
%\(\Ch{\denote{\Xlo\phi \wedge \psi}^{\Modele,g}}\) : \psframe[fillstyle=crosshatch,hatchcolor=lightgray,linewidth=.6pt](0,-.1)(.6,.3)
\end{tabular}
\caption{Intension d'une phrase exprimant $\vrb\phi$ en présupposant $\vrb\psi$}\label{F:prop4}
\end{center}
\end{figure}


\begin{exo}\label{exo:4winp}
Démontrez %précisément 
que les écritures
\pagesolution{crg:4winp}%
\(\denote{\vrb\phi}^{\Modele,w,g}=1\) et \(w \in
\Ch{\denote{\vrb\phi}^{\Modele,g}}\) disent la même chose.
\begin{solu} (p.~\pageref{exo:4winp})\label{crg:4winp}

La démonstration est immédiate :
en sémantique intensionnelle, comme le pose la notation \ref{nota:4.3} p.~\pageref{nota:4.3} (avec la simplification de \S\ref{ss:s&ig}, p.~\pageref{ss:s&ig}),
\(\denote{\Xlo\phi}^{\Modele,w,g}=1\) signifie que \vrb\phi\ est vraie dans $w$ ; 
d'après le point \ref{pt:prop°}, p.~\pageref{pt:prop°}, \(\Ch{\denote{\Xlo\phi}^{\Modele,g}}\) est l'ensemble de tous les mondes où \vrb\phi\ est vraie ; donc si 
\(w \in
\Ch{\denote{\Xlo\phi}^{\Modele,g}}\) c'est que $w$ est un monde où \vrb\phi\ est vraie.
\end{solu}
\end{exo}


%\newpage %****************************************

\subsection{Expression des intensions dans le langage}
%-----------------------------------------------------

Nous allons ici aborder un problème qui avait déjà été identifié, et
en grande partie expliqué, par \citet{Frege:SuB} : le problème de ce
qu'il appelait les \emph{dénotations indirectes}\is{denotation@dénotation!\elid\ indirecte}.  La notion d'intension
introduite précédemment va nous permettre de formaliser une solution à
ce problème, en reprenant notamment le système que propose \citet{PTQ}.

\subsubsection{Le problème des complétives}
%''''''''''''''''''''''''''''''''''''''''''
\label{sss:completives}

Pour nous sensibiliser au problème des dénotations indirectes,
commençons par nous interroger sur la manière dont nous pourrions
traduire, et donc analyser, compositionnellement des phrases qui
contiennent une subordonnée complétive comme \ref{x:complive} :

\ex. \label{x:complive}
Charles pense qu'Alice est en colère.


En admettant que \sicut{Alice est en colère} se traduise simplement
par \(\Xlo\prd{colère}(\cns a)\), et en tenant compte du fait que
\sicut{penser} est un verbe transitif, nous pouvons tout d'abord être
tenté de proposer la traduction suivante :

\ex. \label{x:complive'}
\(\Xlo\prd{penser}(\cns c,\prd{colère}(\cns a))\)


Mais nous savons que cela n'est qu'une tentation à laquelle nous ne devrions
pas oser céder, car la syntaxe de {\LO} ne nous autorise pas, pour
l'instant, à écrire \ref{x:complive'}.  En effet
\(\Xlo\prd{colère}(\cns a)\) est une formule et pour cette raison elle
ne peut pas apparaître en tant qu'argument d'un prédicat binaire comme
\prd{penser}.

Cependant, s'il le faut, nous pouvons toujours envisager d'amender la
syntaxe de {\LO} en ajoutant une règle qui autorise
\ref{x:complive'} et ainsi permette de traduire le sens de
\ref{x:complive}.  Une telle règle serait de la forme : si $\vrb\alpha$
est un terme, si $\vrb P$ est un \emph{certain} prédicat binaire (comme
\prd{penser}, \prd{croire}, \prd{dire}, \prd{vouloir}...) et si $\vrb\phi$
est une formule, alors \(\Xlo P(\alpha,\phi)\) est une formule bien formée
de {\LO}.   Cela nous permettrait d'écrire dans {\LO} ce que l'on appelle
des expressions du \emph{second ordre}, c'est-à-dire qui enchâssent des
formules comme arguments de prédicats.

Mais si nous nous accordons le droit d'écrire \ref{x:complive'},
nous devons aussitôt vérifier la justesse de ses conditions de
vérité.  {A priori}, nous pouvons interpréter \ref{x:complive'} par la
règle sémantique (\RSem\ref{RIprd2}b) (p.~\pageref{RIprd2}) ;  ses
conditions de vérité s'établiront alors comme suit :

\ex.
 \(\denote{\Xlo\prd{penser}(\cns c,\prd{colère}(\cns a))}^{\Modele,w,g}=1\)
\\ ssi 
\(\tuple{\denote{\cns c}^{\Modele,w,g},\denote{\Xlo\prd{colère}(\cns
  a)}^{\Modele,w,g}} \in \denote{\prd{penser}}^{\Modele,w,g}\).


Nous savons que \(\denote{\cns c}^{\Modele,w,g} = \Obj{Charles}\), mais
que vaut \(\denote{\Xlo\prd{colère}(\cns a)}^{\Modele,w,g}\) ? Cela dépend
de $w$, mais ce qui est sûr, c'est que c'est une valeur de vérité, $0$
ou $1$, puisqu'il s'agit d'une formule.  Par conséquent
\ref{x:complive'} est vraie dans $w$ ssi \tuple{\Obj{Charles},1} ou
\tuple{\Obj{Charles},0} (selon qu'Alice  est ou non en colère dans $w$)
appartient à la dénotation de \prd{penser} dans $w$.  Or cela
présuppose qu'un prédicat comme \prd{penser} dénoterait une relation entre
des individus (comme \Obj{Charles}) et des valeurs de vérité ($0$ ou
$1$).  Mais cela est évidemment absurde ! On ne pense pas des valeurs de
vérité ; les objets de pensée sont des objets bien plus sophistiqués,
ce sont... eh bien ce que l'on appelle justement des «pensées».
De manière générale, ces verbes qui enchâssent une complétive et qui sont pour la plupart des verbes d'attitude propositionnelle\is{attitude propositionnelle} (cf. p.~\pageref{VAttProp}) expriment une attitude cognitive d'un individu vis-à-vis du contenu d'une phrase (déclarative). 
Et le contenu d'une phrase, c'est son sens (ou son intension), c'est-à-dire une 
\emph{proposition}.

Si nous considérons ainsi que le second argument de
\prd{penser} est une proposition, c'est-à-dire que \prd{penser} dénote
une relation entre un individu
%%, en l'occurrence \Obj{Jean} pour \ref{x:complive'}, 
et  une fonction de \Unv W vers \set{0;1} (ou un ensemble de mondes possibles),
%% , en l'occurrence les mondes où \Obj{Marie} est en colère, 
nous pouvons  arriver à une
analyse sémantique correcte pour une phrase comme \ref{x:complive'}\footnote{Car les propositions, en tant que fonctions de \Unv W vers \set{0;1}, sont bien présentes et accessibles dans le modèle.}.
Il suffit tout simplement de considérer que dans tout monde $w$,
\prd{penser} met en relation chaque individu du modèle avec chaque
proposition que cet individu estime être vraie dans $w$.  
Concrètement, il faut que la dénotation de \prd{penser} soit un
ensemble de couples de la forme \tuple{\Obj{x},\Obj{p}}, où \Obj{p} est
une proposition.
De même, la dénotation de \prd{vouloir} dans $w$ sera l'ensemble de tous les couples \tuple{\Obj x,\Obj p} tels que \Obj x souhaite voir \Obj p devenir vraie dans $w$. C'est là le principe de base de l'interprétation de tout verbe d'attitude propositionnelle.
%un ensemble de mondes possibles. \fixme{***}

Pour autant, tout n'est pas encore réglé pour {\LO} car 
notre règle (\RSem\ref{RIprd2}b) ne peut pas interpréter correctement 
\ref{x:complive'}. 
En effet (\RSem\ref{RIprd2}b) calcule --~comme il se doit~-- le couple formé de \emph{l'extension} des arguments du prédicat d'attitude propositionnelle. Or nous venons de voir que pour déterminer l'extension de la phrase globale,  ce n'est pas l'extension de la complétive qu'il faut prendre en compte, mais son intension. En pratique, il y a plusieurs façons de régler cela. Une première solution «simple» consiste à dire que puisque nous avons eu recours à une règle syntaxique spéciale pour former \ref{x:complive'}, il est normal d'introduire aussi une règle sémantique \alien{ad hoc} pour interpréter ce type de formule. 
Cette règle est la suivante : %si \vrb P est un prédicat d'attitude propositionnelle, 
\(\denote{\Xlo P(\alpha,\phi)}^{\Modele,w,g}=1\) ssi 
\(\tuple{\denote{\vrb\alpha}^{\Modele,w,g},\denote{\vrb\phi}^{\Modele,g}}\in\denote{\vrb P}^{\Modele,w,g}\).
Ce qui est crucial ici, c'est que la règle nous fait calculer $\denote{\vrb\phi}^{\Modele,g}$ et non $\denote{\vrb\phi}^{\Modele,w,g}$.
Une autre façon de procéder consiste à enrichir {\LO}, pour le rendre plus encore expressif, et en contrepartie nous pourrons interpréter \ref{x:complive'} avec (\RSem\ref{RIprd2}b) sans ajouter de règle supplémentaire. C'est ce que nous allons voir à présent.


\subsubsection{Dénotation indirecte : {$\Intn$ et $\Extn$}}
%'''''''''''''''''''''''''''''''''''''''''''''''''''''''''
\label{sss:^v}

\citet{Frege:SuB}
identifiait le problème qui nous occupe
en disant
que
dans certains contextes, des expressions de la
langue ont une dénotation inhabituelle, indirecte.
Et il ajoutait que la dénotation indirecte d'une expression, c'est précisément \emph{son sens habituel} --~autrement dit, pour nous, son intension.\is{intension}
Et c'est bien ce que nous avons observé \alien{supra} : une subordonnée complétive est une phrase qui ne dénote pas ce que dénote habituellement une phrase (\ie\ une valeur de vérité) mais  une proposition, c'est-à-dire son propre sens.
Cela d'ailleurs ne concerne peut-être pas seulement les subordonnées, et de manière générale, nous pouvons en bref faire le constat que parfois certaines expressions dénotent leur sens.

\largerpage

Or notre système sémantique intensionnel nous permet de formaliser ce phénomène en le répercutant au niveau des traductions dans {\LO}.
Fondamentalement les expressions de {\LO} ont pour vocation de représenter le sens d'expressions de la langue ; mais comme le sens détermine la dénotation (pour tout indice), dans nos calculs, ces expressions nous servent aussi à représenter les dénotations. 
Supposons maintenant qu'une expression $E$ de la langue se traduise ordinairement par \vrb\alpha\ dans {\LO} ; nous saurons alors que la dénotation de $E$ (par rapport à $w$) sera la dénotation de \vrb\alpha\ (par rapport à $w$). Et si ensuite il se trouve que $E$ apparaisse dans un environnement où elle a sa dénotation indirecte, nous ne voudrons pas alors qu'elle se retrouve avec la même dénotation que \vrb\alpha. Il nous suffira donc d'assigner à $E$ une autre traduction. 
Et cette traduction devra être une expression de {\LO} qui \emph{dénote l'intension}\is{intension} de \vrb\alpha. 

La sémantique intensionnelle dispose d'un opérateur qui réalise exactement cela. Il se note $\Xlo\Intn$ et se place devant une expression de {\LO} pour former l'expression qui dénote le sens de la première\footnote{Dans ces pages, j'appellerai le symbole $\Xlo\Intn $ l'opérateur  d'\emph{intensionnalisation}. Il n'y a pas vraiment, dans la littérature, de terme consacré pour le nommer, même si on trouve parfois celui d'\emph{intenseur}.\is{intenseur}}. 
Nous l'introduisons dans {\LO} via les deux règles suivantes.

\label{p:^}
%%Si $\alpha$ est une expression de {\LO},
%%$\Intn\alpha$ est l'expression de {\LO} qui \kwo{dénote le sens} (l'intension) de $\alpha$.


\begin{defi}[Syntaxe et sémantique de {$\Intn$}]
\begin{enumerate}[resume*=RglSyn1] %[(\RSyn1)]{\setcounter{enumi}{\value{RglSynt}}}
  \item
Si $\Xlo\alpha$ est une expression bien formée de {\LO}, alors
$\Xlo\Intn\alpha$ est aussi une expression bien formée de {\LO}.
\label{Syn^}
\setcounter{RglSynt}{\value{enumi}}
\end{enumerate}
\begin{enumerate}[resume*=RglSem2] %[(\RSem1)]{\setcounter{enumi}{\value{RglSem}}}
\item \(\denote{\Xlo\Intn\alpha}^{\Modele,w,g}= \) la fonction \(w'\longmapsto
 \denote{\Xlo\alpha}^{\Modele,w',g} \).
\\
Autrement dit : \(\denote{\Xlo\Intn\alpha}^{\Modele,w,g}=\denote{\Xlo\alpha}^{\Modele,g}\).
\label{Sem^}
\setcounter{RglSem}{\value{enumi}}
\end{enumerate}
\end{defi}


À sa manière, $\Xlo\Intn$ est lui aussi un opérateur modal. 
Mais il est plus «puissant» que $\Xlo\peut$ et $\Xlo\doit$, car (\RSyn\ref{Syn^}) parle d'expressions bien formées, et pas seulement de formules : $\Xlo\Intn$ peut aussi se placer devant une constante, un prédicat, une variable... Par commodité, nous appellerons ici \emph{expression intensionnelle}\is{expression!\elid\ intensionnelle} toute expression de {\LO} de la forme $\Xlo\Intn\alpha$.

La règle (\RSem\ref{Syn^}) montre que, comme il se doit, la dénotation de $\Xlo\Intn\alpha$ ne dépend pas du monde dans lequel nous l'évaluons.
Autrement dit, quel que soit $w$, \(\denote{\Xlo\Intn\alpha}^{\Modele,w,g}\) aura toujours la même valeur, puisque $\Xlo\Intn\alpha$ dénote le \emph{sens} de $\Xlo\alpha$ et que, comme nous l'avons vu, le sens est constant d'un monde à l'autre.
Ainsi \(\denote{\Xlo\Intn\prd{colère}(\cns
  a)}^{\Modele,w,g}\) est une proposition, que nous pouvons ramener à l'ensemble de tous les mondes dans lesquels
Alice est en colère. 

Nous pouvons maintenant traduire dans {\LO} des phrases comme \ref{x:complive} en réaménageant la règle syntaxique \alien{ad hoc} de la manière suivante :
si $\vrb\alpha$
est un terme, si $\vrb P$ est un  prédicat d'attitude propositionnelle et si $\vrb\phi$
est une formule, alors \(\Xlo P(\alpha,\Intn\phi)\) est une formule bien formée
de {\LO}.
Notons qu'il faudra aussi veiller à ajouter une règle similaire
pour certains prédicats ternaires, etc.  Nous verrons dans le chapitre
suivant comment reformuler de telles règles de manière plus efficace, générique
et élégante.  
En attendant, nous pouvons déjà traduire \ref{x:complive} par :

\ex. \label{x:complive''}
\(\Xlo\prd{penser}(\cns c,\Intn\prd{colère}(\cns a))\)

Cette formule s'interprète directement au moyen de (\RSem\ref{RIprd2}b) qui va tester si la dénotation de \cns c (un individu) est bien en relation de «croyance» avec la dénotation de $\Xlo\Intn\prd{colère}(\cns a)$ (une proposition), conformément à ce que nous avons posé comme type de dénotation pour \prd{penser}.



L'opérateur {$\Xlo\Intn$} permet donc d'intégrer littéralement la proposition d'analyse de Frege formulée en termes de dénotations indirectes.  Même si l'on peut parfois s'en passer pour analyser certains phénomènes (cf.\ par exemple la piste suggérée en \S\ref{sss:completives}),  il revêt une importance notoire en sémantique intensionnelle, en particulier depuis les travaux de \citet{PTQ} qui a généralisé fondamentalement l'usage de la notion de dénotation indirecte et d'expressions intensionnelles (nous y reviendrons dans le chapitre~\ref{ch:ISS}).

Et si, dans une telle perspective, le recours à $\Xlo\Intn$ est amené à se multiplier, il s'avère assez naturellement utile de munir notre système de l'opérateur inverse, c'est-à-dire l'opérateur qui restitue l'extension, dans le monde
d'évaluation courant, de toute expression intensionnelle. C'est donc en quelque sorte l'opérateur {«d'extensionnalisation»} ; il se note {$\Xlo\Extn$} et nous l'introduisons dans {\LO} par les deux règles suivantes.


\begin{defi}[Syntaxe et sémantique de {$\Extn$}]
\begin{enumerate}[resume*=RglSyn1] %[~(\RSyn1)]{\setcounter{enumi}{\value{RglSynt}}}
  \item
Si $\Xlo\alpha$ est une expression intensionnelle de {\LO} (\ie\ de la forme $\Xlo\Intn\beta$),
alors $\Xlo\Extn\alpha$ est aussi une expression de \LO. \label{Synv}
\setcounter{RglSynt}{\value{enumi}}
\end{enumerate}
\begin{enumerate}[resume*=RglSem2] %[~(\RSem1)]{\setcounter{enumi}{\value{RglSem}}}
\item \(\denote{\Xlo\Extn\alpha}^{\Modele,w,g} = \denote{\Xlo\alpha}^{\Modele,w,g}(w) \) \label{Semv}
\setcounter{RglSem}{\value{enumi}}
\end{enumerate}
\end{defi}

La règle (\RSyn\ref{Synv}) dit que {$\Xlo\Extn$} est supposé ne se placer que devant $\Xlo\Intn$ (mais il acquerra plus de liberté formelle dans le prochain chapitre), ce qui est normal étant donnée son interprétation. Celle-ci, (\RSem\ref{Semv}), nous fait simplement calculer la valeur que \vrb\alpha\ fournit pour le monde $w$. Car, par hypothèse, $\vrb\alpha$ est une expression intensionnelle, donc $\denote{\Xlo\alpha}^{\Modele,w,g}$ est une fonction sur \Unv W à laquelle on peut donner un indice $w$ en argument. Formellement, il serait inexact de dire que $\denote{\Xlo\Extn\alpha}^{\Modele,w,g}$ est l'extension de \vrb\alpha\ dans $w$ (car l'extension de \vrb\alpha\ est... une intension) ; en fait \vrb\alpha\ étant de la forme $\Xlo\Intn\beta$, $\denote{\Xlo\Extn\alpha}^{\Modele,w,g}$ est précisément l'extension de $\vrb\beta$ dans $w$, \ie\ $\denote{\Xlo\beta}^{\Modele,w,g}$.  Autrement dit, par définition, $\Xlo\Extn$ annule l'effet de $\Xlo\Intn$.


\begin{theo}\label{theo:v^}
Pour toute expression \vrb\alpha\ de {\LO} et pour tout modèle $\Modele$, tout monde $w$ et toute assignation $g$ : \(\denote{\Xlo\Extn\Intn\alpha}^{\Modele,w,g}=\denote{\Xlo\alpha}^{\Modele,w,g}\).
\end{theo}

\sloppy
La démonstration est immédiate. Par (\RSem\ref{Semv}), \(\denote{\Xlo\Extn\Intn\alpha}^{\Modele,w,g}=\denote{\Xlo\Intn\alpha}^{\Modele,w,g}(w)\), 
et par (\RSem\ref{Sem^}), \(\denote{\Xlo\Intn\alpha}^{\Modele,w,g}=\denote{\Xlo\alpha}^{\Modele,g}\) ; donc \(\denote{\Xlo\Extn\Intn\alpha}^{\Modele,w,g}=\denote{\Xlo\alpha}^{\Modele,g}(w)\), ce qui par définition vaut \(\denote{\Xlo\alpha}^{\Modele,w,g}\).

\fussy

Certes, au stade où nous en sommes, l'utilité pratique de $\Xlo\Extn$ peut sembler un peu spécieuse : si $\Xlo\Extn$ ne sert qu'à annuler $\Xlo\Intn$, autant ne pas mettre $\Xlo\Intn$ dès le départ... Mais nous verrons par la suite que le rôle de $\Xlo\Extn$ ne se réduit pas à ce que présente le théorème ci-dessus et qu'il peut avoir d'autres usages plus pertinents. En particulier, nous verrons que la combinaison inverse ne s'annule pas toujours : $\Xlo\Intn\Extn\alpha$ n'est pas nécessairement équivalent à~$\Xlo\alpha$. 




\subsubsection{Retour aux lectures \alien{de re} et \alien{de dicto}}
%'''''''''''''''''''''''''''''''''''''''''''''''''''''''''''''''''''''
\label{sss:derededicto}


Il est temps pour nous d'entreprendre de boucler la boucle en revenant à l'ambiguïté \alien{de re} \vs\ \alien{de dicto} qui nous occupait en début de chapitre. 
\largerpage

Nous avons maintenant à notre disposition des éléments d'analyse sémantique qui nous permettent \emph{presque} de résoudre le problème. «Presque», car nous allons voir que tout n'est pas encore gagné, en particulier en ce qui concerne les {\GN} définis ; il nous faudra encore avancer dans le perfectionnement de {\LO} (chapitre~\ref{ch:types}) pour parvenir à un traitement suffisamment satisfaisant du problème.  Mais nous pouvons déjà avoir un bon aperçu de l'analyse formelle de l'ambiguïté lorsqu'elle concerne les {\GN} indéfinis (que nous avions esquissée en \S\ref{ss:re/dicto}).
Revenons pour cela à l'exemple \ref{x:acvm}.

\ex. 
Alice croit qu'un vampire l'a mordue (pendant la nuit).
\label{x:acvm}


Cette phrase comporte une subordonnée complétive (qui est d'ailleurs ce qui correspond au contexte opaque responsable de l'ambiguïté), que nous allons traduire en utilisant {$\Xlo\Intn$}. 
Commençons par la lecture \alien{de dicto} de \sicut{un vampire}\is{de dicto@\alien{de dicto}}. Rappelons que dans ce cas, \ref{x:acvm} s'interprète comme : Alice croit que \emph{ce qu'elle pense être un vampire} l'a mordue. Autrement dit, l'existence du vampire qui l'a mordue est «localisée» dans les croyances d'Alice (et il se peut tout à fait que dans le monde d'évaluation les vampires n'existent pas). La traduction de \ref{x:acvm} est alors \ref{x:acvmd} :

\ex. 
\(\Xlo\prd{croire}(\cns{a}, \Intn\exists x\,[\prd{vampire}(x) \wedge \prd{mordre}(x,\cns{a})] )\)
\label{x:acvmd}


Le second argument de \prd{croire} est \(\Xlo\Intn\exists x\,[\prd{vampire}(x) \wedge \prd{mordre}(x,\cns{a})]\) qui dénote la proposition équivalant à l'ensemble de tous les mondes possibles dans lesquels il y a un vampire qui a mordu Alice.  Et \ref{x:acvmd} est vraie dans un monde $w$ ssi Alice est, dans $w$, en relation de croyance avec cette proposition --~c'est-à-dire qu'elle estime que $w$ appartient à cet ensemble de mondes (mais rien n'oblige à ce que $w$ soit réellement un élément de cet ensemble, pour peu qu'Alice ait une vision erronée de l'état du monde $w$).

Avec la lecture \alien{de re}\is{de re@\alien{de re}} de \sicut{un vampire},  \ref{x:acvm} signifie quelque chose comme : il y a un vampire (réel) tel qu'Alice croit qu'il l'a mordue. 
Cela nous donne immédiatement la clé pour traduire cette interprétation dans {\LO} :

\ex. 
\(\Xlo\exists x\,[\prd{vampire}(x) \wedge \prd{croire}(\cns{a}, \Intn\prd{mordre}(x,\cns{a}))]\)
\label{x:acvmr}


Dans \ref{x:acvmr}, ce que croit Alice, c'est la proposition dénotée par \(\Xlo\Intn\prd{mordre}(x,\cns{a})\), c'est-à-dire l'ensemble de tous les mondes dans lesquels $g(\vrb x)$ a mordu Alice, où $g$ est l'assignation courante (au moment où l'on interprète cette expression).  Dans tous ces mondes, $g(\vrb x)$ peut être ou non un vampire (selon les mondes), en revanche, $g(\vrb x)$ sera toujours le même individu (car l'interprétation de la proposition dépend de $g$, \ie\ elle n'affecte pas $g$). 
L'identité de cet individu est fixé en amont de la formule par $\Xlo\exists x$ (qui modifie l'assignation de départ pour donner une certaine valeur à \vrb x)
et par $\Xlo\prd{vampire}(x)$ (qui demande que cette valeur de $\vrb x$, $g(\vrb x)$, soit un vampire dans le monde où l'on évalue globalement \ref{x:acvmr}).
Ainsi \ref{x:acvmr} est vraie dans un monde $w$ ssi il existe dans $w$ un individu qui est vampire et tel qu'Alice croit qu'il l'a mordue%
\footnote{\label{fn:dere-exister}Les lecteurs attentifs auront cependant remarqué que \ref{x:acvmr} n'implique pas en l'état l'existence réel du vampire, dans le sens du prédicat \prd{exister} que nous avons introduit en \S\ref{sss:mondepossible}. En effet, dans $w$, le vampire \vrb x pourrait être un individu fictif. Est-ce à dire que la lecture \alien{de re} doit être corrigée en \(\Xlo\exists x\,[\prd{vampire}(x) \wedge \prd{exister}(x) \wedge \prd{croire}(\cns{a}, \Intn\prd{mordre}(x,\cns{a}))]\) pour correctement l'opposer à la lecture \alien{de dicto} ? Ce n'est pas certain (même si c'est une question qui mériterait d'être approfondie). Si \ref{x:acvmr} n'implique pas l'existence d'un vampire dans $w$, elle reste néanmoins compatible avec une telle existence. Et cela peut suffire pour la lecture \alien{de re}. Ce qui importe surtout pour cette lecture, c'est que le locuteur qualifie \vrb x de vampire dans le monde d'évaluation (que ce vampire soit fictif ou réel) ; alors que pour la lecture \alien{de dicto} \vrb x est qualifié de vampire au sein des croyance d'Alice ; ainsi avec \ref{x:acvmr}, Alice peut ne pas penser que celui qui l'a mordue est un vampire.}.

L'exercice~\ref{exo:redictoSue} p.~\pageref{exo:redictoSue}  permet d'appréhender plus en détail de mécanisme interprétatif qui distingue les lectures \alien{de re} et \alien{de dicto}.
Les analyses \ref{x:acvmd} et \ref{x:acvmr} nous montrent que, comme nous l'avions prévu en début de chapitre, l'ambiguïté \alien{de re} \vs\ \alien{de dicto} peut s'expliquer en termes de différence de portées relatives du {\GN} et de $\Xlo\Intn$ : nous obtenons la lecture \alien{de dicto} lorsque le {\GN} est interprété dans la portée de $\Xlo\Intn$, et la lecture \alien{de re} lorsqu'il est interprété à l'extérieur.  
C'est en fait l'opérateur $\Xlo\Intn$ (ainsi également que les opérateurs modaux) qui crée dans {\LO} ces contextes opaques qui nous occupent depuis le début.
Nous comprenons maintenant pourquoi : hors de la portée de $\Xlo\Intn$, le {\GN} est interprété par rapport au monde d'évaluation globale de la phrase (et que, par défaut, le locuteur assimilera au monde réel), alors que dans la portée de $\Xlo\Intn$, il est interprété par rapport à tous les mondes qui rendent vraie la proposition enchâssée.
Mais l'exercice \ref{exo:homfem} va nous montrer que finalement l'histoire n'est peut-être pas aussi simple que cela. 
Par ailleurs, comme annoncé précédemment, cette analyse bloque encore sur les {\GN} définis, mais, en l'occurrence, à cause d'une limite d'expressivité formelle de {\LO}.


Reprenons notre exemple récurrent \ref{x:OEd^} :

\ex. \label{x:OEd^}
\OE dipe$_1$ voulait épouser sa$_1$ mère.


Sans entrer dans les détails de l'analyse syntaxique et sémantique, nous allons d'abord faire l'hypothèse --~très raisonnable~-- que la subordonnée infinitive [\sicut{épouser sa$_1$ mère}] se traite comme les complétives que nous avons vues jusqu'ici (ainsi cette infinitive équivaut à \sicut{qu'il$_1$ épouse sa$_1$ mère}).
En traduisant \sicut{sa$_1$ mère} par \(\Xlo\atoi y\, \prd{mère}(y,\cns{\oe})\), nous obtenons très facilement l'analyse pour la lecture \alien{de dicto} ; il suffit d'avoir \(\Xlo\atoi y\, \prd{mère}(y,\cns{\oe})\) dans la portée de $\Xlo\Intn$ :


\ex. \label{x:fOEd^dd}
\(\Xlo\prd{vouloir}(\cns{\oe},\Intn\prd{épouser}(\cns{\oe},\atoi y\,
  \prd{mère}(y,\cns{\oe})))\)


Pour tout monde $w$,
 \(\Xlo\atoi y\, \prd{mère}(y,\cns{\oe})\)
dénote  l'unique individu de \Unv{A} qui est la
mère d'\OE dipe  \emph{dans $w$}.
Donc \(\Xlo\Intn[\prd{épouser}(\cns{\oe},\atoi
  y\,\prd{mère}(y,\cns{\oe}))]\)  dénote la proposition qui correspond à l'ensemble de tous les mondes $w$ dans
lesquels \OE dipe épouse celle qui est sa  mère dans $w$. Et \ref{x:fOEd^dd} sera vraie dans le monde d'évaluation si, dans ce monde, \OE dipe est en relation de \sicut{vouloir} avec cette proposition (il se trouve que, sémantiquement, le verbe \sicut{vouloir} pose une relation qui exprime non seulement le désir mais aussi la croyance, c'est-à-dire que si un individu veut ou souhaite que la proposition $p$ se réalise, c'est qu'il considère que le contenu de $p$ est compatible avec ses croyances). 


Maintenant, pour rendre compte de la lecture \alien{de re}, il faudrait faire en sorte que 
 \(\Xlo\atoi y\, \prd{mère}(y,\cns{\oe})\) s'interprète en dehors de la portée de $\Xlo\Intn$, mais nous n'avons tout simplement aucun endroit où le placer dans la formule. Car n'oublions pas que  \(\Xlo\atoi y\, \prd{mère}(y,\cns{\oe})\) est un terme, et pas une sous-formule\footnote{Il serait donc complètement fautif de proposer \(\Xlo\atoi y\,\prd{mère}(y,\cns{\oe}) \wedge
\prd{vouloir}(\cns{\oe},\Intn\prd{épouser}(\cns{\oe},y))\), 
car ce n'est pas une expression bien formée de {\LO} (on ne peut connecter par $\Xlo\wedge$ que des \emph{formules}).  
De même \(\Xlo\atoi y\,[\prd{mère}(y,\cns{\oe}) \wedge
  \prd{vouloir}(\cns{\oe},\Intn\prd{épouser}(\cns{\oe},y))]\) ne convient pas non plus,
car même si c'est une expression bien formée de {\LO}, il ne s'agit pas globalement d'une formule et cela ne peut être la traduction d'une phrase (cette expression dénote l'unique individu qui est la mère d'\OE dipe et qu'il veut épouser).
}.
La seule option qui s'offre à nous, en désespoir de cause, est la traduction \ref{x:fOEd^dr3}, où  \(\Xlo\atoi y\,\prd{mère}(y,\cns{\oe})\) et \(\Xlo\atoi z\,\prd{vouloir}(\cns{\oe},\Intn\prd{épouser}(\cns{\oe},z))\) seront évalués et identifiés dans le même monde :


\ex. \label{x:fOEd^dr3}
\(\Xlo\atoi y\,\prd{mère}(y,\cns{\oe}) = \atoi z\,\prd{vouloir}(\cns{\oe},\Intn\prd{épouser}(\cns{\oe},z))\)


Mais c'est peu satisfaisant. 
D'abord sémantiquement, même si elle présente les bonnes conditions de vérité, la formule \ref{x:fOEd^dr3} présuppose par ailleurs qu'\OE dipe ne voulait épouser qu'une seule personne, présupposition qui, en dépit de nos connaissances usuelles du monde, n'est pas du tout présente dans \ref{x:OEd^}.
Ensuite, sur le plan compositionnel, cette traduction nous fait passer par une périphrase qui consiste en
\sicut{la mère d'\OE dipe est celle qu'\OE dipe veut épouser}, et il semble peu plausible que la grammaire du français nous mène simplement de \ref{x:OEd^} à \ref{x:fOEd^dr3}. Nous préférerions une traduction dans {\LO} dont la structure logique soit plus proche de celle de \ref{x:OEd^}. 
Nous aboutissons donc à un semi-échec, et qui est, en grande partie, lié à un problème de compositionnalité ; les chapitres \ref{ch:types} et \ref{ch:ISS} se consacrent intensivement à ce problème, et les développements de {\LO} que nous y aborderons vont nous permettre de donner à \ref{x:OEd^} une traduction qui soit plus dans l'esprit de \ref{x:acvmr}, c'est-à-dire avec la contribution sémantique du {\GN} correctement placée en dehors de la portée de $\Xlo\Intn$ (sans passer par l'égalité artificielle de \ref{x:fOEd^dr3}).

Notons, pour terminer, qu'à ce stade, il peut être intéressant d'examiner le problème en revisitant l'approche alternative que nous avions envisagée au chapitre~\ref{ch:gn}, \S\ref{sss:DefPsp}, pour l'analyse des {\GN} définis. 
Selon cette approche, la contribution sémantique de ces {\GN} se réalise au moyen de variables localement libres (et non des $\atoi$-termes) en traitant séparément le contenu présupposé par le défini. Cela change significativement la donne. Je ne vais pas présenter explicitement cette option ici, elle va faire l'objet de l'exercice \ref{exo:redictoPsp}, et nous verrons que la situation se retrouve inversée : la lecture \alien{de re} du défini se traduit simplement, mais nous butons un peu sur sa lecture \alien{de dicto}.

\medskip

% -*- coding: utf-8 -*-
\begin{exo}\label{exo:redictoSue}
Reprenons en détail l'analyse d'un {\GN} indéfini ambigu comme en :
\pagesolution{crg:redictoSue}
\begin{enumerate}[label=(\alph*)]
\item Sue pense qu'un républicain va remporter l'élection.
\end{enumerate}

Pour simplifier l'exercice, nous considérerons que le groupe verbal
\sicut{va remporter l'élection} se traduit par le prédicat à une place
\prd{élu} (ainsi $\Xlo\prd{élu}(x)$ signifie $\vrb x$ est ou sera élu) ; de plus
nous ne tiendrons pas compte du temps verbal.

Soit le modèle-jouet suivant
\(\Modele=\tuple{\Unv{A},\Unv{W},\FI}\), avec
\(\Unv{A}=\set{\Obj{Barry} ; \Obj{Johnny} ; \Obj{Sue}}\), et
\(\Unv{W}=\set{\w_1; \w_2; \w_3; \w_4; \w_5; \w_6; \w_7; \w_8}\).

On supposera que dans le modèle {\Modele}, on ne peut pas être à la
fois républicain et démocrate, et qu'une seule personne peut remporter
l'élection. 
Le tableau~\ref{t:republ} nous donne les dénotations de \prd{républicain}, \prd{démocrate} et
  \prd{élu} dans {\Modele}.
Et complétons ce modèle, en ajoutant que dans les mondes $\w_1$ et $\w_7$, \Obj{Sue} croit la proposition \set{\w_1;\w_2;\w_3;\w_6}, dans les mondes $\w_2$, $\w_3$ et $\w_4$, elle croit la proposition \set{\w_2;\w_4;\w_6;\w_8}, et dans les mondes $\w_5$, $\w_6$ et $\w_8$, elle croit la proposition \set{\w_1;\w_3;\w_5;\w_7}.
%\fixme{Croyances de Sue}

\begin{table}[h]
\begin{bigcenter}
\caption{Interprétations de \prd{républicain}, \prd{démocrate} et
  \prd{élu} dans {\Modele}}\label{t:republ}
{\small
\begin{tabular}{@{}l@{}l@{}}\lsptoprule
\begin{tabular}{l}
\(\FI(\w_1,\prd{républicain})=\set{\Obj{Barry} ; \Obj{Johnny}}\), \\
\(\FI(\w_1,\prd{démocrate})=\eVide\),\\
\(\FI(\w_1,\prd{élu})=\set{\Obj{Barry}}\) ;
\\%\hline
\(\FI(\w_2,\prd{républicain})=\set{\Obj{Barry} ; \Obj{Johnny}}\), \\
\(\FI(\w_2,\prd{démocrate})=\eVide\),\\
\(\FI(\w_2,\prd{élu})=\set{\Obj{Johnny}}\) ;
\\%\hline
\(\FI(\w_3,\prd{républicain})=\set{\Obj{Barry}}\), \\
\(\FI(\w_3,\prd{démocrate})=\set{\Obj{Johnny}}\),\\
\(\FI(\w_3,\prd{élu})=\set{\Obj{Barry}}\) ;
\\
\(\FI(\w_4,\prd{républicain})=\set{\Obj{Barry}}\), \\
\(\FI(\w_4,\prd{démocrate})=\set{\Obj{Johnny}}\),\\
\(\FI(\w_4,\prd{élu})=\set{\Obj{Johnny}}\) ;
\end{tabular}
&
\begin{tabular}{l}
\(\FI(\w_5,\prd{républicain})=\set{\Obj{Johnny}}\),\\ 
\(\FI(\w_5,\prd{démocrate})=\set{\Obj{Barry}}\),\\
\(\FI(\w_5,\prd{élu})=\set{\Obj{Barry}}\) ;
\\%\hline
\(\FI(\w_6,\prd{républicain})=\set{\Obj{Johnny}}\), \\
\(\FI(\w_6,\prd{démocrate})=\set{\Obj{Barry}}\),\\
\(\FI(\w_6,\prd{élu})=\set{\Obj{Johnny}}\) ;
\\%\hline
\(\FI(\w_7,\prd{républicain})=\eVide\), \\
\(\FI(\w_7,\prd{démocrate})=\set{\Obj{Barry} ; \Obj{Johnny}}\),\\
\(\FI(\w_7,\prd{élu})=\set{\Obj{Barry}}\) ;
\\%\hline
\(\FI(\w_8,\prd{républicain})=\eVide\), \\
\(\FI(\w_8,\prd{démocrate})=\set{\Obj{Barry} ; \Obj{Johnny}}\),\\
\(\FI(\w_8,\prd{élu})=\set{\Obj{Johnny}}\) ;
\end{tabular}
\\\lspbottomrule
\end{tabular} }
\end{bigcenter}
\end{table}

Dans tout cet exercice, nous manipulerons les intensions des formules (\ie\ les propositions) directement comme des ensembles de mondes.


\begin{enumerate}
\item Quelle est l'intension de \(\Xlo\exists x [\prd{républicain}(x)
    \wedge \prd{élu}(x)]\) dans {\Modele} et par rapport à une
    assignation $g$ quelconque ?
\item Quelle est l'intension de \(\Xlo\prd{élu}(x)\) dans {\Modele}, par
  rapport à une assignation $g_1$ telle que $g_1(\vrb x)=\Obj{Barry}$ ? Et par rapport à l'assignation $g_2$ telle que $g_2(\vrb x)=\Obj{Johnny}$ ?
\item Quelle est l'intension de \(\Xlo\prd{penser}(\cns s,\Intn\prd{élu}(x))\) par rapport à $g_1$ ?  Et par rapport à $g_2$ ? (\cns s dénote \Obj{Sue}, naturellement)
\item Quelle est l'intension de \(\Xlo\exists x [\prd{républicain}(x) \wedge \prd{penser}(\cns s,\Intn\prd{élu}(x))]\) ?
\item Quelle est l'intension de \(\Xlo\prd{penser}(\cns s,\Intn\exists x [\prd{républicain}(x) \wedge \prd{élu}(x)])\) ?
\end{enumerate}

\begin{solu}(p.~\pageref{exo:redictoSue})\label{crg:redictoSue}
\begin{enumerate}
\item Nous allons présenter les intensions de formules (\ie\ les propositions) sous la forme d'ensembles de mondes. 
L'intension de \(\Xlo\exists x [\prd{républicain}(x) \wedge \prd{élu}(x)]\) est, par définition (p.~\pageref{pt:prop°}), l'ensemble de tous les mondes dans lesquels celui qui est élu est un républicain. D'après le tableau~\ref{t:republ}, p.~\pageref{t:republ}, il s'agit de l'ensemble \set{\w_1;\w_2;\w_3;\w_6}.

\item Avec $g_1$ qui donne $g_1(\vrb x)=\Obj{Barry}$, l'intension de \(\Xlo\prd{élu}(x)\) est l'ensemble de tous les mondes où \Obj{Barry} est élu. C'est-à-dire : \set{\w_1;\w_3;\w_5;\w_7}. 
Et avec $g_2$ qui donne $g_2(\vrb x)=\Obj{Johnny}$, l'intension de \(\Xlo\prd{élu}(x)\) est l'ensemble de tous les mondes où \Obj{Johnny} est élu : \set{\w_2;\w_4;\w_6;\w_8}.
\item Par rapport à $g_1$, l'intension de \(\Xlo\prd{penser}(\cns s,\Intn\prd{élu}(x))\)  est \set{\w_2;\w_3;\w_4} (d'après les indications de l'énoncé de l'exercice).  Et par rapport à $g_2$, l'intension de la formule  est \set{\w_5;\w_6;\w_8}.
\item Donc dans $\w_2$, $\w_3$ et $\w_4$, \Obj{Sue} pense que \Obj{Barry} va être élu, et dans $\w_5$, $\w_6$ et $\w_8$ elle pense que \Obj{Johnny} va être élu.  Dans $\w_2$, $\w_3$ et $\w_4$, \Obj{Barry} est républicain et dans $\w_5$ et $\w_6$  \Obj{Johnny} est républicain.  Donc l'intension de \(\Xlo\exists x [\prd{républicain}(x) \wedge \prd{penser}(\cns s,\Intn\prd{élu}(x))]\) est \set{\w_2;\w_3;\w_4;\w_5;\w_6}.
\item D'après l'énoncé de l'exercice et le résultat de la question 1, l'intension de la formule \(\Xlo\prd{penser}(\cns s,\Intn\exists x [\prd{républicain}(x) \wedge \prd{élu}(x)])\) est \set{\w_1;\w_7}.
\end{enumerate}
\end{solu}
\end{exo}




\begin{exo}\label{exo:redictoPsp}
Selon l'approche explicitement présuppositionnelle 
\pagesolution{crg:redictoPsp}
(cf.\ \S\ref{sss:DefPsp}, chapitre~\ref{ch:gn}) de l'analyse des définis, une phrase comme \sicut{\OE dipe a épousé sa mère} se traduira par \(\Xlo\prd{épouser}(\cns\oe,x)\) (ou éventuellement par \(\Xlo\prd{épouser}(\cns \oe,x)\wedge \prd{mère}(x,\cns\oe)\)) sachant que par présupposition, nous savons que dans le contexte d'énonciation, \vrb x dénote l'unique individu qui est «mère d'\OE dipe» (formellement la présupposition se traduit donc par \(\Xlo\exists x [\prd{mère}(x,\cns\oe)\wedge \forall y [\prd{mère}(y,\cns\oe)\ssi y=x]]\)).\\
Donnez la traduction de \sicut{\OE dipe voulait épouser sa mère} d'abord pour la lecture \alien{de re} de \sicut{sa mère}. Puis expliquez la difficulté qui se pose pour sa traduction avec la lecture \alien{de dicto}.
%
\begin{solu}(p.~\pageref{exo:redictoPsp})\label{crg:redictoPsp}

%\sloppy
Sachant, par présupposition, que \vrb x dénote la mère d'\OE dipe dans le monde d'évaluation, la lecture \dere\ de la phrase se traduit par \(\Xlo\prd{vouloir}(\cns{\oe},\Intn\prd{épouser}(x))\) (ou éventuellement \(\Xlo\prd{mère}(x,\cns{\oe}) \wedge \prd{vouloir}(\cns{\oe},\Intn\prd{épouser}(x))\)).  


Pour la lecture \dedicto, le mieux que nous pouvons proposer (en traduisant le \GN\ par une variable libre) serait \(\Xlo\prd{vouloir}(\cns{\oe},\Intn[\prd{mère}(x,\cns{\oe}) \wedge \prd{épouser}(x)])\).  Mais du fait de la présupposition, nous savons déjà que \vrb x dénote la mère réelle d'\OE dipe ; cela ne convient donc pas à l'interprétation recherchée. Pour cette lecture, nous sommes en fait obligés de suspendre la présupposition (cf. \S\ref{sss:ptépsp}, p.~\pageref{p.suspen}). C'est-à-dire que nous devons l'empêcher de se projeter (\ie\ de figurer dans le contexte général de la phrase), et de la confiner à l'intérieur de la traduction de la subordonnée (pour que les informations relatives à l'identité de la mère d'\OE dipe soient localisées dans les croyances d'\OE dipe).  Autrement dit, la traduction correcte devrait être : \(\Xlo\prd{vouloir}(\cns{\oe},\Intn\exists x [[\prd{mère}(x,\cns\oe)\wedge \forall y [\prd{mère}(y,\cns\oe)\ssi y=x]] \wedge \prd{épouser}(x)])\).  Mais ce n'est pas quelque chose qui s'obtient de façon simple en sémantique compositionnelle\footnote{C'est une question qui a directement trait au problème de la projection des présuppositions (cf. p.~\pageref{p.projpsp}) et qui est abordée, notamment, par \citet{Heim:92} et qui nous dirige vers la sémantique dynamique.}.

\fussy
\end{solu}
\end{exo}

% -*- coding: utf-8 -*-
\begin{exo}\label{exo:homfem}
Cet exercice, en plus d'être une application de ce qui est présenté
\pagesolution{crg:homfem}
dans cette section, met le doigt sur un «petit» problème d'analyse
sémantique. 
\begin{enumerate}[label=(\alph*)]
\item {[\itshape{C'est l'anniversaire de Paul. Ses amis ont décidé d'organiser une fête surprise chez lui.  Et ils ont opté pour une soirée déguisée. Tous ses amis hommes ont choisi de se déguiser en femmes : en héroïnes de bande-dessinée. Et leurs déguisements étaient si bien réussis que sur le moment, quand il est arrivé...}}] Paul a cru que tous les hommes qui étaient là étaient des femmes.
\end{enumerate}

La phrase qui nous intéresse ici est la dernière de ce paragraphe ; le
texte en italique est là pour fixer un contexte et orienter la
compréhension.  Cette phrase est ambiguë (au moins théoriquement). %, et l'ambiguïté repose sur les lectures \alien{de re} vs.\ \alien{de dicto} du groupe nominal \sicut{tous les hommes qui étaient là} (ou simplement \sicut{tous les hommes}).

\begin{enumerate}
\item Explicitez les deux lectures en donnant, en français, deux paraphrases (ou gloses) précises et suffisamment distinctes de la dernière phrase de (a).
\item Prouvez qu'il s'agit bien d'une ambiguïté en utilisant la méthode vue au chapitre~\ref{Ch:1}. %en cours.
\item Traduisez les deux lectures en \LO\footnote{Ne traduisez pas la relative \sicut{qui étaient là}, elle n'est pas déterminante pour l'exercice.}.
\item L'ambiguïté de la phrase ne saute pas aux yeux car une des deux lectures est assez peu naturelle. De laquelle s'agit-il ? Et essayer d'expliquer pourquoi elle est si peu naturelle. 
\item Cet exercice montre que la dernière phrase de (a) est intrigante car elle remet en cause quelque chose que nous avons vu dans un chapitre précédent. De quoi s'agit-il ?
\end{enumerate}
\begin{solu}(p.~\pageref{exo:homfem})\label{crg:homfem}
\begin{enumerate}
\item Il s'agit bien sûr d'une ambiguïté \alien{de re}/\alien{de dicto} sur le {\GN} \sicut{tous les hommes (qui étaient là)}.
Pour la lecture \alien{de re} la glose sera :
  \begin{enumerate}
  \item Pour chaque individu qui est un homme et qui était là, Paul a cru qu'il s'agissait d'une femme. 
  \end{enumerate}
Pour la lecture \alien{de dicto} la glose sera :
  \begin{enumerate}
  \item[b.] Paul s'est dit : «~tiens, tous les hommes qui sont là sont des femmes~».
  \end{enumerate}
\item Construisons un modèle $\Modele$ par rapport auquel la phrase avec la lecture \alien{de re} sera vraie et celle avec la lecture \alien{de dicto} sera fausse.  $\Modele$ contient les données suivantes :
\begin{itemize}
\item \Obj{Paul}, un homme ;
\item \Obj{Antoine} et \Obj{Mickaël}, des hommes, amis de \Obj{Paul} ;
\item \Obj{Julie} et \Obj{Sarah}, des femmes, amies de \Obj{Paul} ;
\item \Obj{Antoine} est déguisé en Catwoman ;
\item \Obj{Mickaël} est déguisé en Batgirl ;
\item \Obj{Julie} est déguisée en Iron Man ;
\item \Obj{Sarah} est déguisée en marsupilami ;
\item \Obj{Paul} ne reconnaît aucun de ses amis ;
\item \Obj{Paul} croit que \Obj{Antoine} et \Obj{Mickaël} sont des femmes ;
\item \Obj{Paul} croit que \Obj{Julie} et \Obj{Sarah} sont des hommes ;
\end{itemize}

Avec la lecture \alien{de dicto} de \sicut{tous les hommes}, la phrase est fausse, car les individus qui sont des hommes dans les croyances de Paul sont Julie et Sarah et Paul pense que ces deux individus sont des hommes et non pas des femmes.  Mais avec la lecture \alien{de re} de \sicut{tous les hommes}, la phrase est vraie, car les individus qui sont des hommes dans $\Modele$ sont Antoine et Mickaël et Paul croit bien qu'Antoine et Mickaël sont des femmes\footnote{NB : dans cette démonstration, on exclut Paul de l'ensemble des hommes quand on fait la quantification de \sicut{tous les hommes}. C'est un problème annexe ; en fait il s'agit d'un phénomène courant lorsqu'une phrase contient un quantificateur universel et un GN référentiel, ce dernier se retrouve exclu de la quantification. Cf.\ par exemple \sicut{dans la classe, Jean est plus grand que tout le monde} ; techniquement cette phrase devrait toujours être fausse car Jean fait partie de \sicut{tout le monde} et Jean n'est pas plus grand que lui-même, mais par pragmatique on interprète la phrase en partitionnant l'ensemble des individus en faisant en sorte que \sicut{tout le monde} signifie «tout individu sauf Jean».}. 

\item 
  \begin{enumerate}
  \item \alien{De re} :\\
\(\Xlo\forall x [\prd{homme}(x) \implq \prd{croire}(\cns p,\Intn\prd{femme}(x))]\)
  \item \alien{De dicto} :\\
\(\Xlo\prd{croire}(\cns p,\Intn\forall x [\prd{homme}(x) \implq \prd{femme}(x)])\)
  \end{enumerate}
\item Comme l'indique la glose ci-dessus, la lecture \alien{de dicto} est la moins naturelle des deux car elle attribue à Paul la pensée (i.e. la proposition) qui dit que tous les hommes sont des femmes. Mais dans quels mondes possibles la phrase \sicut{tous les hommes sont des femmes} est vraie ? Probablement aucun (si on ne change pas le sens des mots), et cette phrase est absurde (au sens logique, c'est-à-dire que c'est une contradiction, une phrase qui n'est jamais vraie).  Et il est fort probable que Paul n'a pas ce genre de pensée : cela voudrait dire qu'il considère que le monde dans lequel il se trouve appartient à l'ensemble vide...
\item L'interprétation la plus naturelle est celle avec la lecture \alien{de re} du GN \sicut{tous les hommes}.  Mais, comme le montre la formule (3a), cela veut dire que ce GN est interprété avec une portée large, \emph{en dehors de la proposition syntaxique où il apparaît}. C'est un contre-exemple à l'observation que nous avions faite en \S\ref{sss:limiteportée}, p.~\pageref{pt:Portee2}, et qui disait que les GN quantificationnels forts (par ex. universels) ne peuvent pas «~traverser~»\ une frontière de proposition.
\end{enumerate}
\end{solu}
\end{exo}




\section{Récapitulatif et conclusions}
%------------------------------------
\subsection{{LO} intensionnel}
%''''''''''''''''''''''''''''''''''
\label{ss:LOInt}

Pour conclure ce chapitre, prenons le temps de récapituler les modifications que l'intensionnalisation a apporté à notre langage sémantique {\LO}. 

Rappelons d'abord quelques définitions de base. $\VAR$ est l'ensemble de toutes les variables. 
$\CON_0$ est l'ensemble des constantes d'individus, et pour tout nombre entier $n$ pertinent,  $\CON_n$ est l'ensemble des constantes de prédicats $n$-aires. 
%, $\CON_2$, $\CON_3$... 
$\CON$ regroupe toutes les constantes (\ie\ de $\CON_0$, $\CON_1$, $\CON_2$, $\CON_3$, etc.).
Les éléments de $\CON_0$ et ceux de $\VAR$ sont des termes. 
Les éléments de $\CON$, ceux de $\VAR$ ainsi que les formules sont expressions bien formées de {\LO}.  Toutes les autres expressions bien formées sont spécifiées dans la syntaxe suivante.

\largerpage[-1]

\begin{defi}[Syntaxe de \LO]
%^^^^^^^^^^^^^^^^^^^^
\label{SynPw}
\begin{enumerate}[syn,series=RglSyn2] %[(\RSyn1)]
\item 
\begin{enumerate}
\item Si $\Xlo\alpha$ est un terme et $\Xlo P$ une constante de $\CON_1$, alors $\Xlo P(\alpha)$ est une formule ;
\item Si $\Xlo\alpha$ et $\Xlo\beta$ sont des termes et $\Xlo P$ une constante de $\CON_2$, alors $\Xlo P(\alpha,\beta)$ est une formule ;
\item Si $\Xlo\alpha$, $\Xlo\beta$ et $\Xlo\gamma$ sont des termes et $\Xlo P$ une constante de $\CON_3$, alors $\Xlo P(\alpha,\beta,\gamma)$ est une
formule ;
\item etc.
\end{enumerate}
\label{SynPAppw}
\item Si $\Xlo\alpha$ et $\Xlo\beta$ sont des termes, alors $\Xlo\alpha=\beta$
     est une formule ;
\label{SynP=w}
\item Si $\Xlo\phi$  est une formule, alors
 \(\Xlo\neg\phi\) est une formule ;
\label{SynPNegw}
\item Si $\Xlo\phi$ et $\Xlo\psi$  sont des formules, alors
      \(\Xlo[\phi \wedge \psi], [\phi \vee \psi], [\phi \implq
     \psi]\) et \(\Xlo[\phi \ssi \psi]\) sont des formules ;
\label{SynPConnw}
\item Si $\Xlo\phi$ est une formule  et $\vrb v$ une variable, alors $\Xlo\forall v \phi$ et
     $\Xlo\exists v \phi$ sont des formules ;%
\label{SynPQw}
\item Si $\Xlo\phi$ est une formule  et $\vrb v$ une variable, alors $\Xlo\atoi v \phi$ est un terme ;
\label{SynPiww}
\item Si $\Xlo\phi$ est une formule et si $n$ est un nombre entier, alors $\Xlo\doitn n\phi$ et $\Xlo\peutn{n}\phi$ sont des formules ;%
\label{SynModw}
\item Si $\Xlo\alpha$ est une expression bien formée, alors $\Xlo\Intn\alpha$ est une expression bien formée, et c'est une expression intensionnelle ;
\label{Syn^w}
\item Si $\Xlo\alpha$ est une expression intensionnelle, alors $\Xlo\Extn\alpha$ est une expression bien formée.
\label{Synvw}
\setcounter{RglSynt}{\value{enumi}}
\end{enumerate}
\end{defi}

Nous voyons que les expressions bien formées de {\LO} ne sont plus seulement des formules --~précisément par l'ajout des règles (\RSyn\ref{Syn^w}) et (\RSyn\ref{Synvw}). 
Et notons également que cette syntaxe en soi est incomplète (et donc imparfaite), car si (\RSyn\ref{Syn^w}) nous permet de créer des expressions intensionnelles, aucune autre règle ne les réutilise pour les insérer dans des expressions plus complexes (à part (\RSyn\ref{Synvw}) mais qui le fait de manière encore trop simpliste). 
Nous avions déjà mentionné cela en \S\ref{sss:^v}, et en toute rigueur, nous devrions ajouter des compléments aux règles (\RSyn\ref{SynPAppw}), produisant des formules comme $\Xlo P(\Intn\phi)$, $\Xlo P(\alpha,\Intn\phi)$, $\Xlo P(\alpha,\Intn\phi,\beta)$... pour certaines constantes \vrb P des ensembles $\CON_n$\footnote{Ces constantes seront les traductions de verbes qui prennent une complétive en argument, et ils peuvent être de diverses arités, notamment d'arité 1 pour traduire les tours impersonnels en \sicut{il semble que}, \sicut{il paraît que}, ou d'arité 3 pour les verbes comme \sicut{promettre}, \sicut{annoncer}, etc.}.
Mais autorisons nous à ne pas surcharger davantage cette définition \ref{SynPw}, car nous allons voir dans le prochain chapitre une façon de remettre proprement et rigoureusement de l'ordre dans ces règles (sans avoir donc à faire des cas particuliers pour l'enchâssement de propositions). 

La règle (\RSyn\ref{SynModw}) introduit les opérateurs modaux en adoptant l'option des multimodalités, c'est-à-dire que nous nous autorisons à les multiplier à volonté pour autant d'entiers $n$ qui nous conviennent. Et pour nous aligner avec les notations traditionnelles, nous continuerons à utiliser $\Xlo\doit$ et $\Xlo\peut$ comme de simples variantes typographiques de $\Xlo\doitn0$ et $\Xlo\peutn{0}$.

Nous n'avons pas repris les opérateurs temporels $\mP$ et $\mF$, puisque, comme annoncé, la temporalité sera réintroduite au chapitre \ref{Ch:temps2} (vol.~2) sous une forme légèrement différente.  Notre modèle intensionnel se retrouve ainsi allégé (provisoirement) de \Tps.


\begin{defi}[Modèle intensionnel]
%^^^^^^^^^^^^^^^^^^^^^^^^^^^^^^^^
Un modèle intensionnel $\Modele$ est défini comme une structure \(\tuple{\Unv A,\Unv W, \Unv R,F}\) où \Unv A est un ensemble d'individus, \Unv W un ensemble de mondes possibles, \Unv R un ensemble de relations d'accessibilité sur \Unv W, et {\FI} une fonction d'interprétation qui pour tout monde $w$ de \Unv W et toute constante de {\CON} associe la dénotation de cette constante dans $w$.
\end{defi}


\Unv R est un ensemble fini qui se compose d'une série de $m+1$ relations d'accessibilité que nous allons nommer en les indiçant numériquement successivement de la manière suivante :  $\RK_0$, $\RK_1$, $\RK_2$, ..., $\RK_m$.  
Et par convention, nous posons que $\RK_0$ est la relation d'accessibilité aléthique, c'est-à-dire celle qui relie chaque monde de \Unv W avec tous les autres (elle correspond donc à tout l'ensemble $\Unv W\times\Unv W$).

Et rappelons, pour être tout à fait précis, que l'ensemble d'arrivée de la fonction $\FI$ dépend du type de constante qu'elle interprète. Pour les constantes de $\CON_0$ cet ensemble est \Unv A, pour les constantes de $\CON_1$  c'est $\powerset(\Unv A)$ (l'ensemble de tous les sous-ensembles de \Unv A), pour les constantes de $\CON_2$ c'est $\powerset(\Unv A^2)$ (l'ensemble de tous les ensembles de couples d'éléments de \Unv A), etc. 
Autrement dit, pour les constantes de $\CON_n$, avec $n>0$, l'ensemble d'arrivée de $\FI$ est $\powerset(\Unv A^n)$.


L'interprétation, c'est-à-dire le calcul de la dénotation, d'une expression \vrb\alpha\ de {\LO} se fait par rapport à un modèle \Modele, un monde possible $w$ de \Unv W et une fonction d'assignation $g$. Jusqu'à présent les assignations étaient des fonctions de \VAR\ vers \Unv A ; dorénavant elles seront également définies entre $\mathbb{N}$ (l'ensemble des entiers naturels) et l'intervalle d'entiers $[0,m]$ (où $m$ est l'indice de la  «dernière» relation de \Unv R). Ainsi à tout nombre entier, les fonctions $g$ associent un entier compris entre $0$ et $m$. Et par convention, nous posons la contrainte que pour toute assignation $g$, $g(0)=0$.

\largerpage[-1]

La dénotation de \vrb\alpha\ est donc sa valeur sémantique indicée par ces trois paramètres : $\denote{\vrb\alpha}^{\Modele,w,g}$.  Et par définition, l'intension de \vrb\alpha\ est sa valeur sémantique relative seulement à \Modele\ et $g$, \(\denote{\vrb\alpha}^{\Modele,g}\), c'est-à-dire la fonction sur \Unv W définie par $w\longmapsto \denote{\vrb\alpha}^{\Modele,w,g}$.

\begin{defi}[Interprétation des variables et des constantes]
%^^^^^^^^^^^^^^^^^^^^^^^^^^^^^^^^^^^^^^^^^^^^^^^^^^^^^^^^^^^
  Soit un modèle \(\Modele = \tuple{\Unv{A},\Unv W,\Unv R,\FI}\) et $g$ une fonction
  d'assignation : 
\begin{itemize}
\item si $\vrb v$ est une variable de \VAR, \(\denote{\vrb v}^{\Modele,w,g}=g(\vrb v)\) ;
\item si $\Xlo a$ est une constante de \CON, \(\denote{\Xlo a}^{\Modele,w,g}=\FI(w,\vrb a)\).
\end{itemize}
\end{defi}

\begin{defi}[Interprétation des expressions bien formées]
%^^^^^^^^^^^^^^^^^^^^^^^^^^^^^^^^^^^^^^^^^^^^^^^^^^^^^^^^
\label{RIw}
Soit un modèle \(\Modele = \tuple{\Unv{A},\Unv W,\Unv R,\FI}\) et $g$ une fonction
d'assignation. % de $\VAR$ dans \Unv{A}.
\begin{enumerate}[sem,series=RglSem3] %[(\RSem1)]
  \item \raggedright
\label{RIprdw}
\begin{enumerate}
\item
%Si $P$ est un prédicat à une place et si $\alpha$ est un terme, alors
$\denote{\Xlo P(\alpha)}^{\Modele,w,g}=1$ ssi \(\denote{\Xlo\alpha}^{\Modele,w,g} \in
\denote{\Xlo P}^{\Modele,w,g}\) ; 
%
\item %Si $P$ est un prédicat à deux places et si $\alpha$ et $\beta$ sont des
%termes, alors 
$\denote{\Xlo P(\alpha,\beta)}^{\Modele,w,g}=1$ ssi
\(\tuple{\denote{\Xlo\alpha}^{\Modele,w,g},\denote{\Xlo\beta}^{\Modele,w,g}} \in 
\denote{\Xlo P}^{\Modele,w,g}\) ; 
%
\item %Si $P$ est un prédicat à trois places et si $\alpha$, $\beta$ et
%  $\gamma$ sont des termes, alors 
$\denote{\Xlo P(\alpha,\beta,\gamma)}^{\Modele,w,g}=1$ ssi
\(\tuple{\denote{\Xlo\alpha}^{\Modele,w,g},\denote{\Xlo\beta}^{\Modele,w,g},\denote{\Xlo\gamma}^{\Modele,w,g}}
\in \denote{\Xlo P}^{\Modele,w,g}\) ; 
%
\item etc.
  \end{enumerate}
\item \label{RI=w}
%Si $\alpha$ et $\beta$ sont des termes, alors 
\(\denote{\Xlo\alpha = \beta}^{\Modele,w,g}=1\) ssi
\(\denote{\Xlo\alpha}^{\Modele,w,g}=\denote{\Xlo\beta}^{\Modele,w,g}\) ; 
\item \label{RInegw}
%Si $\phi$ est une formule, alors 
\(\denote{\Xlo\neg\phi}^{\Modele,w,g}=1\) ssi
\(\denote{\Xlo\phi}^{\Modele,w,g}=0\) ; 
\item \label{RIconw}
  %Si $\phi$ et $\psi$ sont des formules, alors 
  \begin{enumerate}
\item $\denote{\Xlo[\phi \wedge \psi]}^{\Modele,w,g}=1$ ssi $\denote{\Xlo\phi}^{\Modele,w,g}=1$ \emph{et} $\denote{\Xlo\psi}^{\Modele,w,g}=1$ ;
\item $\denote{\Xlo[\phi \vee \psi]}^{\Modele,w,g}=1$ ssi $\denote{\Xlo\phi}^{\Modele,w,g}=1$ \emph{ou} $\denote{\Xlo\psi}^{\Modele,w,g}=1$ ;
\item $\denote{\Xlo[\phi \implq \psi]}^{\Modele,w,g}=1$ ssi $\denote{\Xlo\phi}^{\Modele,w,g}=0$ \emph{ou} $\denote{\Xlo\psi}^{\Modele,w,g}=1$ ;
\item $\denote{\Xlo[\phi \ssi \psi]}^{\Modele,w,g}=1$ ssi $\denote{\Xlo\phi}^{\Modele,w,g}=\denote{\Xlo\psi}^{\Modele,w,g}$ ;
  \end{enumerate}
\item\label{RIQgw}
\begin{enumerate}
\item \(\denote{\Xlo\exists v \phi}^{\Modele,w,g} = 1\) ssi 
il existe au moins un individu \Obj{d} de \Unv{A} tel que \(\denote{\Xlo\phi}^{\Modele,w,g_{[\Obj{d}/v]}} = 1\) ;
\item \(\denote{\Xlo\forall v \phi}^{\Modele,w,g} = 1\) ssi pour tout
  individu \Obj{d} de \Unv{A}, \(\denote{\Xlo\phi}^{\Modele,w,g_{[\Obj{d}/v]}} = 1\) ;
\end{enumerate}
\item 
\(\denote{\Xlo\atoi v \phi}^{\Modele,w,g}=\Obj{d}\) ssi \Obj{d} est
  l'unique individu de \Unv{A} tel que \(\denote{\Xlo\phi}^{\Modele,w,g_{[\Obj{d}/v]}}=1\) ; 
\item \label{RIModw}
\begin{enumerate}
\item \(\denote{\Xlo\peutn{n} \phi}^{\Modele,w,g} = 1\) ssi il existe un monde $w'\in \Unv W$ tel que $w\RK_{g(n)} w'$ et \(\denote{\Xlo\phi}^{\Modele,w',g} = 1\) ;
\item \(\denote{\Xlo\doitn n \phi}^{\Modele,w,g} = 1\) ssi pour tout monde $w'\in \Unv W$ tel que $w\RK_{g(n)} w'$, alors \(\denote{\Xlo\phi}^{\Modele,w',g} = 1\) ;
\end{enumerate}
\item \(\denote{\Xlo\Intn\alpha}^{\Modele,w,g} = \denote{\Xlo\alpha}^{\Modele,g}\), c'est-à-dire la fonction \(w'\longmapsto \denote{\Xlo\alpha}^{\Modele,w',g}\) ;
\item \(\denote{\Xlo\Extn\alpha}^{\Modele,w,g} = \denote{\Xlo\alpha}^{\Modele,w,g}(w)\).
\setcounter{RglSem}{\value{enumi}}
\end{enumerate}
\end{defi}

Les règles (\RSem\ref{RIModw}) présentent précisément l'interprétation «contextualisée» des opérateurs modaux. La relation d'accessibilité utilisée pour interpréter $\Xlo\peutn{n}$ et $\Xlo\doitn n$ est $\RK_{g(n)}$ qui dépend bien de $g$. 
Cependant pour $n=0$, \ie\ $\Xlo\peut$ et $\Xlo\doit$, comme $g(0)=0$ et que $\RK_0$ est la relation aléthique, nous savons que $\Xlo\peut$ et $\Xlo\doit$ représentent forcément les modalités aléthiques.



\subsection{Validités et conséquence logique}
%'''''''''''''''''''''''''''''''''''''''''''''''
\label{ss:vclw}

La vérité d'une formule dépend d'un modèle, d'un monde et d'une assignation. 
En adaptant la
définition~\ref{d:Tarski} (p.~\pageref{d:Tarski}) du chapitre précédent, nous pouvons dire qu'une formule \vrb\phi\ est satisfaite par \Modele, $w$ et $g$ ssi \(\denote{\vrb\phi}^{\Modele,w,g}=1\), et nous noterons $\Modele,w,g\satisf\vrb\phi$ ou $\satisf_{\Modele,w,g}\vrb\phi$. %, ~\ref{ch:gn}
À partir de là, la logique modale nous permet de définir un certain nombre de notions de validités\is{validité}.


\begin{defi}[Validités dans {\LO} intensionnel]
Soit $\Modele=\tuple{\Unv A,\Unv W,\Unv R,\FI}$ un modèle intensionnel, $w$ un monde de \Unv W et $g$ une assignation.
\begin{enumerate}
\item \(\Modele, w\satisf \vrb\phi\) : \vrb\phi\ est valide par rapport à \Modele\ et $w$ ssi pour toute assignation $g$, on a \(\Modele, w,g\satisf \vrb\phi\).
\item \(\Modele, g\satisf \vrb\phi\) : \vrb\phi\ est valide par rapport à \Modele\ et $g$ ssi pour tout monde possible $w$ de \Unv W, on a \(\Modele, w,g\satisf \vrb\phi\).
\item \(\Modele\satisf \vrb\phi\) : \vrb\phi\ est valide par rapport à \Modele\ ssi pour tout monde possible $w$ de \Unv W et toute assignation $g$, on a \(\Modele, w,g\satisf \vrb\phi\).
\item \(\satisf \vrb\phi\) : \vrb\phi\ est valide  ssi pour tout modèle \Modele, on a \(\Modele\satisf \vrb\phi\).
\end{enumerate}
\end{defi}

\newpage

La validité 1 rappelle celle vue au chapitre \ref{ch:gn}, et elle dit simplement que \vrb\phi\ est vraie dans le monde $w$ quel que soit le contexte (\ie\ $g$).\is{contexte}
Ainsi, ici encore, la plupart des formules qui contiennent des variables libres (comme par exemple \(\Xlo x=\cns a\) ou \(\Xlo\prd{dormir}(x)\)) ne pourront pas être «valides-1».  Mais comme nous avons pris le parti de gérer l'interprétation des modalités au moyen de $g$, cette définition a également un impact sur les formules qui contiennent un opérateur modal.  Pour un monde $w$ donné, par exemple pour avoir \(\Modele,w\satisf\Xlo\peutn{n}\prd{dormir}(\cns a)\), il faudrait que $\Xlo\prd{dormir}(\cns a)$ soit possible dans $w$ pour \emph{toutes} les valeurs modales proposées par {\Unv R} ; cela dépend crucialement de $w$, mais si (comme il se doit) \Unv R contient une suffisamment grande variété de relations d'accessibilité, il n'est pas complètement impossible de trouver une assignation $g$ (\ie\ une relation $\RK_{g(n)}$)  qui rende $\Xlo\peutn{n}\prd{dormir}(\cns a)$ fausse dans $w$.  Il en va de même pour des formules plus générales, comme par exemple $\Xlo\doitn n\phi\implq\phi$.

La validité 2 est, elle, beaucoup plus forte. 
Elle dit que $\Xlo\phi$ est vraie dans \emph{tous} les mondes de \Unv W, pour une assignation donnée. 
Et du fait que nous avons décidé de mettre dans \Unv W tous les mondes possibles imaginables (car nous n'avions pas de raison d'en exclure), nous voyons que cette validité 2 exclut (\emph{presque}) toutes les vérités contingentes, c'est-à-dire les formules qui, intuitivement,  nous semblent avoir des valeurs de vérité variables.  
\emph{Presque}, car il y a encore quelques exceptions ; par exemple, si nous considérons l'assignation $g_1$ telle que $g_1(\vrb x)=\Obj{Alice}$, alors la formule \(\Xlo x=\cns a\) sera «valide-2» par rapport à $g_1$\footnote{Toujours sous l'hypothèse que les constantes sont des désignateurs rigides et que \cns a dénote \Obj{Alice} dans tous les mondes.}.
Cette formule peut traduire la phrase \sicut{C'est Alice}, qu'il est assez difficile de compter comme une tautologie standard. Pourtant, dans un contexte fixé où il est établi que le démonstratif \sicut{ce} désigne \Obj{Alice}, la phrase se retrouve nécessairement vraie.
En ce qui concerne les formules comprenant des opérateurs modaux, la validité 2 est utile pour mettre en évidence certaines propriétés logiques des différents types de modalité (car $g$ étant fixée, nous savons pour tout opérateur à quelle modalité il renvoie).  Par exemple si $R_{g(n)}$ est une relation d'accessibilité épistémique, alors quelle que soit \vrb\phi, nous aurons $\Modele,g\satisf \Xlo\doitn n\phi\implq\phi$, car si \vrb\phi\ est sue (\ie\ $\Xlo\doitn n\phi$) alors, par définition, \vrb\phi\ est vraie. 
En revanche, si $R_{g(n)}$ est une relation déontique, alors cette validité ne tient plus, car pour n'importe quelle \vrb\phi\footnote{À condition, bien sûr, que \vrb\phi\ ne soit pas contradictoire.}, on peut toujours trouver des mondes où \vrb\phi\ est obligatoire (\ie\ $\Xlo\doitn n\phi$) mais pas respectée.

Les validités 3 et 4 correspondent à la traditionnelle validité logique. 
%, celle qui définit les tautologies classiques.  
Selon ces définitions, ni \(\Xlo x=\cns a\) ni \(\Xlo\doitn n\phi\implq\phi\) ne sont valides. 
Évidemment, ce qui devrait nous préoccuper ici, c'est ce qui distingue la validité 3 et la validité 4. En théorie, elles sont nettement différentes (et la validité 4 est bien sûr la plus forte), mais en pratique, du fait des choix que nous avons fait pour constituer notre système, elles s'avèrent très proches.
D'après les définitions que nous avons posées, nous pourrions décrire  la validité 3 comme correspondant aux nécessités aléthiques \emph{par rapport à un modèle donné} et la validité 4 comme correspondant aux tautologies de la logique classique. 
\is{modalite@modalité!\elid\ aléthique}%
Cependant, formuler les choses ainsi peut paraître incohérent et prêter à confusion, précisément parce qu'en \S\ref{modalflavours1} nous avons défini les nécessités aléthiques comme identiques aux tautologies logiques. 
Nous allons donc prendre le temps d'examiner un peu plus précisément cet apparent paradoxe \alien{infra} (\S\ref{sss:pds}).



\smallskip

À l'instar de la validité, nous pouvons, en sémantique intensionnelle, définir plusieurs conséquences logiques.
Mais nous n'allons pas toutes les passer en revue, car il n'y en a qu'une qui est vraiment intéressante pour nous, et elle est donnée dans la définition \ref{d:conslogw}.


\begin{defi}[Conséquence logique]\label{d:conslogw}\is{consequence logique@conséquence logique}%
Soit \vrbi\phi1, \vrbi\phi2..., \vrbi\phi n et \vrb\psi\ $n+1$ formules de {\LO}. 
On dit que \vrb\psi\ est une conséquence logique de \vrbi\phi1, \vrbi\phi2..., \vrbi\phi n, \ie\ \(\vrbi\phi1, \vrbi\phi2..., \vrbi\phi n \satisf \vrb\psi\), ssi, pour tout modèle \Modele, tout monde $w$ et toute assignation $g$ tels que 
$\Modele,w,g\satisf\vrbi\phi1$,  
$\Modele,w,g\satisf\vrbi\phi2$,... et 
$\Modele,w,g\satisf\vrbi\phi n$,
on a aussi $\Modele,w,g\satisf\vrb\psi$.  
\end{defi}


Nous pouvons également définir l'équivalence logique,\is{equivalence@équivalence!\elid\ logique}
comme nous l'avons fait dans les chapitres précédents (à partir de $\vrb\phi\satisf\vrb\psi$ et $\vrb\psi\satisf\vrb\phi$, par exemple).  L'équivalence logique, comme la validité et la conséquence, concerne les formules. 
Or maintenant que nous somme passés à l'intensionnalité, nous pouvons définir une notion plus puissante, qui s'applique à toute expression de {\LO} et dont l'équivalence logique sera un cas particulier. Pour la distinguer, nous l'appellerons l'\emph{équivalence sémantique}, et elle correspond simplement à l'identité de sens, \ie\ d'intension.

\begin{defi}[Équivalence sémantique]\label{d:eqsemw}
Soit $\Modele$ un modèle intensionnel, et \vrb\alpha\ et \vrb\beta\ deux expressions bien formées de {\LO}.  On dit que \vrb\alpha\ et \vrb\beta\ sont sémantiquement équivalentes dans $\Modele$ ssi, pour toute assignation $g$, \(\denote{\vrb\alpha}^{\Modele,g}=\denote{\vrb\beta}^{\Modele,g}\).
\end{defi}

Pour que cette notion soit particulièrement profitable, il est utile de la poser en généralisant sur les assignations comme le fait la définition (afin notamment d'éviter que \vrb x soit parfois équivalente à \vrb y, ce qui peut être handicapant par endroits). 
Nous pouvons également la généraliser pour tout modèle \Modele, mais à l'arrivée cela ne fera guère de différence (cf. \S\ref{sss:pds}). 

L'équivalence sémantique nous permet de formuler un théorème simple et pratique qui concerne la substitution d'expressions. 
Nous avons vu, au début de ce chapitre, que le principe d'extensionnalité n'est pas respecté dans la langue : si l'on remplace, dans une phrase, une expression par une autre ayant la même dénotation (pour un monde donné donc), la dénotation de la phrase (dans ce même monde) n'est pas forcément préservée. Cela se produit notamment si la substitution est opérée dans la portée d'un opérateur modal ou de $\Xlo\Intn$.  Et donc le principe d'extensionnalité n'est pas non plus respecté dans {\LO} intensionnel, comme il se doit.
Mais maintenant nous pouvons envisager un pendant intensionnel de ce principe, et celui-ci est valide dans {\LO} :
si, dans une phrase, on remplace une expression par une autre ayant \emph{la même intension}, alors l'intension (et donc les extensions) de la phrase ne change pas.  C'est un théorème du système.

\begin{theo}\label{th:syno}
Si $\vrb\alpha$ est une
expression de {\LO}, $\vrb\beta$ une sous-expression de $\vrb\alpha$ et
$\vrb\gamma$  sémantiquement équivalente à $\vrb\beta$,  et
si $\vrb{\alpha'}$ est l'expression \vrb\alpha\ dans laquelle on a remplacé \vrb\beta\ par \vrb\gamma, alors \vrb\alpha\ est sémantiquement équivalente à $\vrb{\alpha'}$.
\end{theo}

Nous pourrions d'ailleurs appeler cela le principe de synonymie.\is{synonymie} 
C'est en effet ce qui qualifie théoriquement les synonymes : deux expressions interchangeables dans tout environnement sans altérer le sens du tout.  Évidemment nous savons bien que dans la langue, au niveau lexical, les substitutions ne sont vraiment possibles que dans certains contextes, mais rarement dans tous (à cause de la polysémie des mots) : la synonymie lexicale exacte n'existe quasiment pas. 
Mais au niveau d'expressions «supra-lexicales», \ie\ d'expressions complexes (des syntagmes ou des phrases), le principe peut s'appliquer raisonnablement et profitablement. 
L'équivalence sémantique n'est pas un mirage.

%% Notons que cette définition utilise la validité 1 vue ci-dessus, précisément pour pouvoir généraliser sur les modèles, les mondes et les assignations.

\subsection{Postulats de signification}
%'''''''''''''''''''''''''''''''''''''''''
\label{sss:pds}\is{postulat de signification|(}

Revenons à ce qui distingue la validité 3 et la validité 4.
En théorie, on peut concevoir une formule qui serait valide-3 (\ie\ toujours vraie dans un modèle donné) mais pas valide-4 (\ie\ telle qu'il existe au moins un modèle et un monde de ce modèle qui la rendent fausse).  
La validité 4 sous-entend donc une pluralité de modèles intensionnels différents. 
Or par hypothèse de travail, pour rendre notre système suffisamment performant, nous avons décidé que l'ensemble \Unv W d'un modèle intensionnel devait contenir \emph{tous} les mondes possibles imaginables (mais logiquement et linguistiquement cohérents) ; il n'y a donc qu'un seul ensemble \Unv W, et par conséquent tous les modèles intensionnels que nous voudrions envisager partagent le même \Unv W.  Et donc s'il existe des modèles intensionnels différents, ils se distinguent essentiellement par leur fonction d'interprétation $\FI$\footnote{Ils peuvent, certes, aussi se distinguer par leur ensemble \Unv R, mais \Unv R dépend aussi en partie du contexte et nous laisserons cela de côté dans la présente discussion, pour ne pas encore la compliquer.}.\is{fonction!\elid\ d'interprétation}
La fonction $\FI$ incorpore (et ainsi détermine) le sens\is{sens} des prédicats de {\LO}, et par conséquent le sens des mots de la langue que nous étudions (par exemple le français).  $\FI$ est donc constitutive de l'identité du système {\LO} qui lui-même se doit d'être le reflet de \emph{la} langue étudiée.
Si nous considérons que cette langue, en tant que système, possède \emph{sa} sémantique, alors nous devrons en conclure qu'il n'y a qu'une seule véritable fonction $\FI$. Ou pour être plus précis, parmi toutes les fonctions d'interprétation théoriquement possibles, il n'y en a qu'une seule valable et digne d'intérêt aux yeux du sémanticien : celle qui reflète correctement le sens des mots de la langue\footnote{Attention, qu'on ne se méprenne pas. Je ne suis pas en train de dire que tout mot de la langue possède un unique sens simple encodé par la «bonne» fonction $\FI$.  Nous savons bien que la plupart des mots sont plus ou moins hautement polysémiques ; mais $\FI$ interprète univoquement les \emph{prédicats} de {\LO}, pas directement les mots.  La polysémie lexicale est une problématique sémantique très importante et extrêmement complexe, qu'il n'est pas question de minimiser ; mais elle dépasse très largement la portée du présent ouvrage, et comme signalé plusieurs fois dans des pages précédentes, nous nous en tenons à la simplification qui consiste à traduire un mot polysémique par différents prédicats selon ses acceptions.

De même, je ne prétends pas que la langue française (par exemple) n'évolue pas et qu'elle ne possède pas de variété.  Il semble raisonnable de considérer que, lorsqu'elle devient très minutieuse et affinée, une étude sémantique ne devrait pas aborder «~\emph{le} français» (abstrait et variable) mais se consacrer à un dialecte particulier et suffisamment bien délimité. Mais là encore nous n'entrerons pas dans ce degré de finesse ici.}. 
Ainsi s'il n'y a qu'un seul \Unv W et une seule $\FI$ à prendre compte, il n'y a finalement qu'un seul modèle intensionnel à envisager en sémantique, et nous devrions donc nous arrêter à la validité 3 (avec ce modèle).
Nous pouvons ainsi éclaircir un peu le paradoxe apparent signalé plus haut, en reformulant les choses de la manière suivante : la validité 4 correspond aux nécessités aléthiques strictement logiques, et la nécessité 3 à ce que nous pourrions appeler les nécessités aléthiques sémantiques, ou pour faire plus simple et moins ambigu, les \emph{nécessités sémantiques}. \is{modalite@modalité!\elid\ aléthique}\is{modalite@modalité!\elid\ sémantique}

%***
Mais comment sommes-nous sûrs que nous manipulons \emph{le} modèle adéquat, c'est-à-dire \emph{la} fonction d'interprétation adéquate ? 
En réalité, nous ne nous en soucions pas vraiment, nous nous contentons de présupposer que tel est le cas, sans tellement examiner de près la fonction $\FI$.
Cependant, n'est-il pas de la responsabilité du sémanticien de spécifier les informations que contient $\FI$ pour faire en sorte qu'il ne s'agit pas de n'importe quelle fonction d'interprétation, mais bien de celle qui est le plus possible conforme à la réalité sémantique de la langue ? On peut assurément penser que oui. Et il se trouve que notre système intensionnel nous permet de contraindre $\FI$ et ainsi de la renseigner (au moins un peu) avec des faits de sémantique lexicale. \is{semantique@sémantique!\elid\ lexicale|sqq}%***

Pour nous sensibiliser à cela, considérons la phrase \Next[a] et sa traduction \Next[b] : 

\ex.
\a. Bob a tué Sam, et Sam n'est pas mort. \label{x:PS1a}
\b. \(\Xlo\prd{tuer}(\cns b,\cns s) \wedge \neg\prd{mort}(\cns s)\) \label{x:PS1b}


\Last[a] semble bien contradictoire. Mais d'un point de vue strictement logique, il n'y a aucun principe (aucune règle) qui permette de démontrer que \Last[b] l'est aussi. 
Parce que \(\Xlo\prd{tuer}(\cns b,\cns s)\) et  \(\Xlo\neg\prd{mort}(\cns s)\) sont deux formules atomiques formellement distinctes, et leur conjonction est donc forcément contingente (il suffit de dresser la table de vérité de \Last[b]). 
Autrement dit, la négation de \Last[b] n'est pas valide-4. 
La validité 4 ne s'occupe pas du sens des mots.  Mais nous sommes en sémantique et nous voudrions pouvoir dire que les négations de \ref{x:PS1a} et \ref{x:PS1b} sont valides.  Pour cela, nous avons la validité 3, du moment que le modèle que nous prenons en compte vérifie un principe qui dit que lorsque quelqu'un se fait tuer, alors il est mort.  Ce principe peut d'ailleurs se formuler très facilement dans {\LO}, c'est :

\newpage

\ex.
\(\Xlo\doit\forall x\forall y[\prd{tuer}(x,y)\implq \prd{mort}(y)]\)\footnote{Pour le coup, l'abandon de la temporalité est un peu préjudiciable pour cet exemple, il serait plus rigoureux de préciser ce principe en : \(\Xlo\doit\forall x\forall y[\prd{tuer}(x,y)\implq \mF\prd{mort}(y)]\), pour indiquer que la mort suit (immédiatement) l'acte de tuer.  Et si nous retenons des mondes où des gens peuvent ressusciter, la formule reste correcte, car pour ressusciter, il faut d'abord être mort (pour que la formule soit vraie à un instant $i$ il faut qu'il y ait au moins un instant ultérieur où \vrb y est mort ; rien n'empêche d'avoir un autre instant encore plus tardif où \vrb y est à nouveau vivant).}


\sloppy

Une formule comme \Last\ est ce que l'on appelle, en sémantique formelle, un \kw{postulat de signification}. 
Un postulat de signification a pour rôle de contraindre la fonction d'interprétation du modèle %en imposant des règles 
au moyen d'informations
sémantiques qui relèvent du lexique. 
Cela se présente typiquement sous la forme d'une %formule 
règle
{«forte»} : une nécessité aléthique ($\Xlo\doit$) avec quantification universelle et implication ou équivalence matérielle. %, afin d'exprimer des informations toujours et partout valides.  
Ces postulats ne sont pas des formules ordinaires que {\LO} serait simplement amené à étudier ; ils ont un statut un peu similaire à des axiomes en ce qu'ils sont \emph{posés} et \emph{imposés} comme valides dans le système (d'où le nom de \emph{postulat}). 
Ce statut leur est octroyé par le sémanticien, qui sait (ou décide, ou découvre, ou conjecture) que ces formules expriment une vérité sémantique.  
Techniquement, pour nous ici, ils peuvent s'appliquer à deux niveaux. 
Par rapport à la validité 3, étant valides-3, ils garantissent (grâce à $\Xlo\doit$) que le modèle est tel que tous ses mondes vérifient ce qui est encodé par les postulats.  Mais ils permettent également de récupérer la validité 4 si nous restreignons sa définition seulement aux modèles intensionnels qui satisfont tous les postulats de signification.  Cela permet d'adoucir notre position : nous n'avons plus à prétendre détenir \emph{le} modèle intensionnel conforme à la langue, nous pouvons travailler sur n'importe lequel, du moment qu'il vérifie l'ensemble de nos postulats.
Et ainsi la validité 3 et la validité 4 finissent factuellement par se confondre.

\fussy

L'expressivité de {\LO} étant ce qu'elle est présentement, nous ne pouvons pas formuler des postulats de signification qui expriment des propriétés génériques\footnote{Et d'ailleurs ça ne serait probablement pas souhaitable ; cf.\ la discussion \alien{infra}.} (\ie\ qui tolèrent des exceptions, comme \sicut{les oiseaux volent}). 
Nous devons %, pour l'instant, 
nous en tenir à des vérités solides et «indéfectibles» qui correspondent à des nécessités aléthiques.  Mais cela permet, par exemple, d'introduire dans le système des relations lexicales comme l'hyperonymie \Next[a], l'antonymie complémentaire \Next[b] ou l'antonymie non complémentaire (appelée aussi \emph{antonymie scalaire}, cf. le chapitre~\ref{Ch:adj}, vol.~2) \Next[c,d]. 

\ex.
\a. \(\Xlo\doit\forall x [\prd{oiseau}(x)\implq\prd{animal}(x)]\)
\b. %
\(\Xlo\doit\forall x\forall y [\prd{présent}(x,y)\ssi\neg\prd{absent}(x,y)]\)
\b. \(\Xlo\doit\forall x [\prd{solide}(x)\implq\neg\prd{fragile}(x)]\)\footnote{NB : ici, bien sûr,  le prédicat \prd{solide} représente le sens qui se rapproche de \sicut{résistant}, il ne désigne pas l'état de la matière qui s'oppose à \sicut{liquide} et \sicut{gazeux}.}
\b. \(\Xlo\doit\forall x [\prd{fragile}(x)\implq\neg\prd{solide}(x)]\)


\sloppy
Pris ensembles, les postulats \Last[c] et \Last[d] ne sont pas équivalents à \(\Xlo\doit\forall x [\prd{solide}(x)\ssi\neg\prd{fragile}(x)]\), car ils laissent la possibilité qu'il y ait des objets du modèle qui ne soient ni solides ni fragiles.  Nous pouvons, en revanche, les réunir dans le postulat
\(\Xlo\doit\forall x \neg[\prd{solide}(x)\wedge\prd{fragile}(x)]\).

\fussy

En \Last[b], \(\Xlo\prd{présent}(x,y)\) et \(\Xlo\prd{absent}(x,y)\)
signifient que \vrb x est, respectivement, physiquement présent et absent à l'endroit \vrb y. 
Mais notons que s'il s'agit de présence/absence physiques, il serait probablement prudent de restreindre la quantification de \Last[b]  aux seuls individus pertinents --~si l'on veut par exemple accepter que des entités immatérielles puissent être ni présentes ni absentes de \vrb y. Cela donnera :  
\(\Xlo\doit\forall x \forall y [[\prd{physique}(x) \wedge \prd{physique}(y)] \implq [\prd{présent}(x,y)\ssi\neg\prd{absent}(x,y)]]\).  
On pourrait même vouloir aller plus loin en précisant que dans ce postulat \vrb y doit être un lieu (en mettant donc \(\Xlo\prd{lieu}(y)\) dans sa restriction). Mais qu'est-ce qu'un lieu ? Tout objet physique ne peut-il pas être envisagé comme un possible espace d'accueil pour une localisation ?


Tout cela montre que les postulats de signification permettent de s'interroger profitablement sur le sens des prédicats et des mots, mais aussi qu'ils sont à établir avec vigilance. 
Car il y a une tentation à laquelle il faut résister : il ne faudrait surtout pas s'aventurer à y encoder des connaissances encyclopédiques. Les postulats de signification ne concernent que les faits linguistiques (\ie\ sémantiques). Les lois de la nature que nous connaissons, de la physique, la chimie, la biologie, etc.\ n'en font pas partie.  Ces lois sont certes valides dans notre monde et  dans les mondes qui y ressemblent suffisamment, mais pas dans \emph{tous} les mondes possibles.  Ce ne sont pas des nécessités aléthiques ou sémantiques ; elles relèvent plutôt des modalités dynamiques (ou, si l'on est scrupuleux, des modalités épistémiques). Et si l'on souhaite les insérer dans notre système, ce ne sera pas au moyen de $\Xlo\doit$ mais d'un autre opérateur de nécessité (un $\Xlo\doitn n$ habilement choisi). 

\largerpage[-1]

Malheureusement, dès que l'on tente d'approfondir la question, force est de constater que la frontière entre faits strictement sémantiques et faits encyclopédiques est souvent très ténue et insaisissable.
Un exemple typique est celui des adjectifs \sicut{mort} et \sicut{vivant}.  Ils semblent être d'honnêtes antonymes complémentaires, et on peut alors proposer le postulat \(\Xlo\doit\forall x [\prd{vivant}(x)\ssi\neg\prd{mort}(x)]\) (ou \(\Xlo\doit\forall x [[\prd{végétal}(x) \vee\prd{animal}(x)] \implq [\prd{vivant}(x)\ssi\neg\prd{mort}(x)]]\) pour être plus précis).  Mais que faire alors des mort-vivants ? Sont-ils morts ou vivants ? Ou (pire) les deux à la fois ? 
Cela peut paraître une question anecdotique, voire futile, mais elle a des implications non négligeables sur le système formel. Évidemment les mort-vivants n'existent pas dans notre monde, et il est donc impossible de répondre sérieusement (\ie\ scientifiquement) à la question. Cependant il ne s'agit pas ici de la trancher biologiquement, mais sémantiquement.  Car les mondes possibles (imaginaires) dans lesquels les mort-vivants existent sont linguistiquement cohérents (il n'y a pas de raison qu'ils ne le soient pas).
Nous sommes donc face au dilemme suivant. Soit les mort-vivants sont, dans «leurs» mondes, à la fois morts et vivants. Dans ce cas, nous devons abandonner le postulat d'antonymie ci-dessus et surtout trouver un moyen de définir les sens de \sicut{mort} et de \sicut{vivant} indépendamment l'un de l'autre \emph{et} indépendamment de considérations biologiques connues (ce qui n'est peut-être pas impossible, mais ce n'est pas simple). 
Soit les mort-vivants sont vraiment morts mais pas vivants ou, inversement, vraiment vivants mais pas morts.  Dans ce cas, nous devrons considérer que l'alliance des deux adjectifs correspond soit à un abus de langage soit (plus innocemment) à un emploi figuré de \sicut{mort} ou de \sicut{vivant}%
\footnote{Sans trop épiloguer sur le sujet, voici néanmoins quelques pistes possibles d'éclaircissement. On peut considérer que les morts-vivants sont vraiment morts mais pas vraiment vivants, qu'ils n'en ont que l'apparence (parce qu'ils ressemblent à des vivants, en déambulant, dévorant des cervelles, etc.). Alternativement, on peut considérer que les mort-vivants sont vraiment vivants, précisément parce qu'ils ne sont \emph{plus} morts. L'anglais, à cet égard, possède un terme : \sicut{undead}, littéralement «dé-mort», c'est-à-dire qu'ils sont revenus de la mort. Ils sont  donc alors à nouveau vivants ; c'est juste qu'ils ont ressuscité, mais... mal.

Ajoutons aussi que ce qui est en jeu ici n'est pas du tout de savoir par quelle règle morpho-lexicale le sens de \sicut{mort-vivant} s'obtient à partir de ceux de \sicut{mort} et \sicut{vivant} : exactement le même raisonnement peut être tenu sur le nom simple \sicut{zombie}, par exemple.}. 
Bref, cet exemple, au delà de son exotisme, montre que parfois il est assez difficile de définir le sens des mots en faisant complètement abstraction d'informations locales à notre monde, comme celles provenant des sciences de la nature. 
Et cela complique la tâche de la sémantique lexicale.

Pour autant, nous n'allons pas rejeter catégoriquement les postulats de signification de notre système, au contraire. Ceux-ci fondent ce que nous avons appelés les nécessités sémantiques (qui pour nous deviennent \alien{de facto} un cas particulier des nécessités aléthiques). Or les nécessités sémantiques existent et elles sont utiles, car elles permettent par exemple de tirer des inférences et de faire des raisonnements à l'aide de la conséquence logique de la définition~\ref{d:conslogw}\footnote{Elles permettent aussi d'expliquer pourquoi, par exemple, il est \emph{impossible} que je dessine l'autoportrait de quelqu'un d'autre.}. 
Illustrons cela avec un exemple qui s'inspire, de manière extrêmement simplifiée (dans la formalisation et dans l'esprit), de l'usage que faisait \citet{PTQ} des postulats de signification.\Andex{Montague, R.}

Il y a des prédicats (généralement verbaux) qui impliquent l'existence de leurs arguments. C'est le cas, par exemple, de ceux qui traduisent des verbes comme \sicut{mordre}, \sicut{boire}, \sicut{nettoyer}, \sicut{plier},  \sicut{dormir}, etc.
Il s'agit ici de l'existence\is{existence!prédicat d'\elid} au sens du prédicat \prd{exister} que nous avons introduit en \S\ref{sss:mondepossible} ; rappelons que \(\Xlo\prd{exister}(x)\) est vraie dans le monde $w$ ssi la dénotation de \vrb x existe de façon immanente voire tangible dans $w$.
Ces implications d'existence peuvent être formalisées par des postulats comme :

\ex. \label{post:existe}
\(\Xlo\doit\forall x\forall y [\prd{mordre}(x,y) \implq [\prd{exister}(x) \wedge \prd{exister}(y)]]\)

\Last\ dit que si \vrb x mord \vrb y dans $w$, c'est que forcément \vrb x et \vrb y existent dans $w$.  Les postulats fonctionnent comme des prémisses toujours disponibles pour un raisonnement. Ainsi à partir de \Next[a] nous pouvons déduire la conséquence logique \Next[b].  C'est-à-dire que s'il est vrai qu'un vampire a mordu Alice dans $w$, alors il ne s'agit pas d'un vampire fictif mais d'un vampire bien réel (dans $w$).

\ex. 
\a. \(\Xlo\exists x [\prd{vampire}(x)\wedge\prd{mordre}(x,\cns a)]\)
\b. \(\Xlo\exists x [\prd{vampire}(x)\wedge\prd{exister}(x)\wedge\prd{mordre}(x,\cns a)]\)


Nous voyons donc que, si jamais \prd{exister} a une utilité fondamentale dans l'interprétation de certaines phrases, ce prédicat n'apparaîtra pas mystérieusement dans une formule : il arrive «automatiquement» avec le prédicat verbal de la phrase.  Et grâce à  \ref{post:existe}, la formule \Last[a] suffit à représenter le sens qui est détaillé dans \Last[b].

\newpage

Des postulats comme \ref{post:existe} peuvent paraître à première vue un peu banals, mais en fait ils sont loin de l'être, dès que nous constatons qu'il y a des prédicats qui eux ne provoquent pas ce type d'implications. Par  exemple, ceux qui traduisent les verbes 
\sicut{aimer}, \sicut{rêver de}, \sicut{chercher}, \sicut{ressembler à}, etc.\ n'impliquent pas nécessairement l'existence de leur deuxième argument. 
Il est même probable que certains (comme \sicut{ressembler}) n'impliquent pas non plus l'existence de leur premier argument.
Ainsi \Next[b] n'est pas une conséquence de \Next[a].

\ex.
\a. \(\Xlo\exists x [\prd{licorne}(x)\wedge\prd{chercher}(\cnsi a1,x)]\)
\b. \(\Xlo\exists x [\prd{licorne}(x)\wedge\prd{exister}(x)\wedge\prd{chercher}(\cnsi a1,x)]\)

Notons cependant que \Last[a] ne suffit pas à rendre compte d'une lecture \alien{de dicto}. %, pas plus que \Last[b] ne suffit entièrement à rendre une lecture \alien{de re}. 
Quant à \Last[b], elle en dit plus que ce qu'est une lecture \alien{de re} (cf. note \ref{fn:dere-exister}, p.~\pageref{fn:dere-exister}).
À cet égard, la stratégie de \citet{PTQ}, bien plus sophistiquée, est plus appropriée.


\is{postulat de signification|)}

\subsection{Sens, propositions et conditions de vérité}
%''''''''''''''''''''''''''''''''''''''''''''''''''''''''
%{Problème de l'hyperintensionnalité}
\is{proposition}\is{conditions de vérité}
\is{sens}

Concluons cette (longue) conclusion en faisant d'abord un petit point notionnel.
Dans ce chapitre, nous avons formellement défini le sens (\ie\ l'intension) d'une phrase (déclarative) comme étant une \emph{proposition}\is{proposition}, c'est-à-dire une fonction de \Unv W dans \set{0;1} (ou un ensemble de mondes possibles).  Auparavant nous manipulions le sens d'une phrase comme étant ses \emph{conditions de vérité}\is{conditions de vérité}.  Est-ce à dire que nous avons changé de définition de ce qu'est le sens d'une phrase ? Non, bien sûr.    
Une proposition est une fonction qui pour tout monde $w$ sait dire si la phrase dont elle est le sens est vraie ou fausse dans ce monde. Comment le sait-elle ? Précisément parce qu'elle «connaît» les conditions de vérité de la phrase, et cela lui suffit pour qu'elle fasse son travail (\ie\ retourner $1$ ou $0$ quand il le faut). 
Autrement dit, une proposition n'est autre qu'un objet qui représente et formalise des conditions de vérité. En bref, une proposition \emph{est} un ensemble de conditions de vérité, et réciproquement.

Cependant, la notion de proposition fait apparaître un problème qui généralement est moins visible lorsque l'on travaille sur des conditions de vérité (parce que celles-ci sont un peu moins formelles). 
Pour nous y sensibiliser, considérons les phrases de \Next.
Quelles sont leurs intensions ?

\ex.
\a. Tous les éléphants sont des éléphants.
\b. Alice a le même âge qu'elle-même.
\b. Melbourne est la capitale de l'Australie, ou pas.


Nous sommes à présent suffisamment habitués pour reconnaître que ces trois phrases sont des tautologies (au sens le plus fort du terme, elles sont valides 4), elles sont toujours vraies. 
Donc, par définition, l'intension de chacune d'elle est la fonction constante qui renvoie toujours $1$, jamais $0$. Et en termes ensemblistes, c'est l'ensemble \Unv W complet. Par conséquent, ces trois phrases ont exactement la même intension, elles ont exactement le même sens...
C'est évidemment là que le paradoxe surgit : ces phrases ont beau être logiquement (et donc sémantiquement) équivalentes, elles ne nous frappent pas comme étant synonymes ou des paraphrases les unes des autres. Ne serait-ce que parce qu'elles ne racontent pas du tout les mêmes choses.
Mais notre système ne nous laisse pas le choix ; c'est un de ses théorèmes : toutes les tautologies du langage sont sémantiquement équivalentes entre elles, ce qui implique nécessairement qu'elles ont le même sens, de même que toutes les contradictions ont le même sens (parce que leur intension est l'ensemble vide).

Ce problème a été depuis longtemps repéré par les philosophes et a donné lieu à l'appellation d'\kw{hyperintensionnalité}.  Cette notion renvoie à l'idée qu'il y a quelque chose qui, dans notre compréhension de phrases, nous permet de les distinguer et qui se situe au delà de leur intension. 
Et cela devrait nous amener à la conclusion que notre définition formelle du sens, par intension, est insuffisante voire erronée.  Cependant, il convient d'être particulièrement rigoureux et méthodique si l'on souhaite se lancer sur la piste de cette critique.

Il peut être, au moins {a priori}, parfaitement légitime de considérer que l'hyperintensionnalité n'est pas un problème fondamental pour un système tel que notre {\LO} intensionnel.  
En effet notre système fait la prédiction que les trois phrases de \Last\ ont le même sens ; ce résultat nous dérange, mais tant qu'il ne produit pas une incohérence interne au système (c'est-à-dire que ce résultat ne vient pas en contradiction avec d'autres prédictions du système), ce n'est pas si dramatique. 
Cela voudra dire que notre système est incomplet, qu'il faudrait lui \emph{ajouter} quelque chose, mais pas qu'il est en soi défaillant.  Et alors, en attendant de le compléter, nous devrons nous contenter d'admettre que : eh bien oui, c'est bizarre, les phrases de \Last\ ont le même sens, mais c'est comme ça, étant donné la définition du sens.  
Après tout notre système intensionnel prédit correctement que toutes les tautologies sont \emph{vides de sens} dans la mesure où elles n'apportent en soi aucune information dans un discours ou une conversation.

En revanche, l'hyperintensionnalité posera un problème préoccupant si nous constatons que le système fait des prédictions qui brisent sa cohérence interne. 
Et c'est dans cette direction qu'il faut orienter la critique. 
Notre système n'est pas nécessairement tenu de prédire une différence entre les phrases \Last, par contre il est tenu de prédire le calcul des valeurs de vérité correctes des phrases dans n'importe quel monde. 
Une mauvaise prédiction serait donc cela : une erreur de calcul de dénotation. 
Et pour en trouver (s'il y en a), nous devons revenir au principe de substitution du théorème \ref{th:syno} (p.~\pageref{th:syno}).  Il dit que nous pouvons interchanger des sous-expressions de même intension sans changer l'intension globale de la phrase. 
Pouvons-nous donc opérer des substitutions qui violent ce principe ? Là encore il faut regarder du côté des verbes d'attitudes propositionnelles :

\ex.
\a. Jean pense que tous les éléphants sont des éléphants.
\b. Jean pense qu'Alice a le même âge qu'elle-même.


Si \LLast[a] a la même intension que \LLast[b], alors \Last[a] doit avoir la même intension que \Last[b], en vertu du principe de substitution. Autrement dit, si Jean croit \LLast[a], il doit forcément croire aussi \LLast[b], et même  toutes les autres tautologies du langage. Mais est-ce le cas ? 
Intuitivement non, même si c'est une question qui n'est pas si évidente à trancher. Car on pourrait objecter que même si les gens pensent ce qu'ils veulent (malheureusement), s'ils pensent, c'est qu'ils sont un minimum rationnels et que donc, sans forcément en être conscient, ils penseront toutes les tautologies possibles. Mais c'est là faire de tout individu pensant un parfait raisonneur ou un logicien accompli, ce qui est une hypothèse un peu audacieuse. 
Et cela impliquerait aussi qu'aucun individu ne pourrait avoir de pensées contradictoires, ce qui semble empiriquement démenti quotidiennement... 
Et de toute manière, si dans \Last\ nous remplaçons \sicut{penser} par \sicut{dire}, l'objection ne tient plus, car rien n'oblige à dire ce que l'on pense ni à dire des choses logiques. 

Par conséquent, \Last\ montre qu'il existe dans la langue ce que l'on appelle des contextes hyperintensionnels, c'est-à-dire sensibles à l'hyperintensionnalité, en l'occurrence des verbes d'attitudes propositionnelles (il y en a peut-être d'autres).  Cela prouve que notre système fait des erreurs. 
Il faudrait donc idéalement le corriger ; pour cela il y a deux objectifs à poursuivre : d'abord formaliser précisément ce qui serait «l'hyperintension» d'une expression, et ensuite redéfinir l'interprétation des verbes d'attitudes propositionnelles en posant que leur argument n'est pas l'intension d'une phrase mais son hyperintension.  Ce n'est pas une tâche triviale, loin de là, mais à l'intérieur de \LO, la correction à apporter ne concerne que l'analyse des verbes propositionnels. Notre système n'est peut-être pas entièrement contaminé. 

Pour cette raison, nous ne nous lancerons pas ici dans une tentative de résoudre ce problème ; nous allons conserver {\LO} tel qu'il est, en restant conscients (et contrits\footnote{Mais modérément.}) de son imperfection. Une autre raison est que le traitement formel de l'hyperintensionnalité est singulièrement complexe, et il existe encore aujourd'hui assez peu de consensus sur les analyses à adopter.  Je vais cependant, pour terminer, évoquer brièvement une stratégie séduisante proposée par \citet{Lewis:70}\Andex{Lewis, D.}. D.~Lewis distingue la notion d'intension de la notion de \emph{signification} (\alien{meaning}, dans le texte), cette dernière correspondant à ce que nous appelons ici l'hyperintension.  
Il définit la signification d'une expression comme une structure (techniquement un arbre) qui contient non seulement l'intension de l'expression mais aussi les intensions de tous ses constituants. Autrement dit, la signification contient, en quelque sorte, l'historique de la construction progressive du sens de l'expression.  Et c'est bien ce qui distingue les phrases de {\LLast} : elles sont équivalentes mais leurs sens n'ont pas été obtenus de la même façon. 
Nous voyons que cette stratégie repose crucialement sur le principe de compositionnalité qui sous-tend le mécanisme de construction du sens à partir (entre autres) du sens des parties.  Et cela nous donne une transition toute trouvée pour passer au chapitre suivant.




%\fussy
