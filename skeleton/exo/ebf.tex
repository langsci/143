% -*- coding: utf-8 -*-
\begin{exo}\label{e:LCPebf}
Parmi les séquences suivantes, lesquelles sont des formules correctes
de {\LO}? \pagesolution{crg:LCPebf}
\begin{exolist}
\item \(\Xlo\exists z\, [[\prd{connaître}(\cnsi{r}2,z) \wedge
\prd{gentil}(z)] \wedge \neg\prd{dormir}(z)]\)
\item \(\Xlo\exists x \forall y \exists z\, [\prd{aimer}(y,z) \vee
\prd{aimer}(z,x)]\)
\item \(\Xlo\forall x y\, \prd{aimer}(x,y)\)
\item \(\Xlo\neg\neg \prd{aimer}(\cnsi{r}1,\cns{m})\)
\item \(\Xlo\exists x \neg\exists z\, \prd{connaître}(x,z)\)
\item \(\Xlo\exists x [\prd{acteur}(x) \wedge \prd{dormir}(x) \vee
  \prd{avoir-faim}(x)] \)
\item \(\Xlo[\prd{aimer}(\cns{a},\cns{b}) \implq \neg\prd{aimer}(\cns{a},\cns{b})]\)
\item \(\Xlo\exists y [\prd{acteur}(x) \wedge \prd{dormir}(x)]\)
\end{exolist}
%'''''''''''
\begin{solu} (p.~\pageref{e:LCPebf}) \label{crg:LCPebf}
%Formules bien formées de {\LO}.

L'exercice consiste à produire l'arbre de construction (cf. p.~\pageref{f:Axfbf}) de chaque séquence en appliquant les règles syntaxiques de la définition \ref{SynP}, p.~\pageref{SynP}.

\begin{exolist}
\item \(\Xlo\exists z\, [[\prd{connaître}(\cnsi{r}2,z) \wedge
\prd{gentil}(z)] \wedge \neg\prd{dormir}(z)]\)

Cette formule est bien formée, on le montre à l'aide de son arbre de
construction:

\begin{center}
{\small
\leaf{\(\prd{connaître}\)}
\leaf{\(\cnsi{r}2\)}
\leaf{\(\vrb z\)}
\branch{3}{\(\Xlo\prd{connaître}(\cnsi{r}2,z)\)}
\leaf{\(\prd{gentil}\)}
\leaf{\(\vrb z\)}
\branch{2}{\(\Xlo\prd{gentil}(z)\)}
\branch{2}{\(\Xlo[\prd{connaître}(\cnsi{r}2,z) \wedge \prd{gentil}(z)]\)}
\leaf{\(\prd{dormir}\)}
\leaf{\(\Xlo z\)}
\branch{2}{\(\Xlo\prd{dormir}(z)\)}
\branch{1}{\(\Xlo\neg\prd{dormir}(z)\)}
\branch{2}{\(\Xlo[[\prd{connaître}(\cnsi{r}2,z) \wedge \prd{gentil}(z)] \wedge \neg\prd{dormir}(z)]\)}
\branch{1}{\(\Xlo\exists z\, [[\prd{connaître}(\cnsi{r}2,z) \wedge
      \prd{gentil}(z)] \wedge \neg\prd{dormir}(z)]\)}
\qobitree}
\end{center}

\item \(\Xlo\exists x \forall y \exists z\, [\prd{aimer}(y,z) \vee \prd{aimer}(z,x)]\)

Cette formule est bien formée:

\begin{center}
{\small
\leaf{\(\prd{aimer}\)}
\leaf{\(\Xlo y\)}
\leaf{\(\Xlo z\)}
\branch{3}{\(\Xlo\prd{aimer}(y,z)\)}
\leaf{\(\prd{aimer}\)}
\leaf{\(\Xlo z\)}
\leaf{\(\Xlo x\)}
\branch{3}{\(\Xlo\prd{aimer}(z,x)\)}
\branch{2}{\(\Xlo[\prd{aimer}(y,z) \vee \prd{aimer}(z,x)]\)}
\branch{1}{\(\Xlo\exists z\, [\prd{aimer}(y,z) \vee \prd{aimer}(z,x)]\)}
\branch{1}{\(\Xlo\forall y \exists z\, [\prd{aimer}(y,z) \vee \prd{aimer}(z,x)]\)}
\branch{1}{\(\Xlo\exists x \forall y \exists z\, [\prd{aimer}(y,z) \vee \prd{aimer}(z,x)]\)}
\qobitree}
\end{center}

\item \(\Xlo\forall x y\, \prd{aimer}(x,y)\)

Cette séquence  n'est pas une formule bien formée.  En effet $\Xlo\forall x$
peut être introduit par la règle (\RSyn\ref{SynPQ}), mais cette règle
doit opérer sur une \emph{formule}.  Or \(\Xlo\forall x y\,
\prd{aimer}(x,y)\) se décomposerait alors en \(\Xlo\forall x\) et \(\Xlo y\,
\prd{aimer}(x,y)\) et cette seconde expression n'est pas une formule
bien formée; on ne peut pas construire \(\Xlo y\, \prd{aimer}(x,y)\),
aucune règle n'autorise à placer une variable seule devant une
expression. 

\item \(\Xlo\neg\neg \prd{aimer}(\cnsi{r}1,\cns{m})\)

Cette formule est bien formée:

\begin{center}
{\small
\leaf{\(\prd{aimer}\)}
\leaf{\(\cnsi{r}1\)}
\leaf{\(\cns{m}\)}
\branch{3}{\(\Xlo\prd{aimer}(\cnsi{r}1,\cns{m})\)}
\branch{1}{\(\Xlo\neg \prd{aimer}(\cnsi{r}1,\cns{m})\)}
\branch{1}{\(\Xlo\neg\neg \prd{aimer}(\cnsi{r}1,\cns{m})\)}
\qobitree}
\end{center}


\item \(\Xlo\exists x \neg\exists z\, \prd{connaître}(x,z)\)

Cette formule est bien formée:

\begin{center}
{\small
\leaf{\(\prd{connaître}\)}
\leaf{\(\Xlo x\)}
\leaf{\(\Xlo z\)}
\branch{3}{\(\Xlo\prd{connaître}(x,z)\)}
\branch{1}{\(\Xlo\exists z\, \prd{connaître}(x,z)\)}
\branch{1}{\(\Xlo\neg\exists z\, \prd{connaître}(x,z)\)}
\branch{1}{\(\Xlo\exists x \neg\exists z\, \prd{connaître}(x,z)\)}
\qobitree}
\end{center}


\item \(\Xlo\exists x [\prd{acteur}(x) \wedge \prd{dormir}(x) \vee
  \prd{avoir-faim}(x)] \)

Cette expression n'est pas une formule bien formée car il manque une
paire de crochets dans \(\Xlo[\prd{acteur}(x) \wedge \prd{dormir}(x) \vee
  \prd{avoir-faim}(x)]\).  En effet les connecteurs comme $\Xlo\wedge$ et
$\Xlo\vee$ sont introduits par les règles (\RSyn\ref{SynPConn}) et ces
règles introduisent en même temps une paire de crochets pour chaque connecteur.

\item \(\Xlo[\prd{aimer}(\cns{a},\cns{b}) \implq \neg\prd{aimer}(\cns{a},\cns{b})]\)

Cette formule est bien formée:

\begin{center}
{\small
\leaf{\(\prd{aimer}\)}
\leaf{\(\cns{a}\)}
\leaf{\(\cns{b}\)}
\branch{3}{\(\Xlo\prd{aimer}(\cns{a},\cns{b})\)}
\leaf{\(\Xlo\prd{aimer}\)}
\leaf{\(\Xlo\cns{a}\)}
\leaf{\(\Xlo\cns{b}\)}
\branch{3}{\(\Xlo\prd{aimer}(\cns{a},\cns{b})\)}
\branch{1}{\(\Xlo\neg\prd{aimer}(\cns{a},\cns{b})\)}
\branch{2}{\(\Xlo[\prd{aimer}(\cns{a},\cns{b}) \implq \neg\prd{aimer}(\cns{a},\cns{b})]\)}
\qobitree}
\end{center}

\item \(\Xlo\exists y [\prd{acteur}(x) \wedge \prd{dormir}(x)]\)

Cette formule est bien formée:

\begin{center}
{\small
\leaf{\(\prd{acteur}\)}
\leaf{\(\Xlo x\)}
\branch{2}{\(\Xlo\prd{acteur}(x)\)}
\leaf{\(\prd{dormir}\)}
\leaf{\(\Xlo x\)}
\branch{2}{\(\Xlo\prd{dormir}(x)\)}
\branch{2}{\(\Xlo[\prd{acteur}(x) \wedge \prd{dormir}(x)]\)}
\branch{1}{\(\Xlo\exists y [\prd{acteur}(x) \wedge \prd{dormir}(x)]\)}
\qobitree}
\end{center}

Remarque: la variable $\vrb y$ <<~utilisée~>> par le quantificateur
$\Xlo\exists$ ne réapparaît pas dans la (sous-)formule qui suit, mais cela
n'empêche pas l'expression d'être une formule bien formée, comme le
permet la règle (\RSyn\ref{SynPQ}).

\end{exolist}
\end{solu}
%'''''''''
\end{exo}
