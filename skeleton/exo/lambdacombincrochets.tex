% -*- coding: utf-8 -*-
\begin{exo}[Désambiguïsation d'une expression sans crochet]\label{lcombin[]}
En partant de l'expression (mal formée) \(\Xlo\lambda x\lambda y
\alpha (\beta)(\gamma)\), 
\pagesolution{lcombin[]solu}
et en considérant que $\Xlo\beta$ et $\Xlo\gamma$ y
sont des arguments fournis par application fonctionnelle, donnez
toutes les manières possibles de réécrire correctement l'expression en
suivant notre régle d'écriture de la définition~\ref{def:f@}.  Vous
supposerez notamment que $\Xlo\alpha$ en soi peut dénoter une fonction (de
diverse arité).
\begin{solu}\label{lcombin[]solu}
(p.~\pageref{lcombin[]})\largerpage

Pour corriger \(\Xlo\lambda x\lambda y \alpha (\beta)(\gamma)\) nous
devons faire des hypothèses successives choisissant quelle fonction s'applique
aux arguments $\Xlo\beta$ et $\Xlo\gamma$, et ensuite essayer de réaliser ces hypothèses via la règle de l'application fonctionnelle (définition \ref{def:f@}, p.~\pageref{def:f@}), qui ajoutera des crochets aux endroits appropriés.
\begin{enumerate}
\item $\Xlo\beta$ est appliqué à la fonction $\Xlo\alpha$.
Nous devrons alors écrire 
\(\Xlo\lambda x\lambda y [\alpha (\beta)](\gamma)\).
\\  
Mais il reste encore
à donner $\Xlo\gamma$ à une fonction :
  \begin{enumerate}
    \item $\Xlo\gamma$ est appliqué à la fonction $\Xlo[\alpha (\beta)]$ (car
    cela peut encore être une fonction, si $\Xlo\alpha$ est une fonction à
    plusieurs arguments -- au moins deux).  Nous écrirons alors :
\(\Xlo\lambda x\lambda y [[\alpha (\beta)](\gamma)]\).
    \item $\Xlo\gamma$ est appliqué à la fonction $\Xlo\lambda y [\alpha
    (\beta)]$ (c'est évidemment une fonction, à cause de $\Xlo\lambda
    y$).  Nous écrirons : 
    \(\Xlo\lambda x[\lambda y [\alpha (\beta)](\gamma)]\).
    \item $\Xlo\gamma$ est appliqué à la fonction $\Xlo\lambda x \lambda y [\alpha(\beta)]$.  Nous écrirons :
\(\Xlo[\lambda x\lambda y [\alpha (\beta)](\gamma)]\).
  \end{enumerate}
\item $\Xlo\beta$ est appliqué à la fonction $\Xlo\lambda y \alpha$.  La
  première correction à faire est donc : 
\(\Xlo\lambda x[\lambda y \alpha (\beta)](\gamma)\).  
Ensuite :
  \begin{enumerate}
    \item $\Xlo\gamma$ est appliqué à la fonction $\Xlo[\lambda y \alpha
    (\beta)]$ (c'est possible si on suppose que $\Xlo\alpha$ en soit est
    une fonction à au moins deux arguments).  Nous écrirons :
\(\Xlo\lambda x[[\lambda y \alpha (\beta)](\gamma)]\).
    \item $\Xlo\gamma$ est appliqué à la fonction 
$\Xlo\lambda x[\lambda y \alpha(\beta)]$.  Nous écrirons :
\(\Xlo[\lambda x[\lambda y \alpha (\beta)](\gamma)]\).
  \end{enumerate}
\item $\Xlo\beta$ est appliqué à la fonction $\Xlo\lambda x \lambda y \alpha$.
Cela donne d'abord : 
\(\Xlo[\lambda x\lambda y \alpha (\beta)](\gamma)\).  
Puis :  
  \begin{enumerate}
    \item $\Xlo\gamma$ est appliqué à la fonction $\Xlo[\lambda x \lambda y \alpha (\beta)]$.  Nous écrirons :
\(\Xlo[[\lambda x\lambda y \alpha (\beta)](\gamma)]\).  Ici c'est la seule
    correction possible.
  \end{enumerate}
\end{enumerate}
Pour récapituler, nous obtenons six combinaisons, toutes
différentes :

\(\Xlo\begin{array}[t]{@{}l@{\qquad}l@{\qquad}l}
\lambda x\lambda y [[\alpha (\beta)](\gamma)]&
\lambda x[\lambda y [\alpha (\beta)](\gamma)]&
{}[\lambda x\lambda y [\alpha (\beta)](\gamma)]\\
\lambda x[[\lambda y \alpha (\beta)](\gamma)]&
{}[\lambda x[\lambda y \alpha (\beta)](\gamma)]&
{}[[\lambda x\lambda y \alpha (\beta)](\gamma)]
\end{array}\)
\largerpage\end{solu}
\end{exo}
