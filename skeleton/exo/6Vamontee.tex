% -*- coding: utf-8 -*-
\begin{exo}\label{exo:6VaM}
Une analyse possible des verbes dits à montée,\is{verbe!\elid\ a montee@\elid\ à montée} 
\pagesolution{crg:6VaM}%
comme \sicut{sembler}, est que leur sujet de surface est originairement celui du verbe infinitif enchâssé et que de là il monte, peut-être en plusieurs étapes, jusqu'à la position Spec du TP principal :
\begin{enumerate}
\item {} [\Stag{TP} Alice$_1$ semble [\Stag{TP} avoir $t_1$ dormi]].
\end{enumerate}
À partir de cette hypothèse et de l'hypothèse que \sicut{sembler} est de type \type{\type{s,t},t}, ne prenant ainsi qu'un seul argument, une proposition, 
donnez la composition sémantique complète de la phrase ci-dessus.
\begin{solu}(p.~\pageref{exo:6VaM})\label{crg:6VaM}
%[\Stag{TP} Alice$_1$ semble [\Stag{VP} $t_1$ dormir]].

Nous allons supposer l'analyse syntaxique donnée en figure \ref{f:VaMontée} où \sicut{Alice} monte de la position de sujet de \sicut{dormir} (Spec de VP) jusqu'à la position de sujet de la phrase (Spec de TP) en trois étapes successives%
\footnote{Peu importe ici que cette analyse soit ou non parfaitement correcte sur le plan syntaxique ; l'objectif de l'exercice est de nous faire manipuler des traces d'un même constituant déplacé plusieurs fois.}.

\begin{figure}[h]
\begin{center}
{\small
\Tree
[.TP
  [.DP$_1$ \rnode{a}{Alice} ]
  [.T$'$ 
    [.VP \rnode{t13}{$t_1$}
      [.V$'$ 
        [.V semble ] 
        [.TP 
          \rnode{t12}{$t_1$}
          [.T$'$ 
            [.T avoir ]
            [.VP 
              \rnode{t11}{$t_1$}
              [.V$'$ dormi ] 
            ]
          ]
        ]
      ]
    ]
  ]
]
}%
\ncbar[angle=-90,linecolor=darkgray,nodesep=2pt,offsetB=-1.5pt]{->}{t11}{t12}%
\ncbar[angle=-90,linecolor=darkgray,nodesep=2pt,offset=-1.5pt]{->}{t12}{t13}%
\ncbar[angle=-90,linecolor=darkgray,nodesep=2pt,offsetA=-1.5pt]{->}{t13}{a}
\end{center}
\caption{Hypothèse syntaxique pour \sicut{Alice semble avoir dormi}}\label{f:VaMontée}
\end{figure}

%\bigskip

Les traces $t_1$ marquent des positions anciennement occupées par [\Stag{DP} Alice]$_1$, c'est pourquoi elles ont le même indice et se traduiront donc de la même manière par $\Xlo\lambda P[P(x_1)]$.  

\begin{enumerate}
\item {} [\Stag{VP} $t_1$ dormi] $\leadsto$
\(\Xlo[\lambda P[P(x_1)](\lambda x\,\prd{dormir}(x))]\)\\
\(=\Xlo\prd{dormir}(x_1)\)
\hfill{\small (\breduc s sur \vrb P puis \vrb x)}

\item Là encore, nous considérons provisoirement que l'auxiliaire n'a pas de contribution sémantique et donc que T$'$ se traduit comme VP, de type \typ t.
Nous allons donc ensuite devoir appliquer la règle de montée du sujet (cf. \ref{ri:MontSuj} p.~\pageref{ri:MontSuj}) qui ajoute $\Xlo\lambda x_1$, ce qui fait que la deuxième trace $t_1$ est traitée comme un constituant déplacé.

\item {} [\Stag{TP} $t_1$ avoir $t_1$ dormi] $\leadsto$
\(\Xlo[\lambda P[P(x_1)](\lambda x_1\,\prd{dormir}(x_1))]\)
\hfill{\small (montée du sujet)}\\
\(=\Xlo[\lambda x_1\,\prd{dormir}(x_1)(x_1)]\)
\hfill{\small (\breduc\ sur \vrb P)}\\
\(=\Xlo\prd{dormir}(x_1)\)\footnote{Notons qu'ici il n'a pas été nécessaire de renommer les occurrences liées de \vrbi x1 avant d'effectuer la \breduc\ puisqu'à l'arrivée la variable argument \vrbi x1 reste libre.}
\hfill{\small (\breduc\ sur \vrbi x1)}

\item \(\sicut{semble}\leadsto \Xlo\lambda p\,\prd{sembler}(p)\), avec $\vrb p\in\VAR_{\type{s,t}}$.  Comme \prd{sembler} est de type \type{\type{s,t},t}\footnote{Le prédicat \prd{sembler} a une sémantique modale qui se rapproche de celle des verbes d'attitude pro\-po\-si\-tion\-nel\-le.% ; il peut d'ailleurs être raisonnable de généraliser en ajoutant un argument  
}, il va falloir utiliser l'application fonctionnelle intensionnelle (définition \ref{d:AFI}, p.~\pageref{d:AFI}).

\item {} [\Stag{V$'$} semble $t_1$ avoir $t_1$ dormi] $\leadsto$
\(\Xlo[\lambda p\,\prd{sembler}(p)(\Intn\prd{dormir}(x_1))]\)\footnote{NB : ici $\Xlo\Intn\prd{dormir}(x_1)$ est en fait la simplification (un peu abusive) de $\Xlo\Intn[\prd{dormir}(x_1)]$ (mais sachant que l'expression est bien formée, il n'y a pas de risque de confondre avec $\Xlo[\Intn\prd{dormir}(x_1)]$).}
\hfill{\small (AFI)}\\
\(=\Xlo\prd{sembler}(\Intn\prd{dormir}(x_1))\)
\hfill{\small (\breduc\ sur \vrb p)}

\item Ici, pour combiner la troisième trace $t_1$ avec V$'$ de type \typ t, nous devons encore une fois ajouter l'abstraction $\Xlo\lambda x_1$, tout en sachant qu'il ne s'agit pas là d'une instance de la règle \ref{ri:MontSuj}, mais d'une règle qui reconnaît que le sujet de \prd{dormir} a quitté la subordonnée pour venir occuper (provisoirement) la position sujet de \sicut{sembler} :\\
{} [\Stag{VP} $t_1$ semble $t_1$ avoir $t_1$ dormi] $\leadsto$\\
\(\Xlo[\lambda P[P(x_1)](\lambda x_1\,\prd{sembler}(\Intn\prd{dormir}(x_1)))]\)
\hfill{\small (ajout de $\Xlo\lambda x_1$)}\\
\(=\Xlo\prd{sembler}(\Intn\prd{dormir}(x_1))\)
\hfill{\small (\breduc s sur \vrb P puis \vrbi x1)}

\item {} [\Stag{TP} Alice $t_1$ semble $t_1$ avoir $t_1$ dormi] $\leadsto$\\
\(\Xlo[\lambda P[P(\cns a)](\lambda x_1\,\prd{sembler}(\Intn\prd{dormir}(x_1)))]\)
\hfill{\small (montée du sujet, règle \ref{ri:MontSuj})}\\
\(=\Xlo\prd{sembler}(\Intn\prd{dormir}(\cns a))\)
\hfill{\small (\breduc s sur \vrb P puis \vrbi x1)}
\end{enumerate}

Remarque : Cet exercice, et principalement l'étape 6 de la dérivation, nous permet de constater que les traces intermédiaires doivent se comporter à la fois comme des traces ordinaires mais aussi comme si elles étaient elles-mêmes des éléments déplacés.  Ce genre de traitement ne va pas entièrement de soi dans la formalisation précise de l'interface syntaxe sémantique (même si l'exercice montre qu'il est réalisable) ; cependant il s'intègre très simplement dans la variante formelle (dite «HK») présentée en \S\ref{sss:VarianteMvt}.

\end{solu}
\end{exo}
