% -*- coding: utf-8 -*-
\chapter{Interface syntaxe-sémantique}
%#####################################
\label{ch:ISS}\is{interface syntaxe-sémantique|sq}
\Writetofile{solf}{\protect\section{Chapitre \protect\ref{ch:ISS}}}


Ce chapitre est la continuation du précédent, en présentant des
applications concrètes du $\lambda$-calcul et de la théorie des types.
Il s'agit de revenir sérieusement sur l'objectif annoncé précédemment, à savoir la construction compositionnelle du sens des phrases à partir de leur analyse syntaxique.  Le \lcalcul\ typé est la boîte à outils sémantique qui nous permet d'accomplir cette tâche, mais nous devons maintenant faire le lien avec les structures que la syntaxe de la langue peut nous proposer. 
L'enjeu ici est donc de développer  et d'explorer un système de règles qui assurent l'interface entre la syntaxe et la sémantique.\is{regle@règle!\elid\ d'interface syntaxe-sémantique}
Cela va nous donner l'occasion d'aborder plusieurs exemples d'application directe du \lcalcul, mais aussi de réfléchir méthodiquement  sur certains problèmes qui se posent à l'analyse sémantique.   Et cela, de manière plus cruciale, va nous conduire à introduire des mécanismes de composition du sens complémentaires (et parfois alternatifs) mais indispensables pour mener à bien notre objectif.


\section{Boîte à outils de l'interface syntaxe-sémantique}
%=========================================================

Par compositionnalité, la syntaxe guide le processus d'analyse sémantique.
Nous avons donc besoin d'un moyen de représenter des structures syntaxiques.
Dès que nous commençons à manipuler de telles structures, nous sommes amenés, par la force des choses, à adopter en même temps certaines \emph{analyses} syntaxiques. 
Dans l'idéal, nous aimerions rester le plus possible agnostique vis-à-vis de tout positionnement théorique sur la syntaxe des langues,
car  le présent ouvrage n'a nullement la vocation (ni la légitimité) de défendre tel ou tel paradigme d'analyse syntaxique. 
Mais en pratique, nous devons bien faire des choix.
À cet égard nous suivrons, dans ses grandes lignes, le cadre génératif de \emph{Principes et Paramètres}\footnote{Voir par exemple \citet{Chom:81},\Andex{Chomsky, N.} ou tout autre texte introductif ou manuel sur la grammaire générative.},\is{Principes et Paramètres}\is{grammaire!\elid\ générative} parce que ce modèle est relativement connu et que ses dispositifs d'analyse syntaxique sont suffisamment simples et assimilables, en particulier pour la tâche qui nous occupe ici. 
Bien sûr, cette section ne peut en aucune manière se substituer à une véritable introduction à la syntaxe, et elle se contente simplement de proposer  quelques hypothèses de représentation formelle des structures que nous serons amenés à manipuler (\S\ref{ss:HypSynt}).  À partir de là, nous mettrons en place un format de notation qui permettra de faire «dialoguer» les analyses de la syntaxe et de la sémantique et qui sera le lieu de l'interface que nous développerons dans ce chapitre
(\S\ref{ss:Arbr2Form}).


\subsection{Quelques hypothèses et notations syntaxiques}
%--------------------------------------------------------
\label{ss:HypSynt}\is{syntagme|sq}

Nous allons dorénavant adopter les notations anglo-saxonnes pour désigner les diverses catégories de syntagmes que nous rencontrerons (au lieu des abréviations comme GN, GV ou P).  
Selon cette convention, les syntagmes sont nommés par des «étiquettes» de la forme $X$P, où P vaut pour l'anglais \alien{phrase} (\ie\ syntagme) et $X$ désigne une catégorie lexicale ou fonctionnelle comme N pour nom commun,\is{nom!\elid\ commun} V pour verbe,\is{verbe} A pour adjectif,\is{adjectif} etc. 
Et ainsi nos syntagmes seront NP (syntagme nominal, \alien{noun phrase}), VP (syntagme verbal, \alien{verb phrase}), AP (syntagme adjectival, \alien{adjective phrase}), PP (syntagme prépositionnel, \alien{preposition phrase})...  
Profitons-en pour introduire immédiatement la catégorie TP qui désignera les phrases\is{phrase} (nous allons voir pourquoi ci-dessous)%
%
\footnote{Je tiens à préciser que l'adoption de ces notations n'est en rien une coquetterie gratuite ; c'est un choix pragmatique et stratégique qui vise à maintenir au mieux l'interopérabilité des propositions d'analyses sémantiques présentées ici avec ce que l'on peut trouver dans une part très importante de la littérature en syntaxe et sémantique formelles que les lecteurs curieux et intéressés seraient amenés à consulter. 
Il serait fastidieux et incommode de s'imposer la vigilance de convertir systématiquement des notations locales en notations qui par ailleurs sont assez standards.}%
.


Les structures syntaxiques sont d'abord assurées par la relation de \emph{constituance} entre syntagmes, qui dit qu'un syntagme (complexe) est composé d'autres syntagmes, par inclusions successives. Nous pouvons en rendre compte au moyen des traditionnelles règles de réécritures\is{regle@règle!\elid\ de reecriture@\elid\ de réécriture} comme en {\Next} :

\ex.
\a. TP {\reecr} NP VP
\b. VP {\reecr} V NP
\b. VP {\reecr} V

\Last[a] dit qu'une phrase (TP) se réécrit (\ie\ se décompose) en un NP (sujet) et un VP, et \Last[b] qu'un VP se réécrit en un V et un NP (objet) ou \Last[c] en un V seul.
Ces relations de constituance peuvent facilement être représentées par des arbres syntaxiques\is{arbre!\elid\ syntaxique} (nous en verrons beaucoup d'exemples) ou, plus sobrement, par un système de parenthésages «à plat» comme par exemple [\Stag{VP} V NP] qui donne les mêmes informations que \Last[b]. 


Il nous sera également utile d'adopter le schéma dit X-barre\footnote{\alien{X-bar} en anglais ; cf.\ entre autres \citet{Chom:70,Jkdf:77}.}\is{X-barre}\Andex{Chomsky, N.}\Andex{Jackendoff, R.} qui révèle une structure hiérarchique au sein des syntagmes.
Un syntagme est toujours un syntagme \emph{de} quelque chose, en l'occurrence d'une catégorie lexicale ou fonctionnelle, que l'on appelle la \emph{tête} du syntagme (N pour NP, V pour VP, etc.).  Cette tête entretient différents types de relations avec ses «compagnons» de syntagme, ce dont rend compte le schéma X-barre, illustré en figure~\ref{F:Xbar} (ci-contre), en introduisant trois niveaux de structure dans un constituant de tête X.  Sous le niveau X$'$, la catégorie  X peut se combiner avec un autre constituant, son \emph{complément} ; \is{complement@complément}%
et au niveau de XP, X$'$ peut se combiner avec un constituant qui joue le rôle de ce que l'on appelle le \emph{spécifieur}. \is{specifieur@spécifieur}%
Dans la figure~\ref{F:Xbar}, \sicut{m} est un n\oe ud terminal de l'arbre et représente une unité lexicale (un «mot») ou morphologique.

\begin{figure}[h!]
\begin{center}
  \begin{tabular}{c@{\qquad\qquad}c}
  \begin{tabular}{l}
    \Tree[.XP \textit{Spec} [.X$'$ [.X \sicut{m} ] \textit{Compl} ] ]
  \end{tabular}
  &
  \begin{tabular}{l}
    XP {\reecr} \emph{Spécifieur} X$'$\\
    X$'$ {\reecr} X \emph{Complément}
  \end{tabular}
\end{tabular}
\caption{Schéma X-barre, en arbre et en règles de réécriture}\label{F:Xbar}
\end{center}
\end{figure}


X est généralement appelé la \emph{projection lexicale}\is{projection syntaxique!\elid\ lexicale} de \sicut{m}, XP la \emph{projection maximale}\is{projection syntaxique!\elid\ maximale} de \sicut{m} (ou de X), et X$'$ la \emph{projection intermédiaire}%
\footnote{La projection lexicale est parfois aussi notée X$^0$ et la projection maximale X$''$.}. 
\is{projection syntaxique!\elid\ intermédiaire}%
Ces trois projections constituent des syntagmes à part entière\footnote{Et dans les pages qui suivent, j'utiliserai indifféremment les termes \emph{syntagmes} et \emph{constituants} comme des synonymes.}, même si ce que nous appelons des syntagmes dans l'usage courant renvoie plus souvent à des projections maximales.

TP est donc la projection maximale de la catégorie T qui correspond à l'inflexion\is{inflexion verbale} verbale, c'est-à-dire le \uline{t}emps de conjugaison du verbe (TP est parfois appelé IP, pour \alien{inflection phrase}\footnote{En réalité, c'est un peu inexact. Pour être un peu plus précis, TP n'est qu'un composant de ce qui constitue la flexion verbale et qui correspond synthétiquement à IP.}).  La tête syntaxique d'une phrase, c'est donc T.
L'analyse syntaxique d'une phrase simple comme \sicut{Alice regarde Bruno} peut ainsi, en première approximation, se représenter par l'arbre \Next[a], où {\scshape -prs.3sg} encode les traits qui caractérisent la flexion\is{flexion} du verbe (\ie\ ici : présent, {3\ieme} personne du singulier).  
Nous nous autoriserons, dans un premier temps, lorsqu'un n\oe ud ne semble pas avoir de contribution sémantique (ou qu'il a une contribution que nous laissons de côté), à l'omettre de l'arbre\footnote{Notons bien que, techniquement, ces «omissions» ne sont pas des suppressions de n\oe uds de la structure syntaxique ; elles consistent simplement à \emph{cacher} ces n\oe uds dans la représentation graphique afin de simplifier nos notations.}, et donc à manipuler des représentations syntaxiques dramatiquement simplifiées comme \Next[b]. 


\ex.
\a. {\small
\Tree[.TP
  [.NP  Alice ]
  [.{T$'$}  [.T  \zcbox{\scshape -prs.3sg} ]  %\zcbox{\itshape inflexion} ]
    [.VP  
      [.V$'$ 
        [.V  regard- ]
        [.NP  Bruno ]
      ]
    ] §\qsetw{7em} 
  ]
]
}
\qquad
b. \hstrab[1.4em] {\small
\Tree[.TP
  [.NP  Alice ]
    [.VP  
        [.V  regarde ]
        [.NP  Bruno ] §\qsetw{3.5em} 
  ]
]
}



Dans l'analyse, la sémantique opère \emph{après} la syntaxe\footnote{Ce n'est qu'une hypothèse de travail, qui vise essentiellement à nous faciliter la tâche. En réalité la syntaxe et la sémantique ont plutôt intérêt à collaborer plus ou moins simultanément.}. L'interface syntaxe-séman\-tique travaille donc sur les produits finis de la syntaxe ; autrement dit, pour le moment nous ne nous préoccupons pas des procédés syntaxiques mis en \oe uvre pour obtenir la structure \Last[a], nous la prenons telle quelle.  
Certes nous verrons bientôt qu'en réalité la syntaxe nous fournit une structure plus élaborée et plus informative que \Last[a] ; par exemple, nous voyons bien qu'en \Last[a], le radical verbal (V) et sa désinence (T) ne sont pas correctement positionnés l'un par rapport à l'autre ; nous y remédierons un peu plus tard, et nous serons souvent amenés, au fil des pages, à perfectionner les structures syntaxiques sur lesquelles s'appuie l'analyse sémantique compositionnelle. 
En attendant, nous allons nous contenter de cette simplification (et même de \Last[b]).

\sloppy

Le schéma X-barre rend compte des structures dans lesquelles une tête syntaxique \emph{sous-catégorise}\is{sous-catégorisation} (\ie\ sélectionne et attend) d'autres constituants (qui seront alors compléments ou spécifieurs).  Il existe un autre type de structures syntaxiques qui ne sont pas prises en charge par ce schéma, et qui concerne les \emph{modifieurs},\is{modifieur} c'est-à-dire ces constituants grammaticalement optionnels, non requis par un autre élément de la phrase, comme par exemple les adjectifs épithètes ou les syntagmes dits circonstanciels.
Ceux-ci s'intègrent à la structure par une autre opération syntaxique, que l'on appelle l'\emph{adjonction}, \is{adjonction}illustrée en figure~\ref{F:Adjonction}. 
Concrètement, lorsqu'un constituant W est adjoint à un constituant Y (qui peut être une projection de n'importe quel niveau), le n\oe ud Y est dupliqué dans l'arbre et W s'insère au même niveau que le Y inférieur, sous le Y supérieur.  

\fussy
%Adjonctions

\begin{figure}[h!]
\begin{center}
\begin{tabular}{c@{\qquad\qquad}c}
  \begin{tabular}{ccc}
\Tree
[.Y Y W §{\qbalance} ]
& ou &
\Tree
[.Y {W} {Y} §{\qbalance} ]
  \end{tabular}
  &
  \begin{tabular}{l}
    Y {\reecr} Y W  \\
    Y {\reecr} W Y
  \end{tabular}
\end{tabular}
\end{center}
\caption{Adjonction, en arbre et en règles de réécriture}\label{F:Adjonction}
\end{figure}


L'exemple {\Next} présente la structure d'un NP dans laquelle un adjectif (AP) est adjoint à la projection N$'$.  Notons au passage que dans ce NP, le déterminant (D) joue le rôle de spécifieur et le syntagme prépositionnel (PP) celui de complément du nom. 
Nous réviserons la structure syntaxique des groupes nominaux à partir de \S\ref{ss:iss:Qu}.

\ex.
{\small
\Tree
[.NP  
  [.D un ]
  [.N$'$ 
    [.AP  \zcbox{fameux} ]
    [.N$'$ 
      [.N \xbox{xxxi}{portrait} ]
      [.PP [.P de ]  \qroof{J. S. Bach}.NP  ]
    ]
  ]
]
}



\subsection{Des arbres aux formules}
%-----------------------------------
\label{ss:Arbr2Form}

Maintenant que nous disposons d'un support syntaxique formel (\ie\ des arbres et leurs règles de construction), nous allons pouvoir élaborer notre mécanisme de composition sémantique de tout constituant interprétable de la phrase.  
Rappelons que le principe consiste, en quelque sorte, à décorer les n\oe uds des arbres avec leur traduction sémantique.  
Chaque traduction s'obtient compositionnellement à partir de celles des n\oe uds inférieurs, dans un mouvement d'analyse ascendant, avec l'hypothèse que nous connaissons initialement la représentation sémantique des unités lexicales (\ie\ les feuilles de l'arbre). 
Nous avons annoncé au chapitre~\ref{ch:types} que, \emph{normalement}, la composition sémantique s'effectue au moyen de l'application fonctionnelle\is{application fonctionnelle}: lorsque deux constituants se combinent, l'un d'eux dénote une fonction et l'autre joue le rôle de l'argument.  
Nos règles d'interface syntaxe-sémantique seront donc crucialement guidées par les types, ceci afin de garantir que les fonctions reçoivent bien des arguments de types appropriés. 
En pratique, cela implique que nous devrons toujours veiller à ce que les combinaisons syntaxiques soient bien en accord avec les types sémantiques des constituants mis en jeu.

Comme chaque syntagme doit recevoir une traduction dans {\LO} et que chaque syntagme est défini par une ou plusieurs règles de réécriture, nos règles d'interface auront simplement à établir une correspondance syntaxe-sémantique pour chaque règle de réécriture de la grammaire.\is{regle@règle!\elid\ d'interface syntaxe-sémantique}
À cette occasion, pour faire le lien entre syntagmes et expressions sémantiques, nous pourrons reprendre les fonctions $\Ftrad$.\is{F@\Ftrad}  Si nous notons [\Stag{X} Y Z] le syntagme X constitué des syntagmes Y et Z, alors nous aurons à définir $\Ftradf{[\Stag{X} Y Z]}$ en fonction de $\Ftradf Y$ et $\Ftradf Z$.
Typiquement, les règles aurons la forme de \Next.

\ex.
Si $\Ftradf Y \in \ME_{\mtype{b,a}}$ et $\Ftradf Z \in \ME_{\mtyp b}$, alors
\(\Ftradf{[\Stag{X} Y Z]}=\xlo[\Ftradf{Y}\xlo(\Ftradf{Z}\Xlo)]\). %ou
%\(\Ftrad(Y Z)=\xlo[\Ftrad(Z)\xlo(\Ftrad(Y)\Xlo)]\)
\label{RIFtrad}


{\Last} dit simplement (et formellement) que la traduction de X est l'application fonctionnelle de celle de Y à celle de Z, si les types de Y et Z le permettent.
Cependant, comme le format de {\Last} nous fait manipuler des écritures un peu austères, nous préférerons présenter les règles d'interface  via les  notations plus «amicales» de {\Next} :\is{regle@règle!\elid\ d'interface syntaxe-sémantique}

%Règles d'interface (= règles de traduction dans \LO) :

\ex. \label{xRI1}
\a. \fsynsem{\begin{tabular}[t]{rccc}
X &\reecr& Y& Z\\
\small\mtyp a && \small\mtype{b,a} & \small\mtyp b\\
$\Xlo[\alpha(\beta)]$&\seecr &$\Xlo\alpha$ & $\Xlo\beta$\\
\end{tabular}}
\quad ou \
b.\hstrab[1.4em]
\fsynsem{\begin{tabular}[t]{rccc} 
X &\reecr& Y& Z\\
\small\mtyp b && \small\mtyp{a} & \small\mtype{a,b}\\
$\Xlo[\beta(\alpha)]$&\seecr &$\Xlo\alpha$ & $\Xlo\beta$\\
\end{tabular}}


\sloppy

Ces règles d'interface superposent les règles de réécriture syntaxiques et les règles de composition sémantiques en intercalant les types des expressions comme conditions de réussite.  Elles peuvent également se représenter graphiquement sous forme de sous-arbres (fig.~\ref{f:schemcompsem2}, page suivante) comme nous l'avions déjà vu au chapitre~\ref{ch:types}.  La règle \Last[a]  indique que si X se réécrit en [Y Z] et si Y se traduit par \vrb\alpha\ de type \mtype{b,a} et Z se traduit par \vrb\beta\ de type \mtyp b, alors X se traduit par $\Xlo[\alpha(\beta)]$ de type \mtyp a.  Elle dit la même chose que \LLast.
Quant à \Last[b], elle présente l'autre possibilité, celle où la fonction est dénotée par le constituant de droite~(Z).

\fussy

\begin{figure}[h]
\begin{center} 
{\large
\Tree
[.X\zbox{\ \(\Xlo[\rnode{ax1}{\alpha}(\rnode{bx1}{\beta})]\)${}_{\mtyp a}$}  
  Y\\$\Xlo\rnode{ay1}{\alpha}$\\\zcbox{${}^{\mtype{b,a}}$}  
  Z\\$\Xlo\rnode{bz1}{\beta}$\\\zcbox{${}^{\mtyp{b}}$}    §{\qbalance}
]  \nccurve[nodesep=2pt,angleA=45,angleB=-90,linecolor=gray]{->}{ay1}{ax1}\nccurve[nodesepA=2pt,nodesepB=1pt,angleA=45,angleB=-90,linecolor=gray]{->}{bz1}{bx1}
}
\qquad\qquad
ou
\qquad
{\large
\Tree
[.X\zbox{\ \(\Xlo[\rnode{bx2}{\beta}(\rnode{ax2}{\alpha})]\)${}_{\mtyp b}$}  
  Y\\$\Xlo\rnode{ay2}{\alpha}$\\\zcbox{${}^{\mtyp a}$}
  Z\\$\Xlo\rnode{bz2}{\beta}$\\\zcbox{${}^{\mtype{a,b}}$}  §{\qbalance}
]  \nccurve[nodesep=2pt,angleA=45,angleB=-110,linecolor=gray]{->}{ay2}{ax2}\nccurve[nodesep=1pt,angleA=135,angleB=-90,linecolor=gray]{->}{bz2}{bx2}
}
\caption{Schémas de composition sémantique dans un sous-arbre syntaxique}\label{f:schemcompsem2}
\end{center}
\end{figure}


Nous voyons donc que le choix entre le schéma \Last[a] et le schéma \Last[b] est décidé par les types des expressions. 
Mais il est également corrélé à la règle syntaxique particulière qu'il s'agit d'interfacer. 
En effet, il peut y avoir de bonnes raisons de penser que, sous l'hypothèse du schéma X-barre, la structure interne des syntagmes contribue à guider la composition sémantique : on peut ainsi prévoir qu'au niveau X$'$, c'est la projection lexicale X qui sera la fonction et le complément qui sera l'argument, et au niveau XP, c'est la projection intermédiaire X$'$ qui sera la fonction et le spécifieur son argument.  Dans les premiers exemples présentés \alien{infra}, nous verrons que cette corrélation n'est pas toujours respectée, parce que nous travaillerons sur des analyses syntaxiques très simplifiées ; mais à terme, elle aura tendance à se confirmer de plus en plus. 



Le schéma X-barre s'accorde donc particulièrement bien avec {\LO} typé puisque les embranchements des structures syntaxiques sont toujours, au maximum, binaires : à chaque «étage» d'un arbre, nous aurons au plus deux constituants, l'un  qui dénote une fonction et l'autre qui dénote son argument immédiat{\interfootnotelinepenalty=10000\footnote{Sauf bien sûr dans le cas de structures non branchantes, générées par une règle de la forme \mbox{X {\reecr} Y}.  Dans ce type de configuration, la représentation sémantique du n\oe ud inférieur remonte simplement telle quelle sur le n\oe ud supérieur.}}.
Cependant il n'y a pas nécessairement de raison de se fermer toutes les portes par principe, et s'il s'avère souhaitable de disposer également de branchements multiples «supra-binaires», il se trouve que le {\lcalcul} nous permet de les traiter. 
Si par exemple un syntagme se décompose directement en trois sous-constituants, alors l'un de ceux-ci devra dénoter une fonction appropriée et les deux autres lui seront fournis comme arguments dans un ordre correctement spécifié par la règle d'interface. 
C'est ce qu'illustre {\Next} qui est un exemple théorique, parmi plusieurs autres, d'une telle règle. 


\ex. \label{xRI1tr}
\fsynsem{\begin{tabular}[t]{rcccc}
X &\reecr& Y& Z & W\\
\small\mtyp a && \small\mtype{c,\mtype{b,a}} & \small\mtyp b & \small\mtyp c\\
$\Xlo[[\alpha(\gamma)](\beta)]$&\seecr &$\Xlo\alpha$ & $\Xlo\beta$&$\Xlo\gamma$\\
\end{tabular}}%


%\largerpage

Le tableau \ref{T:XP-Typ} ci-contre récapitule quelques  correspondances entre catégories syntaxiques et types sémantiques qui seront exploitées dans des règles d'interface spécifiques que nous rencontrerons par la suite.  Certaines de ces correspondances sont provisoires, s'appuyant sur des hypothèses d'analyse que nous avons posées dans les chapitres précédents ; nous aurons plusieurs fois l'occasion de discuter de possibles révisions.

\begin{table}[h!]
\begin{center}
\caption{Correspondances catégories syntaxiques -- types sémantiques}\label{T:XP-Typ}
{\small%\rowcolors{2}{gray!10}{gray!25}
\begin{tabular}{ll}\lsptoprule
\bfseries Catégories syntaxiques & \bfseries Types\\\midrule
Noms propres (PN), pronoms personnels& \typ e \\
Phrases (TP) & \typ t \\
Nom\is{nom} (N, N$'$), verbes intransitifs\is{verbe} (V, VP), adjectifs\is{adjectif} (A, AP) & \type{e,t}\\
Verbes transitifs (V), noms (N) et adjectifs (A) relationnnels\is{nom!\elid\ relationnel}\is{adjectif!\elid\ relationnel} & \type{e,\type{e,t}}\\
Prépositions\is{preposition@préposition} (P)& \eet\\
Verbes ditransitifs (V)& \type{e,\eet}\\
\lspbottomrule
\end{tabular}}\is{verbe!\elid\ intransitif}\is{verbe!\elid\ transitif}\is{verbe!\elid\ ditransitif}
\end{center}
\end{table}

\newpage

Pour terminer, je souhaite ici récapituler et préciser quelques aspects du statut théorique des règles d'interface syntaxe-sémantique.\is{regle@règle!\elid\ d'interface syntaxe-sémantique} 
Les exemples \ref{xRI1} et \ref{xRI1tr} ne sont pas de véritables règles d'interface, mais des méta-règles ou \emph{schémas} de règles (ils utilisent des symboles génériques et anonymes de catégories syntaxiques, X, Y, Z). 
Une véritable règle d'interface sera, comme annoncé \alien{supra}, une instanciation qui décrit une construction syntaxique précise et lui associe une composition sémantique appropriée.  C'est pourquoi cette approche de l'interface syntaxe-sémantique, qui est, dans l'esprit, celle de \citet{Montague:UG,PTQ}, est appelée «de règle à règle» (ang.\ \alien{rule-by-rule}\footnote{C'est une appellation que \citet{Partee:96} attribue à E. Bach.\index{Bach, Emmon}}).
De la sorte, le mécanisme d'analyse sémantique n'exploite pas directement les structures (par exemple les arbres) construites par la syntaxe; il est, en fait, guidé par l'historique des applications des règles syntaxiques qui déterminent  la structure d'une phrase --~ce que l'on appelle parfois en syntaxe son \emph{arbre de dérivation}.\is{arbre!\elid\ de dérivation}  
Au chapitre \ref{ch:types} et dans les premières pages du présent chapitre, j'ai pu donner l'impression d'aller à l'encontre de ce principe,
mais c'est simplement parce que dans une grammaire syntagmatique\is{grammaire!\elid\ syntagmatique} (\ie\ avec des règles de réécriture) l'arbre construit et l'arbre de dérivation sont à peu près similaires, ce qui fait que l'on ne voit guère de différence. 
Cependant cela montre qu'en théorie, cette approche montagovienne de l'interface syntaxe-sémantique peut s'adapter à quasiment n'importe quel modèle d'analyse syntaxique : il suffit de définir des règles d'interfaces qui apparient chaque règle structurelle de la syntaxe avec une règle de composition sémantique,  dans l'esprit de ce qu'exprime la méta-règle \ref{RIFtrad}, voir p.~\pageref{RIFtrad}.

\newcommand{\relstruc}[1]{\ensuremath{R_{#1}}}

À titre d'illustration, je vais présenter rapidement un exemple générique d'interfaçage, formulé de manière volontairement abstraite pour montrer en quoi le mécanisme est relativement  indépendant de la théorie syntaxique choisie et mise en jeu.  
Supposons que la grammaire de la langue comprend une règle qui établit que (sous certaines conditions) une relation structurelle, appelons-la $\relstruc1$, s'applique entre un objet syntaxique Y et un objet syntaxique Z pour former un objet syntaxique X.  Nous pouvons formaliser cela par l'équation $\text X = \relstruc1(\text Y,\text Z)$ ou, pour ajouter un peu de hiérarchie, par $\text X = \tuple{\text Y,\relstruc1\mathpunct{:}\text Z}$ qui indique avec quelle relation Z s'articule par rapport à Y, élément plus important dans la structure. 
Dans nos règles de réécriture, les objets syntaxiques sont des syntagmes, la relation $\relstruc1$ est à chaque fois une simple relation d'adjacence et l'opération syntaxique à l'œuvre est une concaténation%
\footnote{L'opération de concaténation, parfois notée $\oplus$, $\cdot$ ou $\mathbin{\smallfrown}$, est simplement l'assemblage séquentiel de deux objets. Ainsi $\text X = \relstruc1(\text Y,\text Z)$ revient à $\text X = \text Y\oplus\text Z$ qui dit que X est la suite «Y Z». Par la même occasion, cela implique qu'il y a une relation de constituance entre X et la paire Y Z : Y et Z sont des parties de X. Avec la notation $\text X = \tuple{\text Y,\relstruc1\mathpunct{:}\text Z}$, \relstruc1 sera la relation «être immédiatement à droite de».}.\is{concaténation}
Mais on peut décider que $\relstruc1$ représente un autre type de relation, comme des relations qui spécifient le rôle structurel qu'un mot ou un syntagme joue par rapport à un autre mot ou syntagme ; cela pourra ainsi correspondre à des fonctions grammaticales ou des relations de sous-catégorisation%
\footnote{C'est ce que l'on retrouve, sous une forme ou une autre, dans les graphes de grammaires de dépendances\is{grammaire!\elid\ de dépendances} \citep{Tesniere:59}, les f-structures de LFG \citep[\alien{Lexical Functional Grammar},][]{Bresnan:82e},\is{LFG (\alien{Lexical Functional Grammar})} les schémas de dominances immédiates en HPSG \citep[\alien{Head-Driven Phrase Structure Grammar},][]{PolSag:94},\is{HPSG (\alien{Head-Driven Phrase Structure Grammar})} les arbres de dérivation de TAG \citep[\alien{Tree Adjoining Grammar},][]{Jsh:87}.\is{TAG (\alien{Tree Adjoining Grammar})}}. 
Par exemple si Y est un V transitif, Z un N ou un NP et $\relstruc1$ la fonction \emph{objet}, alors $\tuple{\text Y,\relstruc1\mathpunct{:}\text Z}$ représente ce V déjà connecté à son complément d'objet. 
Si Y est aussi relié à un mot ou syntagme W par une relation \relstruc2, on pourra avoir, selon les choix de la grammaire, $\tuple{\tuple{\text Y,\relstruc1\mathpunct{:}\text Z}, \relstruc2\mathpunct{:}W}$ (structure récursive avec deux règles) ou $\tuple{\text Y,\relstruc1\mathpunct{:}\text Z, \relstruc2\mathpunct{:} W}$ (structure plate avec une seule règle). 
L'important est que chaque règle syntaxique particulière indique précisément quels sont les composants qu'elle implique (par exemple Y et Z) et comment ceux-ci s'organisent entre eux dans la structure (ici avec \relstruc1), et que nous sachions dans quel ordre les règles s'enchaînent.  
À partir de là, il est relativement simple de formaliser des règles d'interface compositionnelles ; par exemple elles pourront dire : si $\text X = \tuple{\text Y,\relstruc1\mathpunct{:}\text Z}$% 
%(pour des catégories particulières X, Y et Z et une relation particulière \relstruc1) 
,
et si Y se traduit par \vrb\alpha\ de type \mtype{b,a} et Z par \vrb\beta\ de type \mtyp b, alors X se traduit par $\Xlo[\alpha(\beta)]$ de type \mtyp a ; c'est ce que schématise \ref{xRIx}, qui n'est que la version générique de \ref{xRI1}: %***très proche de ce que nous avons vu***:

\ex. \label{xRIx}
\fsynsem{\begin{tabular}[t]{rc@{}c@{}c@{}c@{\ }c@{}c@{}}
X &=& $\langle$ &Y& , \relstruc1 : &Z&$\rangle$\\
\small\mtyp a &&& \small\mtype{b,a} && \small\mtyp b\\
$\Xlo[\alpha(\beta)]$&\seecr &&$\Xlo\alpha$ && $\Xlo\beta$\\
\end{tabular}}



Il faut cependant préciser que les exemples présentés, notamment \ref{xRI1} et \ref{xRI1tr},  peuvent être un peu trompeurs,  parce qu'ils font apparaître une généralisation qui, bien que prégnante dans le système, n'en est pas pour autant fondamentale dans la théorie.
Dans les pages qui précèdent, il a été suggéré que les compositions sémantiques s'effectuaient au moyen de l'outil «à tout faire» qu'est l'application fonctionnelle,\is{application fonctionnelle} 
et cela sera illustré dans les premiers exemples de \S\ref{ss:exAF}.  
Mais, comme nous le verrons par la suite, l'application fonctionnelle n'est pas nécessairement la seule opération sémantique utilisable; en théorie, d'autres procédés de composition peuvent être convoqués, ils dépendent de la construction syntaxique propre à la règle d'interface. 
Car les règles d'interface peuvent être lues, dans  leur horizontalité, comme des formes de raisonnement (et c'est qu'explicite la formulation \ref{RIFtrad}) : la partie à droite des flèches en donne les prémisses et la partie gauche la conclusion. 
Autrement dit, le mécanisme de composition sémantique est un système d'inférences, qui est, comme il se doit, gouverné par une logique spéciale, dédiée à cette tâche.  Une telle logique %est généralement appelée une «glu logique»\footnote{Car elle indique comment «coller» ensemble les expressions syntaxiques et sémantiques qui composent une phrase.}\is{glu logique} (ou «glu sémantique»); elle 
ne vise pas à établir, par exemple, les conséquences logiques d'une phrase, son rôle est de contrôler et réglementer, dans le métalangage, la bonne conduite de l'analyse syntactico-sémantique. 
Celle que nous utilisons ici est relativement simple et elle exploite abondamment les opérations qu'offre le \lcalcul\ typé, mais il est possible de définir à cet effet des logiques plus complexes qui engendrent des cadres d'analyse plus élaborés.  C'est ce que développe notamment le domaine des \emph{grammaires logiques}\footnote{Voir, par exemple, \citet{Morrill:12} pour un panorama général.}.\is{grammaire!\elid\ logique} 
En bref, ce qu'il faut retenir du principe de l'interface syntaxe-sémantique est que ses règles sont des schémas d'inférences suffisamment bien formalisés pour permettre 1) de reconnaître clairement qu'une règle structurelle s'est appliquée dans l'analyse syntaxique d'une phrase et 2) de prédire clairement quelle opération de composition sémantique doit alors s'appliquer à cette étape de l'analyse et quel résultat précis donne cette opération; cette dernière peut donc être l'application fonctionnelle ou autre chose, du moment qu'elle est précisément définie dans la logique du système.



\section{Compositions sémantiques}
%=================================

Comme annoncé, nous allons voir en section \S\ref{ss:exAF} plusieurs exemples de compositions sémantiques qui s'effectuent directement via l'application fonctionnelle.  Les sections suivantes montreront en quoi cette opération seule n'est pas toujours suffisante et nous verrons diverses manières d'ajuster et de développer nos règles d'interface en restant conforme au principe récapitulé \alien{supra}.



\subsection{Usages de l'application fonctionnelle}
%-------------------------------------------------
\label{ss:exAF}\is{application fonctionnelle}

%Qq cas simples

\subsubsection{Le verbe et ses arguments}
%'''''''''''''''''''''''''''''''''''''''''

Dans le chapitre~\ref{ch:types} (\S\ref{ss:glue}), nous avons vu la dérivation pas à pas d'une phrase simple comme \Next[a], rappelée en \Next[b] (avec nos nouvelles notations et en indiquant directement les résultats des \breduc s).


\ex.
\a. Alice regarde Bruno.
\b.
{\small\Tree
[.{TP\zbox{\ \(\Xlo\prd{regarder}(\cns a,\cns b)\)}}
  [.NP \pile{Alice\\\cns a} ]
  [.{VP\zbox{\ \(\Xlo \lambda x \,\prd{regarder}(x,\cns{b})\)}} 
    [.V \pile{regarde\\\zcbox{\(\Xlo\lambda y \lambda x \,\prd{regarder}(x,y)\)\rule{3em}{0pt}}} ]
    [.NP \pile{Bruno\\\cns b} ]
  ]
]}


Cette dérivation s'appuyait sur les règles d'interface explicitées maintenant en  \Next\ et \NNext. 
\Next\ indique comment un verbe transitif\is{verbe!\elid\ transitif} (donc de type \eet) se combine avec son NP objet dans un VP ; et \NNext\ comment un VP se combine avec le NP sujet dans une phrase.


\ex. \RISS{Verbes transitifs}%
{\begin{tabular}[t]{rccc}
    VP & \reecr & V &NP\\
    \small\et && \small\eet & \small\typ e \\
    $\Xlo[\alpha(\beta)]$ &\seecr & $\Xlo\alpha$ &$\Xlo\beta$
  \end{tabular}} \label{ri:VT}

\ex. \RISS{TP}%
{\begin{tabular}[t]{rccc}
    TP & \reecr & NP &VP\\
    \small\typ t && \small\typ e & \small\et \\
    $\Xlo[\beta(\alpha)]$ &\seecr & $\Xlo\alpha$ &$\Xlo\beta$
  \end{tabular}}\label{v:ri:TP1}


Dans ces règles, remarquons que les NP sont traités comme des expressions de type~\typ e, ce qui convient bien aux exemples que nous examinons ici ; nous aurons cependant à réviser ce traitement à partir de \S\ref{ss:iss:Qu}.
Notons aussi que la règle \Last\ vaut pour toute combinaison d'un VP avec son sujet, que sa tête soit un verbe transitif, intransitif ou ditransitif ; un VP est ici\footnote{Cela changera un peu plus tard, lorsque nous utiliserons des analyses syntaxiques et sémantiques plus perfectionnées.} toujours de type \et. 

\sloppy

Profitons de l'occasion pour dire quelques mots sur les verbes ditransitifs,\is{verbe!\elid\ ditransitif}  
comme \sicut{donner}, \sicut{envoyer}, \sicut{présenter}...
Nous n'allons surtout pas entrer dans les détails de la structure syntaxique de ces verbes,
car c'est un sujet relativement complexe et qui dépasse largement la portée du présent ouvrage.  
À la place,  nous allons nous contenter d'une hypothèse syntaxique très  simplifiée, pour nous préoccuper principalement de ce que le \lcalcul\ nous permet de faire avec ces constructions.  
Considérons donc la structure «plate»  \Next.

\fussy

\ex. 
%\a.
{\small\Tree
[.VP 
  [.V \zcbox{donne} ] 
  NP 
  [.PP [.P à ] NP ]
]
}\label{x:a:Vdit1}
%
%% \hstrab[6em]
%% %
%% b. \hstrab[1.4em]
%% {\footnotesize\Tree
%% [.VP 
%%    [.VP 
%%      [.V \zcbox{donne} ] 
%%      NP ] 
%%    [.PP [.P à ] NP ]
%% ]
%% }

\largerpage

Si nous posons que \sicut{$x$ donne $y$ à $z$} se traduit par $\Xlo\prd{donner}(x,y,z)$, qui est la simplification de $\Xlo[[[\prd{donner}(z)](y)](x)]$, 
alors cela prévoit que \sicut{donner} doit d'abord se combiner avec le PP puis avec le NP.  
Avec une structure plate comme en \Last, c'est directement et simplement pris en charge par la règle d'interface \Next.

\ex.  \RISS{Verbes ditransitifs (structure plate)}%
{\begin{tabular}[t]{rcccc}
    VP & \reecr & V &NP & PP\\
    \small\et && \small\type{e,\eet} & \small\typ e & \small\typ e \\
    $\Xlo[[\alpha(\gamma)](\beta)]$ &\seecr & $\Xlo\alpha$ &$\Xlo\beta$ &$\Xlo\gamma$
  \end{tabular}}


Notons que si, au lieu de \ref{x:a:Vdit1}, nous avions une structure binaire où le V se retrouve, d'une manière ou d'une autre, d'abord groupé avec le PP (comme dans [\Stag{VP} [V PP] NP]), alors il suffirait de diviser \Last\ en deux règles assez similaires à \ref{ri:VT}.  En revanche, si nous avions un structure binaire où V se compose d'abord avec NP, comme dans [\Stag{VP} [V  NP] PP], 
alors il nous faudrait en plus changer l'entrée lexicale sémantique du verbe ditransitif. Par exemple, \sicut{donner} ne se traduirait plus par \prd{donner} (\ie\ $\Xlo\lambda z\lambda y\lambda x\,\prd{donner}(x,y,z)$), mais par 
\(\Xlo\lambda y\lambda z\lambda x\,\prd{donner}(x,y,z)\).  
Nous voyons donc que l'analyse sémantique dépend crucialement des hypothèses d'analyses syntaxiques, mais aussi que le \lcalcul\ nous permet de nous adapter avec assez de souplesse.

Remarquons également que dans la règle \Last, le PP est traité comme étant de type~\typ e. D'ailleurs nous aurions pu, comme le font certaines analyses syntaxiques, analyser ce complément comme un NP où \sicut{à} n'est pas véritablement une préposition, mais un marqueur casuel\footnote{À l'instar de certaines langues où ce complément est un NP datif.\is{datif}}.
Qu'il s'agisse d'un PP de type \typ e ou d'un NP, \sicut{à} aura une contribution sémantique vide, et le \lcalcul\ peut en rendre compte en lui assignant la traduction $\Xlo\lambda x\,x$ (de type \type{e,e}).\label{à:id}




\subsubsection{Le(s) verbe(s) \sicuto{être}}
%'''''''''''''''''''''''''''''''''''''''''''
\label{ss:être}

Il existe une autre catégorie de VP qui n'entre pas exactement dans le cadre de ceux vus ci-dessus.  Ce sont ceux dont la tête est ce que les grammaires traditionnelles appellent un verbe d'état, en particulier le verbe \sicut{être}, accompagné d'un syntagme adjectival qui joue le rôle d'attribut du sujet, comme en \Next.

\ex.
Peter  est courageux. \label{x:PPcourage}
\\
{\small\Tree
[.TP 
  [.NP \zcbox{Peter} ]
  [.VP\zbox{${}_{\et}$} 
    [.V est ]
    [.AP\zbox{${}_{\et}$} courageux ]
  ]
]
}


Un adjectif ou un AP comme \sicut{courageux} n'a pas le même type qu'un NP, il est de type \et\ ($\Xlo\lambda x\,\prd{courageux}(x)$).  
Et un VP, nous l'avons vu, est aussi de type \et.  Par conséquent, le verbe \sicut{être} en \Last\ doit s'analyser comme une expression fonctionnelle qui prend en argument la dénotation d'un prédicat adjectival (AP de type \et) et retourne un prédicat verbal (VP de type \et). Son type est donc naturellement \type{\et,\et}, et sa traduction est la suivante :

\ex.
\(\sicut{être} \leadsto \Xlo\lambda P\lambda x [P(x)]\), 
avec $\vrb P\in\VAR_{\type{e,t}}$ \label{être1}


Nous avons déjà rencontré ce \lterme\ en \S\ref{sss:lectureltermes}, p.~\pageref{x:lPlxPx}. 
C'est une expression qui attend un prédicat et un
terme individuel, et qui applique ce prédicat à ce terme. 
Nous avions vu que cela ne fait que réaliser l'application fonctionnelle, et finalement c'est bien juste ce que fait sémantiquement le verbe \sicut{être} dans ce rôle dit de \emph{copule}\is{copule} : «accoupler» un prédicat adjectival avec le sujet de la phrase. 
C'est ce qu'illustre la dérivation sémantique du VP de \ref{x:PPcourage} où, après \breduc s nous retrouvons le même prédicat que celui du AP\footnote{J'y renomme la variable \lamb-abstraite dans la traduction du AP pour éviter toute confusion.} :

\ex.
\(\begin{array}[t]{l@{\ }c@{\ }ll}
\text{[\Stag{VP} est courageux]}
&\leadsto&
\Xlo[\lambda P\lambda x[P(x)](\lambda y\,\prd{courageux}(y))]\\
&=&
\Xlo\lambda x[\lambda y\,\prd{courageux}(y)(x)]
&\text{\small \breduc\ sur \vrb P}\\
&=&
\Xlo\lambda x\,\prd{courageux}(x)
&\text{\small \breduc\ sur \vrb y}\\
\end{array}
\)


La règle d'interface pour le verbe \sicut{être} est donc {\Next} :

\ex.
\RISS{Copule et attribut du sujet}%
{\begin{tabular}[t]{rccc}
    VP & \reecr & V &AP\\
    \small\et && \small\type{\et,\et} & \small\typ \et \\
    $\Xlo[\alpha(\beta)]$ &\seecr & $\Xlo\alpha$ &$\Xlo\beta$
  \end{tabular}}


A priori, cette règle peut valoir aussi pour les autres verbes copules comme \sicut{paraître}, \sicut{rester}, \sicut{devenir}, \sicut{se retrouver}...
Ceux-ci ont juste un contenu sémantique plus riche que \sicut{être}, et nous pourrions être tentés de leur assigner des traductions comme, par exemple, 
\(\Xlo\lambda P\lambda x\, \prd{paraître}(x,P) \), qui est bien de type \type{\et,\et}.
Cependant, nous verrons en \S\ref{ss:AFInt} que ce genre de traduction n'est pas vraiment satisfaisant sur le plan sémantique.

\sloppy
La règle \Last\ peut également, au prix d'un très petit ajustement, s'appliquer aux constructions attributives où le complément du V n'est pas un syntagme adjectival mais un syntagme prépositionnel, comme dans :

\fussy

\ex.
Alice est dans la cuisine.


Le PP [\sicut{dans la cuisine}] se traduit en $\Xlo\lambda x\,\prd{dans}(x,\atoi y\,\prd{cuisine}(y))$ de type \et\ comme un AP, et peut se combiner de la même manière avec le verbe \sicut{être}.  Cela implique que dans un tel exemple, la préposition\is{preposition@préposition} \sicut{dans} se traduit comme un verbe transitif : $\Xlo\lambda y\lambda x\,\prd{dans}(x,y)$ de type \eet.
Nous avons donc la règle suivante pour les PP :


\ex. \RISS{Syntagmes prépositionnels}%
{\begin{tabular}[t]{rccc}
    PP & \reecr & P &NP\\
    \small\et && \small\eet & \small\typ e \\
    $\Xlo[\alpha(\beta)]$ &\seecr & $\Xlo\alpha$ &$\Xlo\beta$
  \end{tabular}} \label{v:ri:VT}


\smallskip

Il y a au moins un autre emploi du verbe \sicut{être} que nous devons mentionner ici. 
Il s'agit de son emploi dit identificationnel, illustré en \Next.

\largerpage

\ex.
\a. Peter Parker est Spiderman.\\
 \(\Xlo\cns p = \cns s\)\label{x:PP=Spiderman}
\b.
Spiderman est l'ennemi de Dr.\ Octopus.\\
\(\Xlo\cns s = \atoi x\,\prd{ennemi}(x,\cns o)\)


Comme ces occurrences de \sicut{être} se combinent avec un NP, nous pouvons les traiter comme des verbes transitifs de type \eet\ avec la règle \ref{ri:VT}. 
Les traductions en \Last\ nous montrent que leur contribution sémantique est l'opérateur d'identité $\xlo=$, ce qui correspond au \lterme\ {\Next} : 

\ex.
\(\sicut{être} \leadsto \Xlo\lambda y \lambda x [x=y]\)\label{être2}


Nous verrons plus tard des stratégies qui visent à unifier les divers emplois de \sicut{être}, afin d'éviter d'avoir à conserver des traductions distinctes comme \ref{être1} et \ref{être2}.  Il nous faudra aussi donner une analyse pour ses emplois (très courants) tels qu'en {\Next}, que nous ne savons pas traiter encore (cf. \S\ref{ss:iss:Qu}).

\ex.
Spiderman est un super-héros.\label{x:spiderman}


\subsubsection{La négation}
%'''''''''''''''''''''''''''
\is{negation@négation}\label{sss:négation}
Terminons cette section par quelques remarques sur la négation.  
Nous allons le faire en posant une hypothèse d'analyse syntaxique qui en réalité s'avère (doublement) inexacte\footnote{Là encore la raison est qu'une analyse syntaxique correcte de la négation met en jeu des éléments formels plus complexes, que nous pourrons commencer à traiter seulement à partir de \S\ref{ss:iss:Qu}.}.  
Mais l'objectif ici est seulement de nous donner l'occasion d'explorer une application simple du {\lcalcul} ; lorsque nous aurons avancé dans le développement des mécanismes d'interface syntaxe-sémantique, nous pourrons nous rapprocher d'une analyse plus correcte du phénomène. 
Nous allons d'abord considérer qu'en français, bien que la négation se réalise par des éléments disjoints en surface, elle occupera, dans l'arbre syntaxique, un seul constituant, que nous appellerons Neg ; c'est la première approximation. 
La seconde est que nous posons que Neg s'adjoint au VP de la phrase. C'est ce qu'illustre l'analyse \Next.

\ex.  
Alice ne dort pas.
\\
{\small\Tree
[.TP
  [.NP \zcbox{Alice} ]
  [.VP\zbox{${}_{\et}$} 
    [.Neg {ne...pas} ]
    [.VP\zbox{${}_{\et}$} dort ] 
  ]
]
}


Informellement, nous voyons que la négation est une expression qui prend un VP en argument et retourne en résultat un nouveau VP qui est la «forme négative» du premier.  Neg est donc de type \type{\et,\et} et sa traduction est :

\ex.
\(\sicut{ne...pas} \leadsto \Xlo\lambda P \lambda x\neg [P(x)]\), où $\vrb P\in\VAR_{\et}$\footnote{Bien sûr il ne faudrait surtout pas s'égarer à commettre l'erreur de proposer $\Xlo\lambda P\neg P$, qui semble faire le travail attendu, mais qui en réalité n'est pas une expression bien formée de \LO. Car $\xlo\neg$ ne peut se placer que devant une expression de type \typ t et \vrb P est de type \et.}


Cette traduction ressemble un peu à celle de \sicut{être} \ref{être1}, mais pour bien comprendre son fonctionnement nous pouvons voir \Last\ comme dénotant une fonction qui attend la dénotation d'un VP ($\Xlo\lambda P$) et qui ensuite \emph{construit} la représentation d'un VP.  Pour ce faire, on abstrait d'abord l'argument qui correspondra au sujet ($\Xlo\lambda x$) puis on unit le prédicat \vrb P et cet argument \vrb x pour produire une formule que l'on nie ($\Xlo\neg[P(x)]$).


\subsection{Les arguments implicites}
%---------------------------------
\label{ss:ArgImpl}\is{argument!\elid\ implicite}
Nous venons de voir des exemples de compositions de prédicats avec leurs arguments, il est assez normal, à ce stade, de s'intéresser alors aux cas des «arguments absents».  Ce sont ces configurations bien connues, comme en \ref{x:VTabs} et \ref{x:VTa0}, où il manque, par exemple, le complément de verbes transitifs,\is{verbe!\elid\ transitif}  de noms ou d'adjectifs relationnels, etc.\footnote{À cela s'ajoute également les constructions passives sans agent.} --~ce que nous pouvons caractériser techniquement comme un emploi de prédicats qui ne respectent pas, en surface, leur arité.


\ex.  \label{x:VTabs}
\a. Alice a mangé. \label{x:VTabsa}
\b. Alice a lu. \label{x:VTabsb}
\b. Alice a dessiné. \label{x:VTabsc}

\ex. \label{x:VTa0}
\a. Alice a apprécié. \label{x:VTa0a}
%Alice a ouvert. 
\b. Alice a compris. \label{x:VTa0b}
\b. Alice a accepté. \label{x:VTa0c}


\sloppy
L'analyse de ces constructions est un enjeu qui concerne de très près l'interface syntaxe-sémantique.  Nous ne pourrons pas la régler ici car c'est un problème particulièrement complexe, abondamment débattu\footnote{Le phénomène apparaît d'ailleurs sous de nombreux intitulés : compléments ou arguments implicites, inexprimés, absents, omis, manquants, optionnels, effacés, invisibles, latents, sous-entendus... Nommer n'est pas décrire, mais cette variété terminologique laisse entrevoir une assez grande diversité d'analyses et d'approches théoriques. Pour des présentations plus détaillées, on peut se reporter par exemple à \citet{Gillon:12}\Andexn{Gillon, B.} et \citet{Bourmayan:13}.\Andexn{Bourmayan, A.}} et soulevant un certain nombre de questions qui, en s'entrecroisant,  multiplient considérablement les hypothèses d'analyse à envisager pour aboutir à un traitement sémantique compositionnel.  
Mais il est néanmoins dans notre intérêt ici d'examiner au moins quelques unes de ces questions et de faire quelques observations sur ce qu'elles peuvent impliquer dans la formalisation de nos règles d'interface.


Commençons par poser explicitement le problème dans des termes qui nous concernent ici directement.  Nous pouvons supposer que les verbes de \ref{x:VTabs} et \ref{x:VTa0} sont lexicalement transitifs, c'est-à-dire de type \eet, et donc, sous cette hypothèse, l'absence de leur complément d'objet ne nous permet pas de composer le sens de ces phrases par nos règles \ref{ri:VT} et \ref{v:ri:TP1}, p.~\pageref{ri:VT}. 
Par conséquent, quel est le processus de composition qui nous permet d'obtenir leur analyse correcte ?

\fussy



Par ailleurs, comme cela a été reconnu depuis longtemps, il faut remarquer que les exemples de \ref{x:VTabs} et de \ref{x:VTa0} ne s'interprètent pas de la même manière.
Pour \ref{x:VTabs} nous comprenons les phrases comme si elles comportaient, en substitut du complément manquant, une sorte de NP {indéfini} assez sous-spécifié, comme \sicut{quelque chose}.  Nous parlerons d'\emph{arguments implicites indéfinis}, et leur traitement met en jeu une quantification existentielle. Par exemple, la traduction attendue pour \ref{x:VTabsa} sera (minimalement) :

\ex. 
Alice a mangé.\\
\(\Xlo\exists y\,\prd{manger}(\cns a,y)\)


Au contraire, pour \ref{x:VTa0},  les compléments manquants, à l'instar de pronoms «silencieux», sont interprétés en récupérant une information accessible dans le contexte et forcément connue du locuteur.
À terme, et selon les contextes, \ref{x:VTa0a} pourra être glosée au moyen d'expressions démonstratives ou définies comme dans \sicut{Alice a apprécié cela/ce qui a été dit/ce qu'elle a vu/ton attitude/le repas qu'on lui a servi/...}
On parle généralement d'\emph{arguments implicites définis}, et compositionnellement, dans {\LO}, nous pouvons en rendre compte en utilisant une variable libre comme nous le faisons pour les pronoms anaphoriques et déictiques :

\ex.  \label{x:Aapprécié}
Alice a apprécié.\\
\(\Xlo\prd{apprécier}(\cns a,y)\)


Il existe certainement d'autres manières encore d'interpréter les arguments non réalisés%
\footnote{Il est notamment souvent cité l'interprétation dite générique, comme dans la lecture de \sicut{Alice dessine} signifiant qu'elle pratique le dessin comme une activité habituelle. À la suite de ce que nous avons vu au chapitre \ref{ch:gn} \S\ref{AdvQ+Gen} sur la généricité, nous rangerons cette interprétation dans la catégorie des arguments implicites indéfinis --~le sens générique intervenant très probablement à un autre endroit que la détermination du complément. En revanche, dans d'autres langues, comme par exemple en anglais, l'omission du complément peut donner lieu à une interprétation réfléchie : réflexive dans \sicut{Bill shaved this morning} (Bill s'est rasé) ou réciproque dans \sicut{they hugged} (ils se sont étreints). En français, un phénomène un peu analogue (quoique distinct)  peut apparaître avec certains noms relationnels\is{nom!\elid\ relationnel} comme \sicut{ami}, \sicut{collègue}, \sicut{camarade}... : dans \sicut{chaque membre du club peut inviter un ami}, le second argument de \sicut{ami} est compris comme une variable, non pas libre, mais liée à celle  quantifiée avec \sicut{chaque membre}.},  
mais ces deux paradigmes nous montrent déjà deux directions distinctes que peut (et doit) emprunter l'interface syntaxe-sémantique.

\largerpage

Ensuite, l'interface a besoin de savoir comment la syntaxe, de son côté, négocie les compléments manquants.  Ceux-ci sont-ils purement et simplement absents de la structure syntaxique ou représentés sous la forme d'un constituant phonologiquement vide ?  La situation idéale pour nous serait que la syntaxe soit sensible à la distinction entre arguments implicites définis et indéfinis en produisant des analyses différentes et dédiées à chaque interprétation.
Mais si c'est le cas, il reste à savoir sur quels critères la syntaxe discrimine les deux types d'arguments implicites.  
La réponse peut être à chercher du côté du lexique, car on observe d'une part part que les verbes transitifs n'acceptent pas tous aussi bien les arguments implicites, et d'autre part que ce ne sont apparemment pas les mêmes verbes qui acceptent les arguments implicites définis et indéfinis.
Cela va d'ailleurs dans le sens de ce que nous proposions, en première approximation, au chapitre~\ref{LCP} (\S\ref{sss:pred} p.~\pageref{H:aritéfixe}) en introduisant la notion d'arité : nous avons dû poser un prédicat \prdi{fumer}1 pour l'emploi intransitif (\ie\ avec un argument implicite indéfini) et un prédicat \prdi{fumer}2 pour l'emploi transitif.  À présent nous savons établir que ces deux prédicats sont sémantiquement reliés par un postulat de signification\footnote{C'est la stratégie proposée par \citet{FodorFodor:80}.\Andexn{Fodor, J. D.}\Andexn{Fodor, J. A.}} :

\ex.
\(\Xlo\doit\forall x[\exists y\,\prdi{fumer}2(x,y)\ssi\prdi{fumer}1(x)]\)

  
\sloppy
Transposée dans la grammaire, cette approche postule que de nombreux verbes, comme \sicut{manger}, \sicut{boire}, \sicut{fumer}, \sicut{lire}, \sicut{dessiner}..., sont lexicalement ambigus entre une version transitive et une version intransitive.  C'est lexicalement peu économique, mais cela a l'avantage de dispenser la syntaxe (et même la sémantique) de représenter des arguments inexprimés.  
Mais il se trouve que cette hypothèse de l'ambiguïté lexicale fait de mauvaises prédictions, comme le montre \citet{Gillon:12}\Andex{Gillon, B.} via le phénomène des ellipses.  Au chapitre~\ref{Ch:1} \S\ref{s:Ambiguïté}, nous avions vu que l'interprétation d'une ellipse, comme dans la seconde partie de \Next[a], se fait en reprenant \emph{le même} matériau sémantique utilisé auparavant. Dans l'ellipse «reconstruite» de \Next[a], la seconde occurrence de \sicut{lire} doit être transitive puisqu'il y a un complément d'objet (\sicut{l'Illiade}). Par conséquent l'hypothèse de l'ambiguïté de \sicut{lire} ne se conforme pas à la règle d'interprétation des ellipses puisqu'elle oblige à utiliser \prdi{lire}1 puis \prdi{lire}2 \Next[b]. 

\fussy

\ex.
\a. Alice a lu, mais pas l'\emph{Illiade}.
\b. \(\Xlo\prdi{lire}1(\cns a) \wedge \neg\prdi{lire}2(\cns a,\cns i)\)


D'autre part, \citet{Part:84c}\Andex{Partee, B.} favorise l'idée d'omettre aussi  les arguments implicites définis, dans les représentations syntaxiques et sémantique, en utilisant des prédicats d'arité 1 (mais différents de \Last)\footnote{Cette option, elle, ne contrevient pas à la règle d'interprétation des ellipses, justement par ce qu'un énoncé elliptique comme \Next\ n'est pas naturel en français :\ExNBP

\ex. \juge{\zarb} Alice a apprécié, mais pas ce que tu as fait/mais je ne sais pas quoi.\par\vspace{-1\baselineskip}}.  
Elle reconnaît cependant qu'une telle stratégie n'est pas implémentable en l'état dans un système sémantique montagovien traditionnel (comme notre \LO) car un prédicat est une constante, interprétée par \FI\ uniquement, et le «verbe» de \ref{x:Aapprécié}, $\Xlo\lambda x\,\prd{apprécier}(x,y)$ (ou ce qui serait son équivalent unaire), doit s'interpréter aussi par rapport à l'assignation courante $g$ à cause de la variable libre \vrb y. 


Notons qu'il y a une manière assez simple de traiter les arguments implicites indéfinis sans les représenter dans la syntaxe, tout en n'utilisant qu'un seul prédicat sémantique pour traduire le verbe.  
Il suffit d'introduire une règle d'interface spécifique  \ref{ri:VTabs} qui sait repérer un verbe transitif, \ie\ de type \eet, apparaissant seul sous V$'$\footnote{Cette règle reprend la proposition de \citet{Dowty:81}\Andexn{Dowty, D.} notamment.} :


\ex. \RISS{Verbes transitifs sans objet (1)\footnotemark}%
{\begin{tabular}[t]{rcc}
    V$'$ & \reecr & V \\
    \small\et && \small\eet  \\
    $\Xlo\lambda x\exists y[[\alpha(y)](x)]$ &\seecr & $\Xlo\alpha$ 
  \end{tabular}} \label{ri:VTabs}
\footnotetext{On peut prévoir de faire entrer aussi la pragmatique en jeu en traduisant V$'$ par $\Xlo\lambda x\exists y[C(y) \wedge[[\alpha(y)](x)]]$ où \vrb C est une variable libre de type \et\ qui restreint le domaine de quantification de $\Xlo\exists y$.}

Un des avantages de cette règle est que la quantification existentielle sur l'argument \vrb y qu'elle insère a forcément une portée étroite, «collée» au verbe.  C'est confirmé, par exemple, par l'interprétation des phrases négatives comme \sicut{Alice n'a pas mangé} qui se comprend comme \sicut{Alice n'a rien mangé} ($\Xlo\neg\exists y\,\prd{manger}(\cns a, y)$) et jamais comme \sicut{il y a quelque chose qu'Alice n'a pas mangé} ($\Xlo\exists y\neg\prd{manger}(\cns a, y)$).

Parallèlement, et pour en revenir à notre «situation idéale», nous pourrions, pour les arguments implicites définis, envisager l'existence d'une «anaphore zéro» (ang.\ \alien{null anaphora}), c'est-à-dire un constituant syntaxique non prononcé mais qui se comporte comme un pronom. 
Dans cette optique (que l'on retrouve dans les premiers travaux de la grammaire générative par exemple), nous aurions une entité syntaxique, notons-là $\Azero$, qui occuperait la position de complément et se traduirait sémantiquement par une variable localement libre de type \typ e.
Les phrases de  \ref{x:VTa0} pourraient ainsi s'analyser directement avec la règle \ref{ri:VT}.


Cependant, nous aurions peut-être un peu trop bon compte à prédéterminer les analyses sémantiques sur la simple base d'informations lexicales (et de ce qu'en fait la syntaxe).  Si, par exemple, il semble que les verbes qui acceptent les arguments implicites définis s'avèrent appartenir à certaines classes sémantiques particulières (ce sont souvent des verbes qui expriment une opinion, un affect, etc.), il n'est pas sûr que la caractérisation de ces classes soit une tâche simple et surtout qu'elle suffise à prédire les interprétations.  \citet{Fil:86},\Andex{Fillmore, C.} entre autres, a montré que de nombreux verbes acceptent les arguments implicites pour certaines acceptions ou certains usages mais pas pour d'autres, et que des quasi-synonymes se comportent différemment vis-à-vis du phénomène.  Par exemple, \sicut{j'ai aimé} se comprend avec un argument implicite défini pour l'acception \sicut{apprécier} mais pas pour \sicut{éprouver de l'amour} ; l'omission de l'objet est possible dans \sicut{Alice a trouvé}, mais pas dans \zarb\sicut{Alice a découvert}.  D'autre part, les interprétations avec arguments implicites indéfinis semblent un peu plus libres et un peu moins contraintes par le lexique (surtout dans les lectures génériques). Même si, avec certains verbes, ce type d'interprétation se trouve {a priori} difficilement disponible, comme dans \zarb\sicut{Alice a offert}, il est parfois possible, dans certains contextes, de restituer une telle interprétation.  Pour reprendre le cas de \sicut{j'ai aimé}, dans un registre soutenu et stylisé, cela peut se comprendre comme \sicut{j'ai connu l'amour (par le passé)} \ie\ \sicut{j'ai (déjà) aimé quelqu'un}.  
Et si l'objet implicite de \sicut{manger} est le plus souvent indéfini, il peut être défini ou indéfini dans les formes impératives : \sicut{mange !} peut vouloir dire \sicut{mange ce qu'il y a dans ton assiette} (défini) ou \sicut{nourris-toi} (indéfini).
Bref, il est indéniable que des informations contextuelles et des raisonnements pragmatiques jouent un rôle central dans l'interprétation des arguments implicites, y compris pour distinguer les deux catégories.  À cela s'ajoute aussi l'éventuelle complication que peuvent introduire certaines \emph{ellipses},\is{ellipse} qui constituent encore un autre cas d'omission de compléments ou d'arguments en surface.  Les ellipses peuvent être caractérisées, très grossièrement, comme un mécanisme qui aboutit à la \emph{non répétition} d'un matériau linguistique présent par ailleurs dans le discours, comme en \Next.  Là aussi il existe  diverses approches  théoriques du phénomène, mais il est habituellement admis que celui-ci est distinct de celui des arguments implicites discuté ici et qu'il doit donc recevoir un traitement différent.

\ex.
\a. Je demande un vote à main levée.  Non, en fait, j'exige !
\b. --- Ça te dirait une glace ?\\
--- Ah oui, je veux bien.


%Inversement, les arguments implicites indéfinis sont parfois vus comme résultant d'un «emploi intransitif» de verbes transitifs, c'est-à-dire \emph{sans} complément d'objet dans la structure syntaxique.  

Pour conclure, dans les limites de la couverture du présent ouvrage, nous devrons reconnaître que la question des arguments implicites ne peut pas entièrement se résoudre par un système de règles d'interface syntaxe-sémantique comme ceux traditionnellement manipulés dans la littérature.  
Faute de mieux, nous adopterons une position qui ne tranche pas entre les lectures définies et indéfinies des arguments implicites, c'est-à-dire qui génère (et souvent surgénère) une ambiguïté lorsqu'un argument attendu par l'arité d'un prédicat n'apparaît pas dans la forme de surface\footnote{Il ne s'agit pas d'une position de principe, c'est simplement une décision qui ne concerne que le système d'interface syntaxe-sémantique. }.  Il ne sera pas non plus absolument nécessaire ici de prendre définitivement parti  sur l'existence des anaphores zéro dans l'analyse syntaxique des arguments implicites.  Si nous avons de bons arguments pour les utiliser, nous avons vu ci-dessus comment les traiter sémantiquement ; sinon, nous pouvons envisager une règle d'interface \Next\ qui, en complément de \ref{ri:VTabs}, produit la lecture attendue pour les arguments implicites définis. 

\ex. \RISS{Verbes transitifs sans objet (2)}%
{\begin{tabular}[t]{rcc}
    V$'$ & \reecr & V \\
    \small\et && \small\eet  \\
    $\Xlo[\alpha(y)]$ &\seecr & $\Xlo\alpha$ 
  \end{tabular}} \label{ri:VTa0} \\ où \vrb y est une variable libre



\subsection{Le problème des modifieurs}
%---------------------------------------
\label{ss:ISSmodifieurs}\is{modifieur}

Nous avons déjà abordé les adjectifs\is{adjectif|(} en \S\ref{ss:être} dans leur fonction d'attribut du sujet, que l'on appelle également leur emploi \emph{prédicatif}.  
En effet, conformément à ce que nous faisons depuis le chapitre~\ref{LCP}, nous traitons les adjectifs comme des prédicats de type \et\footnote{À part bien sûr les adjectifs relationnels qui sont eux des prédicats de type \eet.} ; 
leur sens est bien une \emph{propriété}.
Mais nous avons alors un problème avec leur fonction d'épithète, c'est-à-dire lorsqu'ils modifient une projection nominale dans un NP. 
Nous nous en rendons compte immédiatement en observant les types dans la structure syntaxique \Next, où nous faisons l'hypothèse que l'adjectif s'adjoint à N$'$. 


\ex.  
{\small\Tree
[.N$'$\zbox{${}_{\et}$}
  [.N$'$\zbox{${}_{\et}$} tigre ]
  [.AP\zbox{${}_{\et}$} édenté ]
]
}


\sloppy 

Pour le N$'$ complet nous voudrions obtenir 
\(\Xlo\lambda x [\prd{tigre}(x) \wedge \prd{édenté}(x)]\) 
(l'ensemble de tous les individus qui sont à la fois tigres et édentés) à partir de
\(\Xlo\lambda x\,\prd{tigre}(x)\) 
et de
\(\Xlo\lambda x\,\prd{édenté}(x)\).  
Mais {\LO} ne nous permet tout simplement pas de combiner par application fonctionnelle deux expressions de type \et\ pour produire une expression de type \et. 

\fussy

Il y a plusieurs stratégies pour régler ce problème.  La première consiste à considérer que les adjectifs (du moins la plupart d'entre eux) existent en deux versions dans le lexique sémantique : une version prédicative, de type \et, qui est celle que nous avons manipulée jusqu'ici (par exemple \(\Xlo\lambda x\,\prd{édenté}(x)\)) ; et une version «épithète».
Cette dernière donnera une traduction qui a pour vocation de se combiner par adjonction avec un N$'$. Son type sera donc \type{\et,\et} : une expression qui attend un N$'$ et retourne un N$'$ complexe.  Concrètement, lorsqu'un adjectif sera employé comme modifieur de nom, il se traduira comme en {\Next} :


\ex.
\(\sicut{édenté} \leadsto
\Xlo\lambda P \lambda x [[P(x)]\wedge \prd{édenté}(x)]
\)\label{e:AdjMod}


Avec ce genre d'entrée lexicale, l'adjectif pourra se combiner avec un N$'$ suivant une règle d'interface standard :

\ex.
\RISS{Adjectif épithète (1)}%
{\begin{tabular}[t]{rccc}
    N$'$ & \reecr & N$'$ &AP\\
    \small\et && \small\et &  \small\type{\et,\et} \\
    $\Xlo[\beta(\alpha)]$
    &\seecr & $\Xlo\alpha$ &$\Xlo\beta$
  \end{tabular}}


\sloppy 
L'application fonctionnelle de \sicut{édenté} à \sicut{tigre} nous donnera bien, après \breduc s, \(\Xlo\lambda x [\prd{tigre}(x)\wedge\prd{édenté}(x)]\).  
En \LLast, l'adjectif \prd{édenté} dénote une fonction qui prend en argument un ensemble d'individus ($\Xlo P$), et retourne l'ensemble de tous les individus qui sont à la fois dans ce premier ensemble et dans l'ensemble de tous les édentés.  Autrement dit, \LLast\ est cette fonction qui effectue l'\emph{intersection} de tout ensemble donné en argument avec l'ensemble des édentés ; formellement 
\(\Ch{\denote{\Xlo[\lambda P \lambda x [[P(x)]\wedge \prd{édenté}(x)](\prd{tigre})]}^{\Modele,w,g}}=\Ch{\denote{\prd{édenté}}^{\Modele,w,g}}\cap\Ch{\denote{\prd{tigre}}^{\Modele,w,g}}\).
Cela reflète tout à fait le comportement \emph{intersectif} de l'adjectif que nous avons vu au chapitre~\ref{Ch:t+m} (\S\ref{ss:ModNonExt}).
Mais de ce fait, en \LLast, l'adjectif \sicut{édenté} n'a pas la même dénotation, et donc pas le même sens, que lorsqu'il se traduit par $\Xlo\lambda x\,\prd{édenté}(x)$.  
C'est là une critique que l'on peut adresser à cette stratégie, puisqu'elle est obligée de poser que chaque adjectif est lexicalement ambigu entre une entrée de type \et\ et une entrée de type \type{\et,\et}. Or ce qui distingue \sicut{édenté} prédicatif de \sicut{édenté} modifieur, ce sont surtout des propriétés combinatoires et pas vraiment une opposition de sens.  Le sens de \sicut{édenté}, dans tous ses emplois, c'est semble-t-il la propriété d'être édenté (\ie\ \(\denote{\prd{édenté}}^{\Modele,g}\)).  

\fussy

C'est pourquoi nous pouvons envisager une autre stratégie, qui élimine l'hypothèse de l'ambiguïté lexicale. 
Elle consiste à poser qu'initialement (c'est-à-dire lexicalement) les adjectifs sont de type \et\ (le type le plus simple), mais que lorsqu'ils sont employés comme épithètes, ils subissent un changement de type et donc de traduction sémantique pour prendre la forme de \ref{e:AdjMod}. 
Il se trouve que l'idée qu'une expression puisse changer de type dans certains environnements n'est pas du tout une audace ad hoc de l'analyse, et nous y reviendrons en \S\ref{s:typeshift}. 

Une troisième stratégie, assez proche de la précédente, est celle proposée par
\citet{HeimKratzer:97}.\Andex{Heim, I.}\Andex{Kratzer, A.}
Elle consiste à ajouter un nouveau mode de composition sémantique pour suppléer à l'application fonctionnelle et qui concerne spécifiquement les cas où un modifieur de type \et\ se combine avec un syntagme de type \et.  Ce mode composition est appelé la \emph{modification de prédicat}\is{modification de prédicat} (ang.\ \alien{predicate modification})\footnote{Cette composition est également appelée (mais plus rarement et plus anciennement) la \emph{\lamb-conjonction}.\is{la (lambda)@$\lambda$ (lambda)!$\lambda$-conjonction}}, et il est défini comme suit\footnote{\citet{HeimKratzer:97} ne présentent pas leur définition de la même manière, parce qu'elles n'adoptent pas l'approche de l'analyse sémantique indirecte (\ie\ avec un langage sémantique intermédiaire). Chez elles, les \lterme s ne \emph{dénotent} pas des fonctions, ils \emph{sont} des fonctions du modèle. Mais la règle \ref{ri:PM} est complètement équivalente à leur définition de la modification de prédicat.} :

\newcommand{\PMHK}{\ensuremath{\mathord{\nplus}}}
%% \ex.
%% Modification de prédicat\\
%% \fsynsem{\begin{tabular}[t]{rccc}
%%     N$'$ & \reecr & N$'$ &AP\\
%%     \small\et && \small\et &  \small\et \\
%%     $\Xlo\alpha\PMHK\beta$
%%     &\seecr & $\Xlo\alpha$ &$\Xlo\beta$
%%   \end{tabular}}

\ex.
\RISS{Modification de prédicat}%
{\begin{tabular}[t]{rccc}
    N$'$ & \reecr & N$'$ &AP\\
    \small\et && \small\et &  \small\et \\
    $\Xlo\lambda x [[\alpha(x)]\wedge[\beta(x)]]$ 
    &\seecr & $\Xlo\alpha$ &$\Xlo\beta$
  \end{tabular}}\label{ri:PM}


\sloppy
Cette règle revient simplement à combiner deux prédicats (\vrb\alpha\ et \vrb\beta) pour construire un nouveau prédicat qui vérifie la conjonction (au sens de $\Xlo\wedge$) des deux premiers. Et en termes ensemblistes, cela revient à construire l'intersection des deux ensembles dénotés par \vrb\alpha\ et \vrb\beta.


Notons que nous pouvons aussi la présenter autrement.  Posons par exemple la constante $\Xlo\prdk{inter}$ de type \type{\et,\type{\et,\et}} en postulant qu'elle est, dans tous les mondes possibles, équivalente à \(\Xlo\lambda Q\lambda P\lambda x [[P(x)]\wedge[Q(x)]]\) --~autrement dit, nous postulons que {\Next} est valide.\label{HINTER}

\fussy


\ex.
\(\Xlo\doit[\prdk{inter} = \lambda Q\lambda P\lambda x [[P(x)]\wedge[Q(x)]]]\)


Dans ce cas, \ref{ri:PM} peut se reformuler en {\Next} : 


\ex.
\RISS{Modification de prédicat}%
{\begin{tabular}[t]{rccc}
    N$'$ & \reecr & N$'$ &AP\\
    \small\et && \small\et &  \small\et \\
    $\Xlo[[\prdk{inter}(\beta)](\alpha)]$
    &\seecr & $\Xlo\alpha$ &$\Xlo\beta$
  \end{tabular}}\label{ri:PM2}


\sloppy

Par \breduc, $\Xlo[\prdk{inter}(\beta)]$ équivaut à 
$\Xlo\lambda P\lambda x [[P(x)]\wedge[\beta(x)]]$ et
$\Xlo[[\prdk{inter}(\beta)](\alpha)]$ à 
$\Xlo\lambda x [[\alpha(x)]\wedge[\beta(x)]]$.
Mais cela montre du même coup que \Last\ est en fait très proche de la deuxième stratégie évoquée ci-dessus, dès lors que nous considérons que $\Xlo\prdk{inter}$ est cette fonction (ou opérateur) qui effectue le changement de type et de traduction de l'AP lorsqu'il est épithète. 
De même, la première stratégie dit simplement qu'un adjectif comme \sicut{édenté} a deux traductions : \prd{édenté} lorsqu'il est prédicatif et $\Xlo[\prdk{inter}(\prd{édenté})]$ lorsqu'il est modifieur.

\fussy

Je voudrais, pour terminer, évoquer une dernière stratégie qui est assez rarement prise en compte\footnote{Notons cependant qu'elle est la stratégie adoptée par \citet{Montague:EFL} et qu'elle est précisément détaillée et motivée par \cite{Kmp:75}.}, mais qui peut avoir des implications intéressantes. 
Elle repose sur l'idée que fondamentalement un adjectif n'est pas la même chose qu'un nom, et que le propre d'un adjectif, après tout, c'est de pouvoir modifier un N.
L'emploi prédicatif pourrait être vu comme secondaire.  D'ailleurs nous savons que certains adjectifs ne possèdent pas cet emploi prédicatif, par exemple les adjectifs intensionnels vus en \S\ref{ss:ModNonExt}, alors que tous les adjectifs peuvent être modifieurs\footnote{En fait, en français, il n'existe que deux cas, très exceptionnels, d'adjectifs ne pouvant pas s'employer comme épithètes: \sicut{quitte} et \sicut{sauf}.}. 
Il s'agit donc de prendre le contre-pied de la deuxième stratégie, en considérant que les adjectifs sont \emph{initialement} de type \type{\et,\et}\footnote{Ou d'un type approchant, cf.\ \alien{infra}.}, et qu'ils passent au type \et\ seulement dans leur emploi prédicatif.
Suivant cette approche, il nous faut alors, cette fois, définir une règle spéciale qui effectue le passage de l'emploi modifieur \type{\et,\et} à l'emploi prédicatif \et. 
Cette opération est assez simple à réaliser : supposons que \vrb\beta\ est la traduction de type \type{\et,\et} d'un adjectif et posons que \vrb C est une variable de type \et, alors $\Xlo[\beta(C)]$ est de type \et. 
Ainsi la règle qui combine un AP avec une copule dans un VP pourra être définie comme suit :

\ex.
\RISS{Adjectifs prédicatifs}%
{\begin{tabular}[t]{rccc}
    VP & \reecr & V & AP\\
    \small\et && \small\type{\et,\et} &  \small\type{\et,\et} \\
    $\Xlo[\alpha([\beta(C)])]$
    &\seecr & $\Xlo\alpha$ &$\Xlo\beta$
  \end{tabular}}
\\
où \vrb C est une variable libre de type \et


Par exemple, l'entrée \(\Xlo\lambda P\lambda x [[P(x)] \wedge \prd{édenté}(x)]\) pour \sicut{édenté} verra son argument \vrb P saturé par la variable \vrb C, et ce qui se composera avec le verbe sera, après \breduc, \(\Xlo\lambda x [C(x) \wedge \prd{édenté}(x)]\).  Nous savons que \vrb C dénote (la fonction caractéristique d') un ensemble d'individus (spécifié par le contexte) et donc ce \lterme\ dénote (la fonction caractéristique de) l'ensemble de tous les individus qui sont édentés et dans la dénotation de \vrb C.  
Cela veut dire que nous laissons à un composant interprétatif ultérieur (peut-être pragmatique) le soin de choisir une assignation $g$ qui donne une valeur adéquate pour \vrb C.
Pour un adjectif intersectif comme \sicut{édenté} ce n'est pas très spectaculaire : le plus simple et le plus sûr est de faire en sorte que $g(\vrb C)$ soit la fonction caractéristique de tout {\Unv A} ; avec une telle assignation \(\Xlo\lambda x [C(x) \wedge \prd{édenté}(x)]\) équivaut à \(\Xlo\lambda x\, \prd{édenté}(x)\).  
Mais il existe des adjectifs pour lesquels la présence de \vrb C peut devenir particulièrement pertinente. 

En \S\ref{ss:ModNonExt} nous avons distingué les adjectifs intersectifs (extensionnels) et les adjectifs intensionnels, sur la base des inférences logiques qu'ils permettaient.  À cet égard, il existe une troisième catégorie que l'on appelle les \kwo{adjectifs subsectifs}\is{adjectif!\elid\ subsectif}.
Ce sont les adjectifs, très courants, comme \sicut{grand}, \sicut{petit}, \sicut{gros}, \sicut{intelligent}, \sicut{cher}, \sicut{chaud}, \sicut{froid}, etc. 
Nous y reviendrons plus en détail au chapitre \ref{Ch:adj} (vol.~2), mais nous pouvons dès à présent rappeler l'observation essentielle que nous avions faite au chapitre \ref{Ch:1} (\S\ref{s:conseql} p.~\pageref{x:cachalot}) : \ref{x:Asub1} $\satisf$ \ref{x:Asub2} (comme avec les intersectifs) mais \ref{x:Asub1} $\not\satisf$ \ref{x:Asub3}.

\largerpage[2]

\ex. \label{x:Asub}
\a. Isidore est un petit cachalot.\label{x:Asub1}
\b. Isidore est un cachalot.\label{x:Asub2}
\b. Isidore est petit.\label{x:Asub3}

De même, la phrase \Next\ n'est absolument pas contradictoire :

\ex. 
Isidore est un petit cachalot mais Isidore n'est pas un petit animal.\label{x:Asub4}



C'est bien connu : quand on juge qu'un individu est petit (ou grand), on le fait généralement \emph{par rapport} à un certain ensemble d'objets, que l'on appelle une \kw{classe de comparaison}\is{classe de comparaison}.
Lorsque l'adjectif est épithète comme en \ref{x:Asub1} et \ref{x:Asub4}, celle-ci est renseignée par la tête nominale ; lorsqu'il est prédicatif \ref{x:Asub3}, elle est pourvue par la variable \vrb C qui dénote un ensemble : c'est alors une classe de comparaison implicite ou sous-entendue. 
L'adjectif \sicut{petit} est donc ainsi un adjectif \emph{relatif}, et \ref{x:Asub} et \ref{x:Asub4} montrent aussi que ce n'est pas un adjectif intersectif : nous ne pouvons pas poser un prédicat \prd{petit} de type \et, car celui-ci dénoterait l'ensemble de tous les individus petits de \Unv A, mais cet ensemble n'existe pas dans l'absolu. 
Ce que fait l'adjectif c'est, à partir d'un ensemble d'objets donné, déterminer le \emph{sous-ensemble} de tous ses éléments jugés petits --~d'où le nom de \emph{sub}sectif.
%Un adjectif subsectif comme \sicut{petit} 
\sicut{Petit} se traduira donc par une constante de type \type{\et,\et} ; nous écrirons  \(\Xlo\lambda P\lambda x [[\prd{petit}(P)](x)]\) (ou \(\Xlo\lambda P\lambda x \,\prd{petit}(x,P)\)).\label{p.petitA}
Une définition du sens de \prd{petit} pourra être quelque chose comme :

\ex. \(\denote{\Xlo[\prd{petit}(P)](x)}^{\Modele,w,g}=1\) ssi la taille de \(\denote{\vrb x}^{\Modele,w,g}\) est sensiblement inférieure à la taille moyenne des individus de \(\Ch{\denote{\vrb P}^{\Modele,w,g}}\).


À partir de là, nous pouvons traduire \ref{x:Asub4} par {\Next} qui n'est pas contradictoire.

\ex.
\(\Xlo[\prd{petit}(\prd{cachalot})](\cns i) \wedge \neg[\prd{petit}(\prd{animal})](\cns i)\)


Et \ref{x:Asub3} se traduira par {\Next} qui sera tantôt vraie tantôt fausse dans un même monde (la taille d'Isidore ne change pas) selon la valeur que $g$ aura assignée à \vrb C.

\ex.
\(\Xlo[\prd{petit}(C)](\cns i)\)


\is{adjectif|)}


% -*- coding: utf-8 -*-
\begin{exo}\label{exo:6adj}
Dérivez compositionnellement \pagesolution{crg:6adj}%
(donc à partir d'hypothèses syntaxiques) la traduction sémantique de \sicut{petit tigre édenté}.
\begin{solu}(p.~\pageref{exo:6adj})\label{crg:6adj}

Il y a a priori deux grands types de structures syntaxiques pour analyser \sicut{petit tigre édenté}, schématisés en (a) et (b) ci-dessous ; sans entrer dans les détails de ces analyses, nous allons retenir la version (b) (qui est la plus couramment adoptée pour le français).

\ex.[(a)]
\small
\Tree[.N$'$ 
  [.AP petit ]
  [.N$'$ 
    [.N$'$ tigre ]
    [.AP édenté ]
  ] 
]
\normalsize\qquad
(b)\quad\small
\Tree[.N$'$ 
  [.N$'$ 
    [.AP petit ]
    [.N$'$ tigre ]
  ] 
    [.AP édenté ]
]
\normalsize

\sloppy

\S\ref{ss:ISSmodifieurs} nous offre plusieurs options d'analyses sémantiques ; choisissons ici celle qui traduit \sicut{petit} par \(\Xlo\lambda P\lambda x [[\prd{petit}(P)](x)]\), de type \type{\et,\et} (cf. p.~\pageref{p.petitA}).  \sicut{Tigre} se traduit par \(\Xlo\lambda x\,\prd{tigre}(x)\) ou, pour simplifier immédiatement, \prd{tigre}.  La traduction de \sicut{petit tigre} est donc \(\Xlo[\lambda P\lambda x [[\prd{petit}(P)](x)](\prd{tigre})]\) par application fonctionnelle, puis \(\Xlo\lambda x [[\prd{petit}(\prd{tigre})](x)]\) par \breduc.  
Nous pouvons traduire \sicut{édenté} comme \sicut{petit} (\ie\ \(\Xlo\lambda P\lambda y [[\prd{édenté}(P)](y)]\)), mais puisque c'est un adjectif intersectif, nous pouvons tout aussi bien le traiter comme de type \et, \(\Xlo\lambda x\,\prd{édenté}(x)\), et par la règle de \emph{modification de prédicat} (règle \ref{ri:PM}, p.~\pageref{ri:PM}), \sicut{petit tigre édenté} se traduira par \(\Xlo\lambda y[[\lambda x [[\prd{petit}(\prd{tigre})](x)](y)]\wedge[\lambda x\,\prd{édenté}(x)(y)]]\), ce qui après \breduc\ donnera :
\(\Xlo\lambda y[ [[\prd{petit}(\prd{tigre})](y)]\wedge\prd{édenté}(y)]\).


À noter que si nous avions traduit \sicut{édenté} par \(\Xlo\lambda P\lambda y [[\prd{édenté}(P)](y)]\), nous aurions obtenu au final : \(\Xlo\lambda y [[\prd{édenté}(\lambda x [[\prd{petit}(\prd{tigre})](x)])](y)]\) (ou \(\Xlo\lambda y [[\prd{édenté}([\prd{petit}(\prd{tigre})])](y)]\) par $\eta$-réduction), ce qui aura les mêmes conditions de vérité que le résultat précédent si nous posons que la dénotation de \(\Xlo\lambda P\lambda y [[\prd{édenté}(P)](y)]\) renvoie l'ensemble qui est l'intersection de l'ensemble des édentés avec l'ensemble qui est la dénotation de l'argument~(\vrb P).

\fussy
\end{solu}
\end{exo}



\subsection{L'application fonctionnelle intensionnelle}
%------------------------------------------------------
\label{ss:AFInt}

Nous pouvons maintenant nous attaquer aux phrases complexes qui enchâssent une subordonnée complétive comme \Next[a] dont l'analyse syntaxique est donnée en \Next[b].  

\ex.  
\a. Charles croit qu'Alice dort.
\b. 
{\small
\Tree
[.TP
  [.NP \zcbox{Charles} ]
  [.VP
    [.V croit ]
    [.CP 
      [.C \zcbox{que} ]
      [.TP 
        [.NP Alice ]
        [.VP dort ]
      ]
    ]
  ]
]
}


Dans cette structure, nous avons introduit une nouvelle catégorie lexicale C, pour \emph{complémenteur}, qui correspond à la conjonction de subordination \sicut{que}.  Et la subordonnée complète s'analyse donc comme la projection maximale CP (ce qui permet de distinguer les phrases «simples» qui, elles, sont des TP). 

Par ailleurs nous avons vu au chapitre \ref{Ch:t+m} qu'un verbe d'attitude propositionnelle comme \sicut{croire} dénote une relation entre un individu et une \emph{proposition}.  
Une proposition est l'intension d'une formule, et cela correspond donc à une expression de type \type{s,t}, en l'occurrence $\Xlo\Intn\prd{dormir}(\cns a)$ dans notre exemple \Last, qui en entier se traduit par \Next. 

\ex.
\(\Xlo\prd{croire}(\cns c,\Intn\prd{dormir}(\cns a))\)


Le prédicat \prd{croire} est donc une constante de type \type{\type{s,t},\et}, le \lterme\ qui lui correspond étant \(\Xlo\lambda p\lambda x\,\prd{croire}(x,p)\), avec $\vrb p \in \VAR_{\type{s,t}}$. 
Et par conséquent, par compositionnalité, nous pouvons faire l'hypothèse que dans \LLast, CP, qui réalise la proposition attendue par le verbe, est de type \type{s,t}. 
Et comme le CP domine un TP qui est lui de type \typ t (comme tous les TP), nous pouvons alors en conclure que le rôle du complémenteur \sicut{que} est de «transformer» un formule de type \typ t (TP) en une proposition de type \type{s,t} (CP).  En bref, \sicut{que} ajoute l'opérateur $\Xlo\textIntn$.

Mais ici les choses sont beaucoup moins simples qu'il n'y paraît et il faut être très vigilant sur ce que peut nous inspirer le \lcalcul.  
On pourrait avoir le réflexe de traduire \sicut{que} par %le \lterme\ 
\(\Xlo\lambda \phi\, \Intn \phi\), avec \vrb\phi\ de type \typ t.
En effet ce \lterme\ \emph{semble} faire ce que nous attendons : il prend en argument une formule \vrb\phi\ et donne en résultat son intension $\Xlo\Intn\phi$. 
Mais cela n'est qu'une illusion d'optique, en réalité ce n'est pas du tout ce que fait ce \lterme. 
Pour nous en rendre compte, considérons d'abord son type : 
\type{t,\type{s,t}}. 
C'est le type des expressions qui dénotent une fonction de \set{0;1} vers $\set{0;1}^{\Unv W}$.
Donc \(\Xlo\lambda \phi\, \Intn \phi\) dénote une fonction qui prend en argument une valeur de vérité, $1$ ou $0$.  Elle doit ensuite livrer une proposition ; mais comment saura-t-elle trouver une proposition idoine à partir de la seule donnée de $1$ ou $0$ ?

\sloppy

Examinons un exemple concret.  Plaçons-nous dans un monde $\w_1$ et supposons que dans $\w_1$ Alice dort effectivement. 
Interprétons maintenant le CP dans $\w_1$ sous la présente hypothèse. 
Par composition de \sicut{que} avec TP, nous obtenons \(\Xlo[\lambda \phi\, \Intn \phi(\prd{dormir}(\cns a))]\). 
Remarquons déjà que nous n'avons pas le droit d'effectuer la \breduc\ ici, car l'argument se retrouverait sous la portée de $\Xlo\textIntn$ (cf.~\ref{ss:lambconv}). 
Mais indépendamment de cela, l'interprétation de l'expression échoue.  Par définition de l'application fonctionnelle (\RSem\ref{SemTApp}),  %nous avons
\(\denote{\Xlo[\lambda \phi\, \Intn \phi(\prd{dormir}(\cns a))]}^{\Modele,\w_1,g} = \denote{\Xlo\lambda \phi\, \Intn \phi}^{\Modele,\w_1,g}(\denote{\Xlo\prd{dormir}(\cns a)}^{\Modele,\w_1,g})\). 
Or, par hypothèse, nous savons que \(\denote{\Xlo\prd{dormir}(\cns a)}^{\Modele,\w_1,g}=1\).  Donc
\(\denote{\Xlo[\lambda \phi\, \Intn \phi(\prd{dormir}(\cns a))]}^{\Modele,\w_1,g} = \denote{\Xlo\lambda \phi\, \Intn \phi}^{\Modele,\w_1,g}(1)\). 
Le problème, nous le constatons, est que finalement nous avons perdu le contenu, \ie\ le sens, de la formule $\Xlo\prd{dormir}(\cns a)$ au cours du calcul ; peu importe le résultat que peut donner la fonction \(\denote{\Xlo\lambda \phi\, \Intn \phi}^{\Modele,\w_1,g}\), elle ne sera pas en mesure de «voir»  que son argument provient des conditions de vérité de $\Xlo\prd{dormir}(\cns a)$, 
car l'application fonctionnelle telle qu'elle est définie par (\RSem\ref{SemTApp}) ne voit que les extensions. 
En temps normal, cela fait tout à fait l'affaire ; en revanche le problème se posera crucialement dans tous les cas où l'analyse compositionnelle aura besoin d'accéder à l'intension d'une expression.\is{intension}
Pour dire les choses simplement, il n'est pas possible de passer de l'extension d'une expression à son intension au moyen d'un \lterme\ (ou de toute expression fonctionnelle catégorématique), parce que l'extension est moins informative de l'intension.

\fussy

Tout n'est pas perdu pour autant, notre système formel permet de résoudre le problème.  Mais pour cela, nous devons introduire une nouvelle règle de composition sémantique, qui s'ajoute à l'application fonctionnelle traditionnelle (et à la modification de prédicat vue en \S\ref{ss:ISSmodifieurs}). 
Cette règle ne modifie rien dans {\LO}, elle intervient en tant que règle d'interface syntaxe-sémantique.  Nous l'appellerons l'\kwo{application fonctionnelle intensionnelle}\is{application fonctionnelle!\elid\ intensionnelle} (AFI). Sa définition est donnée ci-dessous. 

\begin{defi}[Application fonctionnelle intensionnelle]\label{d:AFI}
\begin{tabular}[t]{@{}rccc} 
X &\reecr& Y& Z\\
\small\mtyp a && \small\mtype{\mtype{\typ s,b},a} & \small\mtyp b\\
$\Xlo[\alpha(\Intn\beta)]$&\seecr &$\Xlo\alpha$ & $\Xlo\beta$\\
\end{tabular}

\vspace{-1.5ex}~
\end{defi}


Cette règle dit que si un constituant Y attend comme argument une expression intensionnelle de type \mtype{\typ s,b} et qu'il se combine avec un constituant Z de type \mtyp b, alors la composition sémantique peut se faire par application fonctionnelle sur le n\oe ud supérieur X mais en prenant l'intension de l'argument ($\Xlo\Intn\beta$).

Cette règle ne se heurte pas au problème que nous venons de voir, car c'est une règle d'interface. Et ces règles n'opèrent pas sur les extensions mais sur les expressions de {\LO}, qui --~rappelons-le~-- représentent le sens des constituants.  Même si \vrb\beta\ (ou Z) dénote en soi une valeur de vérité $1$ ou $0$, la dénotation de X ne dépendra pas de cette valeur mais de la dénotation de $\Xlo\Intn\beta$.

Pour nous en convaincre, reprenons notre exemple. 
En utilisant l'AFI, nous obtiendrons un CP de type \type{s,t} à partir d'un TP de type \typ t du moment que nous posons que le complémenteur \sicut{que} est de type \type{\type{s,t},\type{s,t}}.  
Il dénotera ainsi une fonction qui exige que son argument soit de type \type{s,t} ; il ne se combinera donc pas avec la dénotation de TP mais avec son intension.  À part ça, que fait cette fonction ? Rien. Elle attend une proposition et la redonne telle quelle :

\ex.
\(\sicut{que}\leadsto \Xlo\lambda p\, p\), avec $\vrb p\in \VAR_{\type{s,t}}$


Le rôle de \sicut{que} est donc juste, par son type, de forcer l'utilisation de l'application fonctionnelle intensionnelle. 
La dérivation sémantique du CP est illustrée en \Next.

\largerpage[-1]

\ex.
\Tree
[.CP\zbox{\ $\xlo{[\lambda p\, p(\Intn\prd{dormir}(\cns a))]}=\xlo{\Intn\prd{dormir}(\cns a)}$}
  [.C que\\{$\Xlo\lambda p\, p$} ]
  \qroof{\pile{{Alice dort}\\\zcbox{$\Xlo\prd{dormir}(\cns a)$}}}.TP
]


Le CP pourra ensuite se combiner avec le verbe \sicut{croire} comme avec la règle des verbes transitifs \ref{ri:VT} (en ajustant les types en conséquence).
L'AFI de la définition \ref{d:AFI} est une règle générale. 
La règle d'interface précise qui concerne notre CP\footnote{Attention, cette règle ne prétend pas que \emph{tous} les CP seront de type \type{s,t} ; elle ne concerne que les complétives, c'est-à-dire les CP dont la tête C est de type \type{\type{s,t},\type{s,t}}. } est la suivante :

\ex.
\RISS{CP}%
{\begin{tabular}[t]{rccc}
    CP & \reecr & {C} & TP\\
    \small\type{s,t} && \small\type{\type{s,t},\type{s,t}} &  \small\typ t \\
    $\Xlo[\alpha(\Intn\phi)]$
    &\seecr & $\Xlo\alpha$ &$\Xlo\phi$
  \end{tabular}}


Notons qu'à la place, nous aurions pu poser une règle ad hoc (sans AFI) qui ignore toute contribution sémantique du complémenteur et fait simplement remonter l'intension du TP au niveau de CP%
\footnote{C'est-à-dire cette règle : \fsynsem{\begin{tabular}[t]{rccc}
    CP & \reecr & {C} & TP\\
    \type{s,t} &&  &  \typ t \\
    $\Xlo\Intn\phi$
    &\seecr & &$\Xlo\phi$
  \end{tabular}}}.
Mais il est bien préférable de maintenir la règle générale, car l'AFI peut s'utiliser dans de nombreux cas\footnote{En fait, on peut même très bien considérer que l'AFI devrait s'utiliser dans \emph{tous} les cas. Ce serait parfaitement approprié car cela impliquerait que la composition sémantique se fait systématiquement en combinant le \emph{sens} de constituants et pas seulement leur dénotation. Et c'est d'ailleurs précisément ce que fait \citet{PTQ}\Andex{Montague, R.}.}.


\sloppy

Elle sera, par exemple, requise pour traiter les adjectifs intensionnels\is{adjectif!\elid\ intensionnel} comme \sicut{présumé}, \sicut{prétendu}, \sicut{supposé}... que nous avons rencontrés 
\S\ref{ss:ModNonExt}.
Ces adjectifs n'ont normalement pas d'emploi prédicatif (de type \et), mais leur type n'est pas non plus \type{\et,\et}, car ils doivent se combiner avec l'intension du prédicat nominal qu'ils modifient.  Leur type est donc \type{\type{s,\et},\et}.  
Si \vrbS P est une variable de propriété, de type \type{s,\et}, alors \sicut{présumé} se traduira par \(\Xlo\lambda\vrbS P\lambda x [[\prd{présumé}(\vrbS P)](x)]\)\footnote{$\Xlo[[\prd{présumé}(\vrbS P)](x)]$ peut se simplifier en $\Xlo\prd{présumé}(x,\vrbS P)$, mais dans le cas présent, il est plus naturel de garder l'écriture originale (ou $\Xlo[\prd{présumé}(\vrbS P)](x)$) pour montrer que l'adjectif sert bien à modifier le prédicat nominal.}. 
Sans trop entrer dans les détails de la sémantique lexicale de \sicut{présumé}, nous pouvons dire que son interprétation se rapproche de celle des modalités. L'idée est que l'adjectif est associé à une certaine relation d'accessibilité (de «présomption») et $\Xlo[[\prd{présumé}(\vrbS P)](x)]$ est vraie dans un monde donné ssi dans tous les mondes accessibles, $\Xlo[\Extn\vrbS P(x)]$ est vraie.
Même si l'interprétation revient à l'extension de \vrbS P (avec $\Xlo\textExtn$) pour permettre la composition avec \vrb x, il est nécessaire de transmettre à l'adjectif l'intension du prédicat nominal, précisément pour pouvoir évaluer $\Xlo[\Extn\vrbS P(x)]$ dans différents mondes.  
Le NP \sicut{présumé coupable} s'analysera comme en \Next, ce qui peut se simplifier en 
$\xlo{\lambda x [[\prd{présumé}(\Intn\prd{coupable})](x)]}$
par $\eta$-réduction.

\fussy

\ex.
\hstrab[5em]{\small \Tree
[.N$'$\zzbox[tl]{\ \pilel{$\xlo{[\lambda\vrbS P\lambda x [[\prd{présumé}(\vrbS P)](x)](\Intn\lambda y\,\prd{coupable}(y))]}$\\\hstrab[2em] = $\xlo{\lambda x [[\prd{présumé}(\Intn\lambda y\,\prd{coupable}(y))](x)]}$}}
  [.AP \piler{présumé\\\zrbox{$\Xlo\lambda\vrbS P\lambda x [[\prd{présumé}(\vrbS P)](x)]$}} ]
  [.N$'$ \pilel{coupable\\\zbox{$\Xlo\lambda y\,\prd{coupable}(y)$}} ]
]}


%\fixme{***verbes d'état}
Il se trouve que les verbes copules lexicalement plein comme en \Next\ auront une analyse tout à fait similaire, du fait de leur dimension soit modale (\sicut{paraître}, \sicut{sembler}, \sicut{passer pour}, \sicut{s'avérer}...) soit aspectuo-temporelle (\sicut{rester}, \sicut{devenir}, \sicut{se retrouver}...). 

\ex. Alice paraît fatiguée.

En effet en \Last, \sicut{paraître} n'établit pas vraiment, dans un monde $w$, une relation entre l'individu \Obj{Alice} et l'ensemble des individus fatigués de $w$ (d'ailleurs \Obj{Alice} n'en fait peut-être pas partie). 
Il est plus raisonnable d'y voir une relation entre un individu et le sens de \sicut{fatigué}.  À l'instar d'un adjectif intensionnel, le verbe se traduira donc par  \(\Xlo\lambda\vrbS P\lambda x [[\prd{paraître}(\vrbS P)](x)]\) de type \type{\type{s,\et},\et}.
Notons, par la même occasion, que si nous voulons unifier l'analyse de tous les verbes copules, il est alors envisageable d'assigner aussi le type \type{\type{s,\et},\et} à \sicut{être} qui se traduira par \(\Xlo\lambda\vrbS P\lambda x [\Extn\vrbS P(x)]\).


Et l'AFI devra également être utilisée pour introduire compositionnellement les modalités, pour les mêmes raisons que précédemment.  Supposons que nous voulons traduire l'adverbe \sicut{nécessairement} par $\Xlo\doitn{n}$ et que cet adverbe s'adjoint au TP.  
Là encore il ne devra pas se traduire par $\Xlo\lambda\phi\,\doitn{n}\phi$ (de type \type{t,t}, qui dénote une simple fonction de vérité) mais par $\Xlo\lambda p\,\doitn{n}\Extn p$ de type \type{\type{s,t},t}. 
Ce \lterme\ dénote une fonction qui prend en argument une proposition (\ie\ un ensemble de mondes) et qui renvoie $1$ ssi cette proposition est vraie dans tous les mondes accessibles au monde d'évaluation (\ie\ ssi tous les mondes accessibles appartiennent à l'ensemble de mondes dénoté par la proposition). 
L'insertion de l'opérateur $\Xlo\textExtn$  est nécessaire pour obtenir une expression de type \typ t, car la règle (\RSyn\ref{SynTMod}) dit que $\Xlo\doitn n$ ne peut se combiner qu'avec une expression de ce type.


\section{Groupes nominaux et déterminants}
%=========================================
\label{ss:iss:Qu}
\is{groupe nominal|(}

Il est temps de nous attaquer sérieusement à l'analyse sémantique compositionnelle des groupes nominaux.  
Depuis le début du chapitre nous avons adopté une hypothèse d'analyse trop simple, qui ne s'applique qu'aux expressions référentielles et qui n'est pas généralisable à tous les \GN.  Nous allons ici réviser profondément cette analyse pour aboutir à un traitement unifié de toutes les catégories de \GN\ que nous avons vues au chapitre \ref{ch:gn}, et qui reprend, dans ses grandes lignes, la proposition de \citet{PTQ}.\Andex{Montague, R.} 

Par la même occasion, profitons-en tout de suite pour réviser aussi l'analyse syntaxique de ces constituants. 
Nous allons considérer que la tête syntaxique de ce que nous continuerons à appeler un groupe nominal n'est pas son N mais son  déterminant, auquel nous associons la catégorie D.  
Un groupe nominal sera donc dorénavant la projection maximale DP.  Mais les NP ne disparaissent pas pour autant : ils recouvrent le matériau nominal et ses «satellites» (excepté le déterminant) et ils s'intègrent comme compléments de D sous la projection D$'$. 

%\ex. \(\sicut{le}/\sicut{la} \leadsto \Xlo\lambda P \atoi x[P(x)]\), de type \type{\et,e}.

\subsection{Une analyse générale des groupes nominaux}
%-----------------------------------------------------
\label{ss:QG.1}

Jusqu'ici nous avons manipulé les groupes nominaux comme des expressions de type~\typ e.  Et cela se reflétait naturellement dans le type des verbes qui prennent des \GN\ en arguments : \et\ pour les intransitifs, \eet\ pour les transitifs, \type{e,\eet} pour les ditransitifs. 
Le type \typ e fait l'affaire pour les noms propres (constantes d'individus), les pronoms personnels (variables d'individus) et les définis (\atoi-termes).  Mais il ne convient pas pour les groupes nominaux quantificationnels (\sicut{tous les enfants}, \sicut{la plupart des enfants}...) et indéfinis (\sicut{un enfant}, \sicut{plusieurs enfants}...). Car si ceux-ci étaient de type \typ e, cela voudrait dire qu'ils dénotent un individu particulier du modèle, ce qui n'a évidemment pas de sens pour les quantificationnels\footnote{À cet égard nous étions allés jusqu'à dire que les \GN\ quantificationnels ne dénotent pas (du moins pas au même titre que les expressions référentielles).  Nous allons voir ici qu'ils ont néanmoins une dénotation.}, mais également pour les indéfinis car nous avons vu au chapitre \ref{ch:gn} qu'un indéfini comme \sicut{un enfant} permet a priori de consulter dans le modèle \emph{n'importe quel} individu qui est un enfant (et donc pas un individu particulier). 


En fait il est assez facile de dévoiler le type et la représentation sémantique propre d'un DP comme \sicut{tous les enfants} ou \sicut{un enfant} à partir du moment où nous connaissons par avance la traduction dans {\LO} d'une phrase où il intervient. C'est comme une simple équation à une inconnue.
Prenons par exemple la phrase \Next[a] dont nous savons que la traduction devra être \Next[b].
 
\ex.
\a. Tous les enfants dorment. \label{x:tled1}
\b. \(\Xlo\forall x [\prd{enfant}(x)\implq\prd{dormir}(x)]\)\label{x:tled2}

Globalement la phrase est l'assemblage d'un DP sujet et d'un VP. Donc  si dans \Last[a] nous retranchons le VP (\sicut{dorment}), il nous restera... le DP (\sicut{tous les enfants}). 
Par parallélisme, nous pouvons faire la même chose dans la formule \Last[b] car nous connaissons la contribution sémantique du VP : c'est le prédicat $\Xlo\lambda x\,\prd{dormir}(x)$ c'est-à-dire \prd{dormir} (par $\eta$-réduction). 
Donc si dans \Last[b] nous enlevons le prédicat verbal, mécaniquement nous obtiendrons la contribution sémantique du DP sujet. Allons-y, enlevons \prd{dormir} ; ça nous donne : \(\Xlo\forall x [\prd{enfant}(x)\implq\xbox{dormir}{}(x)]\).
Évidemment ce n'est pas une expression bien formée de {\LO}, mais ça nous montre que la traduction du DP correspond à la formule \Last[b] dans laquelle \emph{il manque} le prédicat verbal.  Or depuis le chapitre précédent nous savons représenter des places vides dans {\LO} : avec la \lamb-abstraction.  
Donc formellement la traduction sémantique de \sicut{tous les enfants} est {\Next} :

\ex.
\(\sicut{tous les enfants} \leadsto \Xlo\lambda P\forall x [\prd{enfant}(x)\implq[P(x)]]\)\label{l:touslesenf}


De même le DP \sicut{un enfant} se traduit par :

\ex.
\(\sicut{un enfant} \leadsto \Xlo\lambda P\exists x [\prd{enfant}(x)\wedge[P(x)]]\)\label{l:unenf}


Nous pouvons alors constater immédiatement une répercussion importante pour l'interface syntaxe-sémantique. Ces DP se traduisent par des \lterme s de  type \ett, et ils se combinent avec des VP de type \et.  Par conséquent, et contrairement à ce qui était présenté dans la règle \ref{v:ri:TP1} p.~\pageref{v:ri:TP1}, dans la composition sémantique de [\Stag{TP} DP VP], c'est le DP sujet qui dénote la fonction et le VP qui dénote son argument.  C'est normal, les \lterme s \ref{l:touslesenf} et \ref{l:unenf} sont des expressions qui attendent un prédicat verbal pour former une formule.
La règle d'interface à utiliser doit donc être la suivante :

\ex. \RISS{TP}%
{\begin{tabular}[t]{rccc}
    TP & \reecr & DP &VP\\
    \small\typ t && \small\ett & \small\et \\
    $\Xlo[\alpha(\beta)]$ &\seecr & $\Xlo\alpha$ &$\Xlo\beta$
  \end{tabular}}\label{v:ri:TP2}


Et c'est bien ce que confirme la dérivation de \ref{x:tled1} donnée en \Next.

\ex.
[\Stag{DP} Tous les enfants] [\Stag{VP} dorment] 
\a.[$\leadsto$] \(\Xlo[\lambda P\forall x [\prd{enfant}(x)\implq[P(x)]](\lambda y\,\prd{dormir}(y))]\)
\b.[$=$] \(\Xlo\forall x [\prd{enfant}(x)\implq[\lambda y\,\prd{dormir}(y)(x)]]\) \hfill (\breduc\ sur \vrb P)
\b.[$=$] \(\Xlo\forall x [\prd{enfant}(x)\implq\prd{dormir}(x)]\)
\hfill (\breduc\ sur \vrb y)


Plutôt que de faire coexister les règles \ref{v:ri:TP1} et \ref{v:ri:TP2}, nous allons adopter, dans l'esprit, l'approche de \citet{PTQ}\Andex{Montague, R.} en généralisant «au pire des cas» (\ie\ au cas le plus complexe) en considérant que \emph{tous} les DP sont de type \ett. 
Car nous pouvons très bien également formaliser les expressions référentielles par des \lterme s de type {\ett} : par exemple \ref{l:qgER} pour un nom propre et \ref{l:qgDef} pour un défini. 

\ex.
\a. \(\sicut{Alice} \leadsto \Xlo\lambda P [P(\cns a)]\)\label{l:qgER}
\b. \(\sicut{le crayon} \leadsto \Xlo\lambda P [P(\atoi x\,\prd{crayon}(x))]\)\label{l:qgDef}\footnote{Évidemment, comme d'habitude, une traduction plus précise devra être \(\Xlo\lambda P [P(\atoi x[\prd{crayon}(x)\wedge C(x)])]\).}


D'un point de vue sémantique, les DP, étant de type \ett, dénotent une fonction qui prend en argument la dénotation d'un prédicat unaire et renvoie une valeur de vérité.  Pour simplifier ici j'appellerai la dénotation d'un prédicat une propriété extensionnelle.  La dénotation d'un DP est donc la fonction caractéristique d'\emph{un ensemble de propriétés extensionnelles}.
En effet \(\Xlo\lambda P [P(\cns a)]\) dénote dans $w$ l'ensemble de toutes les propriétés vérifiées par \Obj{Alice} dans $w$.  
Auparavant nous formulions les conditions de vérité de \sicut{Alice dort} en disant que l'individu \Obj{Alice} appartient à l'ensemble des dormeurs ; à présent en composant \(\Xlo[\lambda P [P(\cns a)](\lambda y\,\prd{dormir}(y))]\)\label{p.PaV} les conditions de vérité de la phrase disent littéralement que la propriété (extensionnelle) de dormir appartient à l'ensemble de propriétés que possède Alice. 
Au bout du compte cela revient au même (heureusement), mais l'information sémantique est organisée et conditionnée différemment, d'une manière qui s'avère très pertinente dans certains contextes. 

Supposons que l'on pose la question «qui est Alice ?». Nous n'étudions pas ici la sémantique des phrases interrogatives, mais nous pouvons raisonnablement  considérer que celle-ci s'interroge sur la dénotation du nom \sicut{Alice} (et donc sur l'identité d'Alice).  Il y a deux manières d'y répondre.  On peut désigner l'individu \Obj{Alice} en disant «c'est elle» (ou Alice se manifeste en disant «c'est moi»). 
Dans ce cas \sicut{Alice} a été traité comme une expression de type \typ e puisque sa dénotation a été donnée en présentant un individu de \Unv A. 
Mais il y a une autre façon de comprendre la question et donc d'y répondre, en particulier si on sait déjà qu'\sicut{Alice} désigne \Obj{Alice} (cf.\ «je sais que c'est elle, Alice, mais je veux savoir \emph{qui} elle est»). 
C'est le cas où l'on répond par une \emph{description} d'Alice, un portrait parlé, une biographie, un {curriculum vit\ae}, etc.\ autrement dit en fournissant un ensemble de propriétés satisfaites par Alice.
C'est donc le cas où \sicut{Alice} est traité comme étant de type \ett.  
Plus la réponse fournira de propriétés d'Alice, plus elle permettra de l'identifier précisément.
Car \Obj{Alice} est absolument le seul individu de \Unv A à satisfaire \emph{toutes} les propriétés extensionnelles de l'ensemble  dénoté par $\Xlo\lambda P[P(\cns a)]$\footnote{Ne serait-ce que parce que cet ensemble contient notamment la propriété exprimée par $\Xlo\lambda x[x=\cns a]$, \ie\ la propriété d'être \Obj{Alice}.}.
Comme une propriété extensionnelle s'assimile à un ensemble d'individus, la dénotation de $\Xlo\lambda P[P(\cns a)]$ peut donc s'assimiler à un ensemble d'ensembles d'individus, ce qu'illustre la figure \ref{f:QGA}.


\begin{figure}[h!]
\begin{center}
\scalebox{.8}{%
\begin{pspicture}(-4,-2.5)(4,2.5)
\cnode*(0,0){.07}{A}
\rput(3,2){\rnode{Alice}{\Obj{Alice}}}
\ncline[nodesepA=3pt]{Alice}{A}
%
\psset{linewidth=.6pt}
\rput(0,0){\psellipse(.3,.2)}
\rput{8}(0,0){\psellipse(1.1,2.5)}
\rput{10}(0,0.2){\psellipse(2,1.3)}
\rput{-5}(-1,-.4){\psellipse(2.6,1.1)}
\rput(.4,0){\psellipse(1.2,.6)}
\rput{30}(-1,-.8){\psellipse(1.7,.8)}
\rput{60}(1.1,-.6){\psellipse(1.1,2.4)}
\end{pspicture}
%
}
\end{center}
\caption{L'ensemble des propriétés d'Alice}\label{f:QGA} %(\(\denote{\lambda P
%    [P(a)]}^{\Modele,w,g}\))} 
\end{figure}


De la même façon, la dénotation de \sicut{tous les enfants}, \ie\ \(\Xlo\lambda P\forall x [\prd{enfant}(x)\implq[P(x)]]\) sera vue comme l'ensemble de toutes les propriétés extensionnelles que possèdent tous les enfants.  C'est-à-dire l'ensemble de tous les ensembles qui incluent l'ensemble de tous les enfants (qui est $\FI(w,\prd{enfant})$), cf. fig.~\ref{f:QGAx}.


\begin{figure}[h!]
\begin{center}
\scalebox{.8}{%
\begin{pspicture}(-4,-2.5)(4,2.5)
\pnode(0,0){E}
\rput(3,2){\rnode{enfs}{{$\FI(w,\prd{enfant})$}}}
\ncline[nodesepA=3pt]{enfs}{E}
\rput(0,0){\rnode{E}{\psellipse[fillstyle=hlines*,hatchcolor=gray,hatchsep=1.8pt](1,.7)}}
%
\psset{linewidth=.6pt}
%\rput(0,0){\psellipse(.3,.2)}
\rput{8}(0,0){\psellipse(1.2,2.5)}
\rput{10}(0.1,0){\psellipse(2,1.3)}
%\rput{-5}(-1,0){\psellipse(2.6,1.1)}
%\rput(.4,0){\psellipse(1.2,.6)}
%\rput{30}(-1,-.8){\psellipse(1.7,.8)}
\rput{60}(-.2,.2){\psellipse(1.4,2.4)}
\rput{15}(-.8,-.4){\psellipse(2.2,1.5)}
\end{pspicture}
%
}
\end{center}
\caption{L'ensemble des propriétés de tous les enfants}\label{f:QGAx} %
%  (\(\denote{\lambda P \forall x [\prd{tigre} \implq P(x)]}^{\Modele,w,g}\))}
\end{figure}


Et la dénotation de \sicut{un enfant}, \(\Xlo\lambda P\exists x [\prd{enfant}(x)\wedge[P(x)]]\), sera vue comme l'ensemble de toutes les propriétés extensionnelles qui sont vérifiées par au moins un enfant. C'est-à-dire l'ensemble de tous les ensembles qui ont une intersection non vide avec l'ensemble de tous les enfants, cf.\ fig.~\ref{f:QGEx}.  La dénotation de l'indéfini \sicut{un enfant} contient donc beaucoup plus d'ensembles que celle de \sicut{tous les enfants} puisqu'il suffit qu'un ensemble contiennent ne serait-ce qu'un enfant pour qu'il en fasse partie.


\begin{figure}[h!]
\begin{center}
\scalebox{.8}{%
\begin{pspicture}(-4,-2.5)(4,2.5)
\pnode(0,0){E}
\rput(3,2){\rnode{enfs}{{$\FI(w,\prd{enfant})$}}}
\ncline[nodesepA=3pt]{enfs}{E}
\rput(0,0){\rnode{E}{\psellipse[fillstyle=hlines*,hatchcolor=gray,hatchsep=1.8pt](1,.7)}}
%
\psset{linewidth=.6pt}
%\rput(0,0){\psellipse(.3,.2)}
\rput{8}(0,0){\psellipse(1.2,2.5)}
\rput{10}(0.1,0){\psellipse(2,1.3)}
%\rput{-5}(-1,0){\psellipse(2.6,1.1)}
%\rput(.4,0){\psellipse(1.2,.6)}
%\rput{30}(-1,-.8){\psellipse(1.7,.8)}
\rput{60}(-.2,.2){\psellipse(1.4,2.4)}
\rput{15}(-.8,-.4){\psellipse(2.2,1.5)}
%
\rput(0.1,.35){\psellipse(0.22,0.15)}
\rput(0.5,-0.1){\psellipse(0.22,0.15)}
\rput(-0.66,0.05){\psellipse(0.22,0.15)}
\rput(-0.1,-.3){\psellipse(0.22,0.15)}
%
\rput{-10}(1.55,-.55){\psellipse(2,0.65)}
\rput{-45}(.31,.12){\psellipse(.6,0.35)}
\rput{-36}(1.2,0.1){\psellipse(1.5,1.8)}
\rput{10}(-2.05,-0.2){\psellipse(1.7,0.65)}
\end{pspicture}
%
}
\end{center}
\caption{L'ensemble des propriétés d'un enfant}\label{f:QGEx}
%  (\(\denote{\lambda P \exists x [\prd{tigre} \wedge P(x)]}^{\Modele,w,g}\))}
\end{figure}


Un DP analysé de cette façon sous la forme d'une expression de type \ett\ est %parfois
appelée un \kwo{quantificateur généralisé}\is{quantificateur!\elid\ generalise@\elid\ généralisé}.  Nous y reviendrons de façon plus approfondie et étendue en  \S\ref{s:QG}.
Cette généralisation au type \ett\ de tous les groupes nominaux présente certains avantages (au-delà de l'uniformité de type pour la catégorie syntaxique  DP) ; par exemple, nous verrons que cela facilite la coordination de DP.  Mais nous verrons aussi, en \S\ref{s:typeshift}, qu'en ce qui concerne les expressions référentielles, l'alternative entre les types \typ e et \ett\ reste une option disponible.
\is{groupe nominal|)}

\medskip

% -*- coding: utf-8 -*-
\begin{exo}\label{exo:6VPf}
Nous traitons uniformément tous les DP comme étant de type \ett.  
\pagesolution{crg:6VPf}
Supposons maintenant que nous tenions à tout prix à maintenir que dans la composition du TP, le VP continue à dénoter la fonction et que le DP sujet soit son argument.  Quels devraient être alors le type et la traduction d'un verbe comme \sicut{dormir}?
\begin{solu}(p.~\pageref{exo:6VPf})\label{crg:6VPf}

\sloppy

Si  nous voulons que les VP soient la fonction principale de la phrase, alors ceux-ci devront être de type \type{\ett,t}, attendant un DP sujet de type \ett\ pour produire une expression de type \typ t.  \sicut{Dormir} se traduira alors par \(\Xlo\lambda X[X(\lambda x\,\prd{dormir}(x))]\), avec $\vrb X\in\VAR_{\ett}$.
Ainsi nous pourrons dériver la traduction de \sicut{Alice dort} en composant 
\(\Xlo[\lambda X[X(\lambda x\,\prd{dormir}(x))](\lambda P[P(\cns a)])]\) qui, par \breduc s, se simplifie en 
\(\Xlo[\lambda P[P(\cns a)](\lambda x\,\prd{dormir}(x))]\)\footnote{Ce qui nous fait revenir évidemment à ce dont nous partons p.~\pageref{p.PaV}.}, 
puis en 
\(\Xlo[\lambda x\,\prd{dormir}(x)(\cns a)]\) et
\(\Xlo\prd{dormir}(\cns a)\).

\fussy
\end{solu}
\end{exo}



\subsection{Les déterminants}
%----------------------------
\label{ss:déterminants}

\is{determinant@déterminant}
Maintenant que nous connaissons la traduction sémantique des DP, nous pouvons facilement déduire, toujours par «soustraction», la traduction des déterminants : un déterminant est un DP \emph{sans} son NP.  Le NP correspond à un prédicat nominal de type \et\ que nous pouvons \lamb-abstraire avec une variable \vrb Q de $\VAR_{\et}$.  Ainsi à partir de \ref{l:touslesenf}, \ref{l:unenf} et \ref{l:qgDef}, nous obtenons : 

\ex.
\a. \(\sicut{tous les} \leadsto \Xlo \lambda Q \lambda P \forall x [[Q(x)] \implq [P(x)]]\)
\b. \(\sicut{un} \leadsto \Xlo \lambda Q \lambda P \exists x [[Q(x)] \wedge [P(x)]]\)
\b. \(\sicut{le} \leadsto \Xlo \lambda Q \lambda P  [P(\atoi x[Q(x)])]\)


Un déterminant est donc de type \type{\et,\ett}.  C'est une expression qui attend deux prédicats (celui du NP puis celui du VP) pour produire une formule.
Et sémantiquement, il dénote une fonction qui renvoie une valeur de vérité quand on lui a fourni en arguments deux propriétés extensionnelles.  Nous avions vu au chapitre précédent que cela peut également se voir comme une \emph{relation} entre deux propriétés, c'est-à-dire entre deux ensembles d'individus (cf. théorème~\ref{th:aat} p.~\pageref{th:aat}). 
En effet, en termes ensemblistes et informels, \Last[a] est une fonction qui attend deux ensembles \vrb Q et \vrb P et qui répond $1$ ssi tous les éléments qui sont dans \vrb Q sont aussi dans \vrb P : c'est la relation d'inclusion.
De même, \Last[b] est la relation d'intersection non vide (il y a au moins un élément qui est dans \vrb Q et dans \vrb P).  
Quant à \Last[c] c'est aussi la relation d'inclusion, mais à condition que \vrb Q contienne un et un seul élément.
Et tout cela n'est qu'un écho à ce que nous avions vu au chapitre \ref{ch:gn}
(\S\ref{sss:SyntheseGN}, p.~\pageref{x:QEns1}).

La règle d'interface syntaxe-sémantique des DP est donc la suivante :

\ex. \RISS{DP}%
{\begin{tabular}[t]{rccc}
    DP & \reecr & D &NP\\
    \small\ett && \small\type{\et,\ett} & \small\et \\
    $\Xlo[\alpha(\beta)]$ &\seecr & $\Xlo\alpha$ &$\Xlo\beta$
  \end{tabular}}\label{v:ri:DP}


Ajoutons que c'est très certainement à cet endroit qu'il convient d'introduire compositionnellement les restrictions contextuelles des domaines de quantification,\is{restriction!\elid\ du domaine de quantification} ces variables \vrb C présentées en \S\ref{ss:RestrDQuant}.  Ainsi, pour être complet, un déterminant comme, par exemple, \sicut{tous les} se traduira par \(\Xlo \lambda Q \lambda P \forall x [[[Q(x)]\wedge [C(x)]] \implq [P(x)]]\), où \vrb C est une variable \emph{libre} de type \et.
Dans ce qui suit, nous ne ferons pas apparaître ces restrictions dans les notations, afin simplement de ne pas  surcharger ces dernières. 
Il faut également savoir qu'en toute rigueur, cela ne règle pas entièrement le traitement des restrictions. Car, {a priori}, il n'est pas déraisonnable de supposer que chaque déterminant d'une phrase (et même d'un discours) introduira sa propre restriction \vrbi Ci ; il est donc nécessaire de prévoir un moyen,  à l'interface syntaxe-sémantique, de renouveler la variable \vrbi Ci  à chaque fois qu'un déterminant est intégré dans l'analyse.  En fait, un tel mécanisme est similaire à celui de la traduction compositionnelle des pronoms et que nous évoquerons au chapitre~\ref{Ch:contexte} (vol.~2).

Pour récapituler, voici en \Next\ la dérivation sémantique complète d'une phrase comme \sicut{tous les enfants dorment}, résumée également en figure~\ref{f:Stled} (page suivante). 

\ex.
\a. [\Stag{DP} [\Stag D Tous les] [\Stag{NP} enfants]] 
\a.[$\leadsto$]
\(\Xlo[\lambda Q\lambda P\forall x[[Q(x)] \implq [P(x)]](\lambda y\,\prd{enfant}(y))]\)
\b.[$=$]\(\Xlo\lambda P\forall x[[\lambda y\,\prd{enfant}(y)(x)] \implq [P(x)]]\)
\b.[$=$]\(\Xlo\lambda P\forall x[\prd{enfant}(x) \implq [P(x)]]\)
\z.
\b. [\Stag{TP} [\Stag{DP} Tous les enfants] [\Stag{VP} dorment]]
\a.[$\leadsto$]
\(\Xlo[\lambda P\forall x[\prd{enfant}(x) \implq [P(x)]](\lambda y\,\prd{dormir}(y))]\)
\b.[$=$]
\(\Xlo\forall x[\prd{enfant}(x) \implq [\lambda y\,\prd{dormir}(y)(x)]]\)
\b.[$=$]
\(\Xlo\forall x[\prd{enfant}(x) \implq \prd{dormir}(x)]\)

\begin{figure}[h!]
\begin{center}
{\small
\Tree
[.TP\zbox{ $\Xlo\forall x[\prd{enfant}(x) \implq \prd{dormir}(x)]$}
  [.\zrbox{$\Xlo\lambda P\forall x[\prd{enfant}(x) \implq [P(x)]]$ }DP
    \qroof{\pile{tous les\\\xbox[r]{tous les }{$\Xlo\lambda Q\lambda P\forall x[[Q(x)] \implq [P(x)]]$}}}.D
    [.NP enfants\\\xbox[l]{enfants}{$\Xlo\lambda y\,\prd{enfant}(y)$} ]
  ]
  [.VP dorment\\\zcbox{$\Xlo\lambda y\,\prd{dormir}(y)$} ]
]
}
\caption{Dérivation sémantique de \sicut{tous les enfants dorment}}\label{f:Stled}
\end{center}
\end{figure}

Nous pouvons observer que finalement le déterminant dénote non seulement la fonction principale du groupe nominal sujet (d'ailleurs D est la tête du DP) mais aussi de toute la phrase.  C'est plutôt normal, car la phrase se traduit par une formule qui a une structure logique assez particulière : c'est une quantification universelle sur une implication, et cette structure est prédéterminée par l'emploi du déterminant \sicut{tous les}.



% -*- coding: utf-8 -*-
\begin{exo}\label{exo:6iotad}
Supposons que nous souhaitions maintenir
\pagesolution{crg:6ioatd}%
que les DP définis singuliers soient de type~\typ e.  Quels devraient alors être le type et la traduction du déterminant \sicut{le} ?
\begin{solu}(p.~\pageref{exo:6iotad})\label{crg:6iotad}

\sloppy
Pour obtenir un DP défini singulier de type \typ e, le déterminant \sicut{le} devra être de type \type{\et,e} et se traduire par \(\Xlo\lambda P\atoi x[P(x)]\).
Ainsi, \sicut{le Pape} se traduira par \(\Xlo[\lambda P\atoi x[P(x)](\lambda y\,\prd{pape}(y))]\) qui, par \breduc s, se simplifie en 
\(\Xlo\atoi x[\lambda y\,\prd{pape}(y)(x)]\) et 
\(\Xlo\atoi x\,\prd{pape}(x)\) (qui est bien de type \typ e).

\fussy
\end{solu}
\end{exo}



\subsection{Verbes transitifs}
%-----------------------------
\label{sss:Vtrans}
\is{verbe!\elid\ transitif} 

Maintenant que tous les DP sont uniformément de type \ett, nous nous
trouvons face à un problème qu'il nous faut régler : celui de la
composition d'un verbe transitif avec son complément d'objet. En
effet, nous analysons les verbes transitifs comme des prédicats à deux
arguments, et le type qu'il leur est assigné est donc \eet. 
Mais les types {\eet} et {\ett} ne sont pas compatibles entre eux, et
nous ne pouvons pas combiner un verbe transitif avec un DP objet au
moyen de l'application fonctionnelle de la règle (\RSyn\ref{SynTApp})
(p.~\pageref{SynTApp}) : aucun de ces deux types n'est entièrement
inclus dans la partie gauche de l'autre type.  


\ex.
{\small
\Tree[.TP\zbox{${}_{\typ t}$}
  [.DP\zbox{${}_{\ett}$} Alice ]
  [.VP\zbox{${}_{\et}$} [.V\zbox{${}_{\eet}$} mange ] \qroof{une glace}.{DP\zbox{${}_{\ett}$}} ]
]
}


Il y a, en pratique, plusieurs façons d'attaquer ce problème\footnote{Une tactique alternative sera d'ailleurs évoquée en \S\ref{sss:typesVT}.} ; celle que nous allons examiner ici a une mise en œuvre relativement simple et présente l'avantage de ne pas perturber le principe général des règles de composition à l'interface syntaxe-sémantique.
Elle consiste simplement à changer le type et la traduction des verbes transitifs pour les accorder avec le type \ett\ des DP objets.  Dans le VP, c'est le V qui va dénoter la fonction, puisque c'est lui qui exige un complément d'objet. Il n'attend donc pas un argument de type \typ e mais de type \ett\ pour fournir un VP de type \et.  Le type des verbes transitifs est donc \type{\ett,\et}.  La règle d'interface met ainsi en jeu une habituelle application fonctionnelle :

\ex. \RISS{Verbes transitifs}%
{\begin{tabular}[t]{rccc}
    VP & \reecr & V &DP\\
    \small\et && \small\type{\ett,\et} & \small\ett \\
    $\Xlo[\alpha(\beta)]$ &\seecr & $\Xlo\alpha$ &$\Xlo\beta$
  \end{tabular}} \label{ri:VT2}


Bien entendu, au delà de son type, ce qui importe, c'est la traduction précise que doit recevoir un V transitif.  Illustrons cela avec l'exemple du verbe \sicut{manger} qui, à présent, se traduira comme en \ref{el:manger2} :

\ex.
\(\sicut{manger} \leadsto \Xlo\lambda Y\lambda x[Y(\lambda y\,\prd{manger}(x,y))]\), avec $\vrb Y\in \VAR_{\ett}$\label{el:manger2}




Prenons le temps de décortiquer ce \lterme\ pour expliciter son fonctionnement (moins complexe qu'il n'y paraît).  
D'abord $\Xlo\lambda y\,\prd{manger}(x,y)$ est de type \et, c'est le prédicat \prd{manger} saturé (provisoirement) de son argument sujet \vrb x. \vrb Y étant une «variable de DP» de type \ett, elle joue le rôle de ce qui sera le complément d'objet et l'application $\Xlo[Y(\lambda y\,\prd{manger}(x,y))]$ prépare donc la traduction du VP par combinaison du verbe transitif avec son complément ; elle est de type \typ t. Elle ressemble à la combinaison d'un DP sujet avec un VP de la règle \ref{v:ri:TP2} p.~\pageref{v:ri:TP2}, mais ici le prédicat verbal est abstrait de son argument objet ($\Xlo\lambda y$), pas de son sujet.
Ensuite le \lterme\ \Last\ ne fait qu'abstraire les éléments qui seront rencontrés plus tard par le V : d'abord \vrb Y le DP objet puis \vrb x qui sera saturé par le sujet de la phrase
\footnote{En complément, il n'est pas inutile de savoir que cette nouvelle traduction des verbes transitifs est, en quelque sorte, l'encodage en \lcalcul\ de l'opération de \emph{composition fonctionnelle}\is{composition fonctionnellle} de la fonction dénotée par $\Xlo\lambda x\lambda y\,\prd{manger}(x,y)$ sur celle dénotée par le DP objet. La composition fonctionnelle, bien connue des mathématiciens, et notée par l'opérateur $\rond$, est définie comme suit : si $f$ et $h$ sont deux fonctions (de types appropriés), la composée $f\rond h$ est la fonction \(x\mapsto f(h(x))\). Techniquement, c'est un enchaînement de
fonctions, comme une sorte de «tuyau» qui brancherait la sortie de $h$
sur l'entrée de $f$. Il se trouve que la traduction \ref{el:manger2} est équivalente à ce que l'on pourrait noter par \(\Xlo\lambda Y[\lambda x\lambda y\,\prd{manger}(x,y)\orond Y]\) (je laisse les lecteurs motivés s'en assurer).  Cela indique que la composition fonctionnelle pourrait constituer une autre opération de composition sémantique, complémentaire de l'application fonctionnelle.  Cependant je ne développerai pas davantage cette option ici, sachant que \ref{el:manger2}, en soi, réalise la même opération.
}.


Regardons ce que cela donne pour le VP \sicut{mange une glace} :

\ex.
\(\sicut{mange une glace} \leadsto
\Xlo[\lambda Y\lambda x[Y(\lambda y\,\prd{manger}(x,y))](\lambda P\exists u[\prd{glace}(u) \wedge [P(u)]])]\)
\a.[$=$]
\(\Xlo\lambda x[\lambda P\exists u[\prd{glace}(u) \wedge [P(u)]](\lambda y\,\prd{manger}(x,y))]\)
\hfill{\small(\breduc\ sur \vrb Y)}
\b.[$=$]
\(\Xlo\lambda x\exists u[\prd{glace}(u) \wedge [\lambda y\,\prd{manger}(x,y)(u)]]\)
\hfill{\small(\breduc\ sur \vrb P)}
\b.[$=$]
\(\Xlo\lambda x\exists u[\prd{glace}(u) \wedge \prd{manger}(x,u)]\)
\hfill{\small(\breduc\ sur \vrb y)}

%\sloppy

Pour généraliser, nous pouvons maintenant poser que si \vrb\alpha\ est un prédicat de \LO\ de type \eet\  qui dénote la relation binaire de base correspondant à un verbe transitif $V$ de la langue, 
alors, dans l'analyse, la traduction de $V$ sera \(\Xlo\lambda Y\lambda x[Y(\lambda y\,\alpha(x,y))]\).

\fussy

Cette manière de formaliser les verbes transitifs est, en version simplifiée, celle de \citet{PTQ}\Andex{Montague, R.}%
\footnote{Notons, en passant, que nous pourrions aussi, si nous ne voulions pas nous embêter à traîner le \lterme\ compliqué de \ref{el:manger2}, nous donner directement une constante de type \type{\ett,\et} pour traduire \sicut{manger}. Appelons-la, par exemple, \prd{Manger} ; alors \sicut{manger} se traduirait par \(\Xlo\lambda Y\lambda x\,\prd{Manger}(x,Y)\). Nous pourrions d'ailleurs ajouter le postulat de signification : \(\Xlo\doit\forall x\forall Y[\prd{Manger}(x,Y)\ssi Y(\lambda y\,\prd{manger}(x,y))]\), pour retrouver une notation plus classique.  Mais il faut reconnaître que le gain de simplification serait assez pauvre, car \sicut{Alice mange une glace} se traduirait par \(\Xlo\prd{Manger}(\cns a,\lambda P\exists y[\prd{glace}(y)\wedge[P(y)]])\). Cependant je mentionne cette option car elle n'est pas sans rapport avec ce que propose \citet{PTQ}.\Andex{Montague, R.} }. 
Profitons-en aussi pour mentionner l'analyse qu'il donne de l'emploi dit transitif du verbe \sicut{être} évoqué précédemment dans l'exemple \ref{x:spiderman} rappelé ici en \ref{x:spiderman2}. 

\ex.
Spiderman est un super-héros. \label{x:spiderman2}


Pour traiter cet emploi ainsi que l'emploi identificationnel (cf.\ \ref{x:PP=Spiderman} \sicut{Peter Parker est Spiderman}), Montague propose :

\ex.
\(\sicut{être} \leadsto \Xlo\lambda Y\lambda x[Y(\lambda y[x=y])]\)


La structure de ce \lterme\ est similaire à celle de la traduction de \sicut{manger}.  En traduisant \sicut{un super-héros} par \(\Xlo\lambda P\exists z [\prd{superhéros}(z)\wedge[P(z)]]\), la dérivation de \ref{x:spiderman2} est :

\ex.
\a. [\Stag{VP} est [\Stag{DP} un super-héros]] 
\\ $\leadsto$ \(\Xlo[\lambda Y\lambda x[Y(\lambda y[x=y])](\lambda P\exists z [\prd{superhéros}(z)\wedge[P(z)]])]\)
\\ $=$ \(\Xlo\lambda x[\lambda P\exists z [\prd{superhéros}(z)\wedge[P(z)]](\lambda y[x=y])]\)\hfill{\small(\breduc\ sur \vrb Y)}
\\ $=$ \(\Xlo\lambda x\exists z [\prd{superhéros}(z)\wedge[\lambda y[x=y](z)]]\)\hfill{\small(\breduc\ sur \vrb P)}
\\ $=$ \(\Xlo\lambda x\exists z [\prd{superhéros}(z)\wedge[x=z]]\)\hfill{\small(\breduc\ sur \vrb y)}
\b. [\Stag{TP} Spiderman [\Stag{VP} est  un super-héros]]
\\ $\leadsto$ \(\Xlo[\lambda P[P(\cns s)](\lambda x\exists z [\prd{superhéros}(z)\wedge[x=z]])]\)
\\ $=$ \(\Xlo[\lambda x\exists z [\prd{superhéros}(z)\wedge[x=z]](\cns s)]\)\hfill{\small(\breduc\ sur \vrb P)}
\\ $=$ \(\Xlo\exists z [\prd{superhéros}(z)\wedge[\cns s=z]]\)\hfill{\small(\breduc\ sur \vrb x)}


La formule obtenue n'est pas exactement identique à ce nous avions l'habitude d'écrire pour \ref{x:spiderman2}, à savoir \(\Xlo\prd{superhéros}(\cns s)\), mais les deux sont tout à fait logiquement équivalentes. 
\Last[b] est même un peu plus compositionnelle puisqu'elle fait apparaître une quantification existentielle, comme ce qui est ordinairement impliqué dans les traductions des indéfinis.

\smallskip

% -*- coding: utf-8 -*-
\begin{exo}\label{exo:6deriv}
En adoptant les analyses présentées dans cette section, \pagesolution{crg:6deriv}%
dérivez entièrement et pas à pas la composition sémantique des  phrases :
\begin{enumerate}
\item Tous les enfants mangent une glace.
\item Peter Parker est Spiderman.
\end{enumerate}
\begin{solu}(p.~\pageref{exo:6deriv})\label{crg:6deriv}

Comme d'habitude, nous procédons à des renommage de variables pour éviter tout conflit et toute confusion.
\begin{enumerate}
\item Tous les enfants mangent une glace.
\begin{enumerate}
\item \(\sicut{une} \leadsto \Xlo\lambda Q\lambda P\exists y[[P(y)]\wedge[Q(y)]]\)
\item \(\sicut{glace} \leadsto \Xlo\lambda z\,\prd{glace}(z)\)

\item \(\sicut{une glace} \leadsto \Xlo[\lambda Q\lambda P\exists y[[P(y)]\wedge[Q(y)]](\lambda z\,\prd{glace}(z))]\)\\
\(=\Xlo\lambda P\exists y[[P(y)]\wedge[\lambda z\,\prd{glace}(z)(y)]]\)
\hfill{\small(\breduc\ sur \vrb Q)}\\
\(=\Xlo\lambda P\exists y[[P(y)]\wedge \prd{glace}(y)]\)
\hfill{\small(\breduc\ sur \vrb z)}

\item \(\sicut{mangent} \leadsto \Xlo \lambda Y\lambda v[Y(\lambda u\,\prd{manger}(v,u))]\)

\item \(\sicut{mangent une glace} \leadsto \Xlo [\lambda Y\lambda v[Y(\lambda u\,\prd{manger}(v,u))](\lambda P\exists y[[P(y)]\wedge \prd{glace}(y)])]\)\\
\(= \Xlo \lambda v[\lambda P\exists y[[P(y)]\wedge \prd{glace}(y)](\lambda u\,\prd{manger}(v,u))]\)
\hfill{\small(\breduc\ sur \vrb Y)}\\
\(= \Xlo \lambda v\exists y[[\lambda u\,\prd{manger}(v,u)(y)]\wedge \prd{glace}(y)]\)
\hfill{\small(\breduc\ sur \vrb P)}\\
\(= \Xlo \lambda v\exists y[\prd{manger}(v,y)\wedge \prd{glace}(y)]\)
\hfill{\small(\breduc\ sur \vrb u)}

\item \(\sicut{tous les} \leadsto \Xlo\lambda Q\lambda P\forall x[[Q(x)]\implq[P(x)]]\)
\item \(\sicut{enfants} \leadsto \Xlo\lambda z\,\prd{enfant}(z)\)

\item \(\sicut{tous les enfants} \leadsto \Xlo[\lambda Q\lambda P\forall x[[Q(x)]\implq[P(x)]](\lambda z\,\prd{enfant}(z))]\)\\
\(=\Xlo\lambda P\forall x[[\lambda z\,\prd{enfant}(z)(x)]\implq[P(x)]]\)
\hfill{\small(\breduc\ sur \vrb Q)}\\
\(=\Xlo\lambda P\forall x[\prd{enfant}(x)\implq[P(x)]]\)
\hfill{\small(\breduc\ sur \vrb z)}

\item \(\sicut{tous les enfants mangent une glace} \leadsto\) \\
\(\Xlo[\lambda P\forall x[\prd{enfant}(x)\implq[P(x)]](\lambda v\exists y[\prd{manger}(v,y)\wedge \prd{glace}(y)])]\)\\
\(=\Xlo\forall x[\prd{enfant}(x)\implq[\lambda v\exists y[\prd{manger}(v,y)\wedge \prd{glace}(y)](x)]]\)
\hfill{\small(\breduc\ sur \vrb P)}\\
\(=\Xlo\forall x[\prd{enfant}(x)\implq\exists y[\prd{manger}(x,y)\wedge \prd{glace}(y)]]\)
\hfill{\small(\breduc\ sur \vrb v)}
\end{enumerate}

\item Peter Parker est Spiderman.
\begin{enumerate}
\item \(\sicut{Spiderman} \leadsto \Xlo\lambda P[P(\cns s)]\)
\item \(\sicut{est} \leadsto \Xlo\lambda Y\lambda x[Y(\lambda y[x=y])]\)

\item \(\sicut{est Spiderman} \leadsto \Xlo[\lambda Y\lambda x[Y(\lambda y[x=y])](\lambda P[P(\cns s)])]\)\\
\(=\Xlo\lambda x[\lambda P[P(\cns s)](\lambda y[x=y])]\)
\hfill{\small(\breduc\ sur \vrb Y)}\\
\(=\Xlo\lambda x[\lambda y[x=y](\cns s)]\)
\hfill{\small(\breduc\ sur \vrb P)}\\
\(=\Xlo\lambda x[x=\cns s]\)
\hfill{\small(\breduc\ sur \vrb y)}

\item \(\sicut{Peter Parker} \leadsto \Xlo\lambda P[P(\cns p)]\)

\item \(\sicut{Peter Parker est Spiderman} \leadsto \Xlo[\lambda P[P(\cns p)](\lambda x[x=\cns s])]\)\\
\(=\Xlo[\lambda x[x=\cns s](\cns p)]\)
\hfill{\small(\breduc\ sur \vrb P)}\\
\(=\Xlo[\cns p=\cns s]\)
\hfill{\small(\breduc\ sur \vrb x)}
\end{enumerate}
\end{enumerate}
\end{solu}
\end{exo}


% -*- coding: utf-8 -*-
\begin{exo}\label{exo:6HK}
\sloppy
Certains auteurs,\pagesolution{crg:6HK}
par exemple \citet{HeimKratzer:97},\Index{Heim, I.}\Index{Kratzer, A.} conservent le type \eet\ pour les verbes transitifs et examinent la possibilité d'assigner aux DP  le type \type{\eet,\et} (en plus de \ett).  Quelle serait alors la traduction de \sicut{une glace} dans ce cas ?

\fussy
\begin{solu}(p.~\pageref{exo:6HK})\label{crg:6HK}

Assigner le type \type{\eet,\et} à un DP revient à prévoir que celui-ci attend un V transitif (de type \eet) pour produire un VP de type \et.  Posons la variable \vrb R  de type \eet, pour jouer le rôle du V transitif. \sicut{Une glace} se traduira alors par \(\Xlo\lambda R\lambda x\exists y[\prd{glace}(y)\wedge [[R(y)](x)]]\).
La dérivation du VP \sicut{mange une glace} sera alors la suivante (le  DP objet dénote la fonction et le V transitif est l'argument) :

\begin{enumerate}
\item[] \(\sicut{mange une glace} \leadsto
\Xlo[\lambda R\lambda x\exists y[\prd{glace}(y)\wedge [[R(y)](x)]](\lambda v\lambda u\,\prd{manger}(u,v))]\)\\
= \(\Xlo\lambda x\exists y[\prd{glace}(y)\wedge [[\lambda v\lambda u\,\prd{manger}(u,v)(y)](x)]]\)
\hfill{\small(\breduc\ sur \vrb R)}
\\
= \(\Xlo\lambda x\exists y[\prd{glace}(y)\wedge [\lambda u\,\prd{manger}(u,y)(x)]]\)
\hfill{\small(\breduc\ sur \vrb v)}
\\
= \(\Xlo\lambda x\exists y[\prd{glace}(y)\wedge \prd{manger}(x,y)]\)
\hfill{\small(\breduc\ sur \vrb u)}
\end{enumerate}

\end{solu}
\end{exo}



\sloppy

Terminons par une remarque générale sur les autres unités lexicales qui prennent des DP comme compléments.  En particulier, ce que nous venons de faire pour les verbes transitifs doit aussi concerner les verbes ditransitifs\footnote{Ainsi que les prépositions, les noms et adjectifs relationnels, etc.},\is{verbe!\elid\ ditransitif}  puisque leur complément dit indirect est un DP de type \ett\footnote{Et à l'instar de ce que nous avons suggéré p.~\pageref{à:id}, la préposition (ou marque) \sicut{à} se traduira par l'identité $\Xlo\lambda X\, X$ de type \type{\ett,\ett}. }.  Le type d'un verbe comme \sicut{donner} devra donc être \type{\ett,\type{\ett,\et}} et sa traduction
\(\Xlo\lambda Y\lambda Z\lambda x[Z(\lambda z[Y(\lambda y\,\prd{donner}(x,y,z))])]\)... 
À moins que ce ne soit  
\(\Xlo\lambda Y\lambda Z\lambda x[Y(\lambda y[Z(\lambda z\,\prd{donner}(x,y,z))])]\) ?
C'est une question qui n'est pas aussi simple à trancher qu'il n'y paraît et 
je laisse y réfléchir dans l'exercice ci-dessous.
Elle implique qu'il y a \emph{peut-être} des choix décisifs à faire dans la traduction des verbes ditransitifs. 
Mais nous allons voir dans la section suivante qu'il y a peut-être aussi un moyen d'évacuer cette question à moindre frais.

\fussy

\medskip


% -*- coding: utf-8 -*-
\begin{exo}\label{exo:6Vdit}
Quelles sont les différences sémantiques
\pagesolution{crg:6Vdit}%
 (s'il y en a) entre les \lterme s suivant ?\addtolength{\multicolsep}{-10pt}
\begin{enumerate}
\item \(\Xlo\lambda Y\lambda Z\lambda x[Z(\lambda z[Y(\lambda y\,\prd{donner}(x,y,z))])]\) 
\item \(\Xlo\lambda Y\lambda Z\lambda x[Y(\lambda y[Z(\lambda z\,\prd{donner}(x,y,z))])]\)
\item \(\Xlo\lambda Z\lambda Y\lambda x[Z(\lambda z[Y(\lambda y\,\prd{donner}(x,y,z))])]\) 
\item \(\Xlo\lambda Z\lambda Y\lambda x[Y(\lambda y[Z(\lambda z\,\prd{donner}(x,y,z))])]\)
\end{enumerate}
\begin{solu}(p.~\pageref{exo:6Vdit})\label{crg:6Vdit}

\small\noindent
1. \(\Xlo\lambda Y\lambda Z\lambda x[Z(\lambda z[Y(\lambda y\,\prd{donner}(x,y,z))])]\)
\quad 
2. \(\Xlo\lambda Y\lambda Z\lambda x[Y(\lambda y[Z(\lambda z\,\prd{donner}(x,y,z))])]\)
\\
3. \(\Xlo\lambda Z\lambda Y\lambda x[Z(\lambda z[Y(\lambda y\,\prd{donner}(x,y,z))])]\) 
\quad
4. \(\Xlo\lambda Z\lambda Y\lambda x[Y(\lambda y[Z(\lambda z\,\prd{donner}(x,y,z))])]\)
\normalsize

\sloppy

Ces quatre \lterme s attendent deux quantificateurs généralisés \vrb Y et \vrb Z.  \vrb Y correspond au complément direct du verbe (car il se combine avec une expression qui fait abstraction de \vrb y, le second argument de \prd{donner}) et \vrb Z correspond au complément d'objet indirect (datif).  Ce qui distingue, d'une part, 1 de 3 et, d'autre part, 2 de 4, c'est l'ordre de leurs \labstraction s qui reflète l'ordre dans lequel le verbe rencontre syntaxiquement ses compléments.  Quant à ce qui distingue 1 de 2 (ainsi que 3 de 4) c'est qu'en 1 le quantificateur \vrb Y se trouve dans la portée de \vrb Z, et 2 présente les portées inverses.  Si nous traduisons \sicut{donne un exercice à tous les élèves}, avec 1 (ou 3) nous obtiendrons :
\(\Xlo\lambda x \forall z[\prd{élève}(z)\implq \exists y[\prd{exercice}(y)\wedge \prd{donner}(x,y,z)]]\), et avec 2 (ou 4) : 
\(\Xlo\lambda x \exists y[\prd{exercice}(y)\wedge \forall z[\prd{élève}(z)\implq \prd{donner}(x,y,z)]]\).  Le problème est que ces deux interprétations sont possibles, et donc que nous ne devons pas choisir entre 1 et 2 (ou 3 et 4 selon l'analyse syntaxique) ; en d'autres termes nous devrions postuler une ambiguïté de traduction pour les verbes ditransitifs -- mais ce n'est pas la manière la plus efficace de procéder comme le montre la suite du chapitre.

\fussy

\end{solu}
\end{exo}




%\subsection{Mouvements et montée des quantificateurs}
\section{Vers un traitement approprié de la quantification}
%==========================================================
\label{ss:QR}

\subsection{Le \alien{quantifying-in}}  
%-------------------------------------
\label{sss:Qu-in}\is{quantifying-in@\textit{quantifying-in}|(}

Nous savons maintenant construire compositionnellement la représentation sémantique de plusieurs syntagmes et phrases.  Mais tout n'est pas encore réglé, même pour l'analyse de phrases simples.  Si, avec les règles d'interfaces dont nous disposons, nous dérivons la traduction de \Next[a], nous obtenons la formule \Next[b]. Mais nous avons suffisamment étudié la question (chapitre~\ref{ch:gn}) pour savoir que la phrase a également une autre lecture \Next[c] avec portée inversée des groupes nominaux.
Or nous ne sommes pas en mesure de dériver la formule \Next[c] à partir de \Next[a].

\ex.
\a. Tous les pompiers ont éteint un incendie.\label{x:pompiers1}
\b. \(\Xlo\forall x [\prd{pompier}(x)\implq\exists y [\prd{incendie}(y) \wedge \prd{éteindre}(x,y)]]\)
\b. \(\Xlo\exists y [\prd{incendie}(y) \wedge \forall x [\prd{pompier}(x)\implq\prd{éteindre}(x,y)]]\)


Nous sommes là face à un véritable et épineux problème de compositionnalité. 
Rappelons que le principe de compositionnalité dit que le sens d'une phrase dépend du sens de ses mots et de sa structure syntaxique.  Or dans \Last[a] il n'y pas d'ambiguïté lexicale et \emph{a priori} la phrase n'est pas non plus syntaxiquement ambiguë : elle n'a, \emph{a priori},  qu'une seule structure syntaxique\footnote{Contrairement à une phrase comme \sicut{j'ai vu un singe avec un télescope} à laquelle on peut assigner deux structures syntaxiques différentes.}, c'est \Next. C'est pour cela que nous n'obtenons que la traduction \Last[b]. 

\ex. 
{\footnotesize
\Tree
[.TP
\qroof{tous les pompiers}.DP
  [.T$'$ 
    [.T \zcbox{ont} ]
    [.VP 
      [.V \xbox{xxx}{éteint} ] 
      \qroof{un incendie}.DP ]
  ]
]
}\label{x:SyntPompiers}


Ce dont nous avons besoin pour obtenir \LLast[c] c'est un moyen de délocaliser ou de différer l'interprétation du DP \sicut{un incendie} dans l'interface syntaxe-sémantique, c'est-à-dire de l'insérer dans la traduction sémantique «plus tard» qu'au moment où on le rencontre dans la structure syntaxique.  
Or l'une des contributions les plus fondamentales de \citet{PTQ} est, justement, de poser la formalisation d'un tel mécanisme.  Celui-ci porte le nom de \kwo{quantifying-in}, et je vais présenter ici les grandes lignes de son principe, en simplifiant un peu et en l'adaptant à notre système de notations\footnote{La raison est que \citet{PTQ} formalise son composant syntaxique dans le cadre des grammaires catégorielles,\is{grammaire!\elid\ catégorielle} qui, loin d'être désuet, demande cependant, dans sa manipulation, un peu plus de pratique et de maîtrise que les grammaires syntagmatiques\is{grammaire!\elid\ syntagmatique} que nous utilisons ici. Voir par exemple \citet[chap.~4]{Gamut:2} pour une introduction à ce formalisme.}. Nous verrons ensuite (\S\ref{sss:QR} et \ref{sss:VarianteMvt}) des variantes théoriques assez communes de ce mécanisme, qui s'en distinguent par leurs formalisations syntaxiques tout en en conservant le principe sémantique.

Dans le système décrit par \citet{PTQ}, la dérivation syntaxique et sémantique d'une phrase comme \ref{x:pompiers1} peut s'obtenir (au moins) de deux façons différentes, non équivalentes.  La première consiste à analyser la phrase comme nous l'avons fait jusqu'ici, avec une structure similaire à \ref{x:SyntPompiers}, en combinant le verbe d'abord avec son complément puis avec son sujet.  
Pour obtenir la seconde dérivation, nous allons d'abord introduire une famille de symboles syntaxiques supplémentaires, de la forme \himn i, où $i$ est un indice numérique (autrement dit \himn0, \himn1, \himn2...) ; ces symboles sont de catégorie DP et fonctionnent en fait %un peu 
comme des pronoms\footnote{À cet effet, Montague utilise directement des pronoms de l'anglais, \sicut{he}$_i$ et \sicut{him}$_i$, qui occupent les mêmes positions syntaxiques que les DP pleins.  Pour nos illustrations en français, du fait du système de pronoms clitiques, il est plus pratique de manipuler ces symboles \himn i qui, par la même occasion, anticipent sur la variante de formalisation que nous verrons en \S\ref{sss:QR}.}.   À partir de là, les séquences \ref{x:Q-in1}, par exemple, deviennent des phrases (des TP) bien formées de la langue :

\ex. \label{x:Q-in1}
\a. {} [\Stag{TP} [\Stag{DP} \himn0] [\Stag{T$'$} ont éteint [\Stag{DP} un incendie]]].
\b. {} [\Stag{TP} [\Stag{DP} Tous les pompiers] [\Stag{T$'$} ont éteint [\Stag{DP} \himn1]]].
\b. {} [\Stag{TP} [\Stag{DP} \himn0] [\Stag{T$'$} ont éteint [\Stag{DP} \himn1]]].\label{x:Q-in1c}


La grammaire développée dans \citet{PTQ} comporte ensuite une règle --~celle du \alien{quantifying-in} proprement dit~-- qui construit une structure syntaxique que nous allons noter de la manière suivante : \QUIN{i}{DP}{TP}, où $i$ est là aussi un indice numérique. 
Cette règle assemble un DP et un TP déjà dérivés, et elle 
établit que  globalement cette structure est également un TP bien formé. 
Il ne s'agit pas d'une règle syntagmatique comme celles que nous avons manipulées jusqu'ici, et elle produit une structure syntaxique abstraite ; c'est pourquoi la règle comporte une clause de «linéarisation» qui ajoute que \QUIN{i}{DP}{TP} correspond en surface à la phrase TP dans laquelle on a remplacé \himn i par DP\footnote{En fait, la règle de Montague est plus élaborée : elle dit que le premier \himn i est remplacé par DP, et les suivants (s'il y en a) sont remplacés par des pronoms de forme, de nombre et de genre en accord avec DP.}.
Par exemple, \ref{x:Q-in2} se réalise en surface par \sicut{tous les pompiers ont éteint un incendie}.  

\ex. \QUIN{1}{un incendie}{tous les pompiers ont éteint \himn 1} \label{x:Q-in2}

Notons en passant que \QUIN{0}{un incendie}{tous les pompiers ont éteint \himn 1} est aussi un TP bien formé, mais qui se réalise en \sicut{tous les pompiers ont éteint \himn{1}} puisqu'il n'y a pas de \himn 0 dans la partie droite de la structure.  De plus, comme le \alien{quantifying-in} opère sur un TP déjà construit, nous pouvons appliquer la règle récursivement plusieurs fois ; ainsi en partant de \ref{x:Q-in1c}, selon l'ordre dans lequel sont exécutées les opérations, il est également possible de construire \ref{x:Q-in3a} et \ref{x:Q-in3b} qui se réaliseront elles aussi en \ref{x:pompiers1} :

\ex.
\a. \QUIN{0}{tous les pompiers}{\QUIN{1}{un incendie}{\himn 0 ont éteint \himn 1}}\label{x:Q-in3a}
\b. \QUIN{1}{un incendie}{\QUIN{0}{tous les pompiers}{\himn 0 ont éteint \himn 1}}\label{x:Q-in3b}


La partie essentielle, pour nous, du \alien{quantifying-in} est son versant sémantique, c'est-à-dire la règle d'interface qui indique comment les structures \QUIN{i}{DP}{TP} se traduisent en \LO\ à partir des traductions de DP et de TP. 
À cette fin, nous devons d'abord établir la traduction des \himn i. Ceux-ci, étant analogues à des pronoms, se traduiront au moyen de variables ; et comme ce sont des DP, le type de leur traduction sera {\ett} : ainsi chaque \himn i se traduit en \(\Xlo\lambda P[P(x_i)]\), où \vrbi xi est une variable qui porte \emph{le même indice} que \himn i. 
En répliquant les analyses que nous avons vues précédemment en \S\ref{ss:iss:Qu}, nous trouverons que \sicut{tous les pompiers ont éteint \himn 1} se traduira par \(\Xlo\forall x[\prd{pompier}(x)\implq\prd{éteindre}(x,x_1)]\) et \sicut{\himn0 ont éteint \himn 1} par \(\Xlo\prd{éteindre}(x_0,x_1)\). 

\largerpage[-1]

Ensuite la traduction des structures de \alien{quantifying-in} va exploiter crucialement les indices numériques qui les préfixent.  Si DP se traduit par \vrb\alpha\ (de type \ett) et TP se traduit par \vrb\phi\ (de type \typ t), alors \QUIN{i}{DP}{TP} se traduit par \(\Xlo[\alpha(\lambda x_i\phi)]\).  
Ce n'est pas une application fonctionnelle ordinaire, ce qui est normal puisque les types de \vrb\alpha\ et \vrb\phi\ ne sont pas compatibles. 
L'insertion de l'abstraction $\Xlo\lambda x_1$ sert non seulement à accorder les types des expressions à combiner (car $\Xlo\lambda x_1\phi$ est ainsi de type \et), mais surtout, et c'est le principal, à «libérer» la position occupée par \himn1 et lui faire jouer un rôle de «place vide», ce qui est la vocation première de ces pseudo-pronoms dans le \alien{quantifying-in}.  Cela va permettre de relier correctement le quantificateur généralisé \vrb\alpha\ à la position argumentale qui lui correspond au niveau du prédicat verbal, mais comme \vrb\alpha\ est introduit dans la dérivation \emph{après} l'autre quantificateur de la phrase, nous obtiendrons bien les portées inversées que nous attendions.  
C'est ce que montre l'analyse \ref{x:Q-in4} qui produit la traduction sémantique qui nous manquait pour \ref{x:pompiers1} :

\newpage

\ex. \label{x:Q-in4}
\a. 
 ont éteint \himn1
\a.[$\leadsto$]
\(\Xlo[\lambda Y\lambda x [Y(\lambda y\,\prd{éteindre}(x,y))](\lambda P [P(x_1)])]\)
\b.[$=$]
\(\Xlo\lambda x [\lambda P [P(x_1)](\lambda y\,\prd{éteindre}(x,y))]\)
\b.[$=$]
\(\Xlo\lambda x  [\lambda y\,\prd{éteindre}(x,y)(x_1)]\)
\b.[$=$]
\(\Xlo\lambda x  \,\prd{éteindre}(x,x_1)\)
\z.
\b.
tous les pompiers ont éteint \himn1
\a.[$\leadsto$]
\(\Xlo[\lambda P\forall u[\prd{pompier}(u)\implq [P(u)]](\lambda x  \,\prd{éteindre}(x,x_1))]\)
\b.[$=$]
\(\Xlo\forall u[\prd{pompier}(u)\implq [\lambda x  \,\prd{éteindre}(x,x_1)(u)]]\)
\b.[$=$]
\(\Xlo\forall u[\prd{pompier}(u)\implq \prd{éteindre}(u,x_1)]\)
\z.
\b.
\QUIN{1}{un incendie}{tous les pompiers ont éteint \himn 1} 
\hfill{\small(\alien{quantifying-in})}
\a.[$\leadsto$] \raggedright
\(\Xlo[\lambda P[\exists y[\prd{incendie}(y)\wedge[P(y)]]](\underline{\lambda x_1}\forall u[\prd{pompier}(u)\implq \prd{éteindre}(u,x_1)])]\)
\b.[$=$]
 \(\Xlo\exists y[\prd{incendie}(y)\wedge[\lambda x_1\forall u[\prd{pompier}(u)\implq\prd{éteindre}(u,x_1)](y)]]\)
\b.[$=$]
 \(\Xlo\exists y[\prd{incendie}(y)\wedge \forall u[\prd{pompier}(u)\implq\prd{éteindre}(u,y)]]\)



Nous voyons ainsi qu'en retardant l'intervention du DP \sicut{un incendie} dans l'analyse, le \alien{quantifying-in} ne fait que réaliser dans le processus de dérivation syntaxique ce que nous avions observé dans le chapitre \ref{ch:gn} en disant qu'un groupe nominal pouvait s'interpréter «plus à gauche» que sa position de surface.

%***

Comme signalé ci-dessus, le \alien{quanfying-in} n'est pas une règle syntaxique de même nature que celles que nous représentons sous la forme de règles de réécriture dans ce chapitre.  Mais nous pouvons malgré tout tenter de l'intégrer dans notre format de règles d'interface, car \QUIN{i}{DP}{TP} est, en soi, une structure syntaxique et elle produit un TP.  Cette idée peut alors se synthétiser en écrivant 
\mbox{TP {\reecr} \QUIN{i}{DP}{TP}}, ce qui peut se schématiser graphiquement 
sous une forme arborescente comme \ref{A:Q-in} :

\ex. \label{A:Q-in}
\small%
\Tree[.TP 
  \qroof{un incendie}.{\zrbox{${}_1$(\,}DP} \qroof{\pile{tous les pompiers\\ont éteint \himn1}}.{TP\zbox{\,)}}
]
\normalsize


Avec cette convention, nous sommes alors en mesure de formuler proprement la règle d'interface pour le   \alien{quantifying-in} :

\ex. \RISS{\alien{Quantifying-in}}%
{\begin{tabular}[t]{rcc@{}c@{}c@{\,}c}
    TP & \reecr & ${}_i($ & DP &,&TP\zbox{$\,)$}\\
    \small\typ t &&& \small\ett && \small\typ t \\
    $\Xlo[\alpha(\lambda x_i\phi)]$ &\seecr && $\Xlo\alpha$ &&$\Xlo\phi$
  \end{tabular}} \label{v:ri:Q-in}


Ce qu'exprime \ref{v:ri:Q-in} est le cœur d'un des apports les plus marquants de \citet{PTQ}, et même si le \alien{quantifying-in}, tel qu'il vient d'être présenté dans sa dimension syntaxique, est aujourd'hui un peu délaissé au profit de variantes que nous allons voir ci-dessous, le principe de \ref{v:ri:Q-in} lui n'a jamais été abandonné.  

Il nous faut, en effet, émettre ici quelques petites critiques sur la façon dont le mécanisme du \alien{quantifying-in} s'insère dans la grammaire.  La première est qu'il ouvre la possibilité de surgénérer une pléthore (et même une infinité) de dérivations inutiles (car redondantes) pour une même phrase.  En reprenant les séquences de \ref{x:Q-in1}, et en les complétant, nous constatons qu'il y a maintenant au moins cinq stratégies combinatoires différentes pour analyser \ref{x:pompiers1} ; elles sont résumées en \ref{x:Q-inx} :

\ex. \label{x:Q-inx}
\a. tous les pompiers ont éteint un incendie. \hfill{\small(sans \alien{quantifying-in})}\label{x:Q-inxa}
\b. \QUIN{1}{un incendie}{tous les pompiers ont éteint \himn1}\label{x:Q-inxb}
\b. \QUIN{0}{tous les pompiers}{\himn0 ont éteint un incendie}\label{x:Q-inxc}
\b. \QUIN{0}{tous les pompiers}{\QUIN{1}{un incendie}{\himn0 ont éteint \himn1}}\label{x:Q-inxd}
\b. \QUIN{1}{un incendie}{\QUIN{0}{tous les pompiers}{\himn0 ont éteint \himn1}}\label{x:Q-inxe}

Or \ref{x:Q-inxa}, \ref{x:Q-inxc} et \ref{x:Q-inxd} donneront la même traduction (avec portées de surface), de même que \ref{x:Q-inxb} et \ref{x:Q-inxe} (avec portées inversées).  Autrement dit, nous avons cinq dérivations pour deux traductions.  Ce qui est initialement une vertu du \alien{quantifying-in} (car il est évidemment indispensable de pouvoir diversement analyser une phrase ambiguë) risque d'emballer la machine sans raison.  Car nous pouvons aller plus loin encore : a priori, rien n'empêche de dériver quelque chose comme, par exemple, \ref{x:Q-inx2}, toujours pour notre phrase \ref{x:pompiers1} --~puisque syntaxiquement n'importe quel DP peut s'ajouter sur n'importe quel TP avec la structure \QUIN{i}{DP}{TP}.

\ex. \label{x:Q-inx2}
\QUIN{2}{des spaghettis}{\QUIN{3}{Alice}{\QUIN{0}{tous les pompiers}{\QUIN{1}{un incendie}{\himn0 ont éteint \himn1}}}}


Ce constat ne veut pas dire que le \alien{quantifying-in} est en soi un mauvais mécanisme, mais simplement qu'il serait avantageux de lui ajouter des règles qui en contrôlent l'application pour dispenser l'analyse d'emprunter des détours qui n'ont pas lieu d'être\footnote{Notons que cet effet de surgénération est déjà mentionné dans \citet{PTQ}, mais dans l'optique de l'article cela ne constitue pas un problème disqualifiant : ce qui préoccupe Montague, c'est de pouvoir produire toutes les interprétations possibles, pas de limiter l'application des règles.}.

La seconde critique est directement liée à la première et concerne la compositionnalité de l'analyse, en particulier lorsque nous nous plaçons dans une démarche ascendante (dite \alien{bottom-up} en anglais), c'est-à-dire lorsque nous démarrons l'analyse au niveau des unités lexicales pour les regrouper en syntagmes de plus en plus grands jusqu'à parvenir à la racine TP. 
Dans ce cas, il peut paraître contre-intuitif de commencer l'analyse de \sicut{tous les pompiers ont éteint un incendie} en construisant le {VP} \sicut{éteint \himn1} qui n'est clairement pas une partie de la phrase de départ.  
Comment donc, en pratique, pouvons-nous mener à bien l'analyse si notre donnée de départ est littéralement la chaîne \sicut{tous les pompiers ont éteint un incendie} ?   Eh bien il nous faut d'abord enlever le DP \sicut{un incendie} de la phrase, le remplacer par \himn1, puis poursuivre normalement l'analyse syntaxique, et finalement, c'est-à-dire au sommet de l'arbre construit, faire réapparaître le DP pour le combiner par \alien{quantifying-in} au TP racine (avec l'indice $1$). 
Or il se trouve que le scénario de cette manipulation %technique 
ne fait rien d'autre que résumer le principe de mises en œuvre plus «modernes» du \alien{quantifying-in} et en premier lieu le procédé de la montée des quantificateurs.


\is{quantifying-in@\textit{quantifying-in}|)}



\subsection{Forme logique et montée des quantificateurs}
%-------------------------------------------------------
\label{sss:QR}

Le mécanisme de la \kwo{montée des quantificateurs}\is{montée!\elid\ des quantificateurs} (ou \alien{QR} pour \alien{Quantifier Raising})\is{quantifier raising@\textit{Quantifier Raising}}, introduit par \citet{May:77}, est une instance du \alien{quantifying-in} 
qui utilise des outils habituels de la syntaxe générative, en les mettant au service de l'interface syntaxe-sémantique. 
Ces outils sont les \emph{mouvements}\is{mouvement} de constituants dans la structure syntaxique, et la montée des quantificateurs réalise la manipulation décrite ci-dessus 
en postulant qu'un DP comme \sicut{un incendie} peut subir un mouvement en se déplaçant de sa position d'origine pour monter s'adjoindre à la racine TP de l'arbre.
C'est ce qu'illustre \ref{A:QR1} :

\ex. \label{A:QR1}
\small%
\Tree[.TP 
  \qroof{\rnode{dp}{un incendie}}.{DP} \qroof{\pile{tous les pompiers\\ont éteint \rnode{t}{\_}}}.{TP}
]
\ncbar[nodesep=2pt,angle=-90,linecolor=darkgray,armA=6pt]{->}{t}{dp}
\normalsize

\medskip

\sloppy
Nous voyons évidemment apparaître un net parallélisme avec la structure qui était schématisée en \ref{A:Q-in} : en fait \QRa\ est simplement un «~\alien{quantifying-in} par mouvement syntaxique».   Mais pour aborder proprement sa mise en place formelle dans notre système, nous devons faire un petit point théorique.


%Pour ce faire, nous 
Nous allons adopter ici le modèle grammatical dit «en Y» (ou plus exactement «en~\rotatebox[origin=c]{180}{Y}~»), schématisé en figure~\ref{f:ModY} ci-contre.
Dans ce modèle, la syntaxe opère en deux grandes étapes. D'abord pour produire une \kwo{structure} dite \kwo{profonde},\is{structure!\elid\ profonde} qui pose les fondations de l'architecture syntaxique de la phrase par ce que nous appellerons, pour faire simple, des règles «de base» (typiquement des règles de réécriture).  
Ces structures profondes font ensuite l'objet (si nécessaire) d'une série de «transformations» (typiquement des mouvements\is{mouvement} de constituants) pour produire des \kwo{structures} dites \kwo{de surface},\is{structure!\elid\ de surface} qui donnent à peu près l'organisation finale de phrase telle que nous la voyons dans la forme des énoncés de la langue. 
«À peu près» car les structures de surface doivent encore subir quelques ajustements pour obtenir une structure, la \kwo{forme phonétique},\is{forme!\elid\ phonétique}  qui donnera la chaîne parlée ou écrite qui réalise finalement la phrase.  Mais parallèlement la structure de surface va également donner lieu à une structure qui n'est pas visible dans la forme de la phrase mais qui est sensible et perceptible dans son interprétation : la \kwo{forme logique}.\is{forme!\elid\ logique}

\fussy

\begin{figure}[h!]
\begin{center}
{\small \Tree
[.{Structure Syntaxique\\Profonde} [.{Structure Syntaxique\\de Surface} \pile{Forme\\Logique} \pile{Forme\\Phonétique} ] ] }
\end{center}
\caption{Modèle en Y}\label{f:ModY}
\end{figure}


Les formes logiques s'obtiennent par une série de mouvements «invisibles» mais sémantiquement pertinents.  Formellement ce sont des objets de même nature que  les structures syntaxiques, en l'occurrence des arbres de constituants, éventuellement enrichis de quelques autres dispositifs représentationnels. 
Mais les formes logiques appartiennent plus à l'interface syntaxe-sémantique qu'à la syntaxe proprement dite, elles préparent l'analyse sémantique et ce sont elles, en fait, que nos règles de traduction vont prendre en entrée. 

Comme le montre la figure \ref{f:MvtQR}a, les mouvements syntaxiques sont simplement des déplacements de constituants d'une position de l'arbre vers une autre (généralement plus haute).  
Pour marquer le mouvement dans l'arbre, le constituant déplacé partagera un indice numérique ($i$ dans la figure)  avec sa position de départ qui, elle, sera occupée par une \emph{trace}\is{trace} notée $t_i$.
Et nous devinons naturellement que ces traces $t_i$ vont, sémantiquement, jouer le rôle de nos \himn i introduits précédemment.
%Ce dispositif des indices va jouer un rôle important dans notre traitement sémantique. 


\begin{figure}[h!]
\begin{center}
\begin{tabular}{c@{\hstrab[7em]}c}
\Tree 
[.X 
  {\rnode{Y}{Y$_i$}} 
  \qroof{... \rnode{t}{$t_i$} ...}.Z §{\qbalance}
]
\ncbar[nodesep=2pt,angle=-90,linecolor=darkgray]{->}{t}{Y}
%
&
%
\Tree 
[.TP 
  {\rnode{DP}{DP$_i$}} 
  \qroof{... \rnode{t2}{$t_i$} ...}.TP §{\qbalance}
]
\ncbar[nodesep=2pt,angle=-90,linecolor=darkgray]{->}{t2}{DP}
\\ \\[1ex] 
(a)&(b)
\end{tabular}
\caption{Schéma des mouvements (a) en général et (b) de \alien{QR}}\label{f:MvtQR}
\end{center}
\end{figure}


La montée des quantificateurs (\QRa) est un mouvement qui intervient entre la structure de surface et la forme logique. Elle se résume ainsi : un DP \emph{peut} toujours, facultativement, quitter sa position de surface pour «monter» dans la structure en venant s'adjoindre\footnote{Tous les mouvements ne se font pas par adjonction, mais c'est le cas pour \QRa.} sur la projection TP qui le domine, comme illustré dans le schéma en figure~\ref{f:MvtQR}b.  À l'arrivée, nous voyons que ce qui distingue principalement ce type de structures de celles du \alien{quantifying-in} en \S\ref{sss:Qu-in}, c'est la position de l'indice $i$ qui ici décore le DP déplacé au lieu de porter sur l'ensemble du TP supérieur.
Il est cependant pertinent de remarquer aussi qu'à l'issue du mouvement, une structure comme  la figure~\ref{f:MvtQR}b est de même nature que celles que nous avons manipulées jusqu'ici (c'est-à-dire des arbres de constituants), alors que ce n'était pas exactement le cas avec les \QUIN{i}{DP}{TP}.



Comme avec le \alien{quantifying-in}, puisque \QRa\ est toujours possible, il peut s'appliquer successivement sur différents DP de la phrase pour produire, là aussi, cinq analyses distinctes \ref{x:QR1} de notre phrase \ref{x:pompiers1} :


\ex. \label{x:QR1}
\a.
{}[\Stag{TP} tous les pompiers ont éteint un incendie] \label{x:QR1.0}
\hfill{\small(sans {\QRa}, = \ref{x:SyntPompiers})}
\b. 
{}[\Stag{TP} [\Stag{DP} un incendie]$_1$ [\Stag{TP} tous les pompiers ont éteint $t_1$]] \label{x:QR1a}
\b.
{}[\Stag{TP} [\Stag{DP} tous les pompiers]$_1$ [\Stag{TP} $t_1$ ont éteint un incendie]] \label{x:QR1b}
\b.
{}[\Stag{TP} [\Stag{DP} tous les pompiers]$_2$ [\Stag{TP} [\Stag{DP} un incendie]$_1$ [\Stag{TP} $t_2$ ont éteint $t_1$]]] \label{x:QR1c}
\b.
{}[\Stag{TP} [\Stag{DP} un incendie]$_2$ [\Stag{TP} [\Stag{DP} tous les pompiers]$_1$ [\Stag{TP} $t_1$ ont éteint $t_2$]]] \label{x:QR1d}


Mais à la différence du \alien{quantifying-in}, la surgénération ne va pas plus loin puisqu'il n'est question que de réorganiser des constituants déjà présents dans la phrase ; on ne risque plus d'ajouter des DP aléatoires \alien{ad infinitum}.
D'ailleurs les formes logiques de \ref{x:QR1} ne sont pas indépendantes les unes des autres : \ref{x:QR1a} et \ref{x:QR1b} s'obtiennent, par \QRa, à partir de \ref{x:QR1.0}, \ref{x:QR1c} à partir de \ref{x:QR1a} (où on a monté d'abord l'objet puis le sujet, Fig.~\ref{f:montéesDP}a)  et \ref{x:QR1d} à partir de \ref{x:QR1b} (montée du sujet puis de l'objet, Fig.~\ref{f:montéesDP}b).




\begin{figure}[h!]
\begin{bigcenter}
%% {\footnotesize
%% \Tree
%% [.TP
%% \qroof{u\rnode{DP1}{n ince}ndie}.DP$_1$
%% [.TP
%% \qroof{tous les pompiers}.DP
%%   [.T$'$ 
%%     [.T \zcbox{ont} ]
%%     [.VP 
%%       [.V \xbox{xxx}{éteint} ] 
%%       [.DP \rnode{t11}{$t_1$} ] ]
%%   ]
%% ] §\qsetw{23ex}
%% ]\ncbar[nodesep=2pt,angle=-90,linecolor=darkgray]{->}{t11}{DP1}
%% }
%% %
%% \ %\quad
%% %
%% {\footnotesize
%% \Tree
%% [.TP
%% \qroof{\rnode{DP2}{tous les pompiers}}.DP$_1$
%% [.TP
%% [.DP \rnode{t12}{$t_1$} ] 
%%   [.T$'$ 
%%     [.T \zcbox{ont} ]
%%     [.VP 
%%       [.V \xbox{xxx}{éteint} ] 
%%       \qroof{un incendie}.DP
%%     ]
%%   ]
%% ] §\qsetw{18ex}
%% ]\ncbar[nodesep=2pt,angle=-90,linecolor=darkgray]{->}{t12}{DP2}
%% }
%% %
%% %
%% \\[4ex]
%% %
{\footnotesize
\Tree
[.TP \qroof{\rnode{DP2}{tous les pompiers}}.DP$_2$
[.TP
\qroof{u\rnode{DP1}{n incendie}}.DP$_1$
[.TP
  [.DP \rnode{t22}{$t_2$} ]
  [.T$'$ 
    [.T \zcbox{ont} ]
    [.VP 
      [.V \xbox{xxx}{éteint} ] 
      [.DP \rnode{t11}{$t_1$} ] ]
  ]
] §\qsetw{12ex}
] §\qsetw{21ex}
]\ncbar[nodesep=2pt,angle=-90,linecolor=darkgray]{->}{t11}{DP1}\ncbar[nodesep=2pt,angle=-90,linecolor=darkgray]{->}{t22}{DP2}
}
%
\ %\quad
%
{\footnotesize
\Tree
[.TP
      \qroof{\rnode{DP1}{un incend}ie}.DP
[.TP
\qroof{\rnode{DP2}{tous les pompiers}}.DP$_1$
[.TP
[.DP \rnode{t12}{$t_1$} ] 
  [.T$'$ 
    [.T \zcbox{ont} ]
    [.VP 
      [.V \xbox{xxx}{éteint} ] 
      [.DP \rnode{t32}{$t_2$} ]
    ]
  ]
] §\qsetw{13ex}
] §\qsetw{20ex}
]\ncbar[nodesep=2pt,angle=-90,linecolor=darkgray]{->}{t12}{DP2}\ncbar[nodesep=2pt,angle=-90,linecolor=darkgray]{->}{t32}{DP1}
}\\[4ex]
(a)\hspace{16em}(b)
\end{bigcenter}
\caption{Montées de DP dans \ref{x:pompiers1} }\label{f:montéesDP}
\end{figure}


%\subsubsection{Interprétation des traces et des mouvements}
%''''''''''''''''''''''''''''''''''''''''''''''''''''''''''

Nos règles d'interface syntaxe-sémantique opèrent donc maintenant sur des formes logiques telles que \ref{x:QR1}.
\label{p.trace1}%
Et celle qui  définit l'interprétation (\ie\ la traduction) des traces et des mouvements reprend exactement le principe du \alien{quantifying-in},\is{quantifying-in@\textit{quantifying-in}} en posant que les traces $t_i$  se traduisent par $\Xlo\lambda P[P(x_i)]$ comme nos \himn i précédemment.
Cela nous donne la règle \ref{v:ri:QR}, naturellement identique à \ref{v:ri:Q-in}, aux notations syntaxiques près.


\ex. \RISS{Montée des quantificateurs}%
{\begin{tabular}[t]{rccc}
    TP & \reecr & DP$_i$ &TP\\
    \small\typ t && \small\ett & \small\typ t \\
    $\Xlo[\alpha(\lambda x_i\phi)]$ &\seecr & $\Xlo\alpha$ &$\Xlo\phi$
  \end{tabular}} \label{v:ri:QR}
\hstrab[8em]
\Tree
[.TP\zbox{ $\Xlo[\alpha(\lambda x_i\phi)]$}
DP$_i$\\{$\Xlo\alpha$} TP\\{$\Xlo\phi$} ]


Notons que grâce aux indices $i$, cette règle nous permet d'interpréter localement le mouvement, c'est-à-dire au niveau du site d'adjonction du DP : nous n'avons pas besoin d'aller fouiller dans les profondeurs du TP pour vérifier qu'il contient bien une trace $t_i$ ; 
la simple présence de l'indice $i$ sur DP$_i$ suffit à indiquer que celui-ci a été déplacé\footnote{En réalité c'est une hypothèse un peu trop forte, car les indices peuvent servir à autre chose que marquer les mouvements, notamment pour indiquer le lien entre un pronom et son antécédent.  En toute rigueur, nous devrions donc raffiner la règle d'interface.  Pour ce faire, nous pourrions stocker dans un registre (\ie\ un ensemble) les indices qui correspondent à des mouvements syntaxiques ; et la règle \ref{v:ri:QR} devrait alors vérifier que $i$ appartient à cet ensemble pour s'appliquer.}. 
Nous pouvons également remarquer que dans la règle d'interface \ref{v:ri:QR} (comme dans toutes les autres d'ailleurs), la couche syntaxique n'est pas exactement une règle de réécriture ; il faut plutôt la voir comme un patron structurel qui nous guide dans l'exploration de l'arbre de la forme logique lors de la composition sémantique.  Autrement dit, la ligne «TP {$\rightarrow$} DP$_i$ TP» doit se lire comme «si la forme logique contient le sous-arbre [\Stag{TP}~DP$_i$ TP], alors...». 


L'application de  \ref{v:ri:QR}, sur la forme logique \ref{x:QR1a} (c'est-à-dire avec montée du DP objet),  nous donne la même dérivation sémantique que \ref{x:Q-in4} et donc la lecture avec portées inversées pour \ref{x:pompiers1} :


\ex.
%% \a.
%% {} [\Stag{T$'$} ont éteint $t_1$]
%% \a.[$\leadsto$]
%% \(\Xlo[\lambda Y\lambda x [Y(\lambda y\,\prd{éteindre}(x,y))](\lambda P [P(x_1)])]\)
%% \b.[$=$]
%% \(\Xlo\lambda x [\lambda P [P(x_1)](\lambda y\,\prd{éteindre}(x,y))]\)
%% \b.[$=$]
%% \(\Xlo\lambda x  [\lambda y\,\prd{éteindre}(x,y)(x_1)]\)
%% \b.[$=$]
%% \(\Xlo\lambda x  \,\prd{éteindre}(x,x_1)\)
%% \z.
%% \b.
%% {} [\Stag{TP} tous les pompiers [\Stag{T$'$} ont éteint $t_1$]]
%% \a.[$\leadsto$]
%% \(\Xlo[\lambda P\forall u[\prd{pompier}(u)\implq [P(u)]](\lambda x  \,\prd{éteindre}(x,x_1))]\)
%% \b.[$=$]
%% \(\Xlo\forall u[\prd{pompier}(u)\implq [\lambda x  \,\prd{éteindre}(x,x_1)(u)]]\)
%% \b.[$=$]
%% \(\Xlo\forall u[\prd{pompier}(u)\implq \prd{éteindre}(u,x_1)]\)
%% \z.
%% \b.
{} [\Stag{TP} [un incendie]$_1$ [\Stag{TP} tous les pompiers [\Stag{T$'$} ont éteint $t_1$]]]
\a.[$\leadsto$]
\(\Xlo[\lambda P\exists y [\prd{incendie}(y)\wedge [P(y)]](\uline{\lambda x_1}\forall u[\prd{pompier}(u)\implq \prd{éteindre}(u,x_1)])]\)
\b.[$=$]
\(\Xlo\exists y [\prd{incendie}(y)\wedge [\lambda x_1\forall u[\prd{pompier}(u)\implq \prd{éteindre}(u,x_1)](y)]]\)
\b.[$=$]
\(\Xlo\exists y [\prd{incendie}(y)\wedge \forall u[\prd{pompier}(u)\implq \prd{éteindre}(u,y)]]\)


Dans les pages qui suivent, les mouvements de constituants seront généralement la stratégie qui sera utilisée pour illustrer le \alien{quantifying-in}, d'abord parce que \QRa\ n'est fondamentalement (et sémantiquement) pas distinct du \alien{quantifying-in} (la surgénération en moins), ensuite parce que ce mécanisme s'adapte assez naturellement au système de structures syntagmatiques que nous utilisons ici, et enfin parce que la règle \ref{v:ri:QR} peut s'étendre à d'autres types de mouvements (\ie\ autres que \QRa) qui sont une manière simple de rendre compte de divers phénomènes syntaxiques de dépendances à distance (\S\ref{sss:xMontees}). 
Mais nous n'oublierons pas qu'il existe d'autres façons de procéder techniquement (cf. par exemple \S\ref{sss:VarianteMvt}), qui, pour la plupart, se fondent sur le principe exposé dans \citet{PTQ}.




\subsection{Retour à une possible simplification}
%------------------------------------------------
\label{sss:typesVT}

Il est intéressant, à ce point, d'examiner une hypothèse qui peut aboutir à simplification du typage des verbes (ainsi que des autres unités lexicales relationnelles).  Il ne s'agit que d'une hypothèse, et donc d'une suggestion, car elle a des implications nombreuses que nous n'aurons pas la place ici de discuter en totalité. 

Dans notre interprétation de \QRa, les DP déplacés, de type \ett, laissent une trace\is{trace} $t_i$ qui correspond à une variable \vrbi xi de type \typ e  et qui donc  se traduit compositionnellement par \(\Xlo\lambda P[P(x_i)]\). 
Mais, en voulant développer un mécanisme plus général pour interpréter les mouvements, on pourrait envisager une alternative qui consiste à traduire la trace d'un DP directement par une variable de type \ett, \vrbi Xi.
Cela semble a priori très naturel car nous aurions ainsi une règle simple qui dit que toute trace d'un constituant déplacé de type \mtyp a est, en soi, du même type et se traduit directement par une variable de type \mtyp a.  
Mais dans les faits, il se trouve que cette option ne donnera pas du tout le résultat escompté pour l'analyse de la montée des quantificateurs.
Reprenons notre exemple \ref{x:QR1a}.

\ex.[\ref{x:QR1}]
\a.[b.]
{}[\Stag{TP} [\Stag{DP} un incendie]$_1$ [\Stag{TP} tous les pompiers ont éteint $t_1$]] 


\sloppy
Si $t_1$ se traduit par \vrbi X1 de type \ett, nous obtenons pour  [\Stag{T$'$} ont éteint $t_1$], après \breduc, la traduction \(\Xlo\lambda x [X_1(\lambda y\,\prd{éteindre}(x,y))]\). 
En ajoutant le sujet, nous obtenons \(\Xlo\forall u [\prd{pompier}(u)\implq [X_1(\lambda y\,\prd{éteindre}(u,y))]]\).  
Nous pouvons tout de suite voir qu'il y a là quelque chose qui cloche : \vrbi X1, qui sera remplacée par la traduction de \sicut{un incendie}, %\(\Xlo\lambda P[\exists y [\prd{incendie}(y)\wedge[P(y)]]\), 
se trouve  dans la portée de \sicut{tous les pompiers}.  À terme nous allons retomber sur la lecture avec portée étroite de l'indéfini objet, alors que \ref{x:QR1a} était prévu pour fournir l'autre lecture.
Conclusion : même si le constituant déplacé est de type \ett, sa trace doit donner une variable de type \typ e. 

\fussy

Il y a probablement plusieurs façons d'interpréter ce constat.  
Une des plus simples est de considérer qu'en fait, au niveau du VP, le prédicat verbal n'attend que des arguments de type \typ e. 
Et cela a plusieurs implications.  D'abord les DP qui se trouvent dans le VP seront alors de type \typ e et leurs traces se traduiront directement par \vrbi xi (et non $\Xlo\lambda P[P(x_i)]$).  Ensuite cela permet de revenir à des types plus simples pour les verbes, par exemple \eet\ pour les transitifs --~plus besoin de manipuler de «gros» \lterme s de type \type{\ett,\et}.
Cette hypothèse peut également suggérer une explication formelle de la montée des quantificateurs : les DP quantificationnels et indéfinis étant de type \ett, ils ne peuvent pas rester dans leur position de surface qui est, elle, de type \typ e, et c'est pour cette raison qu'ils montent s'adjoindre sur TP. 
Il est même alors possible d'envisager que, finalement, les DP référentiels peuvent rester de type \typ e (variables, constantes ou \atoi-termes), puisqu'ils ne sont pas sujets aux ambiguïtés de portée et n'ont donc pas à suivre \QRa\footnote{Et comme déjà annoncé, nous reviendrons sur cette idée d'une flexibilité de type des DP en \S\ref{s:typeshift}.}.

Nous voyons ainsi que cette approche n'est pas sans attrait, mais elle n'est pas non plus sans poser quelques problèmes. 
En effet, si les positions argumentales des DP doivent être de type \typ e, alors les DP quantificationnels et indéfinis sont \emph{obligés} de monter.  
Or empiriquement, comme nous l'avons observé  dans le chapitre~\ref{ch:gn},  ce déplacement semble être toujours \emph{facultatif}.  
Lorsqu'une phrase contient plusieurs DP mobiles, le problème ne se posera généralement pas, car tous monteront dans la forme logique et, comme indiqué précédemment, ce qui compte alors, c'est l'ordre dans lequel s'effectuent les mouvements ; nous obtiendrons bien toutes les configurations de portées attendues. 
Mais le problème se posera avec acuité lorsqu'il s'agira de faire apparaître les différentes portées d'un DP par rapport à un opérateur a priori pas ou peu mobile, comme par exemple la négation. 
Nous savons bien que les phrase \Next\ sont ambiguës.

\ex.
\a.
Le relecteur n'a pas vu une coquille dans le manuscrit.
\b.
Je n'interrogerai pas tous les étudiants.


Si les DP doivent nécessairement monter dans la forme logique, alors nous n'obtiendrons que la lecture avec leur portée large sur la négation --~celle-ci se trouvant «quelque part» à l'intérieur du TP sur lequel s'adjoint le DP déplacé. 
C'est d'autant plus préoccupant que pour \Last[b] (et peut-être aussi \Last[a]) c'est l'autre lecture qui semble la plus naturelle.


Par conséquent, si l'on veut sérieusement valider cette hypothèse des DP argumentaux de type \typ e, il sera consciencieux d'étudier en profondeur ses implications et ses interactions avec d'autres phénomènes, tant sur le plan sémantique que syntaxique.  Et c'est ce que nous n'allons pas faire ici, faute de place.  
Je ne ferai qu'annoncer quelques pistes possibles : est-ce que la négation et d'autres opérateurs à portée pourraient être mobiles eux aussi ? est-ce que \QRa\ pourrait parfois monter les DP plus bas que TP (par exemple sur VP) ? est-ce qu'il pourrait aussi exister des mouvements qui \emph{descendent} les DP dans la structure syntaxique ? 
est-ce que les différentes lectures de \Last\ sont toutes liées à la portée et à la position sémantique des DP ?...
À ce stade --~tant que ces questions (et bien d'autres) ne sont pas approfondies\footnote{À ce sujet, on peut, par exemple, consulter le chapitre 8 de \citet{HeimKratzer:97} qui montrent de nombreuses conséquences de cette hypothèse.}~-- il est difficile pour nous de trancher ; la prudence et la rigueur conseilleront de conserver l'hypothèse de \S\ref{sss:Vtrans} des DP argumentaux de type \ett, afin de pouvoir continuer à les interpréter \alien{in situ} (\ie\ sans les monter obligatoirement).
Cependant, comme l'autre hypothèse (DP argumentaux de type \typ e) reste prometteuse, qu'elle est plus conforme à l'intuition et qu'elle a le mérite de nous faire manipuler des notations plus simples, nous allons nous autoriser, quand cela n'a pas d'impact fâcheux, à les utiliser  pour alléger nos écritures.  
Nous garderons simplement en tête le \alien{caveat} suivant : si, par exemple, nous traduisons \sicut{regarder} par $\Xlo\lambda y\lambda x\,\prd{regarder}(x,y)$ en le combinant avec \cns b, il ne s'agira que d'un raccourci pour dire qu'en fait nous avons combiné  $\Xlo\lambda Y\lambda x[Y(\lambda y\,\prd{regarder}(x,y))]$ avec $\Xlo\lambda P[P(\cns b)]$.



\subsection{D'autres mouvements}
%-------------------------------
\label{sss:xMontees}\is{mouvement}


En exploitant judicieusement le principe sémantique du \alien{quantifying-in},
notre mécanisme d'interprétation des constituants déplacés nous donne maintenant les moyens de traiter également d'autres mouvements opérés par la syntaxe, en l'occurrence ces mouvements «visibles» qui interviennent entre la structure profonde et la structure de surface. 
Cela va nous permettre de travailler sur des formes logiques plus riches et plus abouties dans leurs structures syntaxiques.
Nous allons examiner quelques exemples ici.

Une hypothèse syntaxique assez courante considère que le VP constitue le noyau prédicatif central de la phrase en réunissant le verbe et tous ses arguments essentiels, y compris son sujet. 
Dans cette configuration, le sujet du verbe est donc initialement généré dans la position de spécifieur du VP\footnote{Pour être plus précis, cela dépend du V ; avec certains verbes intransitifs, ceux dits \emph{inaccusatifs},\is{inaccusatif} on conçoit habituellement que le sujet est généré plus bas dans la structure du VP. Cela n'a pas une très grande importance pour ce qui nous occupe ici, à savoir simplement le fait que le sujet sorte du VP.} puis il se déplace vers la position de spécifieur du TP \Next.  


\ex. \label{x:montéesujet}
{\small
\Tree
[.TP
  [.DP$_1$ \rnode{A}{Alice} ]
  [.T$'$ [.T a ]
    [.VP 
      [.DP \rnode{t}{$t_1$} ] 
      [.V$'$ [.V appelé ] [.DP Bruno ] ] §\qsetw{5em}
    ]
  ]
]\ncbar[nodesep=2pt,angle=-90,linecolor=darkgray]{->}{t}{A}
}


Cette montée du sujet\is{montée!\elid\ du sujet} \emph{n'est pas} une instance du \alien{quantifying-in} ni de \QRa\ (ses motivations syntaxiques sont différentes, et elle concerne la structure de surface), mais sémantiquement nous l'analyserons de la même manière, en nous adaptant aux conséquences de cette nouvelle analyse.  En effet jusqu'à présent nous considérions que les VP étaient systématiquement de type \et, car il leur manquait leur sujet ; à présent le sujet est contenu dans le VP (sous la forme d'une trace), celui-ci est donc saturé et de type \typ t.
Comme pour l'instant nous n'assignons pas de contribution sémantique à l'inflexion T, le constituant T$'$ est lui aussi de type \typ t, et la règle d'interprétation de la montée du DP sujet est ainsi la suivante :


\ex.
\RISS{Montée du sujet}%
{\begin{tabular}[t]{rccc}
    TP & \reecr & DP$_i$ & T$'$\\
    \small\typ t && \small\ett & \small\typ t \\
    $\Xlo[\alpha(\lambda x_i\beta)]$ &\seecr & $\Xlo\alpha$ &$\Xlo\beta$
  \end{tabular}} \label{ri:MontSuj}
\ ou \
\RISS{}%
{\begin{tabular}[t]{rccc}
    TP & \reecr & DP$_i$ & T$'$\\
    \small\typ t && \small\typ e & \small\typ t \\
    $\Xlo[\lambda x_i\beta(\alpha)]$ &\seecr & $\Xlo\alpha$ &$\Xlo\beta$
  \end{tabular}}


La variante de droite est proposée pour les cas où nous voudrions conserver la possibilité que certains DP sujets restent de type \typ e.  Cela permet de voir que le mécanisme d'interprétation du mouvement reste le même : la \labstraction\ $\Xlo\lambda x_i$ s'effectue toujours sur $\vrb\beta$, la traduction du T$'$ ; simplement l'application fonctionnelle, au niveau de TP, est inversée conformément aux types.

Profitons-en pour remarquer que si VP et T$'$ sont maintenant de type \typ t et si la négation se combine sur l'un de ces constituants, alors nous n'avons plus besoin de la traduction suggérée en \S\ref{sss:négation} p.~\pageref{sss:négation} ; la négation sera sera simplement de type \type{t,t} et se traduira plus naturellement en $\Xlo\lambda\phi\neg\phi$.

\smallskip

Un autre phénomène qui peut s'analyser en termes de mouvements visibles est le cas des pronoms relatifs.\is{pronom!\elid\ relatif|(}  Dans la proposition relative \ref{x:prel1a} le pronom \sicut{que} a la fonction d'objet et on peut raisonnablement considérer qu'il est donc lié, d'une manière ou d'une autre, à la position de complément de V, et, par exemple, qu'il monte de cette position à celle de spécifieur de CP comme en figure \ref{x:prel1b} (ci-contre).\label{p.montprorel} 

\ex. 
que Charles connaît \label{x:prel1a}


\begin{figure}
\begin{center}
\small 
\Tree
[.CP 
  [.DP$_2$ \rnode{q}{que} ]
  [.C$'$ 
    %C 
    [.TP 
      [.DP$_1$ \rnode{C}{Charles} ] 
      [.T$'$ T 
	[.VP  [.DP \rnode{t1}{$t_1$} ]
          [.V$'$
            [.V  connaît ] [.DP  \rnode{t2}{$t_2$} ] 
          ]
        ] 
      ]
    ] 
  ] 
]\ncbar[nodesep=2pt,angle=-90,linecolor=darkgray]{->}{t1}{C}\ncbar[nodesep=2pt,angle=-90,linecolor=darkgray,arm=7pt]{->}{t2}{q}
\normalsize\\
\end{center}
\caption{Syntaxe de la relative \ref{x:prel1a}}\label{x:prel1b}
\end{figure}

Cet exemple nous donne l'occasion de réfléchir un instant sur l'interprétation des pronoms relatifs.  Il existe d'assez nombreuses analyses syntaxiques et sémantiques concurrentes des relatives et nous n'entrerons pas dans les détails ici\footnote{Ce qui est présenté ici reprend, dans l'esprit, les analyses de \citet{PTQ},\Andexn{Montague, R.} \cite{Partee:75}\Andexn{Partee, B.}, \citet{Rodman:76}\Andexn{Rodman, R.}, \citet{HeimKratzer:97}, notamment.}, nous allons simplement retenir l'idée qu'une proposition relative en soi fonctionne sémantiquement comme un syntagme adjectival.  Le CP en figure \ref{x:prel1b} devra donc être de type \et\ ou \type{\et,\et} (cf.\ \S\ref{ss:ISSmodifieurs}).
En fait nous pouvons déjà prévoir la traduction sémantique de la relative \ref{x:prel1a} : dans un DP comme [\Stag{DP} \sicut{une} [\Stag{NP} \sicut{ville que Charles connaît}]], le NP dénote l'ensemble de tous les objets qui sont des villes et qui sont connus de Charles, et  nous en déduisons que la relative dénote donc l'ensemble de tous les objets qui sont connus de Charles, ce qui se traduit par 
\(\Xlo\lambda x[\prd{connaître}(\cns c,x)]\) de type \et.
À partir de là nous pouvons retrouver facilement la contribution du pronom \sicut{que}.  TP et C$'$ sont de type \typ t et se traduisent par \(\Xlo\prd{connaître}(\cns c,x_2)\), puis en se combinant sous CP avec le DP$_2$ déplacé, nous ajoutons la \labstraction\ ce qui donne  \(\Xlo\lambda x_2\,\prd{connaître}(\cns c,x_2)\) de type \et. 
C'est identique (au nom de la variable près) au résultat attendu pour CP, et cela semble dire que le pronom \sicut{que} a une contribution  {«vide»}\footnote{Certaines analyses considèrent d'ailleurs qu'un pronom relatif en soi est précisément l'opérateur qui réalise la \labstraction\  $\Xlo\lambda x_2$ ; cf.\ \citet{HeimKratzer:97} et \S\ref{sss:VarianteMvt} \alien{infra}.} ;  nous pouvons cependant lui assigner compositionnellement le type \type{\et,\et} et le traduire par \(\Xlo\lambda P\,P\) ou, de façon plus analytique et plus explicite, par \(\Xlo\lambda P\lambda x[P(x)]\).\label{p.prorel} 


Et si nous choisissons d'analyser la relative comme de type \type{\et,\et}, ce qui peut être judicieux puisqu'en français les relatives ont souvent un emploi d'épithète\footnote{Mais n'oublions pas qu'il existe aussi d'autres emplois, comme, par exemple, les relatives dites sans antécédent : \sicut{qui vole un œuf vole un bœuf}.}, alors sa traduction devra être \(\Xlo\lambda P\lambda x[[P(x)]\wedge\prd{connaître}(\cns c,x)]\).  Et dans ce cas le pronom relatif se traduira par 
\(\Xlo\lambda Q\lambda P\lambda x[[P(x)]\wedge[Q(x)]]\).

\is{pronom!\elid\ relatif|)}

% -*- coding: utf-8 -*-
\begin{exo}\label{exo:6mvt}
Détaillez la composition sémantique
\pagesolution{crg:6mvt}%
 de \ref{x:montéesujet}  \sicut{Alice a appelé Bruno} et du DP \sicut{tous les écrivains que Charles connaît}. 
%%
%%
\begin{solu}(p.~\pageref{exo:6mvt})\label{crg:6mvt}
\begin{enumerate}
\item Nous reprenons l'analyse syntaxique de \ref{x:montéesujet}, p.~\pageref{x:montéesujet}, rappelée ici :\\ {}
[\Stag{TP} [\Stag{DP} \sicut{Alice}]$_1$ [\Stag{T$'$} \sicut{a} [\Stag{VP} $t_1$ [\Stag{V$'$} \sicut{appelé Bruno}]]]], et nous allons traduire tous les DP par des quantificateurs généralisés de type \ett.  
\begin{enumerate}
\item \(\text{[\Stag{V$'$} appelé Bruno]} \leadsto
\Xlo[\lambda Y\lambda x[Y(\lambda y\,\prd{appeler}(x,y))](\lambda P[P(\cns b)])]\)\\
\(=\Xlo\lambda x[\lambda P[P(\cns b)](\lambda y\,\prd{appeler}(x,y))]\)
\hfill{\small(\breduc\ sur \vrb Y)}\\
\(=\Xlo\lambda x[\lambda y\,\prd{appeler}(x,y)(\cns b)]\)
\hfill{\small(\breduc\ sur \vrb P)}\\
\(=\Xlo\lambda x\,\prd{appeler}(x,\cns b)\)
\hfill{\small(\breduc\ sur \vrb y)}

\item \(t_1 \leadsto \Xlo\lambda P[P(x_1)]\) 
\hfill{\small(cf. p.~\pageref{p.trace1})}

\item \(\text{[\Stag{VP} $t_1$ appelé Bruno]} \leadsto 
\Xlo[\lambda P[P(x_1)](\lambda x\,\prd{appeler}(x,\cns b))]\)\\
\(=\Xlo[\lambda x\,\prd{appeler}(x,\cns b)(x_1)]\)
\hfill{\small(\breduc\ sur \vrb P)}\\
\(=\Xlo\prd{appeler}(x_1,\cns b)\)
\hfill{\small(\breduc\ sur \vrb x)}

\item Pour l'instant nous considérons que l'auxiliaire n'a pas de contribution sémantique et donc T$'$ se traduit comme VP.

\item \(\text{[\Stag{TP} Alice$_1$ [\Stag{T$'$} a $t_1$ appelé Bruno]]} \leadsto\)\\
\(\Xlo [\lambda P[P(\cns a)](\lambda x_1\,\prd{appeler}(x_1,\cns b))]\) 
\hfill {\small (montée du sujet, cf. \ref{ri:MontSuj}, p.~\pageref{ri:MontSuj})}\\
\(=\Xlo [\lambda x_1\,\prd{appeler}(x_1,\cns b)(\cns a)]\) 
\hfill{\small(\breduc\ sur \vrb P)}\\
\(=\Xlo \prd{appeler}(\cns a,\cns b)\) 
\hfill{\small(\breduc\ sur \vrbi x1)}
\end{enumerate}

\item En nous inspirant de ce qui a été vu dans le chapitre, nous posons l'analyse syntaxique suivante : %\\{}
[\Stag{DP} \sicut{tous les} [\Stag{NP} [\Stag{NP} \sicut{écrivains}] [\Stag{CP} \sicut{que}$_2$ [\Stag{TP} \sicut{Charles}$_1$ [\Stag{VP} $t_1$ \sicut{connaît} $t_2$]]]]].\\
Les traces $t_1$ et $t_2$ sont traduites respectivement par $\Xlo\lambda P[P(x_1)]$ et $\Xlo\lambda P[P(x_2)]$.
\begin{enumerate}
\item \(\text{[\Stag{V$'$} connaît $t_2$]} \leadsto
\Xlo[\lambda Y\lambda x[Y(\lambda y\,\prd{connaître}(x,y))](\lambda P[P(x_2)])]\)\\
\(=\Xlo\lambda x[\lambda P[P(x_2)](\lambda y\,\prd{connaître}(x,y))]\)
\hfill{\small(\breduc\ sur \vrb Y)}\\
\(=\Xlo\lambda x[\lambda y\,\prd{connaître}(x,y)(x_2)]\)
\hfill{\small(\breduc\ sur \vrb P)}\\
\(=\Xlo\lambda x\,\prd{connaître}(x,x_2)\)
\hfill{\small(\breduc\ sur \vrb y)}

\item \(\text{[\Stag{VP} $t_1$ connaît $t_2$]} \leadsto
\Xlo[\lambda P[P(x_1)](\lambda x\,\prd{connaître}(x,x_2))]\)\\
\(=\Xlo[\lambda x\,\prd{connaître}(x,x_2)(x_1)]\)
\hfill{\small(\breduc\ sur \vrb P)}\\
\(=\Xlo\prd{connaître}(x_1,x_2)\)
\hfill{\small(\breduc\ sur \vrb x)}

\item \(\text{[\Stag{TP} Charles$_1$ $t_1$ connaît $t_2$]} \leadsto\)\\
\(\Xlo[\lambda P[P(\cns c)](\lambda x_1\,\prd{connaître}(x_1,x_2))]
\)
\hfill{\small(montée du sujet, p.~\pageref{ri:MontSuj})}\\
\(=\Xlo[\lambda x_1\,\prd{connaître}(x_1,x_2)(\cns c)]
\)
\hfill{\small(\breduc\ sur \vrb P)}\\
\(=\Xlo\prd{connaître}(\cns c,x_2)
\)
\hfill{\small(\breduc\ sur \vrbi x1)}

\item \(\text{que} \leadsto \Xlo \lambda P\lambda x[P(x)]\)
\hfill{\small(cf. p.~\pageref{p.prorel})}%
\footnote{Nous pourrions aussi choisir de traduire le pronom relatif par $\Xlo\lambda Q\lambda P[[P(x)]\wedge[Q(x)]]$ comme suggéré dans le chapitre. Dans ce cas, nous n'aurons pas à appliquer la règle de modification de prédicat.}

\item \(\text{[\Stag{CP} que$_2$ Charles$_1$ $t_1$ connaît $t_2$]} \leadsto\)\\
\(\Xlo[\lambda P\lambda x[P(x)](\lambda x_2\,\prd{connaître}(\cns c,x_2))]\)
\hfill{\small(montée du pronom, p.~\pageref{p.montprorel})}\\
\(=\Xlo\lambda x[\lambda x_2\,\prd{connaître}(\cns c,x_2)(x)]\)
\hfill{\small(\breduc\ sur \vrb P)}\\
\(=\Xlo\lambda x\,\prd{connaître}(\cns c,x)\)
\hfill{\small(\breduc\ sur \vrbi x2)}\\
CP est de type \et\ comme il se doit, et va se combiner avec [\Stag{NP} \sicut{écrivains}] également de type {\et} ($\Xlo\lambda x\,\prd{écrivain}(x)$) ; nous devons donc appliquer la règle de modification de prédicat vue en \S\ref{ss:ISSmodifieurs} (règle \ref{ri:PM}, p.~\pageref{ri:PM}):

\item \(\text{[\Stag{NP} [\Stag{NP} écrivains] que$_2$ Charles$_1$ $t_1$ connaît $t_2$]} \leadsto \)\\
\(\Xlo\lambda y[[\lambda x\,\prd{écrivain}(x)(y)]\wedge [\lambda x\,\prd{connaître}(\cns c,x)(y)]]\)
\hfill{\small(modification de prédicat)}\\
\(=\Xlo\lambda y[\prd{écrivain}(y)\wedge \prd{connaître}(\cns c,y)]\)
\hfill{\small(\breduc s sur \vrb x)}

\item \(\text{[\Stag{DP} tous les écrivains que$_2$ Charles$_1$ $t_1$ connaît $t_2$]} \leadsto \)\\
\(\Xlo [\lambda Q\lambda P\forall x[[Q(x)]\implq[P(x)]](\lambda y[\prd{écrivain}(y)\wedge \prd{connaître}(\cns c,y)])]\)\\
\(=\Xlo \lambda P\forall x[[\lambda y[\prd{écrivain}(y)\wedge \prd{connaître}(\cns c,y)](x)]\implq[P(x)]]\)
\hfill{\small(\breduc\ sur \vrb Q)}\\
\(=\Xlo \lambda P\forall x[[\prd{écrivain}(x)\wedge \prd{connaître}(\cns c,x)]\implq[P(x)]]\)
\hfill{\small(\breduc\ sur \vrb y)}
\end{enumerate}
\end{enumerate}
\end{solu}
\end{exo}


% -*- coding: utf-8 -*-
\begin{exo}\label{exo:6VaM}
Une analyse possible des verbes dits à montée,\is{verbe!\elid\ a montee@\elid\ à montée} 
\pagesolution{crg:6VaM}%
comme \sicut{sembler}, est que leur sujet de surface est originairement celui du verbe infinitif enchâssé et que de là il monte, peut-être en plusieurs étapes, jusqu'à la position Spec du TP principal :
\begin{enumerate}
\item {} [\Stag{TP} Alice$_1$ semble [\Stag{TP} avoir $t_1$ dormi]].
\end{enumerate}
À partir de cette hypothèse et de l'hypothèse que \sicut{sembler} est de type \type{\type{s,t},t}, ne prenant ainsi qu'un seul argument, une proposition, 
donnez la composition sémantique complète de la phrase ci-dessus.
\begin{solu}(p.~\pageref{exo:6VaM})\label{crg:6VaM}
%[\Stag{TP} Alice$_1$ semble [\Stag{VP} $t_1$ dormir]].

Nous allons supposer l'analyse syntaxique donnée en figure \ref{f:VaMontée} où \sicut{Alice} monte de la position de sujet de \sicut{dormir} (Spec de VP) jusqu'à la position de sujet de la phrase (Spec de TP) en trois étapes successives%
\footnote{Peu importe ici que cette analyse soit ou non parfaitement correcte sur le plan syntaxique ; l'objectif de l'exercice est de nous faire manipuler des traces d'un même constituant déplacé plusieurs fois.}.

\begin{figure}[h]
\begin{center}
{\small
\Tree
[.TP
  [.DP$_1$ \rnode{a}{Alice} ]
  [.T$'$ 
    [.VP \rnode{t13}{$t_1$}
      [.V$'$ 
        [.V semble ] 
        [.TP 
          \rnode{t12}{$t_1$}
          [.T$'$ 
            [.T avoir ]
            [.VP 
              \rnode{t11}{$t_1$}
              [.V$'$ dormi ] 
            ]
          ]
        ]
      ]
    ]
  ]
]
}%
\ncbar[angle=-90,linecolor=darkgray,nodesep=2pt,offsetB=-1.5pt]{->}{t11}{t12}%
\ncbar[angle=-90,linecolor=darkgray,nodesep=2pt,offset=-1.5pt]{->}{t12}{t13}%
\ncbar[angle=-90,linecolor=darkgray,nodesep=2pt,offsetA=-1.5pt]{->}{t13}{a}
\end{center}
\caption{Hypothèse syntaxique pour \sicut{Alice semble avoir dormi}}\label{f:VaMontée}
\end{figure}

%\bigskip

Les traces $t_1$ marquent des positions anciennement occupées par [\Stag{DP} Alice]$_1$, c'est pourquoi elles ont le même indice et se traduiront donc de la même manière par $\Xlo\lambda P[P(x_1)]$.  

\begin{enumerate}
\item {} [\Stag{VP} $t_1$ dormi] $\leadsto$
\(\Xlo[\lambda P[P(x_1)](\lambda x\,\prd{dormir}(x))]\)\\
\(=\Xlo\prd{dormir}(x_1)\)
\hfill{\small (\breduc s sur \vrb P puis \vrb x)}

\item Là encore, nous considérons provisoirement que l'auxiliaire n'a pas de contribution sémantique et donc que T$'$ se traduit comme VP, de type \typ t.
Nous allons donc ensuite devoir appliquer la règle de montée du sujet (cf. \ref{ri:MontSuj} p.~\pageref{ri:MontSuj}) qui ajoute $\Xlo\lambda x_1$, ce qui fait que la deuxième trace $t_1$ est traitée comme un constituant déplacé.

\item {} [\Stag{TP} $t_1$ avoir $t_1$ dormi] $\leadsto$
\(\Xlo[\lambda P[P(x_1)](\lambda x_1\,\prd{dormir}(x_1))]\)
\hfill{\small (montée du sujet)}\\
\(=\Xlo[\lambda x_1\,\prd{dormir}(x_1)(x_1)]\)
\hfill{\small (\breduc\ sur \vrb P)}\\
\(=\Xlo\prd{dormir}(x_1)\)\footnote{Notons qu'ici il n'a pas été nécessaire de renommer les occurrences liées de \vrbi x1 avant d'effectuer la \breduc\ puisqu'à l'arrivée la variable argument \vrbi x1 reste libre.}
\hfill{\small (\breduc\ sur \vrbi x1)}

\item \(\sicut{semble}\leadsto \Xlo\lambda p\,\prd{sembler}(p)\), avec $\vrb p\in\VAR_{\type{s,t}}$.  Comme \prd{sembler} est de type \type{\type{s,t},t}\footnote{Le prédicat \prd{sembler} a une sémantique modale qui se rapproche de celle des verbes d'attitude pro\-po\-si\-tion\-nel\-le.% ; il peut d'ailleurs être raisonnable de généraliser en ajoutant un argument  
}, il va falloir utiliser l'application fonctionnelle intensionnelle (définition \ref{d:AFI}, p.~\pageref{d:AFI}).

\item {} [\Stag{V$'$} semble $t_1$ avoir $t_1$ dormi] $\leadsto$
\(\Xlo[\lambda p\,\prd{sembler}(p)(\Intn\prd{dormir}(x_1))]\)\footnote{NB : ici $\Xlo\Intn\prd{dormir}(x_1)$ est en fait la simplification (un peu abusive) de $\Xlo\Intn[\prd{dormir}(x_1)]$ (mais sachant que l'expression est bien formée, il n'y a pas de risque de confondre avec $\Xlo[\Intn\prd{dormir}(x_1)]$).}
\hfill{\small (AFI)}\\
\(=\Xlo\prd{sembler}(\Intn\prd{dormir}(x_1))\)
\hfill{\small (\breduc\ sur \vrb p)}

\item Ici, pour combiner la troisième trace $t_1$ avec V$'$ de type \typ t, nous devons encore une fois ajouter l'abstraction $\Xlo\lambda x_1$, tout en sachant qu'il ne s'agit pas là d'une instance de la règle \ref{ri:MontSuj}, mais d'une règle qui reconnaît que le sujet de \prd{dormir} a quitté la subordonnée pour venir occuper (provisoirement) la position sujet de \sicut{sembler} :\\
{} [\Stag{VP} $t_1$ semble $t_1$ avoir $t_1$ dormi] $\leadsto$\\
\(\Xlo[\lambda P[P(x_1)](\lambda x_1\,\prd{sembler}(\Intn\prd{dormir}(x_1)))]\)
\hfill{\small (ajout de $\Xlo\lambda x_1$)}\\
\(=\Xlo\prd{sembler}(\Intn\prd{dormir}(x_1))\)
\hfill{\small (\breduc s sur \vrb P puis \vrbi x1)}

\item {} [\Stag{TP} Alice $t_1$ semble $t_1$ avoir $t_1$ dormi] $\leadsto$\\
\(\Xlo[\lambda P[P(\cns a)](\lambda x_1\,\prd{sembler}(\Intn\prd{dormir}(x_1)))]\)
\hfill{\small (montée du sujet, règle \ref{ri:MontSuj})}\\
\(=\Xlo\prd{sembler}(\Intn\prd{dormir}(\cns a))\)
\hfill{\small (\breduc s sur \vrb P puis \vrbi x1)}
\end{enumerate}

Remarque : Cet exercice, et principalement l'étape 6 de la dérivation, nous permet de constater que les traces intermédiaires doivent se comporter à la fois comme des traces ordinaires mais aussi comme si elles étaient elles-mêmes des éléments déplacés.  Ce genre de traitement ne va pas entièrement de soi dans la formalisation précise de l'interface syntaxe sémantique (même si l'exercice montre qu'il est réalisable) ; cependant il s'intègre très simplement dans la variante formelle (dite «HK») présentée en \S\ref{sss:VarianteMvt}.

\end{solu}
\end{exo}





\subsection{Variantes et alternatives}
%-------------------------------------
\label{sss:VarianteMvt}


\sloppy 

Le mécanisme d'interprétation des mouvements présenté dans la section \S\ref{ss:iss:Qu} joue un rôle crucial dans le processus d'analyse à l'interface syntaxe-sémantique, et pour conclure, il me semble intéressant de mener une petite réflexion sur certains aspects formels et procéduraux du mécanisme.

\fussy

Commençons par ce qui peut sembler, à première vue, une anodine variante d'écriture.  Pour interpréter les mouvements, nous avons besoin, lors du parcours de l'arbre, de repérer les constituants déplacés, et nous le faisons grâce aux indices numériques qui viennent, précisément, marquer ces constituants.  Ainsi pour \alien{QR}, comme nous l'avons vu, dans la configuration syntaxique [DP$_i$ TP] nous savons que DP a été extrait de TP (qui contient donc une trace $t_i$) et donc qu'il faudra ajouter $\Xlo\lambda x_i$ à la traduction de TP dans la composition sémantique.  C'est là ce que nous appellerons la version traditionnelle de la gestion des indices numériques dans la syntaxe. 
La variante que nous allons examiner a été introduite et défendue  par \citet{HeimKratzer:97} et nous l'appellerons la version HK. % de la gestion des indices.
Elle consiste à utiliser les indices $i$ non pas pour marquer un constituant déplacé, mais pour marquer le constituant dont celui-ci a été extrait, en l'occurrence TP dans notre exemple de \alien{QR}.
Comme ce marquage prend ainsi une autre signification, nous allons le représenter en plaçant $i$ en position d'indice à gauche de l'étiquette du constituant qu'il marque.  Ainsi si nous rencontrons ${}_i$TP dans l'arbre, cela veut dire que TP est le constituant qui contient la trace $t_i$.  Par cette astuce graphique, nous voyons que les deux versions semblent se distinguer par vraiment peu de chose : d'une part nous avons 
[DP$_i$~TP], d'autre part  [DP~${}_i$TP].  
Mais en réalité, la version HK se trouve avoir des propriétés qui la rendent certainement préférable à la version traditionnelle.

\largerpage

D'abord, écrire ${}_i$TP plutôt que DP$_i$ s'avère, du point de vue de l'analyse sémantique, un peu plus logique, plus pratique et surtout plus compositionnel car dans ${}_i$TP, $i$ vient indiquer l'endroit précis où doit se placer le $\Xlo\lambda x_i$ indispensable à l'analyse.  Ainsi nous aurons une règle simple qui dit que si TP se traduit par \vrb\phi, alors ${}_i$TP se traduit par $\Xlo\lambda x_i\,\phi$\footnote{\citeauthor{HeimKratzer:97} vont d'ailleurs plus loin dans cette idée en traitant $i$ comme un quasi constituant syntaxique (elles écrivent [DP [$i$ TP]]) qui possède une traduction sémantique propre : $\Xlo\lambda x_i$. Formellement il ne s'agit pas, en soi, d'une véritable expression interprétable --~$\denote{\Xlo\lambda x_i}^{\Modele,w,g}$ n'est pas défini et la composition de [$i$ TP] se fait par \emph{concaténation} graphique~--, et c'est pourquoi $i$ doit être vu comme un opérateur syncatégorématique (cf. \S\ref{sss:Categ} p.~\pageref{sss:Categ}).}. Avec la version traditionnelle (cf. \ref{v:ri:QR}), pour placer $\Xlo\lambda x_i$ nous utilisons en somme une petite règle de déduction qui dit : si on rencontre DP$_i$ dans l'arbre syntaxique, alors il faut ajouter $\Xlo\lambda x_i$ à la traduction du «n\oe ud-frère» de DP$_i$.  Mais il peut exister des configurations syntaxiques (que nous n'avons pas encore rencontrées, mais ça viendra) dans lesquelles cette recette produit de mauvais résultats.  C'est, par exemple, ce qui se passe si un constituant déplacé atterrit dans une position qui n'est pas directement «s\oe ur» du constituant d'où il a été extrait.  C'est ce qu'illustre le schéma \ref{x:MonteeXYZ}.

\ex.  \label{x:MonteeXYZ}
%\a.
\small
\Tree
    [. [.Y W \rnode{X}{X$_i$} ] [.Z {$\cdots$} [. [. \rnode{t}{$t_i$} {$\cdots$} ] {$\cdots$} ] ] ]
\normalsize\ncbar[nodesep=2pt,angle=-90,linecolor=darkgray,arm=5pt]{->}{t}{X}
%
%% \rule{1cm}{0pt}
%% b.\qquad
%% \small
%% \Tree
%%     [. [.Y W \rnode{X}{X} ] [.{${}_i$Z} {$\cdots$} [. [. \rnode{t}{$t_i$} {$\cdots$} ] {$\cdots$} ] ] ]
%% \normalsize\ncbar[nodesep=2pt,angle=-90,linecolor=darkgray,arm=5pt]{->}{t}{X}
\\


Dans cette structure, X$_i$ doit d'abord se composer avec W. C'est faisable, mais la règle de la version traditionnelle ne doit pas encore s'appliquer.  Or c'est ce qu'elle prévoit de faire puisque X$_i$ porte l'indice ; mais $\Xlo\lambda x_i$ doit intervenir plus tard, en se plaçant sur Z puisque c'est le constituant qui contient la trace $t_i$.  Nous sommes bloqués\footnote{La stratégie qui, pour sauver la version traditionnelle, viserait à faire remonter $i$ sur Y n'est probablement pas viable, car comment pourra-t-on (simplement) distinguer les cas où l'indice doit remonter (et jusqu'où) de ceux où il doit rester sur l'élément déplacé (comme dans le \alien{QR} ordinaire) ?}. 
La version HK, en revanche, indiquera ${}_i$Z au lieu de Z dans \ref{x:MonteeXYZ}, et ainsi $\Xlo\lambda x_i$ sera automatiquement bien placé.  
Par conséquent, pour les besoins de l'analyse sémantique, cette version s'avère plus appropriée.


Cependant il ne faut alors pas perdre de vue qu'il demeure une question non triviale, c'est : comment techniquement obtenons-nous des marquages comme ${}_i$Z ou ${}_i$TP ?%
\footnote{Notons que nous pourrions, certes, facilement obtenir ce marquage en revenant au \alien{quantifying-in} original.  Il suffirait de remanier superficiellement nos notations en remplaçant TP {\reecr} \QUIN{i}{DP}{TP} par TP {\reecr} DP ${_i}$TP. Mais i) nous retrouverions alors le problème de la surgénération arbitraire, et ii) nous aurions toujours à résoudre celui des configurations comme  \ref{x:MonteeXYZ} pour lesquelles le \alien{quantifying-in} n'est pas adapté.}
Car du point de vue de la syntaxe, c'est la version traditionnelle qui est la plus naturelle :  formellement nous pouvons supposer que dans le processus de mouvement, les constituants déplacés sont indicés, qu'ils emportent leur indice avec eux tout en en laissant une copie sur leur trace.  C'est simple et intuitif, et c'est ce qui apparaît directement  dans le schéma \ref{x:MonteeXYZ}.
Avec la version HK, en revanche, pour marquer ${}_i$Z, il faut être en mesure de détecter dans l'arbre le plus grand constituant (en termes d'inclusion) qui contient la trace $t_i$ mais pas le constituant déplacé X$_i$.  C'est peut-être moins simple à mettre en \oe uvre.  Nous pourrions nous détourner de la question en estimant que tout cela c'est l'affaire de la théorie syntaxique, qu'elle se débrouille pour nous donner ce dont nous avons besoin en sémantique.
Mais soyons un peu charitables et essayons d'imaginer un algorithme à l'interface syntaxe-sémantique qui nous donnerait ces marquages ${}_i$Z (ou quelque chose d'équivalent).



Pour ce faire, nous devons d'abord perfectionner un peu nos structures syntaxiques.  Nous allons poser qu'un n\oe ud de l'arbre porte une étiquette de catégorie syntaxique (DP, NP, N, TP...) enrichie de trois informations : un indice numérique $i$ qui est propre au constituant, et deux ensembles d'indices $D$ et $S$.  Nous appellerons cela un système indiciel et noterons  \sysind{i}{D}{S} (par exemple \icat{VP}{4}{\set{1;3;4}}{\set{2}}). 
L'ensemble $D$ contiendra l'indice de tous les constituants que le n\oe ud domine dans l'arbre, et $S$ l'indice de toutes les traces qu'il domine.
Nous posons également que tout n\oe ud d'un arbre possède un système indiciel.
Nous amorçons la procédure en posant également que les n\oe uds feuilles de l'arbre (avant que les mouvements se produisent) ont typiquement un système de la forme \sysind{i}{\set{i}}{\Evide}. 
Les mouvements s'opèrent comme décrit ci-dessus : un constituant X déplacé emporte avec lui son système \sysind{i}{D}{S} et laisse une trace $t$ dont le système sera \sysind{i}{\set i}{\set{i}} (\ie\ le même $i$ pour X et $t$).
Nous indiquons ensuite comment les indices se propagent dans la structure syntaxique au moyen de la règle générale d'interface {\Next} (les systèmes indiciels sont intercalés entre les catégories et les types pour plus de lisibilité) :


\ex.
\fsynsem{\begin{tabular}[t]{cccc}
    X & \reecr & Y & Z \\
    \sysind{j}{D\cup D'\cup\set{j}}{S\cup S'} && \sysind{k}{D}{S} & \sysind{l}{D'}{S'}\\
    \mtyp a && \mtype{b,a} & \mtyp{b}\\
    $\Xlo[\alpha(\beta)]$ &\seecr& $\Xlo\alpha$ & $\Xlo\beta$
\end{tabular}}


Cette règle nous dit que chaque constituant, X, Y et Z, possède son propre indice, respectivement $j$, $k$ et $l$. Mais X récupère les contenus des ensembles $D$, $D'$, $S$ et $S'$ de ses descendants Y et Z, en en faisant l'union respective.  Ainsi dans l'exemple \ref{x:MonteeXYZ} ci-dessus, si W porte \sysind{n}{\set{n}}{\Evide} et X \sysind{i}{\set i}{\Evide}, alors Y aura \sysind{m}{\set{n;i;m}}{\Evide}, la trace $t$ aura le système \sysind{i}{\set i}{\set i}, et donc Z aura quelque chose de la forme \sysind{l}{\set{\dotsc i;l\dotsc}}{\set{\dotsc i \dotsc}}.
Cette règle \Last\ est une règle par défaut, c'est-à-dire qu'elle s'applique toujours (là où elle peut) \emph{sauf} si une règle plus précise prend priorité sur elle.  
Les règles prioritaires sont justement celles qui s'occupent des constituant déplacés. 
En voici un exemple :

\ex.
\fsynsem{\begin{tabular}[t]{c@{}c@{}c@{\ }c@{\ }c}
    X & \reecr & Y & Z \\
    \sysind{j}{D\cup D'\cup\set{j}}{S\cup S'-\set i} && \sysind{k}{D}{S} & \sysind{l}{D'}{S'}& \uline{si $i\in D\cap S'$}\\
    \mtyp a && \mtype{\mtype{c,b},a} & \mtyp{b}\\
    $\Xlo[\alpha(\lambda x_i\,\beta)]$ &\seecr& $\Xlo\alpha$ & $\Xlo\beta$ & avec $\vrbi xi \in \VAR_{\mtyp c}$
\end{tabular}}\label{RI:QRretrieve}


Cette règle \Last\ est plus précise que \LLast\ car elle comporte une condition : $i\in D\cap S'$.  Cette condition demande de vérifier d'une part que $i$ est dans $S'$, c'est-à-dire que $i$ est bien l'indice d'une trace dominée par Z, et d'autre part que $i$ est aussi dans $D$, c'est-à-dire que $i$ est l'indice d'un constituant plein dominé par (ou identique à) Y. 
Et c'est bien la configuration que nous cherchons pour insérer $\Xlo\lambda x_i$ : la rencontre du constituant déplacé dans Y avec le constituant Z qui contient la trace ; cette règle est donc équivalente au marquage ${}_i$Z (qu'il n'est pas indispensable de noter ici).
La mise à jour du système indiciel de X se fait alors avec 
$S\cup S'-\set i$ (où $-$ est l'opération de différence ensembliste),\is{difference ens@différence ensembliste} c'est-à-dire que l'on retire $i$ de l'ensemble $S'$, car à partir d'ici la trace $t_i$ n'est plus «active», elle vient d'être saturée.  Notons que les mouvements où Y est lui-même le constituant déplacé (comme dans \alien{QR}) correspondent au cas particulier où $k=i$.

%% \ex.
%% \fsynsem{\begin{tabular}[t]{ccccl}
%%     TP & \reecr & DP  & TP &\\
%%     \sysind{j}{D\cup D'\cup\set{j}}{S\cup S'-\set{i}} && \sysind{k}{D}{S} & \sysind{l}{D'}{S'} & \uline{si $i\in D\cap S'$}\\
%%     \typ t && \ett & \typ{t}&\\
%%     $\Xlo[\alpha(\lambda x_i\,\phi)]$ &\seecr& $\Xlo\alpha$ & $\Xlo\phi$&
%% \end{tabular}}

Si j'ai pris le temps de détailler un peu cette suggestion d'algorithme, c'est notamment pour montrer qu'\emph{en pratique}, ce traitement sémantique des mouvements s'avère finalement assez analogue à ce qui a été proposé dans une approche fameuse et concurrente à \alien{QR}, développée par 
\citet{Cooper:75,Cooper:83}\Andex{Cooper, R.} et qui a pris le nom de 
\kwo{Cooper Storage}\is{Cooper Storage@\alien{Cooper Storage}} (littéralement «le stockage à la Cooper»).  Il s'agit d'une méthode alternative pour entreprendre techniquement le \alien{quantifying-in}\is{quantifying-in@\textit{quantifying-in}} de Montague.

Cette approche considère qu'il n'y a pas vraiment lieu de considérer que les ambiguïtés de portées se reflètent dans la structure syntaxique, il s'agit plutôt d'ambiguïtés purement sémantiques, qui sont donc à traiter entièrement dans le processus d'interprétation.  Autrement dit, il n'y a pas de mouvement de LF.  Le principe du \alien{Cooper Storage} est, dans ses grandes lignes, le suivant.  Lorsqu'un DP quantificationnel ou indéfini, de traduction \vrb\alpha, se rencontre dans la structure syntaxique, l'analyse sémantique se trouve face à deux options.  La première consiste à insérer normalement \vrb\alpha\ dans la composition sémantique.
Par rapport à \alien{QR} c'est ce qui correspond à l'absence de mouvement.
La seconde option consiste à remiser \vrb\alpha\ pour l'utiliser plus tard dans la dérivation et à la place on insère une variable indicée \vrbi xi dans la composition.  C'est ce qui correspond aux mouvements de quantificateurs.
Pour implémenter ce mécanisme, les représentations sémantiques doivent être un peu plus structurées : les traductions sémantiques habituelles sont accompagnées d'un ensemble que l'on appelle le \alien{store} (la réserve) et dans lequel on stocke les quantificateurs mis de côté.
Formellement un \alien{store} est un ensemble de couples \tuple{\vrb\alpha,\vrbi xi} où \vrb\alpha\ est la traduction d'un quantificateur et \vrbi xi la variable qui a été introduite à sa place dans la composition.  Les \alien{stores} se propagent dans l'arbre syntaxique exactement comme les $S$ de l'algorithme précédent.  Ce n'est pas un hasard : avec les \vrbi xi, les \alien{stores} enregistrent les traces rencontrées, ce qui est aussi le rôle de nos $S$.
Les quantificateurs sont ensuite réintroduits dans la traduction sémantique principale par une opération de déchargement du \alien{store} (\alien{retrieval}) qui intervient, facultativement, sur certains n\oe uds de l'arbre (typiquement ceux de type \typ t) : un ou plusieurs \tuple{\vrb\alpha,\vrbi xi} sont choisis et retirés du \alien{store} et insérés dans la composition par une opération de \alien{quantifying-in} (\ie\ en ajoutant l'abstraction $\Xlo\lambda x_i$). 
Notons que les informations véhiculées par les ensembles $D$ de notre algorithme ne sont plus nécessaires ici car nous n'avons plus besoin de détecter les quantificateurs déplacés et indicés dans l'arbre : ils sont déjà rangés dans le \alien{store}, directement appariés avec leur trace. 
En fait les couples \tuple{\vrb\alpha,\vrbi xi} donnent l'information qui est encodée dans la condition $i\in D\cap S'$ de \ref{RI:QRretrieve}.

%\citet{RuysWinter:11}

Bien évidemment le \alien{Cooper Storage} et \alien{QR} ne sont pas la même chose, ils ont des fondements théoriques différents, voire opposés --~surtout en ce qui concerne leurs prémisses syntaxiques. 
Mais dans leurs mises en \oe uvre pratique à l'interface syntaxe-sémantique, au vu de l'algorithme suggéré \alien{supra}, nous pouvons constater qu'ils ont un fonctionnement assez similaire%
\footnote{Ce qui n'est pas étrange puisque tous deux dérivent du \alien{quantifying-in}. Et en exagérant un peu, on pourrait même dire que \alien{QR} est une sorte de \alien{Cooper Storage} qui utilise certaines positions de l'arbre en guise de \alien{store} pour y stocker les quantificateurs...}, et pour ce qui nous occupe ici (\ie\ l'analyse sémantique) nous pouvons finalement rester plutôt neutres sur le parti à prendre vis-à-vis de ces deux approches --~même si, pour de simples raisons de commodité, nous continuerons à utiliser les notations des mouvements dans les pages qui suivent.
De même, de façon plus générale, il faut savoir que toutes les théories syntaxiques n'adhèrent pas nécessairement à l'hypothèse des mouvements, et là aussi nous pouvons rester relativement agnostiques.  Ce dont nous avons essentiellement besoin, c'est un dispositif qui nous informe sur un \emph{lien} entre un constituant dans une certaine position et une autre position de la structure.  Finalement peu importe que cette information s'obtienne par des  traces indicées, par des \alien{stores} ou par tout autre mécanisme.




\section{Quantificateurs généralisés et structures tripartites}
%====================================
\label{s:QG}\is{quantificateur!\elid\ generalise@\elid\ généralisé|(}


Depuis le début de cet ouvrage nous avons examiné de nombreux groupes nominaux et déterminants (en nous intéressant de près à leurs propriétés sémantiques dans le chapitre~\ref{ch:gn}) mais prenant toujours bien soin d'éviter de nous interroger sur la traduction sémantique de la plupart d'entre eux.  
En dehors de \sicut{le}, \sicut{un}, \sicut{tous les}, \sicut{chaque}, \sicut{aucun} et de certains possessifs, notre langage {\LO} était incapable de donner une représentation formelle de la vaste majorité des divers déterminants de la langue.  Nous allons maintenant commencer à régler leur compte, grâce à la notion de \kwo{quantificateurs généralisés}. 
Nous avons déjà rencontré les quantificateurs généralisés en \S\ref{ss:QG.1} avec la proposition de Montague : ce sont les traductions de DP de type \ett.
Mais leur étude systématique a été introduite en sémantique formelle par 
\citet{BarwiseCooper:81} et \citet{KeenanStavi:86}\Andex{Barwise, J.}\Andex{Cooper, R.}\Andex{Keenan, E.}\Andex{Stavi, J.}
%\footnote{Voir aussi \citet{Westerstahl:89} pour une présentation étendue.}
à la suite, mais aussi relativement indépendamment, des travaux de Montague.  Ils ne sont pas intrinsèquement liés au \lcalcul, et ils auraient d'ailleurs pu être présentés bien plus tôt dans ces pages (dès le chapitre~\ref{ch:gn}) en  étendant et complexifiant la syntaxe et la sémantique de {\LO}.  Je choisis cependant de les introduire à cet endroit-ci car il se trouve que notre {\LO} typé, en l'état, nous les offre {«gratuitement»} : en fait ils sont déjà présents dans notre système, formellement nous n'avons rien à ajouter à notre langage, il nous reste juste à les dévoiler.



\subsection{L'avènement des déterminants dans LO}
%----------------------------------------
\label{ss:QGDet}

Nous avons vu que, compositionnellement, les déterminants se traduisent dans {\LO} par des expressions de type \type{\et,\ett} et, en termes ensemblistes, ils dénotent ainsi des relations entre deux ensembles d'individus (cf.\ \S\ref{ss:déterminants}, p.~\pageref{ss:déterminants} et \S\ref{sss:SyntheseGN}, p.~\pageref{x:QEns1}).  C'est ce que rappelle \ref{x:4dets}, où \vrb P et \vrb R sont des variables de type \et.

\ex. \label{x:4dets}
\a. \sicut{tous les} $\leadsto$ \xlo{$\lambda R\lambda P \forall x
  [[R(x)] \implq [P(x)]]$}  \label{x:4detsa}
\hfill \(\Ch{\denote{\vrb R}}^{\Modele,w,g}\inclus\Ch{\denote{\vrb P}}^{\Modele,w,g} \hphantom{\,=\eVide}\)
% \(\denote{\xlo R}^{\Modele,w,g}\inclus\denote{\xlo P}^{\Modele,w,g}\)
\b. \sicut{un} $\leadsto$ \xlo{$\lambda R\lambda P \exists x
  [[R(x)] \wedge [P(x)]]$}
\hfill \(\Ch{\denote{\vrb R}}^{\Modele,w,g}\cap\Ch{\denote{\vrb P}}^{\Modele,w,g}\neq\eVide\)
\b.
\sicut{aucun} $\leadsto$ \xlo{$\lambda R\lambda P \neg\exists x
  [[R(x)] \wedge [P(x)]]$}
\hfill \(\Ch{\denote{\vrb R}}^{\Modele,w,g}\cap\Ch{\denote{\vrb P}}^{\Modele,w,g}=\eVide\)
% \(\denote{\xlo R}^{\Modele,w,g}\cap\denote{\xlo P}^{\Modele,w,g}\neq\Evide\)
\b. \sicut{le} $\leadsto$ \xlo{$\lambda R\lambda P [P(\atoi x [R(x)])]$}
\hfill \(\Ch{\denote{\vrb R}}^{\Modele,w,g}\inclus\Ch{\denote{\vrb P}}^{\Modele,w,g}\hphantom{\,=\eVide}\) 
\\{\vstrab[0pt]}\hfill{\small(si \(\Ch{\denote{\vrb R}}^{\Modele,w,g}\) contient exactement 1 élément)}


\sloppy

Maintenant, notre langage {\LO} typé étant ce qu'il est, rien ne nous empêche d'envisager par ailleurs des \emph{constantes} de type \type{\et,\ett} ; elles dénoteront également des relations entre deux ensembles.
Par exemple, posons la constante \prd{Tous} de ce type ; c'est une sorte de «super-prédicat» qui prend en arguments deux expressions de type \et.
Nous pouvons alors écrire \(\Xlo[[\prd{Tous}(R)](P)]\) avec \vrb R et \vrb P de type \et.  Et si nous ajoutons une définition sémantique lexicale qui dit que pour tout monde possible et pour toute assignation, \prd{Tous} dénote la fonction qui renvoie $1$ ssi la dénotation (ensembliste) de son premier argument est incluse dans la dénotation de son second, alors il se trouve que \prd{Tous} (ou $\Xlo\lambda R\lambda P[[\prd{Tous}(R)](P)]$) est complètement équivalent à \ref{x:4detsa} et constitue donc une traduction directe du déterminant \sicut{tous les}.
Et c'est précisément en cela que réside le traitement des déterminants dans l'approche  des quantificateurs généralisés : les analyser  comme des constantes catégorématiques de type \type{\et,\ett} en leur assignant une dénotation relationnelle appropriée.

\fussy

Pour l'occasion, avec ce genre de traduction, nous n'appliquerons pas notre habituelle règle de suppression des crochets (cela produirait des écritures peu intuitives comme \(\Xlo\prd{Tous}(P,R)\)), nous nous contenterons de simplifier en \(\Xlo\prd{Tous}(R)(P)\) ou \(\Xlo[\prd{Tous}(R)](P)\) ou éventuellement \(\Xlo[\prd{Tous}(R)(P)]\).
Ainsi \ref{x:QGx1a} pourra se traduire par \ref{x:QGx1b} ou plus simplement, par $\eta$-réduction, par \ref{x:QGx1c}, ce qui est, convenons-en, une écriture particulièrement transparente pour représenter les mêmes conditions de vérité que \ref{x:QGx1d}.


\ex. 
\a. Tous les enfants dorment. \label{x:QGx1a}
\b. \(\Xlo\prd{Tous}(\lambda x\,\prd{enfant}(x))(\lambda x\,\prd{dormir}(x))\) \label{x:QGx1b}
\b. \(\Xlo\prd{Tous}(\prd{enfant})(\prd{dormir})\) \label{x:QGx1c}
\b. \(\Xlo\forall x [\prd{enfant}(x)\implq\prd{dormir}(x)]\) \label{x:QGx1d}


\sloppy
Et bien entendu nous pouvons, de la même manière, poser les constantes \prd{Un} et \prd{Aucun} en leur associant les dénotations données en \ref{x:def1QG}
sous forme de fonctions de $(\set{0;1}^{\set{0;1}^{\Unv A}})^{\set{0;1}^{\Unv A}}$
ou, de façon plus pratique et plus commune, en leur associant les définitions \ref{x:def2QG} :

\fussy

\ex.
Quels que soient $w$ et $g$ :  \label{x:def1QG}
\a. \(\denote{\Xlo\prd{Tous}}^{\Modele,w,g} =
f \mapsto (h \mapsto 1 \text{ ssi }
\set{\Obj x \in \Unv A\tq f(\Obj x)=1} \inclus \set{\Obj x \in \Unv A\tq h(\Obj x)=1})\)
\b.
\(\denote{\Xlo\prd{Un}}^{\Modele,w,g}=
f \mapsto (h \mapsto 1 \text{ ssi }
\set{\Obj x \in \Unv A\tq f(\Obj x)=1} \cap \set{\Obj x \in \Unv A\tq h(\Obj x)=1} \neq \Evide)\)
\c.
\(\denote{\Xlo \prd{Aucun}}^{\Modele,w,g} = 
f \mapsto (h \mapsto 1 \text{ ssi }
\set{\Obj x \in \Unv A\tq f(\Obj x)=1} \cap \set{\Obj x \in \Unv A\tq h(\Obj x)=1} = \Evide)\)

\ex.
Quels que soient $w$ et $g$ : \label{x:def2QG}
\a. \(\denote{\Xlo\prd{Tous}(R)(P)}^{\Modele,w,g}=1\) ssi
\(\Ch{\denote{\vrb R}}^{\Modele,w,g} \inclus \Ch{\denote{\vrb P}}^{\Modele,w,g}\)
\b.
\(\denote{\Xlo\prd{Un}(R)(P)}^{\Modele,w,g}=1\) ssi
\(\Ch{\denote{\Xlo R}}^{\Modele,w,g} \cap \Ch{\denote{\Xlo P}}^{\Modele,w,g} \neq \Evide\)
\c.
\(\denote{\Xlo \prd{Aucun}(R)(P)}^{\Modele,w,g}=1\) ssi
\(\Ch{\denote{\Xlo R}}^{\Modele,w,g} \cap \Ch{\denote{\Xlo P}}^{\Modele,w,g} = \Evide\)


Si $\Xlo\delta$ %$\Xlo\qnt{Q}$
est une constante de type \type{\et,\ett}, une formule de la forme \(\Xlo\delta(R)(P)\) est habituellement appelée une \kwo{structure tripartite}\is{structure!\elid\ tripartite} --~pour des raisons évidentes.  Par analogie avec la langue naturelle, $\Xlo\delta$ est appelé le \kwo{déterminant}\is{determinant@déterminant} ; le premier argument \vrb R est appelé la \kwo{restriction}\is{restriction!\elid\ d'un quantificateur généralisé} de la quantification et est fourni par le NP ; le second argument \vrb P est appelé la \kwo{portée nucléaire}\is{portee@portée!\elid\ nucléaire}, ou simplement \kwo{portée}, et provient du prédicat verbal (VP). 
La combinaison du déterminant et de sa restriction, $\Xlo[\delta(R)]$ de type \ett, est ce que l'on appelle un \kw{quantificateur} (\kwo{généralisé}) ; cela correspond bien sûr la traduction d'un DP.

Cette façon de représenter les déterminants est loin d'être une simple variante élégante de notation. 
Car maintenant tout type de relation entre deux ensembles que nous pouvons définir dans le modèle pourra servir de dénotation à un déterminant. 
En particulier nous pouvons maintenant \emph{compter} dans {\LO}.  Pour cela, il suffit d'utiliser dans le modèle la fonction mathématique de cardinal\is{cardinal!\elid\ d'un ensemble} d'un ensemble, $\Card{\cdot}$.  Si $A$ est un ensemble, $\Card A$ désigne le nombre d'éléments que contient $A$.  
Les déterminants cardinaux\is{cardinal!determinant@déterminant \elid} comme \sicut{deux}, \sicut{trois}, \sicut{quatre}... 
seront ainsi définis de la manière suivante : 

\ex.
\a.
\(\denote{\Xlo \prd{Deux}(R)(P)}^{\Modele,w,g}=1\) ssi
\(\Card{\Ch{\denote{\Xlo R}}^{\Modele,w,g} \cap \Ch{\denote{\Xlo P}}^{\Modele,w,g}} \geq \nbr2 \) \label{xd:Deux}
\b.
\(\denote{\Xlo \prd{Trois}(R)(P)}^{\Modele,w,g}=1\) ssi
\(\Card{\Ch{\denote{\Xlo R}}^{\Modele,w,g} \cap \Ch{\denote{\Xlo P}}^{\Modele,w,g}} \geq \nbr3 \), etc.

{}\ref{xd:Deux} dit que, par exemple, $\Xlo\prd{Deux}(\prd{enfant})(\prd{dormir})$ est vrai ssi l'intersection de l'ensemble des enfants et de l'ensemble des dormeurs contient au moins\footnote{Voir \S\ref{ss:implicatures} p.~\pageref{ss:implicatures} sur la sémantique en \sicut{au moins} des cardinaux qui explique l'utilisation de $\geq$ et non $=$.} deux éléments ; 
\prd{Deux} dénote donc la relation d'avoir (au moins) deux éléments en commun. 


Examiner de cette façon les cardinalités nous permet de définir de nombreux déterminants, comme \sicut{plusieurs}, \sicut{au moins $n$}, \sicut{au plus $n$}, \sicut{moins de $n$}, \sicut{plus de $n$}, \sicut{exactement $n$}, où $n$ est un nom de nombre comme \sicut{deux}, \sicut{trois}, \sicut{dix}, \sicut{cent}, etc.

\ex.
\a.
\(\denote{\Xlo \prd{Plusieurs}(R)(P)}^{\Modele,w,g}=1\) ssi
\(\Card{\Ch{\denote{\Xlo R}}^{\Modele,w,g} \cap \Ch{\denote{\Xlo P}}^{\Modele,w,g}} > \nbr1\)\footnote{Bien entendu, ici il ne faudra pas confondre, dans le métalangage, la valeur de vérité $1$  et le nombre $\nbr1$ utilisé pour évaluer le cardinal de l'ensemble. }
\b.
\(\denote{\Xlo \prd{Au-moins-deux}(R)(P)}^{\Modele,w,g}=1\) ssi
\(\Card{\Ch{\denote{\Xlo R}}^{\Modele,w,g} \cap \Ch{\denote{\Xlo P}}^{\Modele,w,g}} \geq \nbr2\)
\b.
\(\denote{\Xlo \prd{Au-plus-deux}(R)(P)}^{\Modele,w,g}=1\) ssi
\(\Card{\Ch{\denote{\Xlo R}}^{\Modele,w,g} \cap \Ch{\denote{\Xlo P}}^{\Modele,w,g}} \leq \nbr2\)
\b.
\(\denote{\Xlo \prd{Moins-de-deux}(R)(P)}^{\Modele,w,g}=1\) ssi
\(\Card{\Ch{\denote{\Xlo R}}^{\Modele,w,g} \cap \Ch{\denote{\Xlo P}}^{\Modele,w,g}} < \nbr2\)
\b.
\(\denote{\Xlo \prd{Plus-de-deux}(R)(P)}^{\Modele,w,g}=1\) ssi
\(\Card{\Ch{\denote{\Xlo R}}^{\Modele,w,g} \cap \Ch{\denote{\Xlo P}}^{\Modele,w,g}} > \nbr2\)
\b.
\(\denote{\Xlo \prd{Exactement-deux}(R)(P)}^{\Modele,w,g}=1\) ssi
\(\Card{\Ch{\denote{\Xlo R}}^{\Modele,w,g} \cap \Ch{\denote{\Xlo P}}^{\Modele,w,g}} = \nbr2\)


Nous disposons ainsi de déterminants \emph{pluriels}.  D'ailleurs nous pouvons également ajouter \prd{Des} et \prd{Quelques} qui auront ici la même définition que \prd{Plusieurs}\footnote{Cela ne veut bien sûr pas dire qu'en français \sicut{des}, \sicut{quelques} et \sicut{plusieurs} sont strictement synonymes ; simplement ils sont équivalents en termes de conditions de vérités. De la même manière, \sicut{deux} et \sicut{au moins deux} (ainsi que \sicut{plus d'un}, si on compte des objets entiers) sont également équivalents dans {\LO}, même s'ils se distinguent par diverses propriétés pragmatiques.}.
On remarquera au passage que les définitions \Last[b-f] peuvent être un peu décevantes dans la mesure où elles ne sont pas pleinement compositionnelles.  En effet, on pourrait souhaiter qu'un déterminant comme, par exemple, \sicut{moins de deux} s'analyse en deux parties, \sicut{moins de} et \sicut{deux}, et ainsi que la définition de son sens dépende de celle de \sicut{deux}, autrement dit de \prd{Deux}.  Mais techniquement, cela n'est pas possible dans ce système, notamment parce que la définition \ref{xd:Deux} de \prd{Deux} contient «un peu trop» d'informations que nous ne pouvons pas séparer : une condition sur la cardinalité ($\nbr2$) mais aussi une condition d'existence. 
Cependant nous verrons (chapitre~\ref{GN++}, vol.~2) qu'il est néanmoins possible de rétablir de la compositionnalité pour ces déterminants en enrichissant l'ontologie de notre système. 


Au chapitre \ref{ch:gn}, \S\ref{sss:effportee}, nous avions fait l'hypothèse que \sicut{la plupart} signifiait \sicut{plus de la moitié}.  Nous pouvons maintenant implanter cette hypothèse dans {\LO} avec :


\ex.
\(\denote{\Xlo \prd{Plupart}(R)(P)}^{\Modele,w,g}=1\) ssi
\(\Card{\Ch{\denote{\Xlo R}}^{\Modele,w,g} \cap \Ch{\denote{\Xlo P}}^{\Modele,w,g}} >
\Card{\Ch{\denote{\Xlo R}}^{\Modele,w,g} - \Ch{\denote{\Xlo P}}^{\Modele,w,g}}\)
% ssi 

\sloppy
Dans cette définition, $-$ représente l'opération de différence ensembliste\is{difference ens@différence ensembliste} : $A - B$ contient tous les éléments qui sont dans $A$ \emph{sauf} ceux qui sont aussi dans $B$.  Ainsi \(\Xlo\prd{Plupart}(\prd{enfant})(\prd{dormir})\) est vrai ssi l'ensemble des individus qui sont des enfants et qui dorment contient plus d'éléments que l'ensemble de ceux qui sont des enfants qui ne dorment pas.
Notons que \Last\ pouvait aussi être formulée de manière équivalente en : \(\denote{\Xlo \prd{Plupart}(R)(P)}^{\Modele,w,g}=1\) ssi
\(\Card{\Ch{\denote{\Xlo R}}^{\Modele,w,g} \cap \Ch{\denote{\Xlo P}}^{\Modele,w,g}} > \frac{\nbr1}{\nbr2}\Card{\Ch{\denote{\Xlo R}}^{\Modele,w,g}}\).
Cela montre également que nous pouvons facilement intégrer les déterminants qui expriment des fractions comme \sicut{la moitié de}, \sicut{un tiers de}, \sicut{trois quarts de}, etc. :

\fussy

\ex.
\a. \(\denote{\Xlo \prd{Moitié}(R)(P)}^{\Modele,w,g}=1\) ssi
\(\Card{\Ch{\denote{\Xlo R}}^{\Modele,w,g} \cap \Ch{\denote{\Xlo P}}^{\Modele,w,g}} \geq
\frac{\nbr1}{\nbr2}\Card{\Ch{\denote{\Xlo R}}^{\Modele,w,g}}\)
\b. \(\denote{\Xlo \prd{Trois-quarts}(R)(P)}^{\Modele,w,g}=1\) ssi
\(\Card{\Ch{\denote{\Xlo R}}^{\Modele,w,g} \cap \Ch{\denote{\Xlo P}}^{\Modele,w,g}} \geq
\frac{\nbr3}{\nbr4}\Card{\Ch{\denote{\Xlo R}}^{\Modele,w,g}}\), etc.


Les déterminants définis\is{defini@défini!article \elid} \sicut{le} et \sicut{les} (ainsi que les composés comme \sicut{les deux}, \sicut{les trois}, etc.) peuvent également être traités en utilisant la cardinalité, mais ici il est nécessaire de prendre une précaution.
En effet si nous posions simplement 
\(\denote{\Xlo \prd{Le}(R)(P)}^{\Modele,w,g}=1\) ssi
\(\Card{\Ch{\denote{\Xlo R}}^{\Modele,w,g}}=\nbr1\)
et
\(\Ch{\denote{\Xlo R}}^{\Modele,w,g} \inclus \Ch{\denote{\Xlo P}}^{\Modele,w,g}\),
nous n'obtiendrions pas le même comportement sémantique que ce que nous avons avec $\Xlo\atoi$.
Par exemple dans un monde où il n'y a pas de roi, alors $\Xlo\prd{Le}(\prd{roi})(\prd{chauve})$ serait faux alors que $\Xlo\prd{chauve}(\atoi x\,\prd{roi}(x))$ n'est pas défini.
Pour intégrer la présupposition, il convient d'enchâsser les conditions de la définition des déterminants définis de la manière suivante\footnote{Nous reviendrons au chapitre~\ref{GN++} (vol.~2) sur les propriétés interprétatives du défini pluriel \sicut{les} qui ne sont pas entièrement restituées par cette définition.} :

\ex.  \label{DET:Le}
\a.
\(\denote{\Xlo \prd{Le}(R)(P)}^{\Modele,w,g}\) est défini ssi
\(\Card{\Ch{\denote{\Xlo R}}^{\Modele,w,g}}=\nbr1\), et dans ce cas :\\
\(\denote{\Xlo \prd{Le}(R)(P)}^{\Modele,w,g}=1\) ssi
\(\Ch{\denote{\Xlo R}}^{\Modele,w,g} \inclus \Ch{\denote{\Xlo P}}^{\Modele,w,g}\)
\b.
\(\denote{\Xlo \prd{Les}(R)(P)}^{\Modele,w,g}\) est défini ssi
\(\Card{\Ch{\denote{\Xlo R}}^{\Modele,w,g}}>\nbr1\), et dans ce cas :\\
\(\denote{\Xlo \prd{Les}(R)(P)}^{\Modele,w,g}=1\) ssi
\(\Ch{\denote{\Xlo R}}^{\Modele,w,g} \inclus \Ch{\denote{\Xlo P}}^{\Modele,w,g}\)
\b.
\(\denote{\Xlo \prd{Les-deux}(R)(P)}^{\Modele,w,g}\) est défini ssi
\(\Card{\Ch{\denote{\Xlo R}}^{\Modele,w,g}}=\nbr2\), et dans ce cas :\\
\(\denote{\Xlo \prd{Les-deux}(R)(P)}^{\Modele,w,g}=1\) ssi
\(\Ch{\denote{\Xlo R}}^{\Modele,w,g} \inclus \Ch{\denote{\Xlo P}}^{\Modele,w,g}\), etc.


Il y a cependant des déterminants de la langue qui ne se laissent pas aussi facilement capter par le mécanisme des quantificateurs généralisés.  C'est le cas notamment de \sicut{beaucoup de} et \sicut{peu de}.  Certains auteurs les excluent purement et simplement du système sur l'argument que l'information quantitative exprimée par ces déterminants n'est pas d'ordre vériconditionnel, mais relève d'un jugement de valeur personnel\footnote{Voir par exemple \citet[pp. 256--258]{KeenanStavi:86}.}.  D'autres auteurs leur reconnaissent une très forte et complexe dépendance contextuelle.
En effet lorsque l'on juge qu'une certaine collection d'individus «est beaucoup», c'est généralement par rapport à une certaine quantité qui n'est pas explicitée dans la phrase qui exprime ce jugement.  Cette quantité serait à retrouver dans le contexte\is{contexte} par raisonnement pragmatique ou par spéculation sur l'opinion du locuteur.  Pour rendre compte de cette idée, \citet{Westerstahl:85}\Andex{Westerst{\aa }hl, D.}  propose plusieurs analyses dont voici deux exemples particulièrement pertinents (en les simplifiant un peu) :

\ex.
\a. \(\denote{\Xlo \prd{Beaucoup}_{\mathbf1}^n(R)(P)}^{\Modele,w,g}=1\) ssi
\(\Card{\Ch{\denote{\Xlo R}}^{\Modele,w,g} \cap \Ch{\denote{\Xlo P}}^{\Modele,w,g}} \geq n\)
\b. \(\denote{\Xlo \prd{Beaucoup}_{\mathbf2}^k(R)(P)}^{\Modele,w,g}=1\) ssi
\(\Card{\Ch{\denote{\Xlo R}}^{\Modele,w,g} \cap \Ch{\denote{\Xlo P}}^{\Modele,w,g}} \geq k \Card{\Ch{\denote{\Xlo R}}^{\Modele,w,g}}\)


Dans ces définitions $n$ est un nombre positif quelconque (mais généralement assez élevé) et $k$ est un ratio, c'est-à-dire un nombre décimal compris entre $0$ et $1$ (autrement dit un pourcentage).  Il faut tout de suite noter qu'en toute rigueur, ces deux définitions n'entrent pas dans notre système {\LO} : $n$ et $k$ sont des valeurs fournies par le contexte, ce sont des paramètres nécessaires à l'interprétation des deux déterminants et dans les écritures $\Xlo\prd{Beaucoup}_{\mathbf1}^n$ et $\Xlo\prd{Beaucoup}_{\mathbf2}^k$ ce sont donc des variables, mais des variables de \emph{nombres}.  Or pour l'instant, dans {\LO} nous ne disposons pas de telles variables\footnote{Cela n'a rien de bloquant, nous pouvons assez facilement ajouter des expressions de nombres dans {\LO}. Et c'est que nous ferons rigoureusement dans le chapitre~\ref{Ch:adj} (vol.~2).}.

\sloppy

Cependant {\Last} est intéressant en ce qu'il propose deux interprétations possibles (et donc une ambiguïté) pour \sicut{beaucoup}.  \prdi{Beaucoup}1 donnerait en quelque sorte un interprétation \emph{cardinale} de \sicut{beaucoup}.
\(\Xlo\Xlo \prd{Beaucoup}_{\mathbf1}^n(\prd{enfant})(\prd{dormir})\) est vrai si le nombre d'enfants qui dorment atteint ou dépasse un seuil $n$ jugé comme particulièrement élevé par le locuteur. 
%En pratique, ce seuil $n$ n'est bien sûr \emph{pas} absolu ; l
Le locuteur  fixe ce seuil en fonction de nombreuses informations présentes dans le contexte, comme par exemple le nombre total de personnes dans la situation, le nombre d'enfants censés dormir, l'heure de la journée, les diverses attentes des interlocuteurs, etc., bref, tout un tas de critères qui peuvent être utilisés comme des normes propres à s'appliquer aux présentes circonstances.
Mais compositionnellement $n$ est simplement traité comme un pronom dont l'antécédent est à retrouver dans le contexte.  
Il en va de même pour le $k$ de \prdi{Beaucoup}2 qui propose cette fois une interprétation \emph{proportionnelle} de \sicut{beaucoup}.
\(\Xlo\Xlo \prd{Beaucoup}_{\mathbf2}^k(\prd{enfant})(\prd{dormir})\) est vrai si le nombre d'enfants qui dorment atteint ou dépasse un certain pourcentage du nombre total d'enfants.  Autrement dit le nombre d'enfants qui dorment est comparé au nombre d'enfants qui ne dorment pas, comme avec \prd{Plupart}.  On pourrait supposer que normalement $k$ aura une valeur supérieure ou égale à $0,5$ (\ie\ au moins la moitié), mais en réalité cela n'est pas du tout nécessaire : comme pour $n$, la valeur appropriée de $k$ dépendra de diverses informations contextuelles et extralinguistiques.  Par exemple si nous nous intéressons aux élèves d'une classe donnée, la formule \(\Xlo\Xlo \prd{Beaucoup}_{\mathbf2}^k(\prd{élève})(\prd{gaucher})\) pourra être pertinente pour $k$ valant $0,2$ ou $0,15$, sachant que la proportion moyenne de gauchers de la population est d'environ 10\%.
Ces paramètres contextuels $n$ et $k$ rendent compte du caractère dit \kwo{relatif} de la sémantique de \sicut{beaucoup de} et \sicut{peu de} (ce dernier s'interprétant comme la négation du précédent), même s'il faut reconnaître qu'ils ne suffisent pas à rendre compte de son caractère vague\is{vague} (cf.\ \S\ref{s:Ambiguïté}), qui est une dimension du sens notablement plus complexe à formaliser.


Une autre catégorie de déterminants souvent laissée de côté dans l'étude des quantificateurs généralisés, pour des raisons similaires de forte dépendance contextuelle, est celle des démonstratifs \sicut{ce}/\sicut{cette}/\sicut{ces}. 
Nous reviendrons sur leur cas au chapitre \ref{Ch:contexte} (vol.~2).

\fussy

%***
\smallskip

Terminons cette présentation avec quelques exemples qui montrent comment les quantificateurs généralisés cohabitent avec le \lcalcul\ dans {\LO}. 
D'abord signalons que nous continuerons à utiliser $\Xlo\forall$, $\Xlo\exists$ et $\Xlo\atoi$ car nous en avons pris l'habitude et ces symboles permettent des écritures analytiques très standards. 
Nous réserverons les constantes de type \type{\et,\ett} pour les déterminants qui ne peuvent pas se traduire autrement, comme par exemple \sicut{la plupart}.
Compositionnellement, comme déjà mentionné, elles seront traitées dans l'analyse d'un DP de la même manière que celle présentée en \S\ref{ss:déterminants} avec la règle \ref{v:ri:DP} (p.~\pageref{v:ri:DP}) ; il y aura juste moins de $\beta$-réductions à effectuer, mais certaines simplifications sont possibles par $\eta$-réduction ($\Xlo\lambda x\,\prd{enfant}(x)$ équivaut à \prd{enfant}).
Voici par exemple le détail de la composition de \ref{x:QGex1a} qui comporte un VP complexe :

\ex.
\a. La plupart des enfants connaissent Charles. \label{x:QGex1a}
\b. {}[\Stag{DP} [\Stag{D} La plupart des] [\Stag{NP} enfants]]
\\$\leadsto$ \(\Xlo[\lambda R\lambda P\,\prd{Plupart}(R)(P)(\lambda x\,\prd{enfants}(x))]\)
\hfill règle \ref{v:ri:TP2}
\\$=$ \(\Xlo\lambda P\,\prd{Plupart}(\lambda x\,\prd{enfants}(x))(P)\)
\hfill \breduc\ sur \vrb R
\\$=$ \(\Xlo\lambda P\,\prd{Plupart}(\prd{enfants})(P)\)
\hfill $\eta$-réduction
\b. {}[\Stag{VP}  connaissent Charles]
\\$\leadsto$ \(\Xlo\lambda x\,\prd{connaître}(x,\cns c)\)
\b.  [\Stag{TP} [\Stag{DP} La plupart des enfants] [\Stag{VP} connaissent Charles]]
\\$\leadsto$ \(\Xlo[\lambda P\,\prd{Plupart}(\prd{enfants})(P)(\lambda x\,\prd{connaître}(x,\cns c))]\)
\hfill règle \ref{v:ri:TP2}
\\
$=$ \(\Xlo\prd{Plupart}(\prd{enfant})(\lambda x\,\prd{connaître}(x,\cns c))\)
\hfill \breduc\ sur \vrb P


\sloppy
Remarquons que comme $\Xlo\prd{connaître}(x,\cns c)$ provient de $\Xlo[[\prd{connaître}(\cns c)](x)]$, alors par $\eta$-réduc\-tion $\Xlo\lambda x[[\prd{connaître}(\cns c)](x)]$ peut se simplifier en $\Xlo\prd{connaître}(\cns c)$ et \Last[b] en 
\(\Xlo\prd{Plupart}(\prd{enfant})(\prd{connaître}(\cns c))\).
En revanche, si le quantificateur est complément d'objet, cette simplification ne sera pas disponible :

\ex.
\a. Charles connaît la plupart des enfants.
\b. \(\Xlo\prd{Plupart}(\prd{enfant})(\lambda y\,\prd{connaître}(\cns c,y))\)

\fussy


La restriction peut aussi être un \lterme\ complexe, notamment dans le cas de quantificateurs qui enchâssent une relative restrictive :

\ex.
\a. La plupart des enfants qui sont couchés dorment.
\b. \(\Xlo\prd{Plupart}(\lambda x[\prd{enfant}(x)\wedge\prd{couché}(x)])(\prd{dormir})\)


Et bien sûr nous pouvons rendre compte des ambiguïtés de portée en appliquant (ou non) la montée des quantificateurs, de façon similaire à ce que nous avons vu en \S\ref{sss:QR}.  Cela donne :


\ex. La plupart des élèves ont chanté trois chansons.
\a.
\(\Xlo\prd{Plupart}(\prd{élève})(\lambda x[\prd{Trois}(\prd{chanson})(\lambda y\,\prd{chanter}(x,y))])\)
%\\
%\(\Xlo\prd{Tout}(\prd{élève})(\lambda x\,[\prd{Un}(\prd{poésie})(\prd{apprendre}(x))])\)\\
%{\small(sachant que $\Xlo\lambda y\,\prd{apprendre}(x,y)$ $\equiv$ $\Xlo\prd{apprendre}(x)$)}
\b.
\(\Xlo\prd{Trois}(\prd{chanson})(\lambda y[\prd{Plupart}(\prd{élève})(\lambda x\,\prd{chanter}(x,y))])\)


\sloppy 
Dans \Last[a], qui présente l'interprétation linéaire, \(\Xlo\lambda y\,\prd{chanter}(x,y)\) dénote l'ensemble des choses chantées par \vrb x ;
la formule \(\Xlo[\prd{Trois}(\prd{chanson})(\lambda y\,\prd{chanter}(x,y))]\) est  vraie ssi l'intersection de l'ensemble des chansons et de l'ensemble de ce qui a été chanté par \vrb x contient au moins trois éléments, \ie\ \vrb x a chanté trois chansons ; \(\Xlo\lambda x[\prd{Trois}(\prd{chanson})(\lambda y\,\prd{chanter}(x,y))]\) dénote l'ensemble de tous ceux qui ont chanté (au moins) trois chansons ; et donc \Last[a] est vraie ssi il y a plus d'élèves qui ont chanté trois chansons que d'élèves qui ont chanté moins de trois chansons (ou aucune).



Dans \Last[b], qui présente la lecture avec portées inversées,
\(\Xlo\lambda x\,\prd{chanter}(x,y)\) dénote l'ensemble de tous ceux qui ont chanté \vrb y (qui est fort probablement une chanson) ;
\(\Xlo[\prd{Plupart}(\prd{élève})(\lambda x\,\prd{chanter}(x,y))]\) est vraie ssi il y a plus d'élèves qui ont chanté \vrb y que d'élève qui ne l'ont pas chantée ;
\(\Xlo\lambda y[\prd{Plupart}(\prd{élève})(\lambda x\,\prd{chanter}(x,y))]\) dénote l'ensemble de toutes les choses qui ont été chantées par la plupart des élèves ; et \Last dit que l'intersection de cet ensemble avec l'ensemble des chansons contient au moins trois éléments, autrement dit : il y a trois chansons que la plupart des élèves ont chantées.

\fussy

\medskip

% -*- coding: utf-8 -*-
\begin{exo}\label{exo:6QG}
  Traduisez dans {\LO} les phrases suivantes :\pagesolution{crg:6QG}
  \begin{enumerate}
    \item Il y a quatre 2\textsc{CV} vertes dans le parking.
    \item Moins de la moitié des candidats ont répondu à toutes les questions.
    \item Trois stagiaires apprendront deux langues étrangères.\label{xx:QG3x2}
  \end{enumerate}
  Explicitez les conditions de vérités des formules obtenues pour \ref{xx:QG3x2}.  %Qu'observez-vous ?
\begin{solu}(p.~\pageref{exo:6QG})\label{crg:6QG}
  \begin{enumerate}
    \item Il y a quatre 2\textsc{CV} vertes dans le parking.\\
\(\Xlo\prd{Quatre}(\lambda x[\prd{2cv}(x)\wedge\prd{vert}(x)])(\lambda x\,\prd{dans}(x,\atoi y\,\prd{parking}(y)))\)

Évidemment le déterminant \prd{Quatre} est calqué, \alien{mutatis mutandis},  sur \prd{Deux} et \prd{Trois} vus en \S\ref{ss:QGDet}, p.~\pageref{xd:Deux}.

    \item Moins de la moitié des candidats ont répondu à toutes les questions.\\
Commençons par traduire \sicut{ont répondu à toutes les questions}, c'est un VP de type {\et} : \(\Xlo\lambda x\forall y[\prd{question}(y)\implq\prd{répondre}(x,y)]\). Ensuite, nous devons fournir une définition du déterminant \sicut{moins de la moitié} :\\
\(\denote{\Xlo \prd{Moins-de-la-moitié}(R)(P)}^{\Modele,w,g}=1\) ssi
\(\Card{\Ch{\denote{\Xlo R}}^{\Modele,w,g} \cap \Ch{\denote{\Xlo P}}^{\Modele,w,g}} <
\frac{\nbr1}{\nbr2}\Card{\Ch{\denote{\Xlo R}}^{\Modele,w,g}}\).
\\
À partir de là, nous pouvons traduire la phrase :\\
\(\Xlo\prd{Moins-de-la-moitié}(\prd{candidat})(\lambda x\forall y[\prd{question}(y)\implq\prd{répondre}(x,y)])\)
\\
Cette formule dit que le nombre d'éléments dans l'ensemble des candidats qui ont répondu à toutes les questions est inférieur à la moitié du nombre total de candidats. Mais il y a une autre interprétation, avec les portées inversées des quantificateurs :\\
\(\Xlo\forall y[\prd{question}(y)\implq\prd{Moins-de-la-moitié}(\prd{candidat})(\lambda x\,\prd{répondre}(x,y))]\)\\
Cette traduction dit que pour chaque question, il y a, à chaque fois, moins de la moitié des candidats qui y ont répondu.  Cette lecture exclut par exemple les scénarios où il y  aurait eu quelques questions très faciles auxquelles la plupart des candidats ont su répondre.


    \item Trois stagiaires apprendront deux langues étrangères.

\(\Xlo\prd{Trois}(\prd{stagiaire})(\lambda x\,\prd{Deux}(\lambda y[\prd{langue}(y)\wedge\prd{étranger}(y)])(\lambda y\,\prd{apprendre}(x,y)))\)\\
Cette formule est vraie ssi l'intersection de l'ensemble des stagiaires et de ceux qui ont appris deux langues étrangères contient (au moins) 3 individus.  Il peut donc y avoir 6 langues différentes en jeu dans ce genre de situations.

\(\Xlo\prd{Deux}(\lambda y[\prd{langue}(y)\wedge\prd{étranger}(y)])(\lambda y\,\prd{Trois}(\prd{stagiaire})(\lambda x\,\prd{apprendre}(x,y)))\)\\
Cette traduction présente la lecture avec portées inversées que l'on obtient en appliquant \QRa\ (cf. \S\ref{ss:QR}).  Elle est vraie ssi l'intersection de l'ensemble des langues étrangères et de l'ensemble des choses qui ont été apprises par trois stagiaires contient (au moins) 2 éléments.  Il peut donc y avoir au total 6 stagiaires dans l'histoire (par exemple trois qui apprennent le japonais, et trois autres qui apprennent le russe). 
  \end{enumerate}
\end{solu}
\end{exo}




\subsection{Propriétés des déterminants}
%---------------------------------------
\label{ss:PtésDet}

Les travaux sur les quantificateurs généralisés se sont beaucoup attachés à étudier les propriétés sémantiques des déterminants au sein des structures tripartites.
En effet les relations ensemblistes dénotées par les déterminants ont diverses propriétés mathématiques qui donnent lieu à des caractérisations et des classifications sémantiquement pertinentes.  Nous n'allons pas ici en faire un inventaire exhaustif%
\footnote{Voir \citet{BarwiseCooper:81},\Andexn{Barwise, J.}\Andexn{Cooper, R.} 
\citet{KeenanStavi:86},\Andexn{Keenan, E.}\Andexn{Stavi, J.}
\citet{Westerstahl:89},\Andexn{Westerst{\aa }hl, D.} 
\citet[chap.~14]{PtMW:90},\Andexn{Partee, B.}\Andexn{ter Meulen, A.}\Andexn{Wall, R.} entre autres, pour des études approfondies.},
mais seulement présenter quelques unes de celles les plus souvent impliquées dans les analyses sémantiques.


Une des principales propriétés caractéristiques des déterminants est la \kw{conservativité}.  Pour la définir, introduisons d'abord un raccourci de notation dans {\LO} qui simplifiera les écritures : si \vrb\alpha\ et \vrb\beta\ sont de type \et, alors $\Xlo\alpha\gand\beta$ est de type \et\ et est sémantiquement équivalent à $\Xlo\lambda x[\alpha(x)\wedge\beta(x)]$.  On reconnaît bien sûr l'opération d'intersection entre ensembles : \(\Ch{\denote{\Xlo\alpha\gand\beta}}^{\Modele,w,g} = \Ch{\denote{\Xlo\alpha}}^{\Modele,w,g} \cap \Ch{\denote{\Xlo\beta}}^{\Modele,w,g}\).

\begin{defi}[Conservativité]\label{d:conservativité}
Un déterminant $\Xlo\delta$ est \kwo{conservatif} ssi pour tout $w$ et $g$:\\ \(\denote{\Xlo\delta(\alpha)(\beta)}^{\Modele,w,g}=\denote{\Xlo\delta(\alpha)(\alpha\gand\beta)}^{\Modele,w,g}\).
\end{defi}

Pour évaluer une structure tripartite $\Xlo\delta(\alpha)(\beta)$, on doit consulter les éléments qui sont dans la restriction \vrb\alpha\ et ceux qui sont dans la portée  \vrb\beta.  Ce que dit la définition~\ref{d:conservativité}, c'est que si \vrb\delta\ est conservatif, alors quand on consulte $\Ch{\denote{\Xlo\beta}}^{\Modele,w,g}$, on a en fait seulement besoin de prendre en compte les éléments qui sont déjà aussi dans la restriction $\Ch{\denote{\Xlo\alpha}}^{\Modele,w,g}$.  Autrement dit, on ne regarde jamais le contenu de $\Ch{\denote{\Xlo\beta}}^{\Modele,w,g}-\Ch{\denote{\Xlo\alpha}}^{\Modele,w,g}$. 
C'est une propriété très naturelle.  En effet lorsque l'on interprète par exemple \sicut{quelques enfants dorment}, parmi l'ensemble de tous les individus qui dorment, seuls comptent ceux qui sont des enfants, les autres dormeurs ne sont pas pertinents\footnote{\citet{BarwiseCooper:81} décrivent d'ailleurs cette propriété en disant que le quantificateur $\Xlo\delta(\alpha)$ «vit de» ou «se nourrit de» \vrb\alpha\ (ang.\ \alien{lives on \vrb\alpha}).}.
Et il a été observé empiriquement que la conservativité est une propriété universelle et inhérente des déterminants des langues naturelles : tout déterminant est conservatif.

Pour tester cette hypothèse, la définition~\ref{d:conservativité} nous offre un test linguistique assez simple qui consiste à comparer une phrase qui se traduit par $\Xlo\delta(\alpha)(\beta)$ avec une périphrase qui correspond à $\Xlo\delta(\alpha)(\alpha\gand\beta)$, comme ce qu'illustre {\Next} où $D$ est un déterminant singulier ou pluriel :

\ex.
\a.  \sicut{$D$ enfant dort} est équivalent à \sicut{$D$ enfant est un enfant qui dort}
\b.  \sicut{$D$ enfants dorment} est équivalent à \sicut{$D$ enfants sont des enfants qui dorment}


Les périphrases sont stylistiquement un peu lourdes, mais le test ne concerne que l'équivalence en termes de conditions de vérité.
C'est que l'on observe en \Next\ où les phrases (i) et (ii) de chaque paire sont logiquement équivalentes.

\ex.
\a.
\a. Un enfant dort.
\b. Un enfant est un enfant qui dort.
\z.
\b.
\a. La plupart des enfants dorment.
\b. La plupart des enfants sont des enfants qui dorment.
\z.
\b.
\a. Moins de cinq enfants dorment.
\b. Moins de cinq enfants sont des enfants qui dorment.
\z.


\sloppy

Si l'on trouve une expression de la langue qui peut se traduire par une relation ensembliste qui n'est pas conservative, alors, en vertu de l'hypothèse d'universalité de la conservativité pour les déterminants, on devra en conclure que cette expression n'est pas sémantiquement (et syntaxiquement) un déterminant.
L'exemple typique de cela est le cas de \sicut{seul} (et \sicut{only} en anglais) placé devant un groupe nominal, comme dans \sicut{seuls les enfants dorment}.
Nous nous doutons bien que ni \sicut{seul} ni \sicut{seul les} ne sont des déterminants, car ils n'en ont pas exactement les propriétés grammaticales, mais après tout, à l'instar de ce que nous faisons pour \sicut{tous les}, nous pourrions proposer une constante \prd{Seuls-les} de type \type{\et,\ett} telle que 
\(\denote{\Xlo\prd{Seuls-les}(R)(P)}^{\Modele,w,g}=1\) ssi
\(\Ch{\denote{\vrb P}}^{\Modele,w,g} \inclus \Ch{\denote{\vrb R}}^{\Modele,w,g}\).
Cela nous donnera les bonnes conditions de vérité puisque $\Xlo\prd{Seul-les}(\prd{enfant})(\prd{dormir})$ sera vrai ssi l'ensemble des dormeurs est inclus dans l'ensemble des enfants.
Cependant il sera préférable de ne pas retenir cette stratégie et de s'orienter plutôt vers une analyse plus compositionnelle (\ie\ où \sicut{seul} reçoit une interprétation propre indépendamment du déterminant), car \prd{Seul-les} n'est pas conservatif et par conséquent \sicut{seul les} ne fait pas partie de la catégorie des déterminants.  On s'en rend compte en français avec {\Next} : \Next[b] peut être vrai s'il y a aussi des adultes qui dorment (car ceux-ci ne sont pas des enfants qui dorment).

\fussy

\ex.
\a. Seuls les enfants dorment.
\b. Seuls les enfants sont des enfants qui dorment.



Une autre propriété intéressante et simple à caractériser est la \kwo{symétrie}\is{symetrie@symétrie}.

\begin{defi}[Symétrie]
Un déterminant $\Xlo\delta$ est \kwo{symétrique} ssi pour tout $w$ et $g$:\\ \(\denote{\Xlo\delta(\alpha)(\beta)}^{\Modele,w,g}=\denote{\Xlo\delta(\beta)(\alpha)}^{\Modele,w,g}\).
\end{defi}

\largerpage

Autrement dit, si \vrb\delta\ est symétrique, nous pouvons permuter la portée et la restriction dans une structure tripartite, nous aurons toujours les mêmes conditions de vérité.  Et cela nous donne un test simple pour vérifier la symétrie d'un déterminant D de la langue : il suffit d'invertir les contenus du NP sujet et du VP et de s'assurer que les deux phrases sont logiquement équivalentes\footnote{Nous serons bien conscients que les deux phrases ne sont pas équivalentes sur le plan pragmatique ou communicationnel ; là encore, nous ne regardons que l'équivalence logique.}, comme par exemple :

\ex.
\a.
\sicut{$D$ linguiste est alcoolique} est équivalent à \sicut{$D$ alcoolique est linguiste}
\b.
\sicut{$D$ linguistes sont alcooliques} est équivalent à \sicut{$D$ alcooliques sont linguistes}

Nous constatons assez rapidement qu'il y a des déterminants qui ne sont pas symétriques : dans \Next, les phrases (i) et (ii) de chaque paire ne sont pas équivalentes.  Mais de surcroît il se trouve qu'empiriquement on observe que seuls les indéfinis\is{indefini@indéfini} sont symétriques, \NNext. 

\ex.
\a.
\a. Tous les linguistes sont alcooliques.
\b. Tous les alcooliques sont linguistes.
\z.
\b.
\a. La plupart des linguistes sont alcooliques.
\b. La plupart des alcooliques sont linguistes.
\z.
\b.
\a. Le linguiste est alcoolique.
\b. L'alcoolique est linguiste.
\z.


\ex.
\a.
\a. Un linguiste est alcoolique.
\b. Un alcoolique est linguiste.
\z.
\b.
\a. Plusieurs linguistes sont alcooliques.
\b. Plusieurs alcooliques sont linguistes.
\z.
\b.
\a. Au moins trois linguistes sont alcooliques.
\b. Au moins trois alcooliques sont linguistes.
\z.
\b.
\a. Aucun linguiste n'est alcoolique.
\b. Aucun alcoolique n'est linguiste.
\z.


Au passage nous pouvons remarquer que la symétrie nous donne un argument supplémentaire pour ranger \sicut{aucun} parmi les indéfinis. 


Au chapitre précédent, \S\ref{sss:CL+Mon}, nous avons abordé la notion de monotonie\is{monotonie}  (dé)crois\-sante.
Il est intéressant de la reprendre ici car les déterminants illustrent ce phénomène avec une assez grande variété.
Comme les déterminants dénotent des fonctions à deux arguments, il y a deux positions où observer la monotonie : dans la restriction et dans la portée.
Pour ce qui est de la restriction, on parle généralement de monotonie à gauche, et pour la portée, de monotonie à droite\footnote{Dans \citet{BarwiseCooper:81}, la monotonie croissante à gauche est appelée \emph{persistance} et la monotonie décroissante à gauche \emph{anti-persistance}.}.
En reprenant la définition \ref{d:Monotonie} p.~\pageref{d:Monotonie}, nous pouvons alors caractériser quatre cas de figure :


\begin{defi}[Monotonie des déterminants]
  Soit $\vrb\delta\in\ME_{\type{\et,\ett}}$, et soit
  \vrb\alpha, \vrb\beta\ et $\vrb\gamma\in \ME_{\et}$ tels que $\vrb\alpha\satisf\vrb\beta$. 
\begin{enumerate}
  \item \vrb\delta\ est \kwo{monotone croissant à gauche} ssi $\xlo{\delta(\alpha)(\gamma)}\satisf\xlo{\delta(\beta)(\gamma)}$.
  \item \vrb\delta\ est \kwo{monotone décroissant à gauche} ssi $\xlo{\delta(\beta)(\gamma)}\satisf\xlo{\delta(\alpha)(\gamma)}$.
  \item \vrb\delta\ est \kwo{monotone croissant à droite} ssi $\xlo{\delta(\gamma)(\alpha)}\satisf\xlo{\delta(\gamma)(\beta)}$.
  \item \vrb\delta\ est \kwo{monotone décroissant à droite} ssi $\xlo{\delta(\gamma)(\beta)}\satisf\xlo{\delta(\gamma)(\alpha)}$.
\end{enumerate}
\end{defi}


De là, en reprenant les définitions sémantiques posées en \S\ref{ss:QGDet}, il est possible de démontrer logiquement que tel déterminant possède (ou non) telle et telle propriétés de monotonie.  Nous n'allons pas ici entrer dans les détails de ces démonstrations, mais simplement observer quelques exemples à partir de tests appliqués à des expressions de la langue.


Pour la monotonie à gauche, partons par exemple de la conséquence $\prd{lion}\satisf\prd{animal}$.  Nous voyons que les déterminants indéfinis \sicut{un}, \sicut{plusieurs}, \sicut{au moins deux} sont monotones croissants à gauche \Next[a-c], alors que les autres déterminants de \Next\ ne le sont pas.

\ex.
\a. Un lion s'est échappé $\satisf$ Un animal s'est échappé
\b. Plusieurs lions  se sont échappés $\satisf$ Plusieurs animaux se sont échappés
\b. Au moins deux lions se sont échappés $\satisf$ Au moins deux animaux se sont échappés
\b. Tous les lions  se sont échappés $\not\satisf$ Tous les animaux se sont échappés
\b. Les lions  se sont échappés $\not\satisf$ Les animaux se sont échappés
\b. Aucun lion ne s'est échappé $\not\satisf$ Aucun animal ne s'est échappé
\b. Moins de trois lions se sont échappés $\not\satisf$ Moins de trois animaux se sont échappés
\b. Exactement dix  lions  se sont échappés $\not\satisf$ Exactement dix animaux se sont échappés
\b. La plupart des lions  se sont échappés $\not\satisf$ La plupart des animaux se sont échappés


Parmi les déterminants non monotones croissants à gauche, \sicut{tous les}, \sicut{les}, \sicut{aucun}, \sicut{moins de $n$} sont monotones décroissants %à gauche
\Next[a-d].
Et nous remarquons que \sicut{la plupart} et \sicut{exactement $n$} ne sont ni l'un ni l'autre, \ie\ ils ne sont pas monotones à gauche.

\ex.
\a. %Un animal s'est échappé $\not\satisf$ Un lion s'est échappé
%\b. Plusieurs animaux  se sont échappés $\not\satisf$ Plusieurs lions se sont échappés
%\b. Au moins deux animaux se sont échappés $\not\satisf$ Au moins deux lions se sont échappés
%\b.
Tous les animaux  se sont échappés $\satisf$ Tous les lions se sont échappés
\b. Les animaux  se sont échappés $\satisf$ Les lions se sont échappés
\b. Aucun animal ne s'est échappé $\satisf$ Aucun lion ne s'est échappé
\b. Moins de trois animaux se sont échappés $\satisf$ Moins de trois lions se sont échappés
\b. Exactement dix  animaux  se sont échappés $\not\satisf$ Exactement dix lions se sont échappés
\b. La plupart des animaux  se sont échappés $\not\satisf$ La plupart des lions se sont échappés



Pour examiner la monotonie à droite, prenons la conséquence initiale \sicut{boire du whisky} $\satisf$ \sicut{boire de l'alcool}\footnote{Car nous savons que $\prd{whisky}\satisf\prd{alcool}$ et donc que $\Xlo\lambda x\exists y[\prd{whisky}(y)\wedge\prd{boire}(x,y)] \color{black}\satisf \Xlo\lambda x\exists y[\prd{alcool}(y)\wedge\prd{boire}(x,y)]$.}.
Cette fois, les déterminants monotones croissants à droite sont plus nombreux : \sicut{un}, \sicut{plusieurs}, \sicut{au moins $n$}, \sicut{tous les}, \sicut{les}, \sicut{la plupart}, \Next[a-e,i].  



\ex.
\a. Un invité a bu du whisky $\satisf$ Un invité a bu de l'alcool
\b. Plusieurs invités ont bu du whisky $\satisf$ Plusieurs invités ont bu de l'alcool
\b. Au moins deux invités ont bu du whisky $\satisf$ Au moins deux invités ont bu de l'alcool
\b. Tous les invités ont bu du whisky $\satisf$ Tous les invités ont bu de l'alcool
\b. Les invités ont bu du whisky $\satisf$ Les invités ont bu de l'alcool
\b. Aucun invité n'a bu du whisky $\not\satisf$ Aucun invité n'a bu de l'alcool
\b. Moins de trois invités ont bu du whisky $\not\satisf$ Moins de trois invités ont bu de l'alcool
\b. Exactement dix invités ont bu du whisky $\not\satisf$ Exactement dix invités ont bu de l'alcool
\b. La plupart des invités ont bu du whisky $\satisf$ La plupart des invités ont bu de l'alcool


\newpage

Quant à \sicut{aucun} et \sicut{moins de $n$}, ils sont monotones décroissants à droite.  Et finalement \sicut{exactement $n$} n'est monotone d'aucun côté.

\ex.
\a. Aucun invité n'a bu de l'alcool $\satisf$ Aucun invité n'a bu du whisky
\b. Moins de trois invités ont bu de l'alcool $\satisf$ Moins de trois invités ont bu du whisky
\b. Exactement dix invités ont bu de l'alcool $\not\satisf$ Exactement dix invités ont bu du whisky

%\input{tabDetQG}


\is{quantificateur!\elid\ generalise@\elid\ généralisé|)}


\section{Flexibilité et changements de types}
%============================================
\label{s:typeshift}

Depuis le début de ce chapitre, nous partons du principe que 
les types sémantiques sont directement corrélés aux catégories syntaxiques des expressions qu'ils étiquettent ainsi qu'à leurs  propriétés distributionnelles. 
C'est assez normal puisque les types servent à réglementer la composition sémantique qui, elle-même, est guidée par la structure syntaxique.
Cela a pour implication, au moins par défaut, que les types et les traductions sémantiques des expressions de base sont fixés dès le départ dans le lexique et qu'a priori toutes les expressions d'une catégorie morpho-syntaxique donnée se verront assigner le même type.  
Dans les pages précédentes, cependant, la possibilité d'associer à la catégorie DP le type \typ e ou le type \ett\ a été évoquée plusieurs fois, et c'est cette question de la flexibilité des types que nous allons aborder dans cette section.
La flexibilité des types\is{flexibilité des types} repose sur l'idée que certaines expressions peuvent recevoir différentes traductions formelles sans que cela donne lieu à de véritables ambiguïtés lexicales.
Nous allons prendre ici le temps nécessaire d'en exposer précisément les tenants et les aboutissants, en présentant les thèses et les analyses des travaux fondateurs de  
\citet{PartRooth:83} et \citet{Partee:87}\Andex{Partee, B.}\Andex{Rooth, M.}, car il s'agit d'une dimension fondamentale et indispensable du processus d'analyse à l'interface syntaxe-sémantique et elle doit être mise en \oe uvre avec rigueur et méthode.  Cela va également être l'occasion d'examiner comment notre système peut intégrer l'analyse de constructions que nous n'avons pas encore abordées comme la coordination et les verbes transitifs intensionnels.  C'est ce par quoi nous allons commencer.




\subsection{Coordination et connecteurs généralisés}
%---------------------------------------------------
\label{ss:ConConGen}\is{coordination|(}

Les connecteurs $\Xlo\wedge$ et $\Xlo\vee$ sont vérifonctionnels, et si nous traduisons les conjonctions de coordinations \sicut{et} et \sicut{ou}  respectivement par $\Xlo\lambda\psi\lambda\phi[\phi\wedge\psi]$ et $\Xlo\lambda\psi\lambda\phi[\phi\vee\psi]$ de type \type{t,\type{t,t}}, 
nous obtenons des traductions qui ne peuvent coordonner que des expressions de type \typ t, typiquement des phrases.  Et nous savons bien que \sicut{et} et \sicut{ou} peuvent coordonner de nombreuses autres catégories syntaxiques : DP, NP, VP, V, AP, PP, CP, etc.   Pour ces catégories les deux traductions précédentes ne sont pas utilisables\footnote{Sauf éventuellement pour certains VP par l'hypothèse de montée du sujet, \S\ref{sss:xMontees}.}, mais le \lcalcul\ nous permet néanmoins de composer leur coordination.  Regardons quelques exemples.

Pour ce faire, nous allons d'abord poser une hypothèse très simple et très rudimentaire sur la syntaxe de la coordination en français.  
Nous allons considérer que la coordination de deux constituants de catégorie X donne lieu à une structure syntaxique plate, c'est-à-dire à trois branches (ou plus) comme illustré en {\Next}%
\footnote{\label{fn:CoordP}Cette hypothèse est uniquement motivée par des raisons de commodité et de simplicité d'écriture pour la présente exposition. Les analyses sémantiques utilisées ici sont passablement compatibles avec une version binaire et plus hiérarchisée comme, par exemple, [\Stag{X} X [\Stag{CoordP} Coord X]].}.
%%
%%
%%\ex.
%%{\small\Tree
%%[.X  X Coord X ] }
%%
%%
À l'interface syntaxe-sémantique une conjonction Coord dénotera donc une fonction à deux arguments fournis en même temps.


\ex.
\fsynsem{\begin{tabular}[t]{rcccc}
    X & \reecr & X & Coord & X \\
    $\Xlo[[\gamma(\beta)](\alpha)]$ & \seecr & $\Xlo\alpha$ & $\Xlo\gamma$ & $\Xlo\beta$
  \end{tabular}}


Par exemple pour une coordination d'adjectifs prédicatifs comme en \Next[a], le V$'$ s'analysera comme en \Next[b] :

\ex.
\a. La serviette est propre et sèche.
\b.  \small
\Tree
[.V$'$ [.V \zcbox{est} ]
  [.AP\zbox{\ $\Xlo\lambda x[\prd{propre}(x)\wedge\prd{sec}(x)]$}
    [.AP propre\\\zcbox{$\Xlo\lambda x\,\prd{propre}(x)$} ] [.{\zcbox{Coord}} et ] [.AP sèche\\\zcbox{$\Xlo\lambda x\,\prd{sec}(x)$} ]
  ]
]
\normalsize


\sloppy

Les AP sont de type {\et}, le coordonnant \sicut{et} sera donc de type \type{\et,\type{\et,\et}} et se traduira par un \lterme\ qui produit un prédicat complexe en faisant la conjonction des deux prédicats donnés. C'est une opération que
nous avons déjà rencontrée (cf.\ \prdk{inter} p.~\pageref{HINTER}) :

\fussy

\ex.
\(\sicut{et} \leadsto \Xlo\lambda Q\lambda P\lambda x[[P(x)]\wedge[Q(x)]]\)
\label{x:et-ET}


Si nous voulons coordonner deux DP \Next, nous procéderons de manière similaire.  Le sujet de \Next\ se traduira par \(\Xlo\lambda P[[P(\cns g)] \wedge \prd{Treize}(\prd{nain})(P)]\) de type {\ett} (c'est un quantificateur généralisé qui dénote l'ensemble de toutes les propriétés satisfaites conjointement par Gandalf et treize nains).

\ex.
Gandalf et treize nains se sont invités chez Bilbo.


\sloppy
Les deux DP coordonnés se traduisent respectivement par $\Xlo\lambda P[P(\cns g)]$ et $\Xlo\lambda P[\prd{Treize}(\prd{nain})(P)]$, et le coordonnant \sicut{et} se traduit comme en \Next, recevant le type \type{\ett,\type{\ett,\ett}}, où \vrb X et \vrb Y sont des
variables de $\VAR_{\ett}$ :

\fussy

\ex.
\(\sicut{et} \leadsto \Xlo\lambda Y\lambda X\lambda P[[X(P)]\wedge[Y(P)]]\)
\label{x:et-ETT}


Remarquons en passant que cet exemple illustre l'une des motivations de la proposition de Montague de traiter uniformément tous les DP comme des quantificateurs généralisés.  Il est normal et logique que la coordination s'effectue entre expressions de même type, et si le nom propre \sicut{Gandalf} avait été traduit simplement par \cns g de type \typ e, il n'aurait pas été possible de le coordonner avec le quantificateur \sicut{treize nains}.
La traduction \Last\ résout ce problème dès lors que les noms propres (et tout autre DP référentiel) reçoivent le type \ett.


Mais il y a une autre observation plus déterminante à faire ici : nous constatons que les conjonctions \sicut{et} et \sicut{ou} recevront une traduction différente pour chaque type de constituants qu'elles peuvent coordonner.  Nous pouvons d'ailleurs facilement anticiper que pour coordonner des V transitifs ou des subordonnées complétives, les traductions seront encore différentes.  Concrètement cela veut dire que notre lexique sémantique devra prévoir une ambiguïté, et plus précisément une homonymie très multiple, pour ces deux conjonctions.
En pratique, comme le montrent les exemples précédents, ce n'est pas une situation insurmontable, mais d'un point de vue théorique et descriptif, c'est assez insatisfaisant.
Nous avons la forte intuition que \sicut{et} (et \sicut{ou}) est toujours la même conjonction de coordination quelles que soient les catégories qu'elle coordonne, avec une contribution sémantique stable dérivée de $\Xlo\wedge$ (respectivement $\Xlo\vee$)\footnote{En fait, il y a quelques exceptions pour \sicut{et}, ce sont ses emplois dits collectifs comme dans \sicut{Shaun et Sean se sont battus/rencontrés/ont déplacé un piano...}  Nous y reviendrons au chapitre \ref{GN++} (vol.~2).}.  Il serait donc plus naturel de les traiter sémantiquement de manière unifiée.
%car ce n'est pas très naturel. 




À cet effet, il existe
une stratégie, introduite notamment par \citet{Gzdr:80}\Andex{Gazdar, G.} puis reprise méthodiquement %attentivement 
par  \citet{PartRooth:83},   %\citet{GroeSto:89}
qui consiste à ajouter dans {\LO} de nouveaux symboles syncatégorématiques en \emph{supplément} des connecteurs vérifonctionnels.  
Ce sont ce que l'on appelle des \kwo{connecteurs généralisés}\is{connecteur!\elid\ généralisé}.  Nous utiliserons ici les symboles 
$\Xlo\gand$\footnote{Oui c'est le même symbole qu'en \ref{ss:PtésDet} \alien{supra}, et ce n'est pas un hasard. J'ajoute que les symboles introduits ici sont propres à cet ouvrage ; il existe diverses variantes de notations des connecteurs généralisés dans la littérature, toutes graphiquement assez proches.}, $\Xlo\gor$, $\Xlo\gneg$ pour, respectivement, la conjonction, la disjonction et  la négation\footnote{La négation généralisée peut s'avérer utile pour traiter des négations qui ne portent pas directement sur des formules, cf.\ par exemple avec les adjectifs \sicut{non-violent}, \sicut{pas cher} ou les noms comme \sicut{non-événement}...  Notons que nous pourrions très bien introduire de la même façon une implication matérielle généralisée ; quant à l'équivalence matérielle généralisée, nous l'avons déjà, c'est $\Xlo=$ (cf.\ \S\ref{ss:SemLOTypé} p.~\pageref{H=<->}).}
généralisées.
Leur définition, donnée ci-dessous, reprend le principe de la conséquence généralisée  que nous avons vue en \S\ref{sss:CL+Mon}, et  réutilise donc l'ensemble des types booléens $\TypesB$\is{type!\elid\ booléen} défini p.~\pageref{d:TBooleen}.  Rappelons que les types booléens sont ces types qui finissent toujours par «aboutir à~\typ t~».



\begin{defi}[Connecteurs généralisés]\label{d:ConnGen}
\begin{itemize}
\item[\small(Syn)]  Si $\mtyp b \in \TypesB$ et si \vrb\alpha\ et $\vrb\beta\in\ME_{\mtyp b}$, alors
  $\Xlo\gneg\alpha$, $\Xlo[\alpha\gand\beta]$ et $\xlo{[\alpha\gor\beta]}\in\ME_{\mtyp b}$.
\item[\small(Sem)]  Soit %\mtyp a un type de $\TypesB$ et
  $\vrb\alpha,\vrb\beta \in \ME_{\mtyp b}$, pour tout \Modele, $w$ et $g$ :
\begin{enumerate}
\item si $b=\typ{t}$ :
  \begin{enumerate}
  \item $\denote{\Xlo\gneg \alpha}^{\Modele,w,g} = \denote{\Xlo\neg \alpha}^{\Modele,w,g}$
  \item $\denote{\Xlo\alpha \gand \beta}^{\Modele,w,g} = \denote{\Xlo\alpha \wedge \beta}^{\Modele,w,g}$
  \item $\denote{\Xlo\alpha\gor \beta}^{\Modele,w,g} = \denote{\Xlo\alpha\vee \beta}^{\Modele,w,g}$
%%$\denote{\alpha\gimplq \beta}^{\Modele,w,g} = \denote{\alpha\implq \beta}^{\Modele,w,g}$,   
  \end{enumerate}
\item sinon, \ie\ si  $b=\mtype{a,c}$, et si $\vrb x \in \VAR_{\mtyp a}$ (et n'est pas libre dans \vrb\alpha\ et \vrb\beta) :
  \begin{enumerate}
  \item $\denote{\Xlo\gneg \alpha}^{\Modele,w,g} = \denote{\Xlo\lambda x \gneg [\alpha(x)]}^{\Modele,w,g}$
  \item $\denote{\Xlo\alpha \gand \beta}^{\Modele,w,g} = \denote{\Xlo\lambda x [[\alpha(x)] \gand [\beta(x)]]}^{\Modele,w,g}$
  \item $\denote{\Xlo\alpha\gor \beta}^{\Modele,w,g} = \denote{\Xlo\lambda x [[\alpha(x)] \gor [\beta(x)]]}^{\Modele,w,g}$ 
    %% $\denote{\alpha\gimplq \beta}^{\Modele,w,g} = \denote{\lambda x [\alpha(x)\gimplq \beta(x)]}^{\Modele,w,g}$,
  \end{enumerate}
\end{enumerate}
\end{itemize}
\end{defi}


Comme avec la conséquence généralisée, l'idée est de saturer progressivement les arguments de \vrb\alpha\ et \vrb\beta\ au moyen d'une variable de type approprié (variable qui est «re-abstraite» globalement avec les $\Xlo\lambda x$) jusqu'à arriver au type \typ t pour y appliquer nos connecteurs vérifonctionnels habituels%
\footnote{Deux remarques techniques mais importantes sur la définition~\ref{d:ConnGen} :  d'abord comme les définitions sémantiques sont présentées de manière récursive avec des équivalences %(\ie\ identités de dénotation)
  entre les expressions à interpréter et des \lterme s de \LO, on pourrait penser que les connecteurs généralisés ne sont en fait que des raccourcis graphiques de notations (justement parce qu'on les interprète en les remplaçant par d'autres expressions du langage).  C'est parce que j'ai choisi ici un format de présentation relativement simplifié mais facile à lire et à manipuler. En réalité les connecteurs généralisés ont une définition propre et directe dans le modèle (\ie\ non substitutive), mais celle-ci s'appuie sur des propriétés de structure algébrique du domaine des fonctions ; j'ai préféré la laisser de côté ici pour ne pas complexifier davantage la présente exposition. Cf.\ cependant \citet{PartRooth:83} et les références qui y sont citées pour plus de détails.

De plus la définition~\ref{d:ConnGen} est incomplète pour {\LO} (mais elle ne l'est pas pour {\LOz}), car par exemple \type{s,t} est aussi un type booléen, mais on ne sature pas (et n'abstrait pas) les arguments des intensions avec des variables. En toute rigueur, il faut donc ajouter des règles spéciales pour le cas $\mtyp b = \type{s,\mtyp a}$ : $\denote{\Xlo\alpha \gand \beta}^{\Modele,w,g} = \denote{\Xlo\Intn[\Extn\alpha \gand \Extn\beta]}^{\Modele,w,g}$, etc.}%
.
Ainsi, à présent, \sicut{et} et \sicut{ou} se traduiront toujours par $\Xlo\gand$ et $\Xlo\gor$ respectivement, quelque soit le type des constituants coordonnés, du moment que les deux aient le même type (et qu'il soit un type booléen).
On dit que ces opérateurs sont \kwo{polymorphiques}\is{polymorphique}, c'est-à-dire qu'ils peuvent se combiner avec des expressions de divers types tout en gardant une même sémantique générale.
Reprenons nos exemples précédents, leurs nouvelles traductions seront (en faisant quelques $\eta$-réductions) :

\ex.
\a. \(\sicut{propre et sèche} \leadsto \Xlo[\prd{propre}\gand\prd{sec}]\)
\b. \(\sicut{Gandalf et treize nains} \leadsto \Xlo[\lambda P[P(\cns g)]\gand\lambda P[\prd{Treize}(\prd{nain})(P)]]\)

\sloppy
En appliquant les définitions (ou plus exactement les substitutions) de la définition~\ref{d:ConnGen}, nous pouvons voir que ces traductions sont équivalentes à celles proposées précédemment.  Ainsi $\Xlo[\prd{propre}\gand\prd{sec}]$ équivaut à $\Xlo\lambda x [\prd{propre}(x)\gand\prd{sec}(x)]$ et, comme $\Xlo\prd{propre}(x)$ et $\Xlo\prd{sec}(x)$ sont de type \typ t, cela équivaut à $\Xlo\lambda x [\prd{propre}(x)\wedge\prd{sec}(x)]$.  
De même pour \Last[b] qui, par définition, équivaut à $\Xlo\lambda Q[[\lambda P[P(\cns g)](Q)]\gand[\lambda P[\prd{Treize}(\prd{nain})(P)](Q)]]$, ce qui par \breduc\ se simplifie en
$\Xlo\lambda Q[[Q(\cns g)]\gand[\prd{Treize}(\prd{nain})(Q)]]$, équivalent finalement à $\Xlo\lambda Q[[Q(\cns g)]\wedge[\prd{Treize}(\prd{nain})(Q)]]$.

\fussy

Puisque $\Xlo\gand$ et $\Xlo\gor$ sont syncatégorématiques ($\denote{\Xlo\gand}^{\Modele,w,g}$ et $\denote{\Xlo\gor}^{\Modele,w,g}$ ne sont pas définis), cela veut dire que  \sicut{et} et \sicut{ou} ne sont plus en soi  des expressions interprétables comme nous l'entendons habituellement. Autrement dit \sicut{et} et \sicut{ou}  ne peuvent pas se traduire directement et compositionnellement par $\Xlo\gand$ et $\Xlo\gor$.  Ces traductions doivent donc être introduites syncatégorémiquement via des règles d'interface syntaxe-sémantique comme en \Next.

\ex.
\fsynsem{\begin{tabular}[t]{rcc@{ }c@{ }c}
    X & \reecr & X & [\Stag{Coord} et] & X \\
    \mtyp b && \mtyp b && \mtyp b \\
    $\Xlo[\alpha\gand\beta]$ & \seecr & $\Xlo\alpha$ &  & $\Xlo\beta$
  \end{tabular}}
\quad
\fsynsem{\begin{tabular}[t]{rcc@{ }c@{ }c}
    X & \reecr & X & [\Stag{Coord} ou] & X \\
    \mtyp b && \mtyp b && \mtyp b \\
    $\Xlo[\alpha\gor\beta]$ & \seecr & $\Xlo\alpha$ &  & $\Xlo\beta$
  \end{tabular}}


Ces règles montrent que les conjonctions de coordination \sicut{et} et \sicut{ou} n'ont pas de type (et pas de traduction propre) et que chacune doit être traitée par une règle d'interface qui lui est spécifique (la couche syntaxique des règles doit aller regarder quelle conjonction précise est présente sous la projection Coord).  Nous y avons gagné en uniformité, mais un peu perdu en compositionnalité (en particulier si l'on adopte une structure syntaxique hiérarchique binaire pour traiter la coordination, cf. note~\ref{fn:CoordP} p.~\pageref{fn:CoordP}).
Pour sauver la compositionnalité, nous pouvons être tentés de proposer les traductions $\Xlo\lambda v\lambda u[u\gand v]$ et $\Xlo\lambda v\lambda u[u\gor v]$ pour \sicut{et} et \sicut{ou}.  Mais ce ne sont pas véritablement des \lterme s précis : ils n'ont pas de type propre tant que nous ne connaissons pas le type de \vrb u et \vrb v.  Ce sont, au mieux, deux «formats de \lterme s».
Mais, et c'est une caractéristique du polymorphisme, nous savons tout de même que ces traductions n'auront jamais n'importe quel type : ce sera toujours \mtype{b,\mtype{b,b}} pour $\vrb u, \vrb v \in \VAR_{\mtyp b}$ (et avec $b\in\TypesB$). 
C'est un peu comme si, dans l'interprétation, nous manipulions \mtyp b comme une «variable de types» (qui peut, virtuellement, parcourir tout $\TypesB$). 
De la sorte, ces formats $\Xlo\lambda v\lambda u[u\gand v]$ et $\Xlo\lambda v\lambda u[u\gor v]$ vont générer chacun toute une famille de \lterme s précis utilisables en fonction du type des constituants coordonnés.  Certes, techniquement cela nous fait revenir aux traductions \ref{x:et-ET} et \ref{x:et-ETT} proposées initialement, \emph{mais} cela nous permet néanmoins de donner un traitement «semi-compositionnel»%
\footnote{Concrètement cela veut dire que le lexique sémantique ne saura pas assigner une traduction exacte à, par exemple, la conjonction \sicut{et}, mais il lui attribuera la traduction générique $\Xlo\lambda v\lambda u[u\gand v]$ de type \mtype{b,\mtype{b,b}} et celle-ci s'instanciera précisément lors de l'application de la règle \ref{ri:Coord} en «découvrant» la valeur de \mtyp b (on pourra alors éventuellement renommer \vrb u et \vrb v pour rester cohérent avec les notations habituelles).}
de la coordination au moyen d'une seule règle générique à l'interface syntaxe-sémantique \ref{ri:Coord} qui vaut pour toute catégorie X (grammaticalement coordonnable) et où l'on prévoit que \vrb\gamma\ sera $\Xlo\lambda v\lambda u[u\gand v]$ ou $\Xlo\lambda v\lambda u[u\gor v]$ :


\ex.
\RISS{Coordination}{\begin{tabular}[t]{rcccc}
    X & \reecr & X & {Coord} & X \\
    \mtyp b && \mtyp b &\mtype{b,\mtype{b,b}}& \mtyp b \\
    $\Xlo[[\gamma(\beta)](\alpha)]$ & \seecr & $\Xlo\alpha$ & $\Xlo\gamma$ & $\Xlo\beta$
  \end{tabular}}\label{ri:Coord}


\sloppy

D'une certaine manière, ce traitement de la coordination par les connecteurs généralisés nous donne déjà un aperçu du principe de la flexibilité des types : la possibilité que certaines expressions de la langue reçoivent,  en fonction de leur environnement syntaxique,  des traductions de  différents types mais qui conservent un noyau sémantique commun.  C'est ce qui se passe avec $\Xlo\lambda v\lambda u[u\gand v]$ qui s'incarne sous divers types, mais qui contribue toujours à insérer $\Xlo\gand$ (et à terme $\Xlo\wedge$) dans la composition sémantique.
Cependant si \citet{PartRooth:83} utilisent les connecteurs généralisés,  c'est surtout pour introduire une réflexion sur un autre phénomène de flexibilité de types, plus saillant que celui-ci, et qui concerne les DP et les V.  Nous allons l'aborder dans la section suivante, et pour enchaîner, terminons ici en examinant le cas de la coordination de deux V transitifs\is{verbe!\elid\ transitif}  comme en \Next.

\fussy

\ex.
Laura écrit et compose des chansons.\label{x:LauraM}


Si, pour faire simple, nous considérons qu'un V transitif se traduit par un prédicat de type \eet\ (par exemple $\Xlo\lambda y\lambda x\, \prd{écrire}(x,y)$), alors le V coordonné [\Stag{V}~\sicut{écrit et compose}] se traduit par
\(\Xlo[\prd{écrire}\gand\prd{composer}]\) (\ie\ $\Xlo[\lambda y\lambda x\, \prd{écrire}(x,y)\gand\lambda y\lambda x\, \prd{composer}(x,y)]$).  Appliquons la définition~\ref{d:ConnGen} pour retrouver le \lterme\ traditionnel équivalent (sur la droite est indiqué le type des termes connectés par $\Xlo\gand$) :

\ex.
\a. \(\Xlo[\prd{écrire}\gand\prd{composer}]\) équivaut à : \hfill\eet
\b. \(\Xlo\lambda y[[\prd{écrire}(y)]\gand[\prd{composer}(y)]]\), qui  équivaut à : \hfill \et
\b. \(\Xlo\lambda y\lambda x[[[\prd{écrire}(y)](x)]\gand[[\prd{composer}(y)](x)]]\)%
\footnote{Remarque : ici nous avons appliqué une deuxième fois la règle d'équivalence de la définition, mais cette règle s'est appliquée sur $\Xlo[[\prd{écrire}(y)]\gand[\prd{composer}(y)]]$, pas sur tout le \lterme\ précédent, c'est pourquoi $\Xlo\lambda x$ apparaît bien à droite de $\Xlo\lambda y$.}, qui  équivaut à : \hfill\typ t
\b. \(\Xlo\lambda y\lambda x[\prd{écrire}(x,y)\wedge\prd{composer}(x,y)]\)


Pour dériver la traduction de \ref{x:LauraM}, comme \Last\ est de type \eet\ et que le DP objet est de type \ett, il faudra obligatoirement procéder à \alien{QR}, et au final nous obtiendrons comme il se doit :

\ex.
\(\Xlo\exists x [\prd{chanson}(x) \wedge[\prd{écrire}(\cns l,x)\wedge\prd{composer}(\cns l,x)]]\)
\label{x:LauraM1}


Mais justement, pour éviter un \alien{QR} obligatoire, nous pouvons avoir intérêt à reprendre les traductions de type \type{\ett,\et} des V transitifs, comme
\(\Xlo\lambda Y\lambda x[Y(\lambda y\,\prd{écrire}(x,y))]\).  
Voyons ce que donne alors la coordination :

\ex.
\a.  \(\Xlo[\lambda Y_1\lambda x_1[Y_1(\lambda y_1\,\prd{écrire}(x_1,y_1))]\gand\lambda Y_2\lambda x_2[Y_2(\lambda y_2\,\prd{composer}(x_2,y_2))]]\)
\\\stx\hfill {\small\type{\ett,\et}}
\b. \raggedright 
\(=\Xlo\lambda Y[[\lambda Y_1\lambda x_1[Y_1(\lambda y_1\,\prd{écrire}(x_1,y_1))](Y)] \gand \coupe \hstrab[2ex]
[\lambda Y_2\lambda x_2[Y_2(\lambda y_2\,\prd{composer}(x_2,y_2))](Y)]]\) 
\hfill {\small\et}
\b. \(= \Xlo\lambda Y[\lambda x_1[Y(\lambda y_1\,\prd{écrire}(x_1,y_1))]\gand\lambda x_2[Y(\lambda y_2\,\prd{composer}(x_2,y_2))]]\)\\
\stx\hfill {\small\breduc}
\b. \(=\Xlo\lambda Y\lambda x[[\lambda x_1[Y(\lambda y_1\,\prd{écrire}(x_1,y_1))](x)] \gand \coupe \hstrab[2ex]
[\lambda x_2[Y(\lambda y_2\,\prd{composer}(x_2,y_2))](x)]]\) 
\hfill {\small\typ t}
\b. \(=\Xlo\lambda Y\lambda x[[Y(\lambda y_1\,\prd{écrire}(x,y_1))]\wedge[Y(\lambda y_2\,\prd{composer}(x,y_2))]]\)
%\stx
\hfill {\small\breduc}


Cette fois pour traduire \ref{x:LauraM} nous n'avons pas besoin d'effectuer \alien{QR}, et dans ce cas nous obtenons :

\ex.
\(\Xlo\exists x[\prd{chanson}(x)\wedge\prd{écrire}(\cns l,x)]\wedge\exists x[\prd{chanson}(x)\wedge\prd{composer}(\cns l,x)]\)
\label{x:LauraM2}


\newpage
\sloppy 

Ce n'est pas la même chose que \ref{x:LauraM1}.  Est-ce grave ? Oui. 
Pour s'en convaincre clairement il suffit de comparer les deux traductions que l'on obtient pour l'exemple \Next[a] :  \Next[b] procède de la coordination de deux prédicats de type \eet, et \Next[c] de celle de deux prédicats de type \type{\ett,\et}.

\fussy

\ex.
\a. Charles a épluché et émincé deux oignons.
\b. \(\Xlo\prd{Deux}(\prd{oignon})(\lambda y[\prd{éplucher}(\cns c,y)\wedge\prd{émincer}(\cns c,y)])\)
\b. \(\Xlo\prd{Deux}(\prd{oignon})(\lambda y\,\prd{éplucher}(\cns c,y))\wedge\prd{Deux}(\prd{oignon})(\lambda y\,\prd{émincer}(\cns c,y))\)


\Last[b] dit qu'il y a deux oignons que Charles a épluchés et émincés.  \Last[c] dit qu'il y a deux oignons qu'il a épluchés et qu'il y a deux oignons qu'il a émincés ; ce peut être deux fois les mêmes oignons, mais ce n'est pas nécessaire, il peut y en avoir quatre dans l'histoire.  Or nous n'interprétons jamais \Last[a] de cette façon (possiblement 4 oignons), mais toujours à la façon de \Last[b] (ce sont les mêmes oignons qui sont épluchés et émincés).

Cela nous donne un argument supplémentaire pour nous faire préférer l'option qui traduit les verbes transitifs par des prédicats de type \eet\ à celle qui les traduit par des prédicats de type \type{\ett,\et}. Mais en fait... ce n'est pas si simple.



\is{coordination|)}


\subsection{Verbes transitifs intensionnels}
%-------------------------------------------
\label{ss:VTinten}\is{intensionnel!verbe \elid}
\is{verbe!\elid\ transitif!\elid\ intensionnel} 

Pour  continuer notre exploration de la flexibilité des types,  nous devons faire une petite parenthèse qui va nous donner l'occasion de poursuivre la présentation de certains éléments de l'analyse de \citet{PTQ}\Andex{Montague, R.} et de revenir sur une ambiguïté \alien{de re}/\alien{de dicto}\is{de dicto@\alien{de dicto}}\is{de re@\alien{de re}} que nous avions laissé de côté dans l'analyse compositionnelle.  C'est celle qui apparaît dans la position de complément d'objet de verbes comme \sicut{chercher} :

\ex.
Arthur cherche une licorne. \label{x:Licorne1}


\sloppy

Nous qualifierons ces verbes transitifs\footnote{J'utilise ici \sicut{transitif} dans une acception très relâchée pour désigner des verbes qui prennent (au moins) deux arguments réalisés par des DP ou des PP.} d'\kwo{intensionnels}, justement parce qu'ils installent un environnement opaque  sur leur position de complément, ce qui en permet une lecture \alien{de dicto}.
Font partie de cette catégorie 
\sicut{chercher},
\sicut{désirer},
\sicut{vouloir},
\sicut{exiger},
\sicut{rêver de},
\sicut{avoir besoin de},
\sicut{avoir envie de},
\sicut{avoir peur de}...
Ce sur quoi nous devons réfléchir ici c'est la nature de la dénotation de ces verbes afin de rendre compte convenablement de leur sens.  Et cela se reflète directement sur le type des prédicats à utiliser.



Car nous ne pouvons pas partir de prédicats de type \eet\ comme nous le faisons avec les verbes transitifs ordinaires (extensionnels).  Si, par exemple, \prdi{chercher}1 est un prédicat de ce type, que la traduction du verbe se présente sous la forme \(\Xlo\lambda y\lambda x\,\prdi{chercher}1(x,y)\)  (type \eet) ou \(\Xlo\lambda Y\lambda x[Y(\lambda y\,\prdi{chercher}1(x,y))]\) (type \type{\ett,\et}), nous n'obtiendrons, pour \Last, que la lecture \alien{de re}, c'est-à-dire
\(\Xlo\exists y[\prd{licorne}(y)\wedge\prdi{chercher}1(\cns a,y)]\).
C'est parce que \prdi{chercher}1 dénotera une relation entre deux individus du domaine, et ce n'est pas ce qu'il nous faut pour la lecture \alien{de dicto}. 


Une constante de prédicat de type \type{\ett,\et}, bien que plus élaborée, ne suffira pas encore. Un tel prédicat, par exemple \prdi{chercher}2, dénotera une relation entre un individu (Arthur) et un quantificateur généralisé (l'ensemble de tous les ensembles qui contiennent au moins une licorne).  
Cela nous donnera \(\Xlo\prdi{chercher}2(\cns a,\lambda P\exists y[\prd{licorne}(y)\wedge P(y)]])\) pour la lecture \alien{de dicto}.
Dans un monde $\w_1$ (comme le nôtre) où les licornes n'existent pas, \Last\ peut tout de même être vraie, si Arthur croit que les licornes existent. Si nous faisons usage du prédicat \prd{exister} comme suggéré en \S\ref{sss:mondepossible} (p.~\pageref{prd:exister})\footnote{Si nous n'en faisons pas usage, alors l'option \prdi{chercher}2 de type \type{\ett,\et} est immédiatement disqualifiée car, dans $\w_1$, $\Xlo\lambda P\exists y[\prd{licorne}(y)\wedge P(y)]$ dénotera l'ensemble vide, comme pour tout quantificateur concernant des créatures fictives ; \ref{x:Licorne1} aura alors les mêmes conditions de vérité que \sicut{Arthur cherche un dragon}, \sicut{Arthur cherche un dahu}, etc.\  ce qui n'a pas lieu d'être, y compris pour les lectures \alien{de dicto}. },
alors la dénotation de \prd{licorne} dans $\w_1$ ne sera pas vide, elle contiendra toutes les licornes imaginaires de \Unv A, mais aucune de celles-ci ne sera dans la dénotation de \prd{exister}.  Or la lecture \alien{de dicto} de \ref{x:Licorne1} implique qu'Arthur croit que les licornes existent, par conséquent \(\Xlo\prdi{chercher}2(\cns a,\lambda P\exists y[\prd{licorne}(y)\wedge P(y)]])\) devrait être équivalent à \(\Xlo\prdi{chercher}2(\cns a,\lambda P\exists y[\prd{licorne}(y)\wedge \prd{exister}(y) \wedge P(y)]])\)\footnote{Cf. la définition \ref{xd:chercher} \alien{infra}.}.  Mais ce n'est pas le cas, parce que \(\Ch{\denote{\Xlo\lambda P\exists y[\prd{licorne}(y)\wedge \prd{exister}(y) \wedge P(y)]}^{\Modele,\w_1,g}}\) est l'ensemble vide (dans $\w_1$ il n'y a pas de \vrb y qui à la fois soit une licorne et existe) alors que \(\Ch{\denote{\Xlo\lambda P\exists y[\prd{licorne}(y) \wedge P(y)]}^{\Modele,\w_1,g}}\) contient tous les ensembles qui contiennent au moins une licorne fictive.  Nous aboutissons donc à une contradiction et une incohérence si nous nous attachons à formaliser soigneusement la lecture \alien{de dicto}.
%Par conséquent la dénotation dans $\w_1$ de $\Xlo\lambda P\exists y[\prd{licorne}(y)\wedge P(y)]$ contiendra  
%% Or dans un monde (comme le nôtre) où les licornes n'existent pas mais où \Last\ peut être vraie, ce quantificateur sera l'ensemble vide, comme pour tout DP renvoyant à des créatures fictives ; \Last\ aura alors les mêmes conditions de vérité que «Arthur cherche un dragon», «Arthur cherche un dahu», etc.\  ce qui n'a pas lieu d'être.

  
  En fait, nous nous en doutons bien, les verbes transitifs intensionnels, comme les verbes d'attitudes propositionnelles (dont ils sont très proches), doivent prendre un argument de type intensionnel.  Dans \Last, \sicut{chercher} dénote une relation entre l'individu Arthur et la  propriété d'être une licorne.  Mais attention, il ne s'agit pas exactement de la propriété que nous formalisons par $\Intn\prd{licorne}$, de type \type{s,\et}, car celle-ci n'est pas assez structurée.  En effet nous avons aussi besoin de distinguer les conditions de vérité de \Last\ de celles de \sicut{Arthur cherche deux licornes}, \sicut{Arthur cherche toutes les licornes de la région}, etc.
  C'est pourquoi l'analyse de \citet{PTQ}\Andex{Montague, R.} propose que l'argument du verbe soit ce que nous pouvons appeler un \emph{concept de quantificateur généralisé} ou, dans les termes de \citeauthor{PTQ}, une \emph{propriété de propriétés},\is{propriete@propriété!\elid\ de propriétés}
c'est-à-dire une expression de type \type{s,\ett}%
  \footnote{En fait j'adopte ici la simplification suggérée et argumentée entre autres par \citet[chap.~7, p.~188 et n.~14 p.~250]{DWP:81}.\Andexn{Dowty, D.}\Andexn{Wall, R.}\Andexn{Peters, S.} 
Dans \citet{PTQ}, les quantificateurs dénotent des ensembles de \emph{propriétés de concepts d'individus} et sont donc de type \type{\type{s,\type{\type{s,e},t}},t} et les arguments objets des verbes sont alors de type \type{s,\type{\type{s,\type{\type{s,e},t}},t}}. Nous ne bouderons pas le présent choix de simplification.\label{fn:DWP81simpl}}.
Ainsi, si \vrbS Y est une variable de type \type{s,\ett}, \sicut{chercher} se traduira  par :


\fussy


\ex.
\(\sicut{chercher} \leadsto \Xlo\lambda\vrbS Y\lambda x\,\prd{chercher}(x,\vrbS Y)\)

\largerpage[-1]

Compositionnellement, le DP objet n'aura pas besoin d'être lui-même de type intensionnel, il continuera à être de type \ett\ ($\Xlo\lambda P\exists y[\prd{licorne}(y)\wedge P(y)]$), et la composition s'effectuera par application fonctionnelle intensionnelle (\S\ref{ss:AFInt}). De la sorte, \ref{x:Licorne1} se traduit en \Next\ pour la lecture \alien{de dicto} :


\ex.
\a. \(\sicut{une licorne} \leadsto \Xlo\lambda P\exists y[\prd{licorne}(y)\wedge P(y)]\)
\b. \sicut{cherche une licorne} $\leadsto$
\\*
\(\begin{array}[t]{@{}llr}
\Xlo [\lambda\vrbS Y\lambda x\,\prd{chercher}(x,\vrbS Y)(\Intn\lambda P\exists y[\prd{licorne}(y)\wedge P(y)])]
& \text{(AFI)}\\
= \Xlo\Xlo\lambda x\,\prd{chercher}(x,\Intn\lambda P\exists y[\prd{licorne}(y)\wedge P(y)])&
\text{(\breduc)}
\end{array}
\)
\b.
\(\sicut{Arthur cherche une licorne} \leadsto \Xlo\prd{chercher}(\cns a,\Intn\lambda P\exists y[\prd{licorne}(y)\wedge P(y)])\)\label{x:Licorne1dd}


La lecture \alien{de re} s'obtient par \alien{QR} avec une trace nécessairement de type {\ett} :

\ex.
\a.
\(t_1 \leadsto \Xlo\lambda P [P(x_1)]\)
\b. \(\sicut{cherche $t_1$} \leadsto {\Xlo[\lambda\vrbS Y\lambda x\,\prd{chercher}(x,\vrbS Y)(\Intn\lambda P [P(x_1)])] }\)\\
\(= {\Xlo\lambda x\,\prd{chercher}(x,\Intn\lambda P [P(x_1)])}\)
\b. \(\sicut{ Arthur cherche $t_1$} \leadsto  {\Xlo\prd{chercher}(\cns a,\Intn\lambda P [P(x_1)])}\)
\b. 
  \([\sicut{une licorne}_1\ [\sicut{Arthur cherche $t_1$}]]   \leadsto {} \)\\
  \(\begin{array}[t]{@{}l@{\ }l}
  &\Xlo[\lambda P\exists y[\prd{licorne}(y)\wedge [P(y)]](\lambda x_1\,\prd{chercher}(\cns a,\Intn\lambda P[P(x_1)]))]\\
  =& \Xlo\exists y[\prd{licorne}(y)\wedge [\lambda x_1\,\prd{chercher}(\cns a,\Intn\lambda P[P(x_1)])(y)]] \\
   =& \Xlo\exists y[\prd{licorne}(y)\wedge \prd{chercher}(\cns a,\Intn\lambda P[P(y)])]\footnotemark
\end{array}
\)\footnotetext{Cette dernière \breduc\ sur \vrbi x1 est autorisée car \vrb y étant une variable de type \typ e, interprétée seulement par $g$, elle peut se placer sans risque dans la portée de \xlo{\textIntn}.}


\sloppy\largerpage[-1]

Nous avons déterminé le type des verbes intensionnels, c'est \type{\type{s,\ett},\et}, mais cela ne suffit à définir leur sens.
Nous devons au moins esquisser les conditions de vérité d'une expression de la forme $\Xlo\prd{chercher}(x,\vrbS Y)$. %\fixme{***}
Une propriété de propriétés, comme $\Xlo\Intn\lambda P\exists y[\prd{licorne}(y)\wedge P(y)]$, dénote une fonction qui à chaque monde de \Unv W associe l'ensemble de tous les ensembles qui contiennent au moins une licorne de ce monde.
Il y a certains mondes possibles, comme le nôtre, où les licornes n'existent pas, et la fonction leur associera un ensemble contenant par exemple l'ensemble des équidés, l'ensemble des individus blancs, des individus gris, des individus à crinière, des farouches, des féroces, etc.\ bref tout ensemble correspondant à une propriété extensionnelle d'une ou plusieurs licornes telles qu'elles sont imaginées dans ce monde, \emph{mais pas} la dénotation de \prd{exister}.
Mais il y a d'autres mondes, appelons-les génériquement $w'$, où les licornes existent bel et bien et à ces $w'$ la fonction associera un ensemble plus ou moins similaire au précédent mais comprenant en plus la dénotation de \prd{exister}  dans $w'$ (qui contiendra donc les licornes réelles de $w'$).  Et parmi ces mondes $w'$, il existe en particulier certains mondes $w''$ auxquels la fonction associera un ensemble qui contient aussi l'ensemble de toutes les choses trouvées par Arthur ; ce sont les mondes où Arthur trouve effectivement une licorne.  À partir de là, nous pouvons développer les conditions de vérité de \ref{x:Licorne1dd} : dans un monde quelconque $w$, \ref{x:Licorne1dd} est vraie ssi Arthur pense, dans $w$, que $w$ appartient à l'ensemble de ces mondes $w'$ et Arthur souhaite, dans $w$, que $w$ fasse partie de ces mondes $w''$.  Pour généraliser, nous pouvons poser la définition suivante :

\fussy

\ex. \label{xd:chercher}
\(\denote{\Xlo\prd{chercher}(x,\vrbS Y)}^{\Modele,w,g}=1\)
ssi dans $w$ :
\a.[i.] \(\denote{\vrb x}^{\Modele,w,g}\) pense que $w$ appartient à l'ensemble des mondes $w'$ tels que \(\denote{\Xlo[\Extn\vrbS Y(\prd{exister})]}^{\Modele,w',g}=1\), \pagebreak et
\b.[ii.] \(\denote{\vrb x}^{\Modele,w,g}\) souhaite que $w$ appartienne à l'ensemble des mondes $w''$ tels que 
\(\denote{\Xlo[\Extn\vrbS Y(\lambda y\,\prd{trouver}(x,y))]}^{\Modele,w'',g}=1\).%
\footnote{Nous voyons ainsi que le sens de \prd{chercher} est défini à partir de celui de \prd{trouver} : la définition ramène \sicut{chercher} à \sicut{vouloir trouver}. À ce sujet, il faut admettre que la définition \ref{xd:chercher} est incomplète, car normalement on ne cherche pas quelque chose sans rien faire ; il faudrait ajouter une condition qui dit aussi que \vrb x entreprend dans $w$ une action qui vise à rendre $\Xlo[\Extn\vrbS Y(\lambda y\,\prd{trouver}(x,y))]$ vrai.  }


\sloppy
\citet{PTQ}\Andex{Montague, R.} traite \emph{potentiellement} {tous} les verbes transitifs comme étant de type \type{\type{s,\ett},\et}\footnote{Enfin, plus exactement, il leur assigne le type \type{\type{s,\type{\type{s,\type{\type{s,e},t}},t}}, \type{\type{s,e},t} } ; cf. note \ref{fn:DWP81simpl} \alien{supra}.}
et donc comme dénotant une relation entre un individu et une propriété de propriétés, 
\emph{y compris} les verbes transitifs extensionnels comme \sicut{manger}, \sicut{trouver}, \sicut{capturer}, etc.\  que nous comprenons intuitivement comme dénotant une relation entre deux individus et que nous avons pris l'habitude de traduire ici via le type \eet.
Or il est naturel \emph{et} raisonnable d'estimer qu'une phrase comme par exemple \sicut{Alice mange une tartine} exprime avant tout une relation concrète entre l'individu Alice et l'individu tartine plutôt qu'une relation un peu plus abstraite entre Alice et le concept de quantificateur \sicut{être une tartine}. 
À cet égard, Montague réintroduit les prédicats de type \eet\ sous la forme de variantes de traduction pour les verbes extensionnels (et seulement eux) ; ainsi pour \sicut{manger} qui se traduit par \prd{manger} de type \type{\type{s,\ett},\et} il existe aussi une traduction de type \eet\ notée 
\prdx{manger}.
Les deux  sont sémantiquement reliées entre elles par un postulat de signification qui, en suivant notre simplification des types, peut se formuler comme suit :

\ex.
\(\Xlo\doit\forall x\forall\vrbS Y  [ \prd{manger}(x,\vrbS Y) \ssi [\Extn\vrbS Y(\lambda y\,\prdx{manger}(x,y))] ]\)


\fussy


En fait nous avons intérêt à prendre \Last\ comme une manière de définir le sens «abstrait» de \prd{manger} à partir du sens plus concret de \prdx{manger}, en considérant que celui-ci est le «sens de base» de \sicut{manger}.
Cependant Montague adopte un point de vue qui peut sembler inverse car dans ses analyses, un verbe comme \sicut{manger} est lexicalement %initialement %et compositionnellement
traduit par \prd{manger} puis éventuellement simplifié en \prdx{manger} en vertu du postulat \Last.
C'est conforme à sa stratégie d'uniformiser le type de chaque grande catégorie syntaxique (la généralisation au pire des cas) afin d'obtenir un système régulier et pleinement compositionnel. 
Cela permet notamment de rendre opérationnelle la coordination de n'importe quels verbes transitifs, ce qui nous ramène finalement à notre préoccupation initiale.  Mais auparavant faisons un petit point de notation : pour ne pas perturber nos habitudes d'écriture, nous n'allons pas utiliser les notations «~\prdx{}~» de Montague, nous continuerons à écrire \prd{manger} pour représenter le prédicat de type \eet\ (\ie\ le \prdx{manger} de \Last), et pour la traduction de  
type \type{\type{s,\ett},\et} (\ie\ le \prd{manger} de \Last) nous écrirons 
$\Xlo\lambda\vrbS Y\lambda x[\Extn\vrbS Y(\lambda y\,\prd{manger}(x,y))]$.

\medskip

Dans la section précédente, nous avons vu que des verbes transitifs comme \sicut{éplucher} et \sicut{émincer} (qui, de fait, sont extensionnels) doivent se traduire par des prédicats de type \eet, afin de pouvoir les coordonner correctement, car \sicut{Charles a épluché et émincé deux oignons} n'est pas équivalent à \sicut{Charles a épluché deux oignons et Charles a émincé deux oignons}. Ici nous venons de voir que les verbes transitifs intensionnels doivent se traduire par des prédicats de type \type{\type{s,\ett},\et}. %\footnote{Ou, à la rigueur, par des prédicats de type \type{\ett,\et}.}.
De ce fait, la coordination de deux verbes intensionnels, comme par exemple \sicut{avoir envie et besoin de} se traduira par \(\Xlo[\prd{envie-de}\gand\prd{besoin-de}]\) équivalant à \(\Xlo\lambda\vrbS Y\lambda x[\prd{envie-de}(x,\vrbS Y)\wedge\prd{besoin-de}(x,\vrbS Y)]\).
Cette traduction prédit que \Next[a], pour la lecture \alien{de dicto}, est équivalent à \Next[b],  les deux se traduisant par \Next[c] :


\ex.
\a. Alice a envie et besoin d'un  ordinateur.
\b. Alice a envie d'un ordinateur et Alice a besoin d'un ordinateur.
\b. \raggedright
\(\Xlo\prd{envie-de}(\cns a,\Intn\lambda P\exists y[\prd{ordinateur}(y) \wedge P(y)]) \wedge \coupe 
\prd{besoin-de}(\cns a,\Intn\lambda P\exists y[\prd{ordinateur}(y) \wedge P(y)])\)


Il se trouve que cette prédiction est confirmée : en français \Last[a] et \Last[b] se comprennent de la même façon (pour la lecture \alien{de dicto}) ; la question de savoir en \Last[b] s'il s'agit du même ordinateur ou de deux ordinateurs différents ne se pose pas (car l'existence de l'ordinateur n'est que virtuelle, localisée dans les envies et les besoins d'Alice).  
Et, ce qui est encore plus crucial, nous pouvons faire la même observation lorsque nous coordonnons un verbe intensionnel avec un extensionnel, comme en {\Next} : \Next[a] a les mêmes conditions de vérité que \Next[b]. 


\ex.
\a. Arthur a cherché et trouvé une licorne.
\b. Arthur a cherché une licorne et Arthur a trouvé une licorne.


\sloppy

Ainsi \sicut{trouver}, qui se traduit normalement par $\Xlo\lambda y\lambda x\,\prd{trouver}(x,y)$ de type \eet, devra ici se traduire par $\Xlo\lambda\vrbS Y\lambda x[\Extn\vrbS Y(\lambda y\,\prd{trouver}(x,y))]$  de type 
\type{\type{s,\ett},\et} afin de pouvoir se coordonner avec \sicut{chercher}.
\sicut{Chercher et trouver} se traduit alors par
$\Xlo[\lambda\vrbS Y\lambda x\,\prd{chercher}(x,\vrbS Y) \gand \lambda\vrbS Y\lambda x[\Extn\vrbS Y(\lambda y\,\prd{trouver}(x,y))]$, équivalant à 
$\Xlo\lambda\vrbS Y\lambda x[\prd{chercher}(x,\vrbS Y) \wedge [\Extn\vrbS Y(\lambda y\,\prd{trouver}(x,y))]$.
À terme, \Last[a] se traduira par \Next, qui est aussi la traduction de \Last[b].

\fussy

\ex.
\(\Xlo\prd{chercher}(\cns a,\Intn\lambda P\exists y[\prd{licorne}(y)\wedge P(y)]) \wedge \exists y[\prd{licorne}(y) \wedge \prd{trouver}(\cns a,y)]\)


\sloppy

Récapitulons.  La traduction des verbes transitifs intensionnels est forcément de type \type{\type{s,\ett},\et}.  Mais pour les transitifs extensionnels, nous sommes obligés de prévoir deux traductions : une traduction du type simple \eet, que nous \emph{devons} utiliser notamment pour les coordonner entre eux, et une traduction du type complexe \type{\type{s,\ett},\et}, notamment pour les coordonner avec des verbes intensionnels.  Nous n'avons pas le choix, les deux versions doivent être prises en compte, nous ne pouvons pas tout unifier en généralisant au type le plus complexe.
Et cela peut également avoir une conséquence sur la traduction des DP argumentaux des verbes.  Si nous nous donnons les moyens d'avoir des DP de type \typ e (pour les expressions référentielles), ceux-ci pourront se combiner directement avec les verbes extensionnels de type \eet.  En revanche ils devront obligatoirement passer au type \ett\ lorsqu'ils sont l'argument d'un verbe intensionnel, comme dans \sicut{Alice cherche Dina}. 
%Inversement, 
Nous sommes en plein dans la problématique de la flexibilité et du changement de types.

\fussy


%chap. \ref{Ch:t+m} note \ref{fn:dere-exister} p. \pageref{fn:dere-exister}


\subsection{\alien{Type-shifting}}
%---------------------------------
\is{type-shifting@\textit{type-shifting}|(}

Les observations précédentes ont amené 
\citet{PartRooth:83}\Andex{Partee, B.}\Andex{Rooth, M.}
à introduire un mode opératoire déterminant pour la conception du  mécanisme d'analyse sémantique compositionnelle.  
Le principe est le suivant : dans le lexique sémantique, les expressions de la langue (désambiguïsées) reçoivent une seule traduction, qui rend compte de leur interprétation la plus naturelle et qui correspond, normalement, au type le plus simple que l'on peut leur assigner ; c'est cette traduction qui est introduite par défaut dans l'analyse à l'interface syntaxe-sémantique ; et si dans la composition sémantique il s'avère que cette traduction ne s'accorde pas avec son environnement (à cause d'un conflit de types), alors l'expression subit une opération de changement de type (et donc de traduction) qui résout la discordance.
Ces opérations de changement de type portent traditionnellement  le nom de \kwo{type-shifting}.

Finalement c'est bien ce que nous avons rencontré dans les pages qui précèdent.  Les verbes transitifs extensionnels sont lexicalement de type \eet, et ils s'élèvent au type \type{\type{s,\ett},\et} lorsqu'ils se coordonnent avec un verbe intensionnel (et éventuellement au type \type{\ett,\et} lorsque leur objet est un quantificateur). 
C'est également une façon de revisiter le comportement des adjectifs épithètes que nous avons discuté en \S\ref{ss:ISSmodifieurs} : nous pouvons considérer que les adjectifs (non intensionnels) sont initialement de type \et\ et passent au type \type{\et,\et} lorsqu'ils modifient un N ou un NP. 
Et dans une certaine mesure le polymorphisme des connecteurs généralisés peut être vu comme une intégration «dynamique» d'un processus de \alien{type-shifting} sur des connecteurs originellement de type \type{t,\type{t,t}}.


Un des aspects les plus cruciaux du principe du \alien{type-shifting} est qu'il évite de multiplier les ambiguïtés artificielles dans le lexique : les variantes de traductions sont activées et construites au fil de l'analyse, et le \alien{type-shifting} se présente ainsi comme une stratégie \emph{de dernier recours}, c'est-à-dire à n'utiliser qu'en cas de conflit de types. 
C'est important car le \alien{type-shifting} peut s'avérer un outil très puissant (voire trop) qui risquerait de surgénérer des interprétations indésirables. Il est nécessaire de le discipliner, non seulement en le rendant non prioritaire, mais aussi en limitant son champ d'application.  Nous ne changeons pas le type et la traduction d'une expression juste parce que cela nous arrange.  Il faut que la transformation soit sémantiquement naturelle, qu'elle n'altère pas le sens de base de l'expression et si possible qu'elle soit défendable sur le plan psycholinguistique.  
%**** \fixme{***} **** développer l'argument psycholing ?
De manière générale, le \alien{type-shifting} concerne essentiellement les propriétés combinatoires des expressions, pas leur sens fondamental.
Les principes fondamentaux de cette opération ont été posés par 
\citet{Partee:87}\Andex{Partee, B.}, en l'appliquant essentiellement à l'analyse des DP.  Nous allons voir ici comment sa proposition s'intègre dans notre système~{\LO}. 



\subsubsection{Formalisation}
%''''''''''''''''''''''''''''

Techniquement le \alien{type-shifting} consiste à  passer d'une expression de type \mtyp a à une expression  de type \mtyp b.  L'opération peut donc tout naturellement se réaliser dans {\LO} au moyen d'un \lterme\ de type \mtype{a,b}.
Ce \lterme\ s'appliquera à l'expression de départ et, de la sorte, le résultat sera «fonction» de celle-ci. Cela \emph{peut} contribuer à maintenir raisonnablement le sens initial lors de l'opération.  
De tels \lterme s, prévus à cet usage, sont ce que l'on appelle des \emph{type-shifteurs}.\is{type-shifteur|sqq} 

Une façon de mettre en \oe uvre le \alien{type-shifting} dans notre système est 
de l'intégrer à des règles d'interface syntaxe-sémantique, pour ainsi le déclencher à partir d'une certaine configuration syntaxique et une certaine combinaison de types.
Voici un exemple générique d'une telle règle, qui présuppose l'existence d'un \alienx{type-shifteur} \vrb S de type \mtype{a,b} :

\ex.
\RISS{Exemple de \alien{type-shifting} 1}%
{\begin{tabular}{rccc}
X & \reecr & Y & Z \\
\small \mtyp c  & & \small\mtype{b,c} & \small\mtyp a \\
$\Xlo[\alpha([S(\beta)])]$ &\seecr & \vrb\alpha & \vrb\beta 
  \end{tabular}%
}\label{ri:extypeshift1}


Cette règle présente un cas de conflit de types : Y attend un argument de type \mtyp b et Z est de type \mtyp{a} ; mais en appliquant \vrb S à \vrb\beta\ dans la composition de X nous lui faisons subir un changement de type,  et nous obtenons $\Xlo[S(\beta)]$ de type \mtyp b.  

Insistons sur le fait que, le \alien{type-shifting} étant une stratégie de dernier recours, la légitimité de \ref{ri:extypeshift1} dans la grammaire doit normalement être conditionnée à la présence par ailleurs d'une règle comme \ref{ri:extypeshift0}, c'est-à-dire une règle de composition ordinaire sans conflit de types (X, Y et Z étant les même catégories qu'en \ref{ri:extypeshift1}).
Autrement dit, \ref{ri:extypeshift1} s'appliquera parce que \ref{ri:extypeshift0} a échoué.

\ex.
\fsynsem%
{\begin{tabular}[t]{rccc}
X & \reecr & Y & Z \\
\small \mtyp c  & & \small\mtype{b,c} & \small\mtyp b \\
$\Xlo[\alpha(\beta)]$ &\seecr & \vrb\alpha & \vrb\beta 
  \end{tabular}%
}\label{ri:extypeshift0}


La règle \ref{ri:extypeshift1} opère un changement de type sur l'argument (\vrb\beta).  Nous pouvons aussi concevoir des règles qui opèrent sur l'expression fonctionnelle (\vrb\alpha) comme \ref{ri:extypeshift2}, où cette fois \vrb S est %un \alienx{type-shifteur} 
de type \mtype{\mtype{b,c},\mtype{a,c}}.


\ex.
\RISS{Exemple de \alien{type-shifting} 2}%
{\begin{tabular}{rccc}
X & \reecr & Y & Z \\
\small \mtyp c  & & \small\mtype{b,c} & \small\mtyp a \\
$\Xlo[[S(\alpha)](\beta)]$ &\seecr & \vrb\alpha & \vrb\beta 
  \end{tabular}%
}\label{ri:extypeshift2}


\largerpage

Des règles telles que \ref{ri:extypeshift2} sont parfois considérées comme devant être moins prioritaires que les règles \ref{ri:extypeshift1}, afin de laisser la préférence au \alien{type-shifting} sur les arguments.  C'est cependant ce genre de règles qui s'applique si l'on souhaite élever un V transitif de type \eet\ au type \type{\ett,\et} pour le combiner directement avec un quantificateur généralisé.  Le \alienx{type-shifteur} sera alors de type 
\type{\eet,\type{\ett,\et}}, défini par le \lterme\ \ref{ts:vt-qg}.
Il est courant de poser des raccourcis pour ces \lterme s 
sous la forme de prédicats équivalents, %de même type, 
afin de simplifier les notations\footnote{Les définitions comme \ref{ts:vt-qg} représentent donc des postulats de signification ; j'omets ici les opérateurs modaux dans leurs notations. De plus les noms de raccourcis que j'introduis ici, comme \prdk{vt-qg} et \prdk{vi-int}, sont arbitraires et propres à cet ouvrage.  Ceux de \S\ref{sss:TSDP} sont plus standards.}.  Ainsi $\Xlo[\prdk{vt-qg}(\prd{manger})]$ est la version \type{\ett,\et} de \sicut{manger}.

\ex.
\(\prdk{vt-qg} = \Xlo\lambda V\lambda Y\lambda x[Y(\lambda y[[V(y)](x)])] \) 
\quad avec $\vrb V\in \VAR_{\eet}$ et $\vrb Y\in\VAR_{\ett}$\label{ts:vt-qg}

\sloppy
Ce \lterme\ peut sembler complexe, mais il suffit d'y voir que \vrb V sert à accueillir le prédicat verbal de type \eet\ pour construire des traductions que nous avons plusieurs fois manipulées (cf. \S\ref{sss:Vtrans}).  
Et le \alienx{type-shifteur} qui élève les verbes transitifs extensionnels au type \type{\type{s,\ett},\et} pour les coordonner avec des transitifs intensionnels sera très similaire, sauf qu'il sera de type
\type{\eet,\type{\type{s,\ett},\et}} :


\fussy

\ex.
\(\prdk{vt-int} = \Xlo\lambda V\lambda\vrbS Y\lambda x[\Extn\vrbS Y(\lambda y[[V(y)](x)])]\)
%\hfill%\quad 
avec $\vrb V\in \VAR_{\eet}$ et $\vrbS Y\in\VAR_{\type{s,\ett}}$


Quant au \alienx{type-shifteur} qui transforme un adjectif prédicatif de type \et\ en épithète de type \type{\et,\et}, nous l'avons déjà rencontré en 
\S\ref{ss:ISSmodifieurs} (p.~\pageref{HINTER}), c'est le prédicat que nous avons appelé \prdk{inter} de type \type{\et,\type{\et,\et}} :

\ex.
\(\prdk{inter} = \Xlo\lambda Q\lambda P\lambda x [[P(x)]\wedge[Q(x)]]\)



Notons aussi 
que les règles d'interfaces proposées en \ref{ss:ArgImpl} (pp. \pageref{ri:VTabs} et \pageref{ri:VTa0}) pour traiter les arguments implicites des verbes transitifs pourraient, en quelque sorte, être vues comme une application de \alien{type-shifting} où le conflit de type se manifeste entre le type \et\ attendu pour V$'$ et le type \eet\ fourni par V.  Les \alienx{type-shifteurs} en jeu seraient alors $\Xlo\lambda V\lambda x\exists y[[V(y)](x)]$ et $\Xlo\lambda V[V(y)]$%
\footnote{Cependant, il faut reconnaître que ce second \lterme\ $\Xlo\lambda V[V(y)]$ n'est probablement pas un véritable \alienx{type-shifteur} car il introduit une variable libre, ce que, formellement, il n'est pas censé faire.} 
de type \type{\eet,\et} (avec $\vrb V \in\VAR_{\eet}$).





\subsubsection{Les types des DP}
%'''''''''''''''''''''''''''''''

\citet{Partee:87}\Andex{Partee, B.} rappelle d'abord
que les trois types que l'on peut assigner naturellement aux DP sont :

\begin{itemize}
\item \typ e (le type référentiel), 
\item \et\ (le type prédicatif) et 
\item \ett\ (le type quantificationnel).  
\end{itemize}

Le premier, \typ e, est le type le plus simple avec lequel on peut traduire les noms propres, les pronoms personnels et les descriptions définies ; \ett\ est le type le plus simple pour les quantificateurs généralisés.
Quant au type prédicatif \et, nous n'avons pas eu l'habitude de l'utiliser pour traduire des DP, mais il existe des arguments pour cette option, en particulier concernant les DP indéfinis\footnote{Voir aussi \citet{SorinBeyssade:05}\Andexn{Dobrovie-Sorin, C.}\Andexn{Beyssade, C.} et les références qui y sont citées pour un panorama complet sur ces arguments.}.\is{indefini@indéfini}

\largerpage

En français par exemple, des DP indéfinis peuvent apparaître en position d'attribut du sujet \Next[a] dans la même distribution qu'un NP «nu» \Next[b] ou d'un AP \Next[c].

\ex.
\a.  Charlotte est une actrice.
\b. Charlotte est actrice.
\b. Charlotte est célèbre.


En \ref{sss:Vtrans} (p.~\pageref{x:spiderman2}) nous avons vu une traduction (dans l'esprit de \citealt{PTQ}) du verbe \sicut{être} «transitif» pour analyser \Last[a].  Partee propose une autre stratégie qui est de toujours traduire la copule de la même manière, par $\Xlo\lambda P\lambda x[P(x)]$, et d'analyser tous les attributs par des expressions de type \et. % \ (qui le type naturel des AP et des NP).  
De la sorte, dans \Last[a], l'indéfini \sicut{une actrice} aura une traduction équivalente à $\Xlo\lambda x\,\prd{actrice}(x)$
comme le NP \sicut{actrice} de \Last[b].

Par ailleurs il a été également défendu l'hypothèse, notamment par \citet{McNally:92phd,McNally:98}\Andex{McNally, L.} (dans la lignée de \citealt{Milsark:77}),\Andex{Milsark, G.} que le DP dit pivot des phrases existentielles\is{existentielle (phrase \elid)} comme \Next\ devait s'analyser comme dénotant une propriété.
Dans cette perspective, en simplifiant un peu\footnote{En fait l'analyse de McNally est plus élaborée, mais  ce qui est présenté ici reste un peu dans le même esprit. En particulier, il s'agit de se distinguer de ce que proposent \citet{BarwiseCooper:81} qui traitent le DP pivot comme un quantificateur généralisé ($\Xlo\lambda P\,\prd{Un}(\prd{feu-rouge})(P)$) et \sicut{il y a} comme le prédicat \prd{exister}  destiné à occuper la portée du quantificateur. À l'arrivée on obtient les mêmes conditions de vérité, mais l'analyse compositionnelle est différente. En particulier l'option de Barwise \& Cooper explique moins facilement le contraste entre les groupes nominaux forts et faibles dans les existentielles que nous avions vu en \S\ref{sss:Dfortfaible} puisque tout DP peut se traduire par un quantificateur généralisé.}, 
c'est \sicut{il y a} qui introduit la condition d'existence en se traduisant par $\Xlo\lambda P\exists x[[P(x)]\wedge\prd{exister}(x)]$ et le DP correspond simplement à $\Xlo\lambda x\,\prd{feu-rouge}(x)$.

\ex.
Il y a un feu rouge.


Dans une langue comme le français, si l'on veut analyser directement et compositionnellement les indéfinis par le type \et, il faudra redéfinir la traduction des déterminants concernés, par exemple en posant que \sicut{un} se traduit par $\Xlo\lambda P\lambda x [P(x)]$\footnote{Pour les autres déterminants indéfinis, c'est moins immédiat, mais néanmoins réalisable grâce à la formalisation que nous verrons au chapitre \ref{GN++} (vol.~2).} (ce qui le rendra, de fait, «synonyme» du verbe \sicut{être}, mais ce n'est peut-être pas un hasard).  Mais des DP de type \et\ peuvent aussi s'obtenir par des opérations naturelles de \alien{type-shifting} que décrit \citet{Partee:87}.


\subsubsection{Les changements de type des DP}
%'''''''''''''''''''''''''''''''

\label{sss:TSDP}

Nous allons examiner une par une les diverses opérations sémantiques qui permettent de passer d'un type \typ e, \et\ ou \ett\ à un autre.  Cela va créer un réseau de \alien{type-shifting} en triangle qui sera résumé dans la figure \ref{f:trianglePartee} p. \pageref{f:trianglePartee}.


\paragraph*{De {e} à \texttype{\texttype{e,t},t}}
%¨¨¨¨¨¨¨¨¨¨¨¨¨¨¨¨¨¨¨¨¨¨¨¨¨¨¨¨¨¨¨¨¨¨¨¨¨¨¨¨¨¨¨¨¨¨¨

Suivant la philosophie du \alien{type-shifting}, les DP référentiels auront initialement et par défaut le type \typ e, en se traduisant par des constantes ou variables d'individus ou des \atoi-termes. 
Passer, par exemple, de la constante \cns a au quantificateur correspondant $\Xlo\lambda P[P(\cns a)]$ est une opération relativement simple (nous l'avons appliquée en \S\ref{ss:QG.1}).  
Dans le modèle, cela consiste à rassembler tous les ensembles qui contiennent \Obj{Alice}, et Partee nomme \prdk{lift} le \alienx{type-shifteur} qui effectue cette opération.  Sa définition dans {\LO} est la suivante :

\ex.
\(\prdk{lift} = \Xlo\lambda x \lambda P [P(x)]\)
\hfill type \type{e,\ett}


\prdk{lift} sera à utiliser pour coordonner une expression référentielle avec un quantificateur (cf.\ \sicut{Gandalf et treize nains}, \sicut{le président et tous les ministres}...) ainsi que pour la combiner avec un verbe transitif intensionnel (cf. \sicut{Alice cherche Dina}). 



\paragraph*{De \texttype{\texttype{e,t},t} à e}
%¨¨¨¨¨¨¨¨¨¨¨¨¨¨¨¨¨¨¨¨¨¨¨¨¨¨¨¨¨¨¨¨¨¨¨¨¨¨¨¨¨¨¨¨¨¨¨

\sloppy
Comme \prdk{lift} dénote une fonction, nous pouvons envisager la fonction inverse, que Partee nomme \prdk{lower}.  Être l'inverse de \prdk{lift} signifie que \prdk{lower} réalise les «trajets retours» de \prdk{lift} et satisfait ainsi  le postulat de signification 
\(\Xlo\doit\forall x [\prdk{lower}(\prdk{lift}(x)) = x]\).
Informellement, étant donné la définition de \prdk{lift}, \prdk{lower} peut être vu comme dénotant cette fonction qui prend en argument un ensemble d'ensembles et retourne l'élément (s'il existe) qui appartient à tous ces ensembles.  En \lterme\ cela donne :

\fussy

\ex.
\(\prdk{lower} = \Xlo \lambda X\atoi y \forall P[[X(P)] \implq [P(y)]]\)
\hfill type \type{\ett,e}\label{xd:lower}


Dans cette définition, \vrb X est une variable de quantificateur généralisé (donc d'ensemble d'ensembles) et la condition $\Xlo\forall P[[X(P)] \implq [P(y)]]$ dit que tout «ensemble» \vrb P qui est dans \vrb X contient \vrb y. 
La fonction dénotée par \prdk{lift} est \emph{totale}, c'est-à-dire qu'elle donne un résultat pour tout élément de son domaine $\DoM_{\typ e}$ ; au contraire \prdk{lower} dénote une fonction \emph{partielle}, c'est-à-dire qu'il y a des éléments du domaine $\DoM_{\ett}$ pour lesquels elle n'est pas définie.  
C'est une conséquence de la présence de $\Xlo\atoi y$. 
En effet pour certains quantificateurs il n'existe pas de \vrb y qui satisfasse la condition ci-dessus, et pour d'autres il en existe plusieurs (cf. exercice \ref{exo:LOWER} \alien{infra}).

\paragraph*{De {e} à \texttype{e,t}}
%¨¨¨¨¨¨¨¨¨¨¨¨¨¨¨¨¨¨¨¨¨¨¨¨¨¨¨¨¨¨¨¨¨

Il s'agit de passer d'un individu de \Unv A à un ensemble d'individus.
La manière la plus simple et immédiate de procéder est, à partir d'un individu \Obj x, de retourner le singleton \set{\Obj x}.  Le \alienx{type-shifteur} qui effectue cela est la relation d'identité sur $\DoM_{\typ e}$ (cf. \S\ref{sss:lectureltermes}) :

\ex.
\(\prdk{ident} = \Xlo\lambda y \lambda x [x=y]\)
\hfill type \eet


Pour une valeur donnée de l'argument \vrb y, $\Xlo\lambda x[x=y]$ dénote l'ensemble de «tous les» \vrb x qui sont identiques à \vrb y, autrement dit c'est bien l'ensemble \set{\denote{\vrb y}^{\Modele,w,g}}.
L'application $\Xlo\prdk{ident}(\cns a)$ équivaut donc à $\Xlo\lambda x[x=\cns a]$ qui est la propriété (extensionnelle) \sicut{d'être Alice}.
Élevée au type \et\ par \prdk{ident} une expression référentielle peut ainsi intervenir comme attribut de la copule ($\Xlo\lambda P\lambda x[P(x)]$).


\paragraph*{De \texttype{e,t} à {e}}
%¨¨¨¨¨¨¨¨¨¨¨¨¨¨¨¨¨¨¨¨¨¨¨¨¨¨¨¨¨¨¨¨¨

Le type-shifting de \et\ à \typ e (\ie\ d'un ensemble à un individu) est un peu moins immédiat, mais commençons par un cas très simple, celui où l'ensemble de départ est un singleton. Il suffit alors d'extraire l'unique élément de l'ensemble. C'est la fonction inverse de \prdk{ident} et Partee nomme \prdk{iota} ce \alienx{type-shifteur} :

\ex.
\(\prdk{iota} = \Xlo\lambda P\atoi x[P(x)]\)
\hfill type \type{\et,e}


Ainsi $\Xlo\prdk{iota}(\prd{pape})$ équivaut à $\Xlo\atoi x\,\prd{pape}(x)$ et dénote l'unique individu qui est pape dans le monde d'évaluation.
En soi \prdk{iota} peut servir de traduction pour l'article défini\is{defini@défini!article \elid} \sicut{le}/\sicut{la}%
\footnote{À ce moment-là, \prdk{iota} ne sera pas véritablement un \alienx{type-shifteur}, mais simplement la traduction directe du déterminant défini singulier. En revanche dans les langues qui ne réalisent pas les articles (elles sont nombreuses, comme le japonais, le russe, le latin, etc.), \prdk{iota} en tant que \alienx{type-shifteur} permet de convertir un NP nu en une expression de type \typ e qui pourra être argument d'un verbe. }, afin d'obtenir des DP définis de type \typ e qui se combinent directement avec des verbes de type \et\ ou \eet.
Mais attention, il ne s'agit pas d'un déterminant équivalent à ce que nous avons vu en \S\ref{ss:déterminants} et \S\ref{ss:QGDet} où les déterminants sont de type \type{\et,\ett} et produisent des quantificateurs généralisés. 

Comme \prdk{lower}, \prdk{iota} dénote une fonction partielle (cf. $\Xlo\atoi x$ dans \Last) qui ne retourne un résultat que si son argument est un singleton. 
Cependant il est possible d'envisager d'autres opérations moins restreintes qui produisent un individu à partir d'un ensemble.

En particulier,  à partir d'un ensemble quelconque (mais non vide) il est toujours possible d'extraire un de ses éléments «au hasard».  Celui-ci apparaîtra alors comme un représentant du contenu de l'ensemble.  
Une fonction qui effectue cette opération s'appelle une \kwo{fonction de choix}\is{fonction!\elid\ de choix}.
Et il se trouve que l'utilisation de cet outil formel a été introduite %et défendue 
en sémantique, notamment par  \citet{Reinhart:97}\Andex{Reinhart, T.} et \citet{Winter:97}\Andex{Winter, Y.}, pour traduire les indéfinis et rendre compte des phénomènes de portées exceptionnelles qu'ils manifestent.

Même si traditionnellement les fonctions de choix ne sont pas manipulées comme des \alienx{type-shifteurs} proprement dits, je les présente ici car elles ont un  fonctionnement très proche, du fait qu'elles nous conduisent du type \et\ au type \typ e de façon relativement naturelle.
Par exemple, si \vrb f de type \type{\et,e} dénote une fonction de choix, alors $\Xlo f(\prd{enfant})$ dénotera un certain enfant.  
De cette façon, nous obtenons une traduction de type \typ e pour un indéfini comme \sicut{un enfant}, et nous pouvons écrire directement $\Xlo\prd{dormir}(f(\prd{enfant}))$.  Évidemment il ne suffit pas d'être de type \type{\et,e} pour être une fonction de choix :  comme $\Xlo f(P)$ dénote un élément de la dénotation de \vrb P, alors par définition toute fonction de choix \vrb f doit nécessairement satisfaire la formule $\Xlo P(f(P))$ pour tout prédicat \vrb P.
Et pour chaque prédicat, une fonction de choix donnée choisit toujours le même élément ; ce qui veut dire qu'il existe potentiellement une infinité de fonctions de choix différentes (afin que tout élément de tout ensemble puisse être choisi).  
C'est pourquoi les fonctions de choix sont généralement manipulées sous formes de variables, et  pour obtenir les conditions de vérité correctes de \Next[a], il est nécessaire ensuite de quantifier existentiellement la fonction comme en \Next[b] :

\ex.
\a. Un enfant dort.
\b. $\Xlo\exists f\,\prd{dormir}(f(\prd{enfant}))$\footnote{Cette formule se glose par : il existe une façon de choisir un individu de la dénotation de \prd{enfant} telle que cet individu dort. Remarquons en revanche que si l'indéfini a une lecture spécifique, alors \vrb f peut rester libre, dans une approche qui n'est pas très éloignée de ce que nous proposions en \ref{ss:specificite}.}


L'analyse compositionnelle des indéfinis au moyen de fonctions de choix n'est donc pas complètement élémentaire. Elle prévoit de traduire le déterminant indéfini par \Next\ où \vrb f est une variable localement libre de fonction de choix, mais cette variable doit ensuite être liée par $\Xlo\exists f$.  La question cruciale est alors de savoir comment et surtout où placer cette quantification existentielle ; elle n'a pas vraiment de réponse simple et consensuelle, et cela fait partie des doutes et critiques généralement adressés à cette approche, qui, si elle a des vertus, ne semble pas en mesure de faire complètement l'unanimité dans le domaine. 

\ex.
\(\sicut{un} \leadsto \Xlo\lambda P [f(P)]\)

%$\Xlo f^c$

Les fonctions de choix sont en fait un outil formel qui permet de manipuler l'\emph{axiome du choix} introduit en mathématiques par le logicien E.\ Zermelo.\ia{Zermelo, Ernst}  Cet axiome dit que pour toute collection d'ensembles non vides, il est  toujours possible d'accéder à un élément quelconque de chacun de ces ensembles.   
Il existe un autre moyen que les fonctions de choix d'user de cet axiome dans un langage formel, par l'intermédiaire de l'opérateur $\Xlo\epsi$ (\emph{epsilon})\is{epsilon@$\epsi$ (epsilon)} défini par le mathématicien D.\ Hilbert.\ia{Hilbert, David}  Pour dire les choses sommairement, $\Xlo\epsi$ est aux indéfinis ce que $\Xlo\atoi$ est aux descriptions définies : si $\Xlo\phi$ est de type \typ t et \vrb v une variable de type \typ e, $\Xlo\epsi v\phi$ est une expression de type \typ e et elle dénote un élément \Obj x de \Unv A, quelconque mais tel que \(\denote{\Xlo\phi}^{\Modele,w,g[\vrb v/\Obj x]}=1\).  Ainsi, c'est une sorte d'opérateur de «tirage aléatoire».
Là encore, $\Xlo\epsi$ n'est pas en soi un \alienx{type-shifteur}, mais $\Xlo\lambda P\epsi x[P(x)]$ nous fait passer du type \et\ au type \typ e d'une façon qui correspond à ce que fait le déterminant \sicut{un}.  C'est notamment l'approche défendue par \citet{Heusinger:00}\Andex{von Heusinger, K.} pour rendre compte des propriétés référentielles particulières des indéfinis.
L'opérateur $\Xlo\epsi$ n'est pas confronté au problème de la quantification existentielle des fonctions de choix, mais les deux doivent faire face au problème des ensembles vides : mathématiquement, si \vrb P dénote l'ensemble vide, alors $\Xlo f(P)$ et $\Xlo\epsi x[P(x)]$ ne sont pas définis, mais sémantiquement ces expressions doivent dénoter un élément de \Unv A pour que les formules qui les incluent puissent avoir une valeur de vérité.  De plus, cet élément doit avoir la particularité de n'appartenir à aucun sous-ensemble de \Unv A, ou du moins de ne satisfaire aucun prédicat de \LO\ dans aucun monde possible (ce qui est un véritable paradoxe mathématique). Ce n'est pas un problème entièrement insurmontable, mais qui demande des aménagements {ad hoc} pour garantir le bon fonctionnement formel des \vrb f et $\Xlo\epsi$ en sémantique%
\footnote{\citet{Winter:97} traite ce problème pour les fonctions de choix en leur attribuant le type \type{\et,\ett} et non \type{\et,e}.}.


\paragraph*{De \texttype{e,t} à \texttype{\texttype{e,t},t}}
%¨¨¨¨¨¨¨¨¨¨¨¨¨¨¨¨¨¨¨¨¨¨¨¨¨¨¨¨¨¨¨¨¨¨¨¨¨¨¨¨¨¨¨¨¨¨¨¨¨¨¨¨¨¨¨¨

Cela fait un bon moment dans ces pages que nous produisons compositionnellement des expressions de type \ett\ à partir d'expressions de type {\et} :  par l'utilisation de déterminants de type \type{\et,\ett} pour former des DP (quantificateurs généralisés) à partir de NP.  Les déterminants ne sont bien sûr pas des \alienx{type-shifteurs}, mais nous pouvons nous en inspirer et c'est ce que fait Partee en proposant le \alienx{type-shifteur} \prdk{a} dont la définition n'est autre que celle du déterminant indéfini \sicut{un}.

\ex.
\(\prdk{a} = {\Xlo\lambda Q\lambda P \exists x [[Q(x)] \wedge [P(x)]]} = {\Xlo\prd{Un}} \)
\hfill type \type{\et,\ett}


\prdk{a} fait double emploi avec notre déterminant, mais il peut être utile en tant que \alienx{type-shifteur} dans les langues qui ne réalisent pas l'article indéfini.  

À partir du déterminant défini singulier,\is{defini@défini!article \elid} Partee propose également le \alienx{type-shifteur} \prdk{the} avec la définition suivante :

\ex.
\(\prdk{the} = \Xlo\lambda Q\lambda P\exists x[\forall y [[Q(y)] \ssi y=x]\wedge [P(x)]]\)
\hfill type \type{\et,\ett}


Je conserve ici l'appellation originale \prdk{the} de Partee pour éviter toute confusion avec le déterminant \prd{Le} défini en \ref{DET:Le} p.~\pageref{DET:Le}.  Car les deux ne sont pas équivalents.  En fait \prdk{the} encode l'analyse de Russell que nous avons vue au chapitre~\ref{ch:gn} (\S\ref{sss:Russell}). Si dans le monde d'évaluation il n'existe pas de pape (ou s'il y en a plusieurs), alors $\Xlo\prd{Le}(\prd{pape})$, équivalent à $\Xlo\lambda P[P(\atoi x\,\prd{pape}(x))]$, ne sera pas défini ; en revanche $\Xlo\prdk{the}(\prd{pape})$, équivalent à $\Xlo\lambda P\exists x[\forall y [\prd{pape}(y) \ssi y=x]\wedge P(x)]$ sera toujours défini et dénotera en l'occurrence la fonction qui renvoie toujours $0$ quelle que soit la propriété extensionnelle donnée en argument --~en termes ensemblistes, c'est l'ensemble vide. 
Autrement dit \prdk{the} ne présuppose pas  la condition d'existence et d'unicité des définis, il l'affirme.  Il a donc nécessairement un usage différent de \prd{Le}, et cet usage correspond précisément aux cas où des DP définis\is{defini@défini!groupe nominal \elid} s'avèrent ne pas ne déclencher leurs présuppositions habituelles.
Or c'est possiblement ce qui se passe dans les exemples {\Next} :


\ex.
\a. Je suis le roi de France.\label{x:jesuisleroi}
\b. Alphonse a rencontré la femme de sa vie.
\b. L'élève qui résoudra tous les problèmes du devoir en moins d'une heure aura un point supplémentaire.


\sloppy

Notons que pour toute expression \vrb\alpha\ de type \et, $\Xlo\prd{Le}(\alpha)$ équivaut à $\Xlo\prdk{lift}(\prdk{iota}(\alpha))$ ; dans le système des \alienx{type-shifteurs} \prd{Le} devient donc superflu puisque l'on commence toujours par assigner à une expression sa traduction la plus simple.  Nous avons donc quatre possibilités pour traduire un DP défini : $\Xlo\prdk{iota}(\alpha)$ la traduction de base (présuppositionnelle), $\Xlo\prdk{lift}(\prdk{iota}(\alpha))$ la version quantificateur généralisé, $\Xlo\prdk{ident}(\prdk{iota}(\alpha))$ pour l'emploi prédicatif et $\Xlo\prdk{the}(\alpha)$ pour le quantificateur généralisé non présuppositionnel. 

\fussy

\paragraph*{De \texttype{\texttype{e,t},t} à \texttype{e,t}}
%¨¨¨¨¨¨¨¨¨¨¨¨¨¨¨¨¨¨¨¨¨¨¨¨¨¨¨¨¨¨¨¨¨¨¨¨¨¨¨¨¨¨¨¨¨¨¨¨¨¨¨¨¨¨¨¨

Le dernier changement de type qui complète notre parcours fait passer d'un ensemble d'ensembles d'individus à un ensemble d'individus.  Pour ce faire nous pourrions utiliser une fonction de choix de type \type{\ett,\et}  qui extrait arbitrairement un de ces ensembles, mais pour le coup ce n'est pas très utile et il est plus intéressant de chercher à produire un ensemble plus remarquable par rapport à l'ensemble d'ensembles de départ.  C'est ce que fait le \alienx{type-shifteur} que Partee appelle \prdk{be} et qui se trouve être équivalent à la traduction transitive du verbe \sicut{être} que nous avons vue \ref{sss:Vtrans} (p.~\pageref{x:spiderman2}) :

\ex.
\(\prdk{be} = \Xlo\lambda Y\lambda x [Y(\lambda y [y=x])]\)
\hfill type \type{\ett,\et}\label{xd:BE}


Pour bien comprendre ce que fait \prdk{be} en tant que \alienx{type-shifteur}, prenons le temps de le décomposer un peu informellement.  Nous l'avons déjà vu, $\Xlo\lambda y[y=x]$ dénote le singleton qui contient la dénotation de \vrb x, ce que, pour simplifier, nous noterons ici \set{\vrb x}.  \vrb Y étant une variable de quantificateur généralisé (donc d'ensemble d'ensembles), la condition $\Xlo [Y(\lambda y[y=x])]$ vérifie que le singleton \set{\vrb x} appartient à \vrb Y.  Enfin $\Xlo\lambda x[Y(\lambda y[y=x])]$ construit l'ensemble de tous les \vrb x qui sont l'élément d'un singleton contenu dans \vrb Y. 
Dit autrement, \prdk{be} prend en argument un ensemble d'ensembles \vrb Y, cherche dans \vrb Y tous ses singletons, extrait l'élément de chacun de ces singletons et range tous ces éléments dans un ensemble, qui est le résultat de la fonction.

De la sorte, \prdk{be} dénote une fonction totale, elle donnera toujours un résultat, mais pour certains quantificateurs ce résultat sera trivialisé à l'ensemble vide. 
Pour illustrer cela, regardons d'abord l'application de \prdk{be} sur le quantificateur indéfini \sicut{un enfant} (\ie\ $\Xlo\prd{Un}(\prd{enfant})$ ou $\Xlo\lambda P\exists u[\prd{enfant}(u)\wedge[P(u)]]$).  La dénotation de ce quantificateur contient tous les ensembles qui contiennent au moins un enfant ; parmi ceux-ci il y a donc tous les singletons composés d'un enfant (pour chaque enfant).  L'application de \prdk{be} peut ainsi rassembler tous ces enfants et l'ensemble résultant est donc la même chose que la dénotation du prédicat \prd{enfant} (l'ensemble de tous les enfants).


\ex.
\({\Xlo[\prdk{be}(\lambda P\exists u[\prd{enfant}(u)\wedge[P(u)]])]}\)\\
\(=
{\Xlo[\lambda Y\lambda x [Y(\lambda y [y=x])](\lambda P\exists u[\prd{enfant}(u)\wedge[P(u)]])]}
\)
\\
$=$ 
\(\Xlo\lambda x \exists u [\prd{enfant}(u) \wedge u=x]\)
$=$
\(\Xlo\lambda x\,\prd{enfant}(x)\)


Examinons maintenant l'application de \prdk{be} sur le quantificateur universel \sicut{tous les enfants} (\ie\ $\Xlo\prd{Tous}(\prd{enfant})$ ou $\Xlo\lambda P\forall u[\prd{enfant}(u)\implq[P(u)]]$).  Ce quantificateur dénote l'ensemble de tous les ensembles qui contiennent \emph{tous} les enfants.  Le plus petit de ces ensembles est donc celui contient tous les enfants du monde d'évaluation et rien d'autres (la dénotation de \prd{enfant}), et s'il y a plusieurs enfants dans ce monde, cet ensemble n'est pas un singleton. Autrement dit, \prdk{be} ne trouvera aucun singleton dans le quantificateur et retournera donc l'ensemble vide. 

\ex.
\({\Xlo[\prdk{be}(\lambda P\forall u[\prd{enfant}(u)\implq[P(u)]])]}\)\\
\(=
{\Xlo[\lambda Y\lambda x [Y(\lambda y [y=x])](\lambda P\forall u[\prd{enfant}(u)\implq[P(u)]])]}
\)
\\
$=$ 
\(\Xlo\lambda x \forall u [\prd{enfant}(u) \implq u=x]\)
\quad (\ie\ tous les enfants sont le même individu \vrb x)


Ces résultats semblent se confirmer par l'analyse que Partee propose pour les quantificateurs en position de complément de la copule. 
Rappelons que selon cette approche, \sicut{être} se traduit toujours par $\Xlo\lambda P\lambda x[P(x)]$ de type \type{\et,\et}. Les quantificateurs doivent alors être ramenés au type \et\ par \alien{type-shifting} avec \prdk{be}.  Or en français, les quantificateurs pour lesquels \prdk{be} renvoie l'ensemble vide s'avèrent sémantiquement (et probablement syntaxiquement) inappropriés dans cette position :

\ex.
\a. Alice est une enfant.
\b. \juge{\zarb} Alice est tous les enfants.
\b. \juge{\zarb} Les filles sont tous les élèves de la classe.



Notons aussi que, si le prédicat \prd{rdf} traduit \sicut{roi de France}, alors \prdk{be} appliqué au quantificateur généralisé $\Xlo\prdk{the}(\prd{rdf})$ renverra un singleton (s'il y a un roi de France) ou l'ensemble vide (s'il n'y en a pas --~ou éventuellement s'il y en a plusieurs) et le défini \sicut{le roi de France} traduit par $\Xlo\prdk{be}(\prdk{the}(\prd{rdf}))$ pourra ainsi être employé avec la copule  sans effet présuppositionnel.  Cela permet de dire (au moins dans certains cas)  que \ref{x:jesuisleroi} est fausse. 


\medskip

La figure \ref{f:trianglePartee} rassemble graphiquement les opérations de \emph{type-shifting} du domaine nominal dans le fameux «triangle de Partee»\ia{Partee, Barbara H.} (les flèches en pointillés représentent les fonctions partielles).  Nous verrons d'autres conversions encore entre \typ e et \et\ au chapitre \ref{GN++} (vol.~2).


\newgray{Ogris}{.94}
\newpsstyle{Otype}{linecolor=lightgray,linewidth=1pt,fillstyle=solid,fillcolor=Ogris}
\newpsstyle{shift}{linecolor=darkgray,linewidth=1pt,arrows=->}
\newpsstyle{shiftp}{style=shift,linestyle=dashed,dash=2.7pt 1.8pt}

\begin{figure}[h!]
\begin{center}
\psset{unit=.82cm}
\begin{pspicture}(8,6)
\psset{nodesep=3pt}
\rput(4,5){\circlenode[style=Otype]{ett}{\ett}}
\rput(7,1){\circlenode[style=Otype]{et}{\xbox{\ett}{\et}}}
\rput(1,1){\circlenode[style=Otype]{e}{\xbox{\ett}{\typ e}}}
\ncline[style=shiftp,offset=5pt]{->}{et}{e}\ncput*{\prdk{iota}}
\ncline[style=shift,offset=5pt]{->}{e}{et}\ncput*{\prdk{ident}}
\ncline[style=shift]{->}{et}{ett}\ncput*[nrot=:D]{\prdk{the}}
\ncline[style=shift,offset=10pt]{->}{et}{ett}\ncput*[nrot=:D]{\prdk{a}}
\ncline[style=shift,offset=10pt]{->}{ett}{et}\ncput*[nrot=:U]{\prdk{be}}
\ncline[style=shift,offset=5pt]{e}{ett}\ncput*[nrot=:U]{\prdk{lift}}
\ncline[style=shiftp,offset=5pt]{ett}{e}\ncput*[nrot=:D]{\prdk{lower}}
\end{pspicture}
\end{center}
\caption{Le triangle de Partee}\label{f:trianglePartee}
\end{figure}

\psset{unit=1cm}



Le \alien{type-shifting} formalise la flexibilité des types, c'est-à-dire cette
capacité adaptative des expressions sémantiques dans leur composition avec leur environnement linguistique.  
La construction du sens des phrases est crucialement guidée par la syntaxe, mais par ailleurs, les propriétés sémantiques compositionnelles des expressions ne se reflètent pas entièrement dans leurs propriétés syntaxiques distributionnelles telles qu'elles sont incarnées dans les catégories syntagmatiques : 
les DP ne changent pas de catégories mais ils peuvent changer de types justement parce que la syntaxe impose des configurations qui engendrent des conflits combinatoires dans la sémantique\footnote{Outre DP et NP, on peut voir se multiplier les projections syntaxiques pour analyser les groupes nominaux, comme QP, NumP, DemP, etc. Cela peut éventuellement raffiner la correspondance entre catégories syntaxiques et types sémantiques, mais il ne s'agit pas de flexibilité au sens où nous l'entendons ici.  }.
La discussion précédente (\S\ref{ss:VTinten}) sur les V transitifs intensionnels semble montrer que le \emph{type-shifting} est indispensable dans le processus d'analyse.  Et cela jette un nouvel éclairage sur le problème du typage des V transitifs et de l'obligation d'appliquer \QRa\ même pour les portées de surface (cf. \S\ref{sss:typesVT}).  Nous avons vu ci-dessus comment un verbe lexicalement de type \eet\ peut passer au type \type{\ett,\et} si son complément est un quantificateur généralisé.  Le \alien{type-shifting} a même pu être exploité, par \citet{Hendriks:93},\Andex{Hendriks, H.} pour régler les ambiguïtés et inversions de portée sans mettre en \oe uvre du mouvement de LF\footnote{En gros, l'idée développée par Hendriks, est que, par exemple, appliqué à un V de type \eet, le \alienx{type-shifteur} \(\Xlo\lambda V\lambda Y\lambda X[Y(\lambda y[X(\lambda x[[V(y)](x)])])]\) produit un prédicat de type \type{\ett,\type{\ett,t}} qui se combine avec deux quantificateurs généralisés \emph{en inversant} leurs portées.
}.
Dans les chapitres suivants, nous ferons usage de l'outil du \alien{type-shifting} quand cela sera nécessaire, mais, autant que possible, avec parcimonie.  Car il y a encore d'intéressants travaux à mener pour savoir ce qui peut légitimement compter comme un \alienx{type-shifteur} en sémantique, et nous  garderons en mémoire qu'il s'agit d'une opération de dernier recours, contrôlée formellement dans le système de l'interface syntaxe-sémantique, sans tour de magie commode qui permettrait de débloquer n'importe quel échec de composition.

%*** la flexibilité des types n'est pas un amollissement du système
%c'est contrôlé par le type-shifting qui s'intègre formellement dans l'interface synt-sem 


\newpage%\medskip

% -*- coding: utf-8 -*-
\begin{exo}\label{exo:LOWER}
Démontrez que $\Xlo\prdk{lower}(\lambda P[P(\cns a)])$ équivaut à \cns a.\pagesolution{crg:LOWER}
\\
%\sloppy
Calculez 
\(\Xlo\prdk{lower}(\lambda P\forall x[\prd{enfant}(x)\implq[P(x)]])\)
et
\(\Xlo\prdk{lower}(\lambda P\exists x[\prd{enfant}(x)\wedge[P(x)]])\),
et décrivez la dénotation 
des résultats obtenus.   
%\fussy
\begin{solu}(p.~\pageref{exo:LOWER})\label{crg:LOWER}

\sloppy
Par définition (\ref{xd:lower}, p.~\pageref{xd:lower}), \prdk{lower} équivaut à $\Xlo \lambda X\atoi y \forall P[[X(P)] \implq [P(y)]]$ ; par conséquent $\Xlo\prdk{lower}(\lambda P[P(\cns a)])$ équivaut à 
$\Xlo [\lambda X\atoi y \forall P[[X(P)] \implq [P(y)]](\lambda P'[P'(\cns a))]$ qui, par \breduc s successives, se simplifie en 
$\Xlo\atoi y \forall P[[\lambda P'[P'(\cns a)(P)] \implq [P(y)]]$ 
puis en 
$\Xlo\atoi y \forall P[[P(\cns a)] \implq [P(y)]]$.
Ce \atoi-terme dénote dans \w\ l'individu qui possède \emph{toutes} les propriétés (extensionnelles) que possède Alice dans \w.  Cet individu existe, est unique et est forcément Alice, ne serait-ce parce que parmi les propriétés d'Alice, il y a celle qui s'exprime par $\Xlo\lambda x[x=\cns a]$ et que seule Alice satisfait.  Donc $\Xlo\prdk{lower}(\lambda P[P(\cns a)])$ est bien équivalent à \cns a.


De la même façon, \(\Xlo\prdk{lower}(\lambda P\forall x[\prd{enfant}(x)\implq[P(x)]])\)
équivaut à 
\(\Xlo[\lambda X\atoi y \forall P[[X(P)] \implq [P(y)]](\lambda P'\forall x[\prd{enfant}(x)\implq[P'(x)]])]\) 
qui se réduit en 
\(\Xlo\atoi y \forall P[\forall x[\prd{enfant}(x)\implq[P(x)]] \implq [P(y)]]\).
Ce \atoi-terme dénote l'unique individu qui possède toutes les propriétés communes à  tous les enfants ; mais s'il y a plusieurs enfants dans le monde d'évaluation, il ne peut pas y avoir un seul individu qui satisfait cette condition (chaque enfant a toutes les propriétés de tous les enfants). Donc la dénotation de \(\Xlo\prdk{lower}(\lambda P\forall x[\prd{enfant}(x)\implq[P(x)]])\)
n'est pas définie (tant qu'il y a plusieurs individus dans la dénotation de \sicut{enfant}).


De même, \(\Xlo\prdk{lower}(\lambda P\exists x[\prd{enfant}(x)\wedge[P(x)]])\) se réduit en 
\(\Xlo\atoi y \forall P[\exists x[\prd{enfant}(x)\wedge[P(x)]] \implq [P(y)]]\).
Cette fois ce \atoi-terme dénote l'unique individu qui possède toute propriété satisfaite par au moins un enfant, mais un tel individu n'existe pas (s'il y a plusieurs enfants) car parmi ces propriétés il y en a beaucoup qui sont contradictoires entre elles (par exemple, il y a des enfants bruns, des enfants blonds, des filles, des garçons, etc.). Ainsi, comme précédemment, la dénotation de \(\Xlo\prdk{lower}(\lambda P\exists x[\prd{enfant}(x)\wedge[P(x)]])\) n'est pas définie.

\fussy
\end{solu}
\end{exo}


% -*- coding: utf-8 -*-
\begin{exo}\label{exo:BE-Deux}
Calculez $\Xlo\prdk{be}(\prd{Deux}(\prdk{enfant}))$ et 
\pagesolution{crg:BE-Deux}
décrivez la dénotation du résultat obtenu.
\begin{solu}(p.~\pageref{exo:BE-Deux})\label{crg:BE-Deux}

\sloppy

Par définition (cf.\ \ref{xd:BE}, p.~\pageref{xd:BE}), \prdk{be} est équivalent à \(\Xlo\lambda Y\lambda x [Y(\lambda y [y=x])]\). Sachant que $\Xlo\prdk{be}(\prd{Deux}(\prd{enfant}))$ est de type \ett\ (c'est un quantificateur généralisé), 
\(\Xlo\prdk{be}(\prd{Deux}(\prd{enfant}))\) peut donc se réécrire en :
\(\Xlo[\lambda Y\lambda x [Y(\lambda y [y=x])](\prd{Deux}(\prd{enfant}))]\). 
Par \breduc, cela se simplifie en 
\(\Xlo\lambda x [\prd{Deux}(\prd{enfant})(\lambda y [y=x])]\).
\(\Xlo\prd{Deux}(\prd{enfant})(\lambda y [y=x])\) est une structure tripartite de type \typ t, et sa portée \(\Xlo\lambda y [y=x]\) dénote le singleton \(\set{\denote{\vrb x}^{\Modele,w,g}}\).  Par définition, \(\Xlo\prd{Deux}(\prd{enfant})(\lambda y [y=x])\) est donc vraie ssi l'intersection de l'ensemble de tous les enfants et de l'ensemble \(\set{\denote{\vrb x}^{\Modele,w,g}}\) contient au moins deux éléments ; mais cela est, bien sûr, mathématiquement impossible puisque \(\set{\denote{\vrb x}^{\Modele,w,g}}\) ne contient qu'un seul élément.
Par conséquent, \(\Xlo\prd{Deux}(\prd{enfant})(\lambda y [y=x])\) sera toujours faux et \(\Xlo\lambda x [\prd{Deux}(\prd{enfant})(\lambda y [y=x])]\) dénotera toujours l'ensemble vide.

\fussy

\end{solu}
\end{exo}


% -*- coding: utf-8 -*-
\begin{exo}\label{exo:6FC}
Supposons que $\Xlo f(\prd{enfant})$, 
\pagesolution{crg:6FC}
où \vrb f est une variable de fonction de choix, soit la traduction de base que la grammaire attribue à l'indéfini \sicut{un enfant}.  Quel serait alors le \lterme\ de {\LO}, utilisant donc $\Xlo f(\prd{enfant})$, qui nous donnerait la même dénotation que le quantificateur généralisé $\Xlo \prd{Un}(\prd{enfant})$ ?  En quoi ce passage de la première traduction à la seconde n'est pas, techniquement, un \alien{type-shifting} ?
\begin{solu}(p.~\pageref{exo:6FC})\label{crg:6FC}

\sloppy
$\Xlo f(\prd{enfant})$, de type \typ e, dénote un enfant particulier choisi par \vrb f. Appelons, provisoirement, cet enfant Fanny. Le quantificateur $\Xlo \prd{Un}(\prd{enfant})$ dénote l'ensemble de tous les ensembles qui contiennent au moins un enfant. 
Nous pouvons déjà remarquer que $\Xlo\lambda P[P(f(\prd{enfant}))]$, même si c'est bien un quantificateur généralisé, n'est pas équivalent à $\Xlo \prd{Un}(\prd{enfant})$, puisque le premier dénote l'ensemble de tous les ensembles qui contiennent \Obj{Fanny}, alors que nous cherchons l'ensemble de tous les ensembles qui contiennent n'importe quel enfant.  Pour prendre en compte n'importe quel enfant, nous devons donc faire en sorte que \vrb f ne soit plus libre (afin que sa dénotation ne dépende plus fixement de l'assignation $g$ globale), et nous obtenons cela en quantifiant sur \vrb f.  Mais il ne faut pas se tromper : $\Xlo\forall f[P(f(\prd{enfant}))]$ est une formule qui est vraie ssi \vrb P est une propriété qui est commune à tous les enfants (car $\Xlo\forall f$ va nous faire parcourir toutes les façons de choisir un enfant dans la dénotation de \prd{enfant}).  C'est bien une quantification existentielle qu'il faut utiliser ici : $\Xlo\exists f[P(f(\prd{enfant}))]$ est vraie ssi \vrb P est une propriété satisfaite par au moins un enfant\footnote{NB : $\Xlo[P(\exists f\,f(\prd{enfant}))]$ et $\Xlo[P(\forall f\,f(\prd{enfant}))]$ sont des expressions mal formées de \LO, car $\Xlo f(\prd{enfant})$ n'est pas de type \typ t.}
 ; et donc ce qui équivaut à $\Xlo \prd{Un}(\prd{enfant})$ est $\Xlo\lambda P\exists f[P(f(\prd{enfant}))]$. 

\fussy

L'opérateur qui nous ferait passer du terme $\Xlo f(\prd{enfant})$ au quantificateur généralisé devrait donc être le \lterme\ $\Xlo\lambda x\lambda P\exists f[P(x)]$.  Or cela semble présupposer que nous connaîtrions à l'avance le nom de la variable \vrb f de fonction de choix utilisée «dans \vrb x~», ce qui n'a pas de raison d'être.  Mais, en fait, la situation est encore plus grave car, formellement, dans ce \lterme, $\Xlo\exists f$ ne lie rien du tout, sémantiquement il ne sert à rien et en soi $\Xlo\lambda x\lambda P\exists f[P(x)]$ équivaut  à $\Xlo\lambda x\lambda P[P(x)]$\footnote{Sans compter que dans $\Xlo[\lambda x\lambda P\exists f[P(x)](f(\prd{enfant}))]$, nous n'aurions pas le droit d'effectuer la \breduc\ (alors que c'est précisément ce qu'il nous faudrait) car le \vrb f libre de l'argument ne doit pas se retrouver lié après réduction (déf.~\ref{d:breduc2} p.~\pageref{d:breduc2}).}. 
Par conséquent, si nous admettons que les \alien{type-shifteurs} sont des opérateurs \emph{compositionnels} formalisables par des fonctions, alors le passage de $\Xlo f(\prd{enfant})$ au quantificateur généralisé n'est pas du \alien{type-shifting}.
\end{solu}
\end{exo}




\is{type-shifting@\textit{type-shifting}|)}


\section{Conclusion}
%====================
\label{conclu:ISS}

\sloppy

Dans ce chapitre, nous avons vu de nombreuses applications du \lcalcul\ pour l'analyse sémantique compositionnelle, c'est-à-dire au niveau de l'interface syntaxe-sémantique.   Cela nous a donné l'occasion de constater que les compositions sémantiques ne se font pas uniquement par application fonctionnelle entre des prédicats et des arguments ; pour mener à bien la construction du sens de certaines phrases, des dispositifs supplémentaires sont nécessaires, comme les mouvements (ou leurs équivalents qui diffèrent l'interprétation de certains constituants), l'application fonctionnelle intensionnelle, le polymorphisme, le \alien{type-shifting}.  
Ces dispositifs sont, en partie, tributaires de la syntaxe, ce qui est normal puisque l'interprétation des phrases est conditionnée par leurs structures formelles.
Et nous avons montré comment ils peuvent se formaliser rigoureusement dans le cadre du  \lcalcul.

\fussy

Il existe d'autres mécanismes formels qui peuvent être mis en \oe uvre dans la composition sémantique%
\footnote{En rapport avec des problématiques de sémantique lexicale, citons les mécanismes de \mbox{\emph{co-composition}} et de \emph{coercion} (ou \emph{coercition}) \emph{de types} de \citet{Pust:95}\Andexn{Pustejovsky, J.} ; voir aussi \citet{Asher:11} et \citet{deSwart:11x}\Andexn{de Swart, H.} à ce sujet.}
et parmi ceux-ci il y en a un que nous avons entr'aperçu sans vraiment l'aborder frontalement dans nos formalisations.
Il s'agit de ce que, pour faire simple, j'appellerai ici le phénomène de \emph{coréférence}\is{coreference@coréférence} et qui  se manifeste lorsque, dans l'analyse de certaines constructions, nous constatons que plusieurs éléments de traductions doivent non seulement être identiques mais aussi avoir la même valeur sémantique.  C'est typiquement ce qui se produit lorsque nous interprétons une expression pronominale en relation avec son antécédent.  
Il faut bien savoir que le rôle de la sémantique compositionnelle n'est pas de trouver (\ie\ résoudre) «le bon antécédent» d'un pronom\footnote{Sur cette question, la sémantique et la syntaxe ont, en fait, la tâche d'indiquer les expressions qui, dans une configuration donnée, ne peuvent en aucune manière être l'antécédent d'un pronom donné. En repérant les antécédents impossibles, elles caractérisent ainsi l'ensemble des antécédents \emph{possibles}. },\is{pronom} pas plus qu'elle n'a à lever (toutes) les ambiguïtés.  En revanche, ce qui concerne la sémantique c'est, \emph{sous l'hypothèse} qu'un constituant $X$ est l'antécédent d'un constituant (pronominal) $Y$, 
de déterminer la traduction correcte de $Y$.  
C'est un problème qui se situe au c\oe ur de l'interface syntaxe-sémantique, ne serait-ce que parce que nous représentons ce genre d'hypothèses au moyen des indices numériques\is{indice!\elid\ referentiel@\elid\ référentiel} dans les structures syntaxiques, par des \emph{coindiciations}, comme dans \sicut{Alice$_1$ pense qu'elle$_1$ a raison}.
Dans ce chapitre, nous avons essentiellement utilisé les coindiciations pour interpréter les mouvements et les traces, mais nous n'avons pas complètement exploité toutes les applications de ce dispositif. 
Comme il joue un rôle important pour le traitement de la coréférence et donc pour l'analyse des pronoms, nous l'aborderons en détail dans le chapitre \ref{Ch:contexte} (vol.~2).
Et il se trouve que le mécanisme sémantique qui interprète les coindiciations occupe également une place centrale dans l'interface syntaxe-sémantique basée sur \LOz.  Je souhaiterais donc terminer ici en faisant quelques remarques sur l'analyse compositionnelle dans \LOz\ (que nous avons laissé de côté dans ce chapitre en travaillant seulement dans \LO) et qui nous donneront l'occasion d'annoncer pourquoi un nouveau type de coréférence (et donc de coindiciation) doit être utilisé.


Les procédures d'analyses qui ont été présentées dans ce chapitre peuvent être reprises telles quelles pour mener les analyses sémantiques dans {\LOz}.  Toutes sauf, en fait, l'application fonctionnelle intensionnelle, qui doit être adaptée aux spécificités formelles de \LOz.  Ce point  en rejoint un autre, plus général, qui concerne la gestion des arguments de mondes possibles lors de l'analyse.
Si nous voulons obtenir compositionnellement la traduction \ref{x:LO2.1'b} de \ref{x:LO2.1'a}, deux questions se posent : comment les variables de mondes sont introduites dans la traduction et   lesquelles choisir ? 

\ex.  \label{x:LO2.1'}
\a. Tous les enfants dorment. \label{x:LO2.1'a}
\b. \(\Xloz\forall x [\prdz{enfant}'_w(x)\implq\prdz{dormir}'_w(x)]\) \label{x:LO2.1'b}



Une  stratégie simple peut consister à répliquer directement ce que nous avons vu dans ce chapitre en l'adaptant à {\LOz}.   Comme les compositions sémantiques dans {\LO} ne se soucient pas vraiment de l'indice de monde avec lequel interpréter la phrase, cela voudra dire que dans \LOz\ nous utiliserons toujours la même variable  \vrbz w, qui de fait est une variable quelconque et habituellement libre.  Et de temps en temps cette variable se retrouvera liée lors de la composition, précisément par l'action de l'AFI.   Dans {\LO}, AFI ajoute $\Xlo\Intn$ sur l'argument d'un {\lterme} ; dans \LOz, elle va consister à ajouter la \labstraction\ d'une variable de monde sur cet argument, et comme nous travaillons toujours avec la même variable \vrbz w, nous savons que cette \labstraction\ sera $\Xloz\lambda w$.  Par la suite, nous pourrons accessoirement renommer les occurrences ainsi liées de cette variable (cf.\ théorème \ref{th:renomvar} p. \pageref{th:renomvar}), juste pour faciliter la lisibilité des formules construites.   
Cette stratégie peut suggérer une façon, également simple, de traiter la première question, en introduisant les arguments \vrbz w en même temps que les prédicats auxquels ils s'appliquent dès le niveau des projections lexicales dans l'arbre syntaxique.  En pratique, cela voudra dire que les unités \sicut{enfants} et \sicut{dorment} se traduisent toujours respectivement par $\Xloz\lambda x\,\prdz{enfant}'_w(x)$ et $\Xloz\lambda x\,\prdz{dormir}'_w(x)$ : comme nous utilisons toujours le même \vrbz w, nous n'avons pas vraiment à nous poser la question de son choix.   Et nous voyons que ces unités lexicales sont déjà de type \et\ (et pas de type \type{s,\et} comme les prédicats \prdzz{enfant} et \prdzz{dormir}) puisqu'elles sont déjà saturées par leur argument de monde.  La dérivation de  \ref{x:LO2.1'} peut alors procéder exactement de la même façon que ce que nous avons vu pour {\LO} dans ce chapitre.

Pour illustrer plus précisément le processus d'analyse spécifique à \LOz, avec intervention de l'AFI, nous allons reprendre la dérivation de l'exemple \sicut{Charles croit qu'Alice dort}.  D'après ce qui est dit ci-dessus, la règle d'interface pour l'AFI pourra se formaliser comme suit : 

\ex.
\RISS{AFI dans \LOz}%
{\begin{tabular}[t]{@{}rccc} 
X &\reecr& Y& Z\\
\small\mtyp a && \small\mtype{\mtype{\typ s,b},a} & \small\mtyp b\\
$\Xloz[\alpha(\lambda w\beta)]$&\seecr &$\Xloz\alpha$ & $\Xloz\beta$\\
\end{tabular}}


La dérivation de \Next[a] est donnée dans l'arbre \Next[b], où les prédicats sont dès le départ saturés par \vrbz w (pour simplifier, les noms propres sont traduits directement par des constantes de type \typ e comme suggéré en \S\ref{ss:LO-LO2}).  Comme dans \LO, \sicut{que} se traduit par $\Xloz\lambda p\,p$ de type \type{\type{s,t},\type{s,t}} pour déclencher l'AFI\footnote{\sicut{Que} ne peut pas se traduire par $\Xloz\lambda\phi\lambda w\phi$ de type \type{t,\type{s,t}}, pour les mêmes raisons que celles données en \S \ref{ss:AFInt}.}.
Au niveau du VP supérieur, nous avons renommé la variable liée par $\Xloz\lambda w$ pour plus de clarté (mais c'était facultatif). 


\ex.  
\a. Charles croit qu'Alice dort.
\b. 
{\small
\Tree
[.TP\zbox{\ $\Xloz\prdz{croire}'_w(\cnsz c,\Xloz\lambda w_1\,\prdz{dormir}'_{w_1}(\cnsz a))$}
  [.DP \zcbox{Charles}\\$\cnsz c$ ]
  [.VP\zbox{\ $\Xloz\lambda x\,\prdz{croire}'_w(x,\Xloz\lambda w\,\prdz{dormir}'_w(\cnsz a))$ = $\Xloz\lambda x\,\prdz{croire}'_w(x,\Xloz\lambda w_1\,\prdz{dormir}'_{w_1}(\cnsz a))$}
    [.V croit\\\xbox[r]{croit}{$\Xloz\lambda p\lambda x\,\prdz{croire}'_w(x,p)$} ]
    [.CP\zbox{\ $\Xloz[\lambda p\,p(\lambda w\,\prdz{dormir}'_w(\cnsz a))]$ = $\Xloz\lambda w\,\prdz{dormir}'_w(\cnsz a)$} 
      [.C \zcbox{que}\\\zcbox{$\Xloz\lambda p\,p$} ]
      [.TP\zbox{\ $\Xloz\prdz{dormir}'_w(\cnsz a)$} 
        [.DP Alice\\$\cnsz a$ ]
        [.VP dort\\\xbox[l]{dort}{$\Xloz\lambda x\,\prdz{dormir}'_w(x)$} ]
      ]
    ]
  ]
]
}


Comme dit précédemment, cette stratégie est simple, mais elle l'est finalement trop.  D'abord nous pouvons facilement nous rendre compte qu'utiliser une seule et même variable \vrbz w nous empêchera d'obtenir directement la traduction de la lecture \dere\ de \Next\ que nous avions proposée au chapitre~\ref{ch:types} \S\ref{s:Ty2}.\is{de dicto@\alien{de dicto}}\is{de re@\alien{de re}}

\ex.
\a. \OE dipe voulait épouser sa mère.
\b. \(\Xloz \prdz{vouloir}'_w(\cnsz{\oe},\lambda w_1\,\prdz{épouser}'_{w_1}(\cnsz{\oe},\atoi x\,\prdz{mère}'_{w_1}(x,\cnsz{\oe})))\) \taquet{2cm}{(de dicto)}
\b. \(\Xloz \prdz{vouloir}'_w(\cnsz{\oe},\lambda w_1\,\prdz{épouser}'_{w_1}(\cnsz{\oe},\atoi x\,\prdz{mère}'_{w}(x,\cnsz{\oe})))\)\label{x:dere153c} \taquet{2cm}{(de re)}

\sloppy

Dans \Last[b], la variable \vrbzi{w}1 ne provient que d'un renommage {a posteriori} de variables liées (la formule équivaut à \(\Xloz \prdz{vouloir}'_w(\cnsz{\oe},\lambda w\,\prdz{épouser}'_{w}(\cnsz{\oe},\atoi x\,\prdz{mère}'_{w}(x,\cnsz{\oe})))\)).  Mais ce n'est pas le cas de \Last[c] où l'occurrence de \vrbz w sur \prdzz{mère} est dans la portée de $\Xloz\lambda w_1$ mais celui-ci ne doit surtout pas la lier.  Autrement dit, au cours de l'analyse, avant d'opérer l'AFI, nous  devrons obtenir \(\Xloz\prdz{épouser}'_{w_1}(\cnsz{\oe},\atoi x\,\prdz{mère}'_{w}(x,\cnsz{\oe}))\),  c'est-à-dire avec deux arguments de mondes formellement distincts. 

Nous pourrions nous débarrasser hâtivement de cette complication en estimant que la lecture \dere\ s'obtient en fait par montée du DP concerné (par \alien{quantifying-in} ou \QRa) : ainsi le quantificateur généralisé $\Xloz\lambda P[P(\atoi x\,\prdz{mère}'_{w}(x,\cnsz{\oe}))]$ serait interprété en dehors de la portée de la \labstraction\ introduite par AFI, ce qui donnerait  
\(\Xloz[\lambda P[P(\atoi x\,\prdz{mère}'_{w}(x,\cnsz{\oe}))](\lambda x_1\,\prdz{vouloir}'_w(\cnsz{\oe},\lambda w\,\prdz{épouser}'_{w}(\cnsz{\oe},x_1)))]\)
n'utilisant qu'une seule variable \vrbz w mais équivalant à \Last[c]. 
Cependant il y a tout lieu de penser que cette option est fautive.  
D'abord parce qu'il n'y a pas vraiment de raison que les DP référentiels soient sujet à la montée des quantificateurs.  En outre, analyser ces lectures \dere\ par montée des DP  impliquerait des instances tout à fait exceptionnelles de \QRa\ envoyant les quantificateurs à l'extérieur de leur proposition syntaxique d'origine.  Nous avons vu au chapitre \ref{ch:gn} que cette possibilité était (éventuellement) une singularité des indéfinis mais pas des DP quantificationnels.  
C'est en fait une contrainte fondamentale de l'interface syntaxe-sémantique : il n'est pas permis d'opérer \QRa\ au delà d'un CP correspondant à une proposition finie\footnote{C'est-à-dire dont le verbe est conjugué. Notons aussi que selon cette contrainte, les portées exceptionnelles des indéfinis doivent alors s'expliquer par d'autres mécanismes ; c'est ce qui a motivé l'usage des fonctions de choix, entre autres.}.
Le problème ne se pose pas vraiment pour \Last\ (où la subordonnée est infinitive), mais il est déterminant pour l'exemple qui faisait l'objet de l'exercice \ref{exo:homfem} du chapitre \ref{Ch:t+m} repris ici en \Next.

\fussy


\ex.
Paul a cru [\Stag{CP} que tous les hommes étaient des femmes].

La montée de \sicut{tous les hommes} hors du CP n'est pas autorisée ici.  
Si c'était le cas, alors cela se produirait aussi dans les exemples \Next\ et nous pourrions avoir parfois une portée maximale de \sicut{tous les hommes} faisant ainsi covarier \sicut{un invité}.  Mais une telle interprétation n'est jamais disponible.

\ex.
\a. Un invité a cru [\Stag{CP} que tous les hommes étaient des femmes].
\b. Paul a raconté à un invité [\Stag{CP} que tous les hommes étaient des femmes].


Par conséquent la lecture \dere\ des DP référentiels et quantificationnels ne peut pas s'expliquer par \QRa, et nous devons forcément produire des formules comme \ref{x:dere153c} et \Next\ (sans déplacer les DP).  

\ex.
\(\Xloz\prdz{croire}'_w(\cnsz p,\lambda w_1\forall x[\prdz{homme}'_w(x)\implq\prdz{femme}'_{w_1}(x)])\)

Autrement dit nous devons pouvoir manipuler différentes variables de type \typ s dans l'interface %syntaxe-sémantique, 
et donc, pour en revenir à notre point de départ, la stratégie qui vise à utiliser partout la même variable trouve ici ses limites.
De surcroît, cela implique que \LOz\ a véritablement un avantage expressif sur \LO\ (puisque \LO\ équivaut à l'utilisation de \LOz\ avec une seule variable de type \typ s).

\sloppy
D'ailleurs notons aussi que cela aurait été un peu étrange de traduire les unités lexicales immédiatement par des prédicats saturés pour leur argument de monde.  \LOz\ nous offre la possibilité de manipuler des intensions et il serait assez naturel de penser que le lexique nous livre des traductions qui représentent le sens des mots, comme $\Xloz\lambda w\lambda x\,\prdz{enfant}'_w(x)$ et $\Xloz\lambda w\lambda x\,\prdz{dormir}'_w(x)$ (\ie\ \prdzz{enfant} et \prdzz{dormir}), plutôt que des dénotations (même relatives à un monde quelconque).
Mais si nous traduisons les unités lexicales de la sorte, alors il nous faut un mécanisme qui sature les arguments de monde \emph{pendant} la dérivation à l'interface syntaxe-sémantique. 
C'est une tâche qui est loin d'être triviale et qui a des implications non négligeables sur les analyses syntaxiques et sémantiques.

\fussy

En fin de compte, nous nous retrouvons dans une situation où nous devrons introduire \emph{compositionnellement} diverses variables de monde sachant que certaines devront être liées par des \labstraction s, comme \vrbzi w1 dans \Last, et que d'autres devront être identiques entre elles, c'est-à-dire en relation de coréférence, comme les deux occurrences de \vrbz w dans \Last.  Ce comportement fait penser à celui des pronoms, et c'est pourquoi il peut être pertinent d'envisager le traitement des arguments de monde au moyen d'un système d'indices numériques. Nous prendrons le temps d'examiner ce mécanisme en détail dans le chapitre~\ref{Ch:contexte} du volume 2.

