% -*- coding: utf-8 -*-
\begin{exo}\label{exo:CatGN}
En appliquant les tests que nous avons à notre disposition, 
\pagesolution{crg:CatGN}%
déterminez à quelles classes appartiennent les {\GN} suivants : \sicut{un tiers des candidats}, \sicut{trois quarts des candidats}, \sicut{beaucoup de candidats}.
\begin{solu}(p.~\pageref{exo:CatGN})\label{crg:CatGN}

\sicut{Un tiers des candidats} est un indéfini car, d'un point de vue strictement sémantique, il ne satisfait pas le test de consistance (p.~\pageref{test:contra}) : \sicut{un tiers des candidats sont barbus et un tiers des candidats sont imberbes} peut être jugé vrai dans une situation qui comporte, par exemple, 50\% de barbus et 50\% d'imberbes, si nous considérons qu'\sicut{un tiers} signifie \sicut{au moins un tiers} (cf. \S\ref{ss:implicatures}).

\sicut{Trois quarts des candidats} est quantificationnel, car il satisfait le test de consistance (\sicut{trois quarts des candidats sont barbus et trois quarts des candidats sont imberbes} ne peut pas être vrai), mais il ne satisfait pas le test de complétude (p.~\pageref{test:compl}): \sicut{trois quarts des candidats sont barbus ou trois quarts des candidats sont imberbes} est faux par exemple dans une situation avec 50\% de barbus et 50\% d'imberbes.

Pour \sicut{beaucoup de candidats}, la classification est moins simple, car en fait le déterminant est ambigu.  Commençons avec le test de consistance : \sicut{beaucoup de candidats sont barbus et beaucoup de candidats sont imberbes}.  Supposons que nous sommes dans une situation où il y a 200 candidats barbus et 200 candidats imberbes et que nous estimons que 200 est un nombre important de candidats (par exemple parce que nous en attendions seulement une soixantaine), dans ce cas la phrase pourra être jugée vraie.  Et cela suffit à prouver que le GN\ est un indéfini.  

Mais on peut comprendre le déterminant \sicut{beaucoup de} d'une
autre manière. Dans les phrases ci-dessus, \sicut{beaucoup de} est
interprété comme signifiant «une grande quantité de» et la
«grandeur» de cette quantité dépend du contexte. L'autre interprétation de
\sicut{beaucoup de} est celle qui signifie quelque chose comme «une grande proportion de» ; là encore la grandeur de cette proportion dépend habituellement du contexte.  Selon le seuil que fixe le contexte pour estimer qu'on est en présence d'une \emph{grande} proportion,  \GN\ apparaîtra alors comme quantificationnel (si le seuil est supérieur à 50\%) ou indéfini (si le seuil est inférieur à 50\%) (cf. \S\ref{ss:QGDet}).   Dans certains cas, il est possible cependant d'associer à \sicut{beaucoup de} un sens qui ne dépend pas du contexte en l'interprétant comme signifiant «la plus grande proportion de».  Dans ce cas le \GN\ sera quantificationnel.  En 
effet dans ce cas \sicut{beaucoup de candidats sont barbus et beaucoup de candidats sont imberbes} est contradictoire, car quel que soit le
nombre de candidats au total, ce sont soit les barbus, soit les
imberbes qui représentent la plus grande proportion, mais pas les deux
à la fois.  Quant à \sicut{beaucoup de candidats sont barbus et beaucoup de candidats sont imberbes}, elle est fausse dans le cas où il y a
autant de barbus que d'imberbes dans le groupe de candidats.
\end{solu}
\end{exo}
