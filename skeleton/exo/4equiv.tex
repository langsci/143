% -*- coding: utf-8 -*-
\begin{exo}\label{exo:4equiv}
Chaque formule de la liste suivante est équivalente à une autre formule
\pagesolution{crg:4equiv}
de la liste.  Indiquez, en le démontrant, quelles sont ces paires
d'équivalences.  
\addtolength{\multicolsep}{-10pt}
\begin{multicols}{4}
\begin{exolist}
\item $\Xlo\peut\phi$
\item $\Xlo\doit\phi$
\item $\Xlo\peut\neg\phi$
\item $\Xlo\doit\neg\phi$
\item $\Xlo\neg\peut\phi$
\item $\Xlo\neg\doit\phi$
\item $\Xlo\neg\peut\neg\phi$
\item $\Xlo\neg\doit\neg\phi$
\end{exolist}
\end{multicols}

\smallskip

\noindent Puis corroborez  ces résultats en donnant des traductions en français de chacune des formules.

\begin{solu} (p.~\pageref{exo:4equiv})\label{crg:4equiv}

Les équivalences s'infèrent à partir des règles d'interprétation des modalités  (\RSem\ref{RSemMod}) p.~\pageref{RSemMod} et de celle de la négation (\RSem\ref{RIneg}), \S\ref{s:reglsem} p.~\pageref{RIneg}.
Ce sont les suivantes :

\begin{exolist}
\item $\Xlo\peut\neg\phi$ et $\Xlo\neg\doit\phi$ : la première dit qu'il y a un monde où \vrb\phi\ est fausse, et la seconde qu'il est faux que \vrb\phi\ est vraie dans tous les mondes.

$\Xlo\peut\neg\phi$ {\rtrad} \sicut{il est possible que non {\vrb\phi}}
et
$\Xlo\neg\doit\phi$ {\rtrad} \sicut{il n'est pas nécessaire que {\vrb\phi}}.

\item $\Xlo\doit\neg\phi$ et $\Xlo\neg\peut\phi$ : la première dit que \vrb\phi\ est fausse dans tous les mondes, et la seconde qu'il n'y a pas de monde où \vrb\phi\ est vraie.

$\Xlo\doit\neg\phi$ {\rtrad} \sicut{il est nécessaire que non {\vrb\phi}}
et $\Xlo\neg\peut\phi$ {\rtrad} \sicut{il est impossible que {\vrb\phi}}.

\item $\Xlo\neg\peut\neg\phi$ et $\Xlo\doit\phi$ : la première dit qu'il n'y a pas de monde où \vrb\phi\ est fausse, et la seconde que \vrb\phi\ est vraie dans tous les mondes.

$\Xlo\neg\peut\neg\phi$ {\rtrad} \sicut{il est impossible que non {\vrb\phi}}
et $\Xlo\doit\phi$ {\rtrad} \sicut{il est nécessaire que {\vrb\phi}}.

\item $\Xlo\neg\doit\neg\phi$ et $\Xlo\peut\phi$ : la première dit que \vrb\phi\ n'est pas fausse dans tous les mondes et la seconde qu'il y a un monde où \vrb\phi\ est vraie.

$\Xlo\neg\doit\neg\phi$ {\rtrad} \sicut{il n'est pas nécessaire que non {\vrb\phi}}
et $\Xlo\peut$ {\rtrad} \sicut{il est possible que {\vrb\phi}}.

\end{exolist}


\end{solu}
\end{exo}
