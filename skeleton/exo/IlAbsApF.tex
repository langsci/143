% -*- coding: utf-8 -*-
\begin{exo}\label{exo:Ilabs@}
Détaillez pas à pas le calcul des valeurs suivantes en utilisant
\pagesolution{sol:Ilabs@}%
successivement les règles des définitions \ref{d:slabs} et
\ref{d:Sem@} (ainsi que le théorème \ref{th:Seml@}).  On considérera que
$\w_1$ est le monde décrit dans la \S\ref{ss:SFP} (cf. la
figure~\ref{f:regardf}, p.~\pageref{f:regardf}).
\begin{enumerate}
\item\(\denote{\Xlo\lambda x\lambda y[[\prd{regarder}(y)](x)]}^{\Modele,\w_1,g}\)

\item \(\denote{\Xlo\lambda x[\lambda y[\prd{regarder}(y)](x)]}^{\Modele,\w_1,g}\)
\end{enumerate}
\begin{solu}\label{sol:Ilabs@}
(p.~\pageref{exo:Ilabs@})

%%Pour calculer les valeurs sémantiques de  \lterme s, il y a deux
%%façons de procéder,  selon que l'on part du terme global pour le
%%décomposer ou que l'on ***

\begin{enumerate}
\item \(\denote{\Xlo\lambda x\lambda y[[\prd{regarder}(y)](x)]}^{\Modele,\w_1,g}\)
%%  \begin{enumerate}
%%  \item 
\end{enumerate}
\sloppy

D'après la définition~\ref{d:slabs}, p.~\pageref{d:slabs},
\(\denote{\Xlo\lambda x\lambda y[[\prd{regarder}(y)](x)]}^{\Modele,\w_1,g}\) 
est la fonction qui, à tout objet \Obj x, associe 
\(\denote{\Xlo\lambda y[[\prd{regarder}(y)](x)]}^{\Modele,\w_1,g_{[\Obj
    x/\vrb x]}}\).
De même, \(\denote{\Xlo\lambda y[[\prd{regarder}(y)](x)]}^{\Modele,\w_1,g_{[\Obj
    x/\vrb x]}}\) est la fonction qui, à tout objet \Obj y, associe  
\(\denote{\Xlo[[\prd{regarder}(y)](x)]}^{\Modele,\w_1,g_{[\Obj
    x/\vrb x]_{[\Obj y/\vrb y]}}}\).

Ensuite, d'après la définition~\ref{d:Sem@}, p.~\pageref{d:Sem@},
\(\denote{\Xlo[[\prd{regarder}(y)](x)]}^{\Modele,\w_1,g_{[\Obj
      x/\vrb x]_{[\Obj y/\vrb y]}}}
= %\linebreak
\denote{\Xlo[\prd{regarder}(y)]}^{\Modele,\w_1,g_{[\Obj
      x/\vrb x]_{[\Obj y/\vrb y]}}}(\denote{\Xlo x}^{\Modele,\w_1,g_{[\Obj
      x/\vrb x]_{[\Obj y/\vrb y]}}})
\).
Et comme l'assignation courante spécifie la valeur de \vrb x, nous savons que \(\denote{\Xlo x}^{\Modele,\w_1,g_{[\Obj
      x/\vrb x]_{[\Obj y/\vrb y]}}} = g_{[\Obj
      x/\vrb x]_{[\Obj y/\vrb y]}}(\vrb x) =\Obj x\).
Donc 
\(\denote{\Xlo[[\prd{regarder}(y)](x)]}^{\Modele,\w_1,g_{[\Obj
      x/\vrb x]_{[\Obj y/\vrb y]}}}
= \denote{\Xlo[\prd{regarder}(y)]}^{\Modele,\w_1,g_{[\Obj
      x/\vrb x]_{[\Obj y/\vrb y]}}}(\Obj x)\).

Toujours d'après la définition~\ref{d:Sem@}, nous avons
\(\denote{\Xlo[\prd{regarder}(y)]}^{\Modele,\w_1,g_{[\Obj
      x/\vrb x]_{[\Obj y/\vrb y]}}} = 
\denote{\Xlo\prd{regarder}}^{\Modele,\w_1,g_{[\Obj
      x/\vrb x]_{[\Obj y/\vrb y]}}}
(\denote{\Xlo y}^{\Modele,\w_1,g_{[\Obj
      x/\vrb x]_{[\Obj y/\vrb y]}}})
=
\denote{\Xlo\prd{regarder}}^{\Modele,\w_1,g_{[\Obj
      x/\vrb x]_{[\Obj y/\vrb y]}}}
(\Obj y)
\). 
Et le fait est que 
\(\denote{\Xlo\prd{regarder}}^{\Modele,\w_1,g_{[\Obj
      x/\vrb x]_{[\Obj y/\vrb y]}}}\) est la même chose que 
\(\denote{\Xlo\prd{regarder}}^{\Modele,\w_1,g}\) (car il n'y a plus de
variable ici), c'est-à-dire c'est la
fonction  
décrite dans la
figure~\ref{f:regardf}, p.~\pageref{f:regardf}.
Par conséquent, 
\(\denote{\Xlo[[\prd{regarder}(y)](x)]}^{\Modele,\w_1,g_{[\Obj
      x/\vrb x]_{[\Obj y/\vrb y]}}} = 
\denote{\Xlo\prd{regarder}}^{\Modele,\w_1,g}(\Obj y)(\Obj x)
\), c'est-à-dire la valeur que cette fonction assigne quand on lui
donne \Obj y comme premier argument et \Obj x comme second
argument.  Par exemple, si \Obj x est \Obj{Alice} et \Obj y est
\Obj{Dina}, la valeur sera $0$, etc.

Pour conclure : en reprenant le calcul depuis le début, 
\(\denote{\Xlo\lambda x\lambda y[[\prd{regarder}(y)](x)]}^{\Modele,\w_1,g}\) 
est la fonction qui à tout objet \Obj x associe la fonction qui à tout
objet \Obj y associe la valeur
\(\denote{\Xlo\prd{regarder}}^{\Modele,\w_1,g}(\Obj y)(\Obj x)\). 
Mais attention, cela n'est pas la même fonction que
\(\denote{\Xlo\prd{regarder}}^{\Modele,\w_1,g}\), car les arguments ne
sont pas attendus dans le même ordre\footnote{Rappelons que
  \(\denote{\Xlo\prd{regarder}}^{\Modele,\w_1,g}\)
équivaut à \(\denote{\Xlo\lambda y\lambda
  x[[\prd{regarder}(y)](x)]}^{\Modele,\w_1,g}\), la fonction  qui à
tout objet \Obj y associe la fonction qui à tout 
objet \Obj x associe la valeur 
\(\denote{\Xlo\prd{regarder}}^{\Modele,\w_1,g}(\Obj y)(\Obj x)\).}.
%%  \end{enumerate}

%\fussy

\begin{enumerate}[resume]
\item 
\(\denote{\Xlo\lambda x[\lambda y[\prd{regarder}(y)](x)]}^{\Modele,\w_1,g}\)
\end{enumerate}

D'après la définition~\ref{d:slabs},
\(\denote{\Xlo\lambda x[\lambda y[\prd{regarder}(y)](x)]}^{\Modele,\w_1,g}\)
est la fonction qui, à tout objet \Obj x, associe 
\(\denote{\Xlo[\lambda y[\prd{regarder}(y)](x)]}^{\Modele,\w_1,g_{[\Obj
x/\vrb x]}}\).  Maintenant pour interpréter \(\Xlo[\lambda
  y[\prd{regarder}(y)](x)]\), nous pouvons utiliser le théorème
\ref{th:Seml@}. 
Pour ce faire, nous devons connaître la dénotation de \vrb x par
rapport aux indices via lesquels nous évaluons l'application fonctionnelle  (en
l'occurrence $\w_1$ et $g_{[\Obj x/\vrb x]}$) ;
et nous la connaissons : \(\denote{\Xlo x}^{\Modele,\w_1,g_{[\Obj
x/\vrb x]}} = \Obj x\).
Partant, le théorème nous dit que 
\(\denote{\Xlo[\lambda y[\prd{regarder}(y)](x)]}^{\Modele,\w_1,g_{[\Obj
x/\vrb x]}}
=
\denote{\Xlo[\prd{regarder}(y)]}^{\Modele,\w_1,g_{[\Obj
x/\vrb x]_{[\Obj x/\vrb y]}}}
\).

\sloppy
Ensuite, la définition~\ref{d:Sem@} nous indique que 
\(\denote{\Xlo[\prd{regarder}(y)]}^{\Modele,\w_1,g_{[\Obj
x/\vrb x]_{[\Obj x/\vrb y]}}}
=
\denote{\Xlo\prd{regarder}}^{\Modele,\w_1,g_{[\Obj
x/\vrb x]_{[\Obj x/\vrb y]}}}
(\denote{\Xlo y}^{\Modele,\w_1,g_{[\Obj
x/\vrb x]_{[\Obj x/\vrb y]}}})
\).
Et nous avons \(\denote{\Xlo y}^{\Modele,\w_1,g_{[\Obj
x/\vrb x]_{[\Obj x/\vrb y]}}} = \Obj x\) (à cause de l'assignation $g_{[\Obj
    x/\vrb x]_{[\Obj x/\vrb y]}}$ qui fixe la valeur de \vrb y à \Obj x). 
Par conséquent, 
\(\denote{\Xlo[\lambda y[\prd{regarder}(y)](x)]}^{\Modele,\w_1,g_{[\Obj
x/\vrb x]}}
=
\denote{\Xlo\prd{regarder}}^{\Modele,\w_1,g}(\Obj x)\).

Conclusion : \(\denote{\Xlo\lambda x[\lambda y[\prd{regarder}(y)](x)]}^{\Modele,\w_1,g}\)
est la fonction qui à tout objet \Obj x associe 
\(\denote{\Xlo\prd{regarder}}^{\Modele,\w_1,g}(\Obj x)\). 
C'est globalement une fonction à un seul argument et qui renvoie comme
résultat une fonction elle aussi à un argument.

\fussy
\end{solu}
\end{exo}
