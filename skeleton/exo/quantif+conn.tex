% -*- coding: utf-8 -*-
\begin{exo}[Quantificateurs et connecteurs]
\label{exo:2q+c}
Indiquez (informellement\footnote{C'est-à-dire en français, sans
  entrer dans les détails techniques.}) 
\pagesolution{crg:2q+c}
les conditions de vérité des formules
suivantes :
\addtolength{\multicolsep}{-8pt}
%\smallskip
\begin{multicols}{2}
\begin{exolist}
\item \(\Xlo\exists x [\prd{homard}(x) \wedge \prd{gaucher}(x)]\)
\item \(\Xlo\exists x [\prd{homard}(x) \vee \prd{gaucher}(x)]\)
\item \(\Xlo\exists x [\prd{homard}(x) \implq \prd{gaucher}(x)]\)
\item \(\Xlo\exists x [\prd{homard}(x) \ssi \prd{gaucher}(x)]\)
\item \(\Xlo\forall x [\prd{homard}(x) \wedge \prd{gaucher}(x)]\)
\item \(\Xlo\forall x [\prd{homard}(x) \vee \prd{gaucher}(x)]\)
\item \(\Xlo\forall x [\prd{homard}(x) \implq \prd{gaucher}(x)]\)
\item \(\Xlo\forall x [\prd{homard}(x) \ssi \prd{gaucher}(x)]\)
\end{exolist}
\end{multicols}
%

Quelles sont celles qui peuvent être des traductions de phrases
simples et naturelles du français ?

\begin{solu} (p.~\pageref{exo:2q+c})\label{crg:2q+c}
\begin{exolist}
\item \(\Xlo\exists x [\prd{homard}(x) \wedge \prd{gaucher}(x)]\) :
il existe un individu qui est un homard et qui est gaucher ; en français cela donnera \sicut{il y a un homard gaucher} ou, éventuellement, \sicut{il existe des homards gauchers}.

\item \(\Xlo\exists x [\prd{homard}(x) \vee \prd{gaucher}(x)]\) :
il existe un individu qui est un homard ou qui est gaucher ; cette formule est vraie du moment que les homards existent (même s'il n'y a pas de gauchers dans le modèle)\footnote{Ou inversement, cette formule est vraie aussi du moment que les gauchers existent.}.  Pas de phrase naturelle en français pour cette formule.

\item \(\Xlo\exists x [\prd{homard}(x) \implq \prd{gaucher}(x)]\) :
il existe un individu tel que \emph{si} c'est homard alors il est gaucher ; cette formulation des conditions de vérité est un peu alambiquée, mais il faut se souvenir que cela équivaut à : il existe un individu qui n'est pas un homard ou qui est gaucher (car $\Xlo\phi\implq\psi$ équivaut à $\Xlo\neg\phi\vee\psi$) ; cette formule est vraie du moment qu'il existe des individus (par exemple vous et moi) qui ne sont pas des homards... Pas de phrase naturelle en français pour cette formule.

\item \(\Xlo\exists x [\prd{homard}(x) \ssi \prd{gaucher}(x)]\)
il existe un individu qui est un homard gaucher ou bien qui n'est ni homard ni gaucher. Pas de phrase naturelle en français pour cette formule.

\item \(\Xlo\forall x [\prd{homard}(x) \wedge \prd{gaucher}(x)]\) :
tout individu du modèle est un homard gaucher. Pas vraiment de phrase naturelle en français pour cette formule.

\item \(\Xlo\forall x [\prd{homard}(x) \vee \prd{gaucher}(x)]\) :
tout individu est un homard ou est gaucher ; autrement dit, pour tout individu, si ce n'est pas un homard, alors il doit forcément être gaucher (et vice-versa). Pas de phrase naturelle en français pour cette formule.

\item \(\Xlo\forall x [\prd{homard}(x) \implq \prd{gaucher}(x)]\) :
pour tout individu, s'il est un homard alors il est gaucher. En français : \sicut{tous les homards sont gauchers}.

\item \(\Xlo\forall x [\prd{homard}(x) \ssi \prd{gaucher}(x)]\) :
pour tout individu, si c'est un homard, alors il est gaucher et s'il est gaucher, alors c'est un homard. Pas vraiment de phrase naturelle en français pour cette formule, si ce n'est \sicut{«homard» et «gaucher», c'est la même chose}...

\end{exolist}
\end{solu}
\end{exo}
