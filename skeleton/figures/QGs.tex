\documentclass[11pt,preview,pstricks]{standalone}


\usepackage[T1]{fontenc}
\usepackage{textcomp}  %% font math
\usepackage{amsmath}   %% extensions math
\usepackage{amsfonts}  %% font math (appeler plut�t avant amssymb)
\usepackage{amssymb}   %% font math
\usepackage{stmaryrd}  %% font symboles
\usepackage{pifont}    %% font symboles Dingbats

\input{../pre_fontLib} % Polices

\usepackage{pstricks}
\usepackage{pst-node}
\usepackage{pst-grad}
\usepackage{pst-tree}

\input{macros} %% macros g�n�rales
\renewcommand{\denote}[1]{%
   \ensuremath{\dlb{#1}\drb}}  %% de newtxmath
%   \ensuremath{\llbracket{#1}\rrbracket}}  %% de stmaryrd
\newcommand{\Denote}[1]{%
   \ensuremath{\left\llbracket\begin{array}{@{}c@{}}\text{#1}\end{array}\right\rrbracket}}  %% de stmaryrd
\input{sem} %% macros S�mantique


\begin{document}

\begin{pspicture}(-4,-3)(4,3)
\pnode(0,0){E}
\rput(3,2){\rnode{enfs}{{$\FI(w,\prd{enfant})$}}}
\ncline[nodesepA=3pt]{enfs}{E}
\rput(0,0){\rnode{E}{\psellipse[fillstyle=hlines*,hatchcolor=gray,hatchsep=1.8pt](1,.7)}}
%
\psset{linewidth=.6pt}
%\rput(0,0){\psellipse(.3,.2)}
\rput{8}(0,0){\psellipse(1.2,2.5)}
\rput{10}(0.1,0){\psellipse(2,1.3)}
\rput{-5}(-1,0){\psellipse(2.6,1.1)}
%\rput(.4,0){\psellipse(1.2,.6)}
%\rput{30}(-1,-.8){\psellipse(1.7,.8)}
\rput{60}(-.2,.2){\psellipse(1.4,2.4)}
\rput{15}(-.8,-.4){\psellipse(2.2,1.5)}
\end{pspicture}

\end{document}
