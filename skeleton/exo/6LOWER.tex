% -*- coding: utf-8 -*-
\begin{exo}\label{exo:LOWER}
Démontrez que $\Xlo\prdk{lower}(\lambda P[P(\cns a)])$ équivaut à \cns a.\pagesolution{crg:LOWER}
\\
%\sloppy
Calculez 
\(\Xlo\prdk{lower}(\lambda P\forall x[\prd{enfant}(x)\implq[P(x)]])\)
et
\(\Xlo\prdk{lower}(\lambda P\exists x[\prd{enfant}(x)\wedge[P(x)]])\),
et décrivez la dénotation 
des résultats obtenus.   
%\fussy
\begin{solu}(p.~\pageref{exo:LOWER})\label{crg:LOWER}

\sloppy
Par définition (\ref{xd:lower}, p.~\pageref{xd:lower}), \prdk{lower} équivaut à $\Xlo \lambda X\atoi y \forall P[[X(P)] \implq [P(y)]]$ ; par conséquent $\Xlo\prdk{lower}(\lambda P[P(\cns a)])$ équivaut à 
$\Xlo [\lambda X\atoi y \forall P[[X(P)] \implq [P(y)]](\lambda P'[P'(\cns a))]$ qui, par \breduc s successives, se simplifie en 
$\Xlo\atoi y \forall P[[\lambda P'[P'(\cns a)(P)] \implq [P(y)]]$ 
puis en 
$\Xlo\atoi y \forall P[[P(\cns a)] \implq [P(y)]]$.
Ce \atoi-terme dénote dans \w\ l'individu qui possède \emph{toutes} les propriétés (extensionnelles) que possède Alice dans \w.  Cet individu existe, est unique et est forcément Alice, ne serait-ce parce que parmi les propriétés d'Alice, il y a celle qui s'exprime par $\Xlo\lambda x[x=\cns a]$ et que seule Alice satisfait.  Donc $\Xlo\prdk{lower}(\lambda P[P(\cns a)])$ est bien équivalent à \cns a.


De la même façon, \(\Xlo\prdk{lower}(\lambda P\forall x[\prd{enfant}(x)\implq[P(x)]])\)
équivaut à 
\(\Xlo[\lambda X\atoi y \forall P[[X(P)] \implq [P(y)]](\lambda P'\forall x[\prd{enfant}(x)\implq[P'(x)]])]\) 
qui se réduit en 
\(\Xlo\atoi y \forall P[\forall x[\prd{enfant}(x)\implq[P(x)]] \implq [P(y)]]\).
Ce \atoi-terme dénote l'unique individu qui possède toutes les propriétés communes à  tous les enfants ; mais s'il y a plusieurs enfants dans le monde d'évaluation, il ne peut pas y avoir un seul individu qui satisfait cette condition (chaque enfant a toutes les propriétés de tous les enfants). Donc la dénotation de \(\Xlo\prdk{lower}(\lambda P\forall x[\prd{enfant}(x)\implq[P(x)]])\)
n'est pas définie (tant qu'il y a plusieurs individus dans la dénotation de \sicut{enfant}).


De même, \(\Xlo\prdk{lower}(\lambda P\exists x[\prd{enfant}(x)\wedge[P(x)]])\) se réduit en 
\(\Xlo\atoi y \forall P[\exists x[\prd{enfant}(x)\wedge[P(x)]] \implq [P(y)]]\).
Cette fois ce \atoi-terme dénote l'unique individu qui possède toute propriété satisfaite par au moins un enfant, mais un tel individu n'existe pas (s'il y a plusieurs enfants) car parmi ces propriétés il y en a beaucoup qui sont contradictoires entre elles (par exemple, il y a des enfants bruns, des enfants blonds, des filles, des garçons, etc.). Ainsi, comme précédemment, la dénotation de \(\Xlo\prdk{lower}(\lambda P\exists x[\prd{enfant}(x)\wedge[P(x)]])\) n'est pas définie.

\fussy
\end{solu}
\end{exo}
