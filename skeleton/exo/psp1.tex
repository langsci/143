% -*- coding: utf-8 -*-
\begin{exo}\label{exo:1psp1}
Explicitez les projections de chacune des phrases suivantes.
\begin{enumerate}
\item %\ex.[\ref{x:expsp1}]
Si Pierre était venu, Marie serait partie.

\item %\ex.[\ref{x:expsp2}]
Hélène aussi fait de la linguistique.\label{x:expsp2}

\item %\ex.[\ref{x:expsp3}]
Hélène fait aussi  de la linguistique.\label{x:expsp3}

\item %\ex.[\ref{x:expsp4}]
Marianne a arrêté de fumer.

\item %\ex.[\ref{x:expsp5}]
Lorsque le téléphone a sonné, j'étais dans mon bain.

\item %\ex.[\ref{x:expsp6}]
Laurence a pris seulement une salade.

\item %\ex.[\ref{x:expsp7}]
Antoine n'est plus barbu.

\item %\ex.[\ref{x:expsp8}]
Jean a réussi à intégrer l'ENA.

\item %\ex.[\ref{x:expsp9}]
C'est Pierre qui a apporté des fleurs.\label{x:expsp9}

\item %\ex.[\ref{x:expsp10}]
Même Robert a eu la moyenne au partiel.\label{x:expsp10}

\item %\ex.[\ref{xmoutarde}]
Jean sait que c'est le colonel Moutarde le coupable.
\end{enumerate}
%--------
\begin{solu} (p.~\pageref{exo:1psp1})

La méthode la plus simple pour trouver les projections d'une
phrase $P$ est de la comparer avec sa forme négative en \sicut{il est
  faux que $P$}, comme ce que nous avons vu avec les exemples \ref{xmoutarde}--\ref{xgonc} en \S\ref{ss:projections}, p.~\pageref{p.resNeg}.

\begin{enumerate}
\item %\ex. \label{x:expsp1}
Projection : Pierre n'est pas venu. 

\item %\ex.  \label{x:expsp2}
Projection : Quelqu'un d'autre qu'Hélène fait de la linguistique.

\item %\ex.  \label{x:expsp3}
Projection : Hélène fait autre chose que de la linguistique.

\item %\ex.  \label{x:expsp4}
Projection : Marianne fumait avant.

\item %\ex. \label{x:expsp5}
Projection : Le téléphone a sonné.

\item %\ex.  \label{x:expsp6}
Projection : Laurence a pris une salade (c'est-à-dire la phrase sans \sicut{seulement} ;  remarquons que dans cette
phrase, le  contenu proféré est : Laurence n'a rien pris d'autre qu'une
salade). 

\item %\ex.  \label{x:expsp7}
Projection : Antoine a été barbu.

\item %\ex.  \label{x:expsp8}
Projections : Jean a essayé d'intégrer l'ENA (c'est-à-dire il s'est présenté au
concours d'entrée).  Mais il y a une seconde présupposition déclenchée par \sicut{essayer} qui est : il n'est pas facile d'intégrer l'ENA.

\item %\ex.  \label{x:expsp9}
Projection : Quelqu'un a apporté des fleurs.

\item %\ex.  \label{x:expsp10}
Projection : Robert faisait partie de ceux qui avaient le moins de chances d'avoir la moyenne au partiel.  Notons ici que le test de la négation s'applique un peu
difficilement. Il faut bien prendre soin d'utiliser la formulation en
\sicut{il est faux que}.

\item %\ex.[\ref{xmoutarde}] \a. 
Projections : 1) Quelqu'un est le coupable (c'est-à-dire  il y a un coupable) et 2)
le coupable est le colonel Moutarde. 
\end{enumerate}

Notons que la plupart des projections de cet exercice se trouvent être des présuppositions. Pour certaines, on peut s'en assurer assez facilement en appliquant le test de redondance (\S\ref{sss:ptépsp}, p.~\pageref{p.testAB}) ; pour d'autres, comme \ref{x:expsp9} et \ref{x:expsp10}, le test est moins concluant, ce qui peut soulever la question de leur statut véritablement présuppositionnel. Quant à \ref{x:expsp2} et \ref{x:expsp3}, le test fonctionne bien à condition d'utiliser une formulation plus précise du contenu projectif, par exemple \sicut{Lucie fait de la linguistique et Hélène aussi fait de la linguistique} {\vs} {\zarb}\sicut{Hélène aussi fait de la linguistique et Lucie fait de la linguistique}.

\end{solu}
\end{exo}
