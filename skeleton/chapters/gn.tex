% -*- coding: utf-8 -*-
\chapter{Groupes nominaux et quantification}
%%%%%%%%%%%%%%%%%%%%%%%%%%%%%%%%%%%%%%%%%%%%
\label{ch:gn}
\Writetofile{solf}{\protect\section{Chapitre \protect\ref{ch:gn}}}



Ce chapitre est consacré à une exploration approfondie de la sémantique du domaine nominal, et plus précisément des groupes nominaux (\GN). Le langage du calcul des prédicats, sur lequel est fondé {\LO}, donne une part importante au phénomène de la quantification, qui se réalise principalement dans l'interprétation de certains {\GN} de la langue.  Il est donc assez naturel de s'y attarder un peu à ce stade de l'ouvrage. Dans un premier temps, comme annoncé, nous allons perfectionner la sémantique de la quantification pour la rendre plus satisfaisante sur le plan formel (\S\ref{s:quantif2}). 
Dans la suite du chapitre (\S\ref{s:GNportée} et \ref{s:CatGN}), à partir de notions issues du système formel et d'observations empiriques sur des données linguistiques, nous étudierons un certain nombre de propriétés sémantiques typiquement attachées aux {\GN}. Cela nous donnera l'occasion de mieux appréhender l'interprétation et la traduction en {\LO} de diverses phrases et constructions du français.


\section{Sémantique de la quantification}
%========================================
\label{s:quantif2}\is{quantificateur|(}\is{quantification|(}

Nous l'avons vu, le jeu de règles d'interprétation définies dans le
chapitre précédent est problématique par rapport à l'objectif
théorique et descriptif que nous nous sommes donné.  Le premier
problème est qu'il est incomplet : nous ne savons pas interpréter les
variables.  Or les variables sont des expressions de {\LO} à part
entière, qui peuvent intervenir dans des formules qui représentent le
sens de phrases correctes du français.  C'est par exemple le cas des
formules quantifiées, et dans le chapitre précédent, pour les
interpréter, nous prenions bien soin de nous débarrasser de leurs variables, en
les remplaçant par des constantes, justement parce que nous n'avions pas
de règle permettant de calculer \denote{\vrb x}.  Et nous arrivons là à notre
deuxième problème (qui est lié au premier) : les règles de sémantique
vues jusqu'ici ne sont pas pleinement compositionnelles. Nous avons
une règle syntaxique qui nous permet de construire une formule
quantifiée, par exemple $\Xlo\exists x\phi$, à partir de $\vrb x$ et surtout
à partir de $\vrb\phi$.  En suivant le principe de compositionnalité,
l'interprétation de $\Xlo\exists x\phi$ devrait donc dépendre
l'interprétation de $\vrb\phi$.  Mais nous l'interprétions à partir de
$[\kappa/\vrb x]\Xlo\phi$, ce qui n'est pas la même chose.
Il y a pire : comment interpréter une formule comme \(\Xlo\exists x
[\prd{aimer}(x,y)]\) ?  
C'est une expression bien formée de {\LO} et donc elle \emph{doit} être
interprétable.  Or nous ne savons pas le faire.  Nous verrons
d'ailleurs que nous  rencontrerons de telles formules au
cours de l'interprétation de formules plus complexes.

La première question qui va donc nous occuper dans cette section est :
comment interpréter les variables ? c'est-à-dire : comment calculer
\(\denote{\vrb x}^{\Modele}\) ?

\subsection{Variables et pronoms}
%--------------------------------
\is{variable|(}\is{pronom}

La réponse à la question précédente est toute simple, et c'est : «on
ne peut pas».  Pour un modèle {\Modele} donné, on ne peut
raisonnablement pas  trouver une valeur à \(\denote{\vrb x}^{\Modele}\),\is{valeur!\elid\ d'une variable}
du moins avec le même type de méthode que celle vue dans le chapitre
précédent.  Le reste de cette section va consister à montrer 
%(/ convaincre de) 
i) pourquoi «on ne peut pas» et ii) comment s'en sortir malgré tout.


Nous avons vu déjà que ce qui traduit le mieux les variables (libres) en
français ce sont les pronoms personnels comme \sicut{il}
ou \sicut{elle} (ou \sicut{le}, \sicut{la}, etc.).  Or les pronoms ont
une façon de dénoter qui est très particulière, très différente de
celle des noms propres par exemple.  

Revenons un instant au principe général d'interprétation des phrases.
Imaginons que je parle d'un groupe d'étudiants, c'est-à-dire que mon
modèle, appelons-le $\Modele_0$, contient essentiellement les
individus qui composent ce 
groupe.  Si je prononce la phrase :

\ex.  \label{x:abaille}
Aurélie bâille.\\
\(\Xlo\prd{bâiller}(\cns{a})\)

pour connaître sa dénotation dans $\Modele_0$, il faut et, surtout, il
suffit de connaître certaines informations sur $\Modele_0$ : savoir qui
est Aurélie et si elle fait partie de l'ensemble des bâilleurs de
$\Modele_0$.  Autrement dit, il suffit de savoir comment est le monde
(dont je parle).  De même pour trouver la dénotation de :
\ex.  \label{x:qqbaille}
Quelqu'un bâille.\\
\(\Xlo\exists x \,\prd{bâiller}(x)\)\footnote{Cette traduction est
  simplifiée.  Pour être plus précis, il faudrait plutôt écrire quelque chose
  comme : \(\Xlo\exists x [\prd{humain}(x) \wedge \prd{bâiller}(x)]\).}

Cette phrase est vraie dans $\Modele_0$ si et seulement s'il existe un
individu du domaine qui appartient à l'ensemble des bâilleurs.  Et peu
importe qui est cet individu.  Maintenant examinons le cas où je prononce :
\ex.  \label{x:ellebaille}
Elle bâille.\\
\(\Xlo\prd{bâiller}(x)\)

sans faire aucun geste de démonstration.  Ici, il ne suffit plus de
savoir comment est le monde pour trouver la dénotation de
\ref{x:ellebaille}, nous avons besoin d'une autre information : nous avons besoin
de savoir de qui on parle exactement quand on dit \sicut{elle}.
En effet, supposons que dans $\Modele_0$, Aurélie fait partie des
bâilleurs, mais pas Émilie.  Si en disant \sicut{elle} je pense
à  Aurélie,
alors \ref{x:ellebaille} est vraie, mais si \sicut{elle} c'est
Émilie, alors \ref{x:ellebaille} est fausse (mais dans $\Modele_0$
\ref{x:qqbaille} est indéniablement vraie).  Ainsi --~et c'est ce
qui est important~-- une même phrase, \ref{x:ellebaille}, peut être
vraie ou fausse relativement à un \emph{même} modèle.
Pour dire encore les choses différemment, si nous ne disposons \emph{que} d'un
modèle (\ie\ une description du monde), nous sommes incapables de trouver la valeur de
vérité précise de la phrase \ref{x:ellebaille}.  Il est  même possible de
pousser plus loin : relativement à un modèle seul, comme  $\Modele_0$,
\ref{x:ellebaille} ne dénote pas une valeur de vérité.  Cette
observation est tout à fait cohérente et conforme à l'intuition, bien
qu'elle malmène quelque peu la thèse de Frege : \ref{x:ellebaille}
n'aurait pas exactement la même catégorie sémantique que
\ref{x:abaille} et \ref{x:qqbaille}, puisqu'elles n'ont pas des
types de dénotation comparables.
En fait, nous n'allons pas nous engager d'avantage sur ce chemin de
fracture entre la vision fregéenne et notre observation sur les
phrases à pronoms comme \ref{x:ellebaille} ; au contraire nous allons 
nous diriger vers une conciliation, qui reprend la proposition due au
logicien A.~Tarski \citep{Tarski:44}.\Andex{Tarski, A.}


La solution s'appuie sur les points suivants : une phrase déclarative
comme \ref{x:abaille}, \ref{x:qqbaille} ou \ref{x:ellebaille},
dénote toujours une valeur de vérité ; cette dénotation se calcule
toujours \emph{relativement à} un
état du monde (\ie\ un modèle), \emph{mais pas seulement} ; elle dépend
aussi d'un autre paramètre qui spécifie ce à quoi fait référence
toute occurrence de pronoms, ou plus exactement ce à quoi fait
référence toute variable du langage objet.   Ce paramètre sera
formalisé au moyen d'une \emph{fonction} qui à chaque variable du
langage assigne une valeur dans le domaine du modèle.\is{valeur!\elid\ d'une variable}
Une telle fonction se nomme, donc, une fonction d'assignation de
valeurs aux variables, ou plus simplement une \kwo{fonction
d'assignation}\is{fonction!\elid\ d'assignation} ou encore
une \kwo{assignation}.%
\is{assignation|see{fonction d'assignation}}

\begin{defi}[Fonction d'assignation]
Soit un modèle \(\Modele = \tuple{\Unv{A},\FI}\) et $\VAR$ l'ensemble
des variables de {\LO}. Une fonction d'assignation est une fonction de
$\VAR$ vers \Unv{A}.

Les fonctions d'assignation seront notées ici $g$ (ou $g'$, $g_1$, etc.).
\end{defi}

Voici un exemple.  Partons d'un (petit) modèle $\Modele_1 =
\tuple{\Unv{A}_1, \FI_1}$ qui ne comprend que quatre individus :
\(\Unv{A}_1 = \set{\Obj{Ada}; \Obj{Cordula}; \Obj{Lucette};
\Obj{Van}}\).  Supposons que notre langage objet comporte 
%quatre
%constantes d'individus ($\cns{a}$, $\cns{c}$, $\cns{l}$, $\cns{i}$) et
trois variables, $\vrb x$, $\vrb y$, $\vrb z$.  Une fonction d'assignation projettera
alors chacune de ces variables sur un élément de $\Unv{A}_1$.
Ci-dessous, $g_1$, $g_2$ et $g_3$ en sont des exemples :

\begin{center}
\(g_1:\left[\begin{array}{l}
\vrb x \longmapsto \Obj{Cordula}\\
\vrb y \longmapsto \Obj{Lucette}\\
\vrb z \longmapsto \Obj{Ada}\\
\end{array}\right]\)
\(g_2:\left[\begin{array}{l}
\vrb x \longmapsto \Obj{Ada}\\
\vrb y \longmapsto \Obj{Van}\\
\vrb z \longmapsto \Obj{Ada}\\
\end{array}\right]\)
\(g_3:\left[\begin{array}{l}
\vrb x \longmapsto \Obj{Lucette}\\
\vrb y \longmapsto \Obj{Van}\\
\vrb z \longmapsto \Obj{Cordula}\\
\end{array}\right]\)
\dots
\end{center}


On notera que rien n'empêche une fonction d'assigner la même valeur à
différentes variables (comme $g_2$).  Pour un exemple aussi minimal
que celui-ci, il y a 64 ($4^3$) fonctions d'assignations différentes,
qui présentent toutes les combinaisons possibles de dénotations pour
le trio de variables.


À présent nous calculerons les dénotations des expressions par rapport
à un modèle {\Modele} \emph{et} une fonction $g$ ; nous écrirons
\(\denote{\cdot}^{\Modele,g}\).  C'est la fonction $g$ qui nous donne
la dénotation des variables, alors que la dénotation des constantes
nous est fournie par la fonction d'interprétation {\FI}, c'est-à-dire
par le modèle.  Ainsi la règle sémantique qui interprète les termes
possède en fait deux volets.

\begin{defi}[Interprétation des termes]
  Soit un modèle \(\Modele = \tuple{\Unv{A},\FI}\) et $g$ une fonction
  d'assignation : 
\begin{itemize}
\item si $\vrb v$ est une variable de \LO, \(\denote{\vrb v}^{\Modele,g}=g(\vrb v)\) ;
\item si $\Xlo a$ est une constante de \LO, \(\denote{\Xlo a}^{\Modele,g}=\FI(\xlo a)\).
\end{itemize}
\end{defi}


Par ailleurs, nous gardons les mêmes règles sémantiques que précédemment
(sauf bien sûr pour les quantifications).

La question qui peut se poser ici est :  comment choisir $g$ ? 
%d'où vient-elle ? *** À dévlopper
En fait, d'une certaine manière, nous pouvons dire que nous ne la choisissons pas
et qu'elle nous est donnée (ou que nous nous la donnons)
\emph{arbitrairement}.  Car comme pour les modèles, ce qui compte pour
nous c'est d'être capable d'effectuer les calculs pour n'importe
quelle assignation ; $g$ n'a donc pas à être spéciale {a priori}.
C'est ainsi que nous la considérerons dans ce qui suit : une fonction
d'assignation \emph{quelconque}.  Je reviendrai cependant sur la
question en disant quelques mots sur le statut des fonctions
d'assignation en \S\ref{ss:statutg} \alien{infra}.


Reprenons l'exemple \ref{x:ellebaille} et interprétons le par
rapport à {$\Modele_0$} (= \tuple{\Unv{A}_0,\FI_0}) et une certaine fonction
d'assignation $g$.  Selon la règle d'interprétation  des formules
prédicatives, donnée dans la définition~\ref{RI1}, p.~\pageref{RI1}, 
nous savons que 
\(\denote{\Xlo\prd{bâiller}(x)}^{\Modele_0,g} = 1\) ssi
\(\denote{\Xlo x}^{\Modele_0,g} \in \denote{\prd{bâiller}}^{\Modele_0,g}\) ;
poursuivons, puisque $\vrb x$ est une variable, cela nous donne la
condition : ssi \(g(\vrb x) \in \FI_0(\prd{bâiller})\).


Évidemment, tout cela ressemble fort à l'interprétation de phrases
avec des noms propres (\ie\ des constantes) : $g$ semble jouer un rôle
très semblable à celui de $\FI_0$.  Et d'une certaine manière, cette
analogie n'est pas incongrue : après tout, on peut dire que les
fonctions d'assignation sont aux variables ce que les fonctions
d'interprétation sont aux constantes.  Mais il faut bien se garder de
perdre de vue ce qui les distingue : $\FI_0$ est une partie
constitutive du modèle, alors que $g$ n'en fait pas partie.  Et
rappelons-nous que cela permet d'obtenir plusieurs valeurs sémantiques
différentes pour une phrase par rapport à un modèle donnée : il suffit
pour cela de changer de fonction $g$.  Autrement dit la fonction qui
décrit la dénotation des variables peut varier sans altérer le
modèle.   C'est ce qui fait que les variables sont bien variables :
elles varient indépendamment du modèle ; et
c'est aussi ce qui va permettre de définir précisément la sémantique
des expressions quantifiées.
\is{variable|)}

\subsection{Interprétation des formules quantifiées}
%---------------------------------------------------
\label{ss:IFQ}

Revenons à la comparaison de \ref{x:qqbaille} et \ref{x:ellebaille} :

\ex.[\ref{x:qqbaille}]
Quelqu'un bâille.\\
\(\Xlo\exists x \,\prd{bâiller}(x)\)

\ex.[\ref{x:ellebaille}]
Elle bâille.\\
\(\Xlo\prd{bâiller}(x)\)


Et rappelons que ce qui les oppose en premier lieu, c'est que
la dénotation de \ref{x:qqbaille} n'est pas dépendante d'une
d'assignation, mais simplement du modèle (c'est bien ce qu'engendre la
distinction grammaticale en français entre un pronom indéfini\is{pronom!\elid\ indéfini} et un
pronom anaphorique ou déictique).  Il semble donc inutile de recourir
à une fonction $g$ pour interpréter  \ref{x:qqbaille} contrairement
à \ref{x:ellebaille}.  Cependant, nous remarquons que la traduction de
\ref{x:ellebaille} est une sous-formule de la traduction de
\ref{x:qqbaille}, et donc, par compositionnalité, l'interprétation de
\ref{x:qqbaille} devra bien, à un moment ou à un autre, «passer
par» l'interprétation de \ref{x:ellebaille}, c'est-à-dire se faire
au moyen d'une assignation $g$.
Le principe interprétatif est donc le suivant : pour déterminer leur
dénotation, les expressions quantifiées ont besoin des assignations
car on les interprète justement en quantifiant sur des fonctions
d'assignation.  
C'est ce que nous allons voir en détail dans ce qui
suit, mais auparavant, introduisons un élément de notation
correspondant à un concept qui va nous être utile pour
l'interprétation. 

À partir d'une fonction d'assignation donnée, $g$, nous allons avoir
besoin de considérer certaines «variantes» de $g$.  

\begin{nota}\label{g[d/x]}
Soit $g$ une fonction d'assignation, $\vrb v$ une variable de {\LO} et
\Obj{d} un individu du domaine \Unv{A}.  La fonction notée
$g_{[\Obj{d}/\vrb v]}$ est la fonction d'assignation identique%
\footnotemark\
à $g$
\emph{sauf} que la valeur qu'elle assigne à $\vrb v$ est \Obj{d}.\\
Ainsi pour toute variable $\vrb u$ autre que $\vrb v$,
$g_{[\Obj{d}/\vrb v]}(u)=g(\vrb u)$ et $g_{[\Obj{d}/\vrb v]}(v)=\Obj{d}$, quelle que
soit la valeur de $g(\vrb v)$\footnotemark.
\end{nota}
\addtocounter{footnote}{-1}\footnotetext{Rappelons que deux fonctions sont identiques si elles ont
  mêmes ensembles de départ et d'arrivée et qu'à chaque élément de
  l'ensemble de départ elles assignent exactement les mêmes valeurs.}
\addtocounter{footnote}{1}\footnotetext{Remarquons que rien n'empêche
  d'avoir par ailleurs $g(\vrb v)=\Obj{d}$ ; dans ce cas là,
  $g_{[\Obj{d}/\vrb v]}$ et $g$ sont alors complètement identiques.

Profitons-en aussi pour mentionner la variante de notation $g_{[\vrb v\rightarrow\Obj{d}]}$, que l'on peut rencontrer dans la littérature, et qui représente la même chose que $g_{[\Obj{d}/\vrb v]}$.}

Cette notation nous permet en quelque sorte de contraindre
certaines valeurs d'une assignation.  Ou pour dire les choses plus
précisément, cela nous permet de voir directement dans l'écriture du
nom de l'assignation ($g_{[\Obj{d}/\vrb v]}$) quelle valeur précise est
assignée à telle ou telle variable. Voici un exemple, qui reprend
l'assignation $g_1$ donnée ci-dessus :

\ex.[]
\hspace{-6.5ex}
\(g_1:\left[\begin{array}{l}
\vrb x \longmapsto \Obj{Cordula}\\
\vrb y \longmapsto \Obj{Lucette}\\
\vrb z \longmapsto \Obj{Ada}\\
\end{array}\right]\)
%
\(g_{1[\Obj{Ada}/\vrb x]}:\left[\begin{array}{l}
\vrb x \longmapsto \Obj{Ada}\\
\vrb y \longmapsto \Obj{Lucette}\\
\vrb z \longmapsto \Obj{Ada}\\
\end{array}\right]\)
%
\(g_{1[\Obj{Van}/\vrb y]}:\left[\begin{array}{l}
\vrb x \longmapsto \Obj{Cordula}\\
\vrb y \longmapsto \Obj{Van}\\
\vrb z \longmapsto \Obj{Ada}\\
\end{array}\right]\)


Et dans cet exemple, $g_1$ étant ce qu'elle est, nous avons
$g_{1[\Obj{Ada}/\vrb z]}=g_1$.  Et puisque par exemple $g_{1[\Obj{Van}/\vrb y]}$
est elle-même une assignation, nous pouvons envisager également
$g_{1[\Obj{Van}/\vrb y]_{[\Obj{Ada}/\vrb x]}}$ qui est tout autant une
assignation.  Cela permet de contraindre plusieurs valeurs pour une
assignation de départ.

\ex.[]
\(g_{1[\Obj{Van}/\vrb y]_{[\Obj{Ada}/\vrb x]}}:\left[\begin{array}{l}
\vrb x \longmapsto \Obj{Ada}\\
\vrb y \longmapsto \Obj{Van}\\
\vrb z \longmapsto \Obj{Ada}\\
\end{array}\right]\)


Nous avons maintenant en main les éléments de métalangage nécessaires
pour définir les règles d'interprétation des formules quantifiées par
rapport à un modèle \(\Modele=\tuple{\Unv{A},\FI}\).

\begin{defi}[Interprétation des formules quantifiées]
\begin{enumerate}[semi,resume=RglSem1] 
\item\label{RIQg}
\begin{enumerate}
\item \(\denote{\Xlo\exists v \phi}^{\Modele,g} = 1\) ssi 
il existe au moins un individu \Obj{d} de \Unv{A} tel que \(\denote{\vrb\phi}^{\Modele,g_{[\Obj{d}/\vrb v]}} = 1\) ;
\item \(\denote{\Xlo\forall v \phi}^{\Modele,g} = 1\) ssi pour tout
  individu \Obj{d} de \Unv{A}, \(\denote{\vrb\phi}^{\Modele,g_{[\Obj{d}/\vrb v]}} = 1\).
\end{enumerate}
\setcounter{RglSem}{\value{enumi}}
\end{enumerate}
\end{defi}


D'une certaine manière, ces deux règles ne sont pas spectaculairement
différentes de celles vues au chapitre~\ref{LCP}, p.~\pageref{RI1}.
Ici nous opérons des quantifications (respectivement existentielle et
universelle) sur des individus du domaine, alors que précédemment nous
quantifions sur des constantes.  Mais à présent, nous ne touchons pas
à la forme de $\vrb\phi$.  L'interprétation est bien compositionnelle. 
Et nous nous débarrassons de l'hypothèse qu'à chaque individu de
\Unv{A} correspond une constante de {\LO}.

\sloppy

Illustrons le fonctionnement des règles (\RSem\ref{RIQg}$'$) à l'aide
d'un petit modèle \(\Modele_2 = \tuple{\Unv{A}_2,\FI_2}\), avec
\(\Unv{A}_2=\set{\Obj{Ada}; \Obj{Cordula};\Obj{Lucette};
  \Obj{Van}}\). Nous considérons quatre constantes d'individus
interprétées comme suit : $\FI_2(\cns{a})=\Obj{Ada}$,
$\FI_2(\cns{c})=\Obj{Cordula}$,   $\FI_2(\cns{l})=\Obj{Lucette}$,
$\FI_2(\cns{j})=\Obj{Van}$ ; et regardons simplement l'interprétation
de \prd{aimer} définie ainsi : \(\FI_2(\prd{aimer})=
\set{\tuple{\Obj{Ada}, \Obj{Van}}; \tuple{\Obj{Cordula}, \Obj{Van}};
  \tuple{\Obj{Lucette}, \Obj{Van}}; \tuple{\Obj{Van}, \Obj{Ada}};
  \tuple{\Obj{Van}, \Obj{Van}}}\). Calculons maintenant la valeur
sémantique de la formule \ref{x:Qex1} par rapport à $\Modele_2$ et $g_1$
(telle que définie plus haut).  

\fussy

\ex.  \label{x:Qex1}
\(\Xlo\exists x\, \prd{aimer}(x,\cns{a})\)


\ref{x:Qex1} peut être une traduction (approximative) de
\sicut{quelqu'un aime Ada}.  Appliquons la règle
(\RSem\ref{RIQg}$'$.a) :

\begin{itemize} \sloppy
\item \(\denote{\Xlo\exists x\, \prd{aimer}(x,\cns{a})}^{\Modele_2,g_1}=1\)  ssi 
il existe au moins un individu \Obj{d} de $\Unv{A}_2$ tel que
\(\denote{\Xlo\prd{aimer}(x,\cns{a})}^{\Modele_2,g_{1[\Obj{d}/\vrb x]}} = 1\) ; 
%
\item choisissons \Obj{Van} comme individu \Obj{d} et calculons
  \(\denote{\Xlo\prd{aimer}(x,\cns{a})}^{\Modele_2,g_{1[\Obj{Van}/\vrb x]}}\) ;
  si nous trouvons $1$, nous aurons bien montré que \ref{x:Qex1} est
  vraie ; 
%
\item d'après la règle (\RSem\ref{RIprd}) d'interprétation de {\LO}
  (p.~\pageref{RI1}), 
  nous savons que
  \(\denote{\Xlo\prd{aimer}(x,\cns{a})}^{\Modele_2,g_{1[\Obj{Van}/\vrb x]}}=1\)
  ssi \(\tuple{\denote{\vrb x}^{\Modele_2,g_{1[\Obj{Van}/\vrb x]}},
  \denote{\cns{a}}^{\Modele_2,g_{1[\Obj{Van}/\vrb x]}}} \in
  \denote{\prd{aimer}}^{\Modele_2,g_{1[\Obj{Van}/\vrb x]}}\) ; 
%
\item par définition,
  \(\denote{\vrb x}^{\Modele_2,g_{1[\Obj{Van}/\vrb x]}}=\Obj{Van}\),
  \(\denote{\cns{a}}^{\Modele_2,g_{1[\Obj{Van}/\vrb x]}}\) vaut \Obj{Ada}
  (cf. $\FI_2(\cns{a})$), et
  \(\denote{\prd{aimer}}^{\Modele_2,g_{1[\Obj{Van}/\vrb x]}}\) est donné
  ci-dessus par $\FI_2$ ;
%
\item il reste donc à vérifier que \tuple{\Obj{Van},\Obj{Ada}}
  appartient à $\FI_2(\prd{aimer})$ ; et c'est bien le cas ;
%
\item donc
  \(\denote{\Xlo\prd{aimer}(x,\cns{a})}^{\Modele_2,g_{1[\Obj{Van}/\vrb x]}}=1\),
  et donc \(\denote{\Xlo\exists x\,
  \prd{aimer}(x,\cns{a})}^{\Modele_2,g_1}=1\). 
\end{itemize}

\fussy
Cet exemple illustre aussi le fait que nous avons montré, en «allant
chercher» \Obj{Van} dans le domaine, que \ref{x:Qex1} est vrai par
rapport à $\Modele_2$ et surtout par rapport à une fonction
d'assignation ($g_1$) qui en soi ne parle pas du tout de \Obj{Van}.
Mais c'est normal, car comme nous l'avons vu précédemment,
\emph{globalement} la dénotation d'une formule comme \ref{x:Qex1}
\emph{ne dépend} d'aucune fonction d'assignation : la démonstration
ci-dessus repose surtout sur $g_{1[\Obj{Van}/\vrb x]}$, et nous aurions
donc pu mener la même démonstration en partant de n'importe qu'elle
assignation.

\sloppy

Cette remarque peut nous amener à la réflexion suivante : en fait les
règles (\RSem\ref{RIQg}$'$) interprètent les formules en quantifiant
sur les fonctions d'assignation.  Et nous pourrions ainsi simplifier
leurs énoncés en disant simplement : \(\denote{\Xlo\exists v
\phi}^{\Modele,g} = 1\) ssi il existe au moins une assignation $g'$
telle que \(\denote{\vrb\phi}^{\Modele,g'} = 1\) ; en effet si $\Xlo\exists v
\phi$ est vraie, on trouvera toujours une assignation $g'$ qui assigne
à $\vrb v$ une valeur qui «marche».  De même, nous pourrions simplifier :
\(\denote{\Xlo\forall v \phi}^{\Modele,g} = 1\) ssi pour toute assignation
$g'$, \(\denote{\vrb\phi}^{\Modele,g'} = 1\), car, de fait, $g'$ devra
ainsi passer en revue toute les valeurs possibles pour $\vrb v$.

\fussy

Cette idée est pertinente, mais la simplification suggérée est
en fait trop... simple.   Pour s'en convaincre regardons la formule
\ref{x:Qex3} :

\ex.  \label{x:Qex3}
\(\Xlo\exists x \forall y \, \prd{aimer}(y,x)\)


Cette formule correspond (toujours approximativement) à \sicut{il y a
quelqu'un que tout le monde aime}.  Tentons de calculer sa
dénotation par rapport à $\Modele_2$ et $g_1$ avec la version
simplifiée des règles d'interprétation :

\sloppy

\begin{itemize}
\item \(\denote{\Xlo\exists x \forall y \,
  \prd{aimer}(y,x)}^{\Modele_2,g_1}=1\) ssi il existe une autre
  assignation, appelons-la d'abord $g'$, telle que 
  \(\denote{\Xlo\forall y \,
  \prd{aimer}(y,x)}^{\Modele_2,g'}=1\) ;

\item bien sûr nous aurons intérêt de choisir $g_{1[\Obj{Van}/\vrb x]}$
  pour $g'$ (car en regardant $\Modele_1$ on se doute que c'est
  \Obj{Van} qui est aimé de tous) ; donc maintenant calculons
  \(\denote{\Xlo\forall y \,
  \prd{aimer}(y,x)}^{\Modele_2,g_{1[\Obj{Van}/\vrb x]}}\) ;

\item par la règle simplifiée, nous aurons 
  \(\denote{\Xlo\forall y \,
  \prd{aimer}(y,x)}^{\Modele_2,g_{1[\Obj{Van}/\vrb x]}}=1\) ssi pour {toute}
   fonction d'assignation possible, $g''$,
  \(\denote{\Xlo\prd{aimer}(y,x)}^{\Modele_2,g''}=1\) ;

\item mais maintenant si nous devons regarder \emph{toutes} les
  assignations $g''$, alors nous allons passer en revue toutes les
  valeurs possibles pour $\vrb x$ et $\vrb y$ et nous allons trouver de
  mauvaises conditions de vérité : en examinant toutes les valeurs
  possibles combinées pour $\vrb x$ et $\vrb y$ nous chercherons à montrer que
  tout le monde aime tout le monde, ce qui n'est bien sûr pas le sens
  de \ref{x:Qex3}.
\end{itemize}

\fussy

L'erreur qui a été commise dans ce calcul est que nous avons oublié
que nous avions choisi d'examiner seulement les cas où $\vrb x$ représente
\Obj{Van}, c'est-à-dire que nous avons oublié $g_{1[\Obj{Van}/\vrb x]}$.
Pendant l'interprétation d'une formule, lorsque l'on rencontre une
quantification existentielle, on  contraint (et on modifie)
l'assignation qui est 
initialement donnée en choisissant de fixer une valeur pour la
variable quantifiée, mais on ne touche pas aux valeurs des autres
variables, car celles-ci ont pu être déjà fixées par une passe
précédente d'interprétation.  On transmet ensuite l'assignation que
l'on a contrainte à l'étape suivante de l'interprétation qui doit
(récursivement) se souvenir des choix précédents.  De même, lorsque
l'on rencontre une quantification universelle, on fait varier la
fonction d'assignation que l'on a, mais on la fait varier uniquement
pour la valeur qu'elle peut donner à la variable quantifiée
universellement.  Il y a ensuite autant de passes d'interprétation
qu'il y de a valeurs possibles pour cette variable (en fait autant qu'il
y a d'individus dans \Unv{A}) et chaque passe se souvient de la valeur
provisoirement assignée à la variable quantifiée.

C'est pour ces raisons qu'une reformulation correcte des règles
(\RSem\ref{RIQg}$'$) doit se faire dans les termes suivants :  

\begin{enumerate}[label=\alph*.]
\item 
\(\denote{\Xlo\exists v \phi}^{\Modele,g} = 1\) ssi  
il existe au moins une assignation $g'$ identique à $g$ sauf pour la
valeur qu'elle peut assigner à $\vrb v$ et telle que
\(\denote{\vrb\phi}^{\Modele,g'} = 1\) ; 
\item 
\(\denote{\Xlo\forall v \phi}^{\Modele,g} = 1\) ssi pour toute
assignation $g'$ identique à $g$ sauf pour la valeur qu'elle peut
assigner à $\vrb v$, \(\denote{\vrb\phi}^{\Modele,g'} = 1\).  
\end{enumerate}

Ces assignations
$g'$ «identiques à $g$ sauf pour la valeur qu'elles peuvent
assigner à $\vrb v$~» indiquent clairement qu'il faut se souvenir des
valeurs assignées aux autres variables par $g$.  Et finalement, ces
reformulations sont tout à fait équivalentes à celles de
(\RSem\ref{RIQg}$'$) (cf. la notation~\ref{g[d/x]},
p.~\pageref{g[d/x]}), sauf qu'elles ne font plus explicitement mention 
des individus du domaine.  Si je prends la peine de les présenter ici,
c'est d'abord parce que de nombreux ouvrages de sémantique en font
usage et qu'il est utile de les maîtriser.  Ensuite, il est important
de concevoir l'interprétation des formules quantifiées comme des
quantifications sur les fonctions d'assignation, car certains
phénomènes sémantiques s'expliquent adéquatement avec cette vision.
%(cf. \fixme{***}).  **********************************
Cependant, et cela ne change rien à la vision de la
quantification, dans la suite de cet ouvrage, nous prendrons
l'habitude de manipuler les formulations de (\RSem\ref{RIQg}$'$), qui
restent plus commodes.

Pour conclure sur ce point, reprenons, mais correctement cette fois,
l'interprétation  de \ref{x:Qex3} :


\begin{itemize} \sloppy
\item \(\denote{\Xlo\exists x \forall y \,
  \prd{aimer}(y,x)}^{\Modele_2,g_1}=1\) ssi il existe un individu
  \Obj{d} de $\Unv{A}_2$, tel que \(\denote{\Xlo\forall y \,
  \prd{aimer}(y,x)}^{\Modele_2,g_{1[\Obj{d}/\vrb x]}}=1\) ;

\item choisissons \Obj{Van} pour \Obj{d} et calculons 
  \(\denote{\Xlo\forall y \,
  \prd{aimer}(y,x)}^{\Modele_2,g_{1[\Obj{Van}/\vrb x]}}\) ; 

\item  \(\denote{\Xlo\forall y \,
  \prd{aimer}(y,x)}^{\Modele_2,g_{1[\Obj{Van}/\vrb x]}}=1\) ssi pour
  tout individu \Obj{d} de $\Unv{A}_2$, on a
  \(\denote{\Xlo\prd{aimer}(y,x)}^{\Modele_2,g_{1[\Obj{Van}/\vrb x]_{[\Obj{d}/\vrb y]}}}=1\) ;

\item il nous faut donc examiner successivement
\begin{itemize}
\item \(\denote{\Xlo\prd{aimer}(y,x)}^{\Modele_2,g_{1[\Obj{Van}/\vrb x]_{[\Obj{Ada}/\vrb y]}}}\),
\item \(\denote{\Xlo\prd{aimer}(y,x)}^{\Modele_2,g_{1[\Obj{Van}/\vrb x]_{[\Obj{Cordula}/\vrb y]}}}\), 
\item \(\denote{\Xlo\prd{aimer}(y,x)}^{\Modele_2,g_{1[\Obj{Van}/\vrb x]_{[\Obj{Lucette}/\vrb y]}}}\), et
\item \(\denote{\Xlo\prd{aimer}(y,x)}^{\Modele_2,g_{1[\Obj{Van}/\vrb x]_{[\Obj{Van}/\vrb y]}}}\) ;
\end{itemize}

\item en accélérant un peu le processus, cela nous amène à vérifier
  successivement   que \tuple{\Obj{Ada},\Obj{Van}},
  \tuple{\Obj{Cordula},\Obj{Van}}, \tuple{\Obj{Lucette},\Obj{Van}}, et
  \tuple{\Obj{Van},\Obj{Van}} appartiennent à $\FI_2(\prd{aimer})$ ;
  c'est bien le cas, donc \ref{x:Qex3} est vraie par rapport à
  $\Modele_2$ et à $g_1$.
\end{itemize}

\fussy%

\begin{figure}[h]
\begin{center}
%\input{fig/g1p1.latex}
\pstree[treemode=R,arrows=->,nodesep=3pt,levelsep=*1.8cm]{\Tr{$g_1$}}{%
  \pstree{\Tr[ref=r]{$g_{1[\Obj{Van}/\vrb x]}$}}{%
\Tr[ref=l]{$g_{1[\Obj{Van}/\vrb x]_{[\Obj{Ada}/\vrb y]}}$}
\Tr[ref=l]{$g_{1[\Obj{Van}/\vrb x]_{[\Obj{Cordula}/\vrb y]}}$}
\Tr[ref=l]{$g_{1[\Obj{Van}/\vrb x]_{[\Obj{Lucette}/\vrb y]}}$}
\Tr[ref=l]{$g_{1[\Obj{Van}/\vrb x]_{[\Obj{Van}/\vrb y]}}$}
}%
}
\end{center}
\caption{Parcours des assignations pour l'interprétation de
\ref{x:Qex3}}\label{fig:EA}
\end{figure}

\newpage

Enfin, à titre de comparaison,  nous pouvons regarder dans les grandes
lignes l'interprétation de \ref{x:Qex4} par rapport à $\Modele_2$ et $g_1$ :

\ex.  \label{x:Qex4}
\(\Xlo\forall y \exists x\, \prd{aimer}(y,x)\)


En gardant les mêmes simplifications de traduction que précédemment,
\ref{x:Qex4} signifie à peu près \sicut{tout le monde aime quelqu'un}
(\ie\ \sicut{une personne possiblement différente}), comme le montrent les étapes de calcul suivantes:  

\begin{itemize}
\item \(\denote{\Xlo\forall y \exists x  \,
  \prd{aimer}(y,x)}^{\Modele_2,g_1}=1\) ssi pour tout individu
\sloppy%\tolerance=1500
  \Obj{d} de $\Unv{A}_2$,  nous obtenons: \(\denote{\Xlo\exists x \,
  \prd{aimer}(y,x)}^{\Modele_2,g_{1[\Obj{d}/\vrb y]}}=1\) ;
\item il faudra donc  calculer la valeur
  sémantique de \(\Xlo\exists x \, \prd{aimer}(y,x)\)  successivement avec les
  valeurs \Obj{Ada}, \Obj{Cordula}, \Obj{Lucette} et \Obj{Van} pour
  $\vrb y$, c'est-à-dire avec quatre variantes «en $\vrb y$~»
  successives de $g_1$ ;   
\item pour la variante $g_{1[\Obj{Ada}/\vrb y]}$, il faudra ensuite trouver
  une variante «en $\vrb x$~» de cette assignation pour vérifier la
  quantification existentielle, cette nouvelle variante pourra être
  par exemple $g_{1[\Obj{Ada}/\vrb y]_{[\Obj{Van}/\vrb x]}}$ ; 
\item cette opération sera répétée pour $g_{1[\Obj{Cordula}/\vrb y]}$,
  puis pour $g_{1[\Obj{Lucette}/\vrb y]}$, puis pour $g_{1[\Obj{Van}/\vrb y]}$ ;
\end{itemize}

\begin{figure}[h]
\begin{center}
%\input{fig/g1p2.latex}
\pstree[treemode=R,arrows=->,nodesep=3pt,levelsep=*1.8cm]{\Tr[ref=r]{$g_1$}}{%
  \pstree{\Tr[ref=l]{$g_{1[\Obj{Ada}/\vrb y]}$}}{\Tr[ref=l]{$g_{1[\Obj{Ada}/\vrb y]_{[\Obj{Van}/\vrb x]}}$}}
  \pstree{\Tr[ref=l]{$g_{1[\Obj{Cordula}/\vrb y]}$}}{\Tr[ref=l]{$g_{1[\Obj{Cordula}/\vrb y]_{[\Obj{Van}/\vrb x]}}$}}
  \pstree{\Tr[ref=l]{$g_{1[\Obj{Lucette}/\vrb y]}$}}{\Tr[ref=l]{$g_{1[\Obj{Lucette}/\vrb y]_{[\Obj{Van}/\vrb x]}}$}}
  \pstree{\Tr[ref=l]{$g_{1[\Obj{Van}/\vrb y]}$}}{\Tr[ref=l]{$g_{1[\Obj{Van}/\vrb y]_{[\Obj{Van}/\vrb x]}}$}}
}%
\end{center}
\caption{Parcours des assignations pour l'interprétation de
\ref{x:Qex4}}\label{fig:AE}
\end{figure}


\fussy%

Autrement dit, il faudra choisir \emph{à quatre reprises} une
assignation pour satisfaire (quatre fois) la quantification
existentielle de \ref{x:Qex4} (\ie\ \(\Xlo\exists x \,
\prd{aimer}(y,x)\)).  Chacune de ces  assignations prolonge une des
quatre assignations exigées par la quantification universelle de
\ref{x:Qex4}.  Il se trouve que par rapport à $\Modele_2$ (et à
$g_1$) \ref{x:Qex4} est vraie, comme \ref{x:Qex3}.  Mais si dans
le modèle on avait eu, dans $\FI_2(\prd{aimer})$,
\tuple{\Obj{Lucette},\Obj{Ada}} \emph{au lieu de}
\tuple{\Obj{Lucette},\Obj{Van}}, alors \ref{x:Qex4} aurait été
encore vraie, mais \ref{x:Qex3} fausse.  En effet 
\ref{x:Qex4} sera vraie dans ce nouveau modèle, et relativement à
$g_1$, car on trouvera par exemple  les assignations
$g_{1[\Obj{Ada}/\vrb y]_{[\Obj{Van}/\vrb x]}}$,
$g_{1[\Obj{Cordula}/\vrb y]_{[\Obj{Van}/\vrb x]}}$,
$g_{1[\Obj{Lucette}/\vrb y]_{[\Obj{Ada}/\vrb x]}}$, et
$g_{1[\Obj{Van}/\vrb y]_{[\Obj{Ada}/\vrb x]}}$  pour vérifier
\(\Xlo\prd{aimer}(y,x)\) ; ici la combinaison de valeurs pour les variables
$\vrb x$ et $\vrb y$ reste assez libre.  Elle est beaucoup plus contrainte en
revanche pour \ref{x:Qex3} : on peut au départ choisir librement une
valeur pour $\vrb x$ mais il faut s'y tenir dans la suite de
l'interprétation. 

Ces exemples montrent bien que les règles d'interprétation données ici
rendent compte systématiquement des variations de lectures dues
aux positions relatives des quantificateurs dans la représentation
sémantique d'une phrase.  Pour synthétiser (et schématiser) ces
différentes lectures, les figures \ref{fig:EA} et \ref{fig:AE} représentent
graphiquement les «parcours» successifs des assignations au cours
des étapes de l'interprétation de \ref{x:Qex3} et \ref{x:Qex4}.


\is{quantification|)}


\subsection{Synthèse}
%--------------------

Pour faire le point sur ce mécanisme d'interprétation qui met en jeu
des fonctions d'assignation, récapitulons l'ensemble des règles qui
définissent la sémantique de {\LO}.

\sloppy
Nous  identifions dans le langage de représentation sémantique {\LO},
l'ensemble  {\VAR} qui est l'ensemble des variables de {\LO} et
l'ensemble {\CON} l'ensemble de
ses constantes non logiques (comprenant donc les constantes
d'individus et les constantes de prédicats).  Les règles de formation
des formules de {\LO} sont données par la définition
syntaxique~\ref{SynP}, p.~\pageref{SynP}.

\fussy

\largerpage

\begin{defi}[Interprétation des termes]
  Soit un modèle \(\Modele = \tuple{\Unv{A},\FI}\) et $g$ une fonction
  d'assignation : 
\begin{itemize}
\item si $\vrb v$ est une variable de \VAR, \(\denote{\vrb v}^{\Modele,g}=g(\vrb v)\) ;
\item si $\Xlo a$ est une constante de \CON, \(\denote{\Xlo a}^{\Modele,g}=\FI(\xlo a)\).
\end{itemize}
\end{defi}


\begin{defi}[Interprétation des formules]
\label{RI2}
Soit un modèle \(\Modele = \tuple{\Unv{A},\FI}\) et $g$ une fonction
d'assignation de $\VAR$ dans \Unv{A}.
\begin{enumerate}[sem,series=RglSem2]
  \item 
\label{RIprd2}
\begin{enumerate}
\item
%Si $P$ est un prédicat à une place et si $\alpha$ est un terme, alors
$\denote{\Xlo P(\alpha)}^{\Modele,g}=1$ ssi \(\denote{\Xlo\alpha}^{\Modele,g} \in
\denote{\Xlo P}^{\Modele,g}\) ; 
%
\item %Si $P$ est un prédicat à deux places et si $\alpha$ et $\beta$ sont des
%termes, alors 
$\denote{\Xlo P(\alpha,\beta)}^{\Modele,g}=1$ ssi
\(\tuple{\denote{\Xlo\alpha}^{\Modele,g},\denote{\Xlo\beta}^{\Modele,g}} \in 
\denote{\Xlo P}^{\Modele,g}\) ; 
%
\item %Si $P$ est un prédicat à trois places et si $\alpha$, $\beta$ et
%  $\gamma$ sont des termes, alors 
$\denote{\Xlo P(\alpha,\beta,\gamma)}^{\Modele,g}=1$ ssi
\(\tuple{\denote{\Xlo\alpha}^{\Modele,g},\denote{\Xlo\beta}^{\Modele,g},\denote{\Xlo\gamma}^{\Modele,g}}
\in \denote{\Xlo P}^{\Modele,g}\) ; 
%
\item etc.
  \end{enumerate}
\item \label{RI=2}
%Si $\alpha$ et $\beta$ sont des termes, alors 
\(\denote{\Xlo\alpha = \beta}^{\Modele,g}=1\) ssi
\(\denote{\Xlo\alpha}^{\Modele,g}=\denote{\Xlo\beta}^{\Modele,g}\). 
\item \label{RIneg2}
%Si $\phi$ est une formule, alors 
\(\denote{\Xlo\neg\phi}^{\Modele,g}=1\) ssi
\(\denote{\Xlo\phi}^{\Modele,g}=0\). 
\item \label{RIcon2}
  %Si $\phi$ et $\psi$ sont des formules, alors 
  \begin{enumerate}
\item $\denote{\Xlo[\phi \wedge \psi]}^{\Modele,g}=1$ ssi $\denote{\Xlo\phi}^{\Modele,g}=1$ \emph{et} $\denote{\Xlo\psi}^{\Modele,g}=1$.
\item $\denote{\Xlo[\phi \vee \psi]}^{\Modele,g}=1$ ssi $\denote{\Xlo\phi}^{\Modele,g}=1$ \emph{ou} $\denote{\Xlo\psi}^{\Modele,g}=1$.
\item $\denote{\Xlo[\phi \implq \psi]}^{\Modele,g}=1$ ssi $\denote{\Xlo\phi}^{\Modele,g}=0$ \emph{ou} $\denote{\Xlo\psi}^{\Modele,g}=1$.
\item $\denote{\Xlo[\phi \ssi \psi]}^{\Modele,g}=1$ ssi $\denote{\Xlo\phi}^{\Modele,g}=\denote{\Xlo\psi}^{\Modele,g}$.
  \end{enumerate}
\item\label{RIQg2}
\begin{enumerate}
\item \(\denote{\Xlo\exists v \phi}^{\Modele,g} = 1\) ssi 
il existe au moins un individu \Obj{d} de \Unv{A} tel que \(\denote{\Xlo\phi}^{\Modele,g_{[\Obj{d}/\vrb v]}} = 1\) ;
\item \(\denote{\Xlo\forall v \phi}^{\Modele,g} = 1\) ssi pour tout
  individu \Obj{d} de \Unv{A}, \(\denote{\Xlo\phi}^{\Modele,g_{[\Obj{d}/\vrb v]}} = 1\).
\end{enumerate}
\setcounter{RglSem}{\value{enumi}}
\end{enumerate}
\end{defi}

Comme nous l'avons indiqué, ce qui fonde  le principe interprétatif
dont nous disposons à présent, c'est que la valeur sémantique, \ie\ la
dénotation, d'une expression se définit relativement à un modèle
\emph{et} à une fonction d'assignation donnée.  La notion de vérité
d'une formule (ou d'une phrase) est donc doublement relativisée.  Mais
cela ne nous empêchera pas, de temps à autre, de parler de la vérité
d'une formule par rapport à un modèle seul,... du moins pour certaines
formules de {\LO}.  Avec les éléments formels de métalangage que nous
avons, nous pouvons exprimer la vérité d'une formule de la manière
suivante. 


\begin{defi}[Vérité, ou satisfaction, d'une formule]
\label{d:Tarski}\is{satisfaction} 
$\Modele, g \satisf \vrb\phi$ ssi $\denote{\Xlo\phi}^{\Modele,g}=1$ ; on dira
  alors que {\Modele} et $g$ satisfont $\Xlo\phi$.\\
On peut également noter : $\satisf_{\Modele,g}\Xlo\phi$.
\\
$\Modele \satisf \Xlo\phi$ ssi pour toute fonction d'assignation $g$,
  $\denote{\Xlo\phi}^{\Modele,g}=1$ ; et on dira que {\Modele} satisfait
  $\Xlo\phi$. 
\end{defi}

C'est là ce que l'on nomme habituellement \emph{le principe de satisfaction
de Tarski}.\Andex{Tarski, A.}
Si $\Modele\satisf\vrb\phi$, nous pourrons également nous autoriser à écrire $\denote{\vrb\phi}^{\Modele}=1$, puisqu'alors la vérité de \vrb\phi\ ne dépend d'aucune assignation\footnote{Mais attention, cette notation de $\denote{\vrb\phi}^{\Modele}$ ne devra pas être confondue avec notre ancienne manière de représenter les dénotations au chapitre \ref{LCP}, car maintenant $g$ est un paramètre obligatoire pour noter les dénotations. $\denote{\vrb\phi}^{\Modele}$  est une \emph{généralisation} de la dénotation de \vrb\phi.}.


En utilisant les règles d'interprétation de la définition~\ref{RI2},
on peut montrer facilement que pour qu'un modèle {\Modele} satisfasse à lui seul une formule $\vrb\phi$  (\ie\ $\Modele \satisf \Xlo\phi$),  il
est nécessaire que $\Xlo\phi$ ne contiennent pas de variables
libres, puisque par définition, $\Modele \satisf \Xlo\phi$ vaut pour \emph{toute} assignation%
\footnote{\label{fn:x=x}%
En fait, il  existe quelques formules exceptionnelles qui
  contiennent des variables libres et dont la valeur de
  vérité reste la même quelle que soit l'assignation prise en compte ;
  le meilleur exemple est la formule : $\Xlo x=x$.}.  


Ainsi grâce aux assignations, nous pouvons également nous faire une idée du pendant sémantique des notions de variables libres et variables liées\footnote{Attention, il ne s'agit pas là de définitions, mais simplement de caractéristiques sémantiques de ce que sont les variables libres et les variables liées.}\is{variable!liee@{\elid} liée}\is{variable!{\elid} libre}. 
Si \vrb x apparaît comme variable libre dans \vrb\phi, alors $\denote{\vrb\phi}^{\Modele,g}$ dépend de la valeur que $g$ assigne à \vrb x (sauf cas particuliers, cf. note~\ref{fn:x=x}).
Inversement, si \vrb\phi\ ne contient que des variables liées, alors la valeur de $\denote{\vrb\phi}^{\Modele,g}$ ne dépend pas de $g$. De même, si $\Xlo\ell$ est un lieur\is{lieur} de {\LO}, alors quand \(\denote{\Xlo\phi}^{\Modele,g}\) dépend de la valeur que $g$ assigne à $\vrb x$, \(\denote{\Xlo\ell x\, \phi}^{\Modele,g}\) ne
dépend pas de cette valeur. 


Enfin, à partir de la définition~\ref{d:Tarski}, nous pouvons
redéfinir la notion de conséquence logique, d'une manière plus
générale.  

\begin{defi}[Conséquence logique]
\({\Xlo\phi_1, \phi_2,\dots, \phi_n} \satisf \Xlo\psi\) ssi pour \emph{tout} modèle
{\Modele} et pour \emph{toute} assignation $g$ tels que $\Modele,g \satisf
\Xlo\phi_1$, $\Modele,g \satisf \Xlo\phi_2$,... et $\Modele,g\satisf \Xlo\phi_n$,
on a $\Modele,g \satisf \Xlo\psi$.   On dira que $\Xlo\psi$ est une
conséquence logique de l'ensemble de formules \set{\xlo{\phi_1};
  \xlo{\phi_2};\dots;\xlo{\phi_n}}. 
\end{defi}

\subsection{Sur le statut des fonctions d'assignation}
%-----------------------------------------------------
\label{ss:statutg}
\is{fonction!\elid\ d'assignation}

Les fonctions d'assignation sont au c\oe ur  de la théorie.  Elles
constituent un appareil formel et technique qui permet d'effectuer des
calculs de type quantificationnel (des sortes de dénombrements) afin
d'interpréter des variables et les formules qui les contiennent.
Précisément, elles
permettent de déplacer (d'exporter) les mécanismes quantificationnels
hors du langage objet (où les quantifications ne sont que
représentées, par $\Xlo\exists$ et $\Xlo\forall$) vers le métalangage et le
modèle (où les
quantifications sont alors interprétées) : grâce aux assignations, ce
qui est dit des variables dans {\LO} est transposé, à l'extérieur de
{\LO}, aux valeurs  que peuvent prendre ces variables.  Les
assignations sont donc un outil très précieux et
efficace pour la sémantique. %(on pourrait même aller jusqu'à dire : superbement astucieux).
Mais
au-delà de ce rôle utilitaire, nous pouvons nous demander si ces objets, ces
fonctions d'assignations de valeurs aux variables, que nous manipulons
dans nos calculs sémantiques, correspondent à (ou formalisent) quelque
chose d'un peu plus «tangible» ou d'identifiable dans la réalité
extralinguistique ; ou bien s'il s'agit d'un simple artifice d'aide au
calcul exigé seulement pour les besoins de la théorie.


\largerpage

Rappelons d'abord que les fonctions d'assignations \emph{ne font pas partie}
du modèle : un modèle décrit un état du monde, et une distribution
particulière de valeurs à des variables n'a rien à voir avec
«comment est le monde».  C'est ce que nous avions vu avec les
exemples \ref{x:qqbaille} et \ref{x:ellebaille},
p.~\pageref{x:ellebaille}.   Nous pouvons déjà nous accorder l'opinion que les
assignations ne sont guère objectives, et même très arbitraires.
Lorsqu'une variable représente réellement un pronom,\is{pronom|(} comme en
\ref{x:ellebaille},  les assignations sont, d'une certaine manière,
chargées de nous dire à qui ou à quoi réfère le pronom, par exemple
qui est \sicut{elle} en \ref{x:ellebaille} :


\ex.[\ref{x:ellebaille}]
Elle bâille.\\
\(\Xlo\prd{bâiller}(x)\)


%
Or cette information, qui est \sicut{elle},  n'est contenue ni
dans  la phrase (et sa sémantique), ni dans le modèle : c'est
uniquement l'affaire du locuteur, qui pense à un individu particulier
en disant \sicut{elle}, ou qui a choisi d'appeler  \sicut{elle} un
individu particulier au sujet duquel il souhaite dire quelque chose.
À la rigueur, nous pouvons considérer que c'est  également l'affaire de
l'allocutaire qui, face à \ref{x:ellebaille}, essaie de retrouver
quel individu le locuteur a décidé de désigner par \sicut{elle}.  Mais
il est toujours question d'une certaine disposition mentale du
locuteur, une sorte de «casting» des variables (ou des pronoms)
dont il est le directeur.  C'est pourquoi les fonctions d'assignation
peuvent être vues comme des éléments de formalisation (de certaines
dimensions) des \emph{états cognitifs} du locuteur.  Bien sûr, une
assignation ne suffirait pas à elle seule à modéliser complètement les
connaissances, l'attention ou la conscience d'un locuteur, mais elle
y participe très certainement. 

Il y a un autre angle d'attaque pour concevoir les assignations, qui
n'est pas sans rapport avec le précédent.  Une formule, comme
\ref{x:ellebaille}, qui contient une ou plusieurs variable(s)
libre(s) ne peut être ni vraie ni fausse si on l'interprète
uniquement relativement à un modèle.  Par conséquent, nous avons
établi que la dénotation d'une expression correspond à sa valeur
sémantique par rapport à un modèle et une assignation.  Cependant, si
on ne tient compte que d'un modèle, on peut malgré tout calculer une
valeur sémantique pour l'expression.  Et gardons à l'esprit que dans
une conversation, l'enjeu n'est pas de juger de la vérité ou de la
fausseté de ce que nous dit un locuteur : au contraire, la règle du jeu
de la conversation consiste à accepter (au moins provisoirement) comme
vrai ce que l'on nous dit et, de là, à «affiner» un modèle que
nous supposons\footnote{Nous reviendrons sur cette question d'affinage
du modèle dans le chapitre~\ref{Ch:t+m}.}.  Et donc nous ne connaissons pas
non plus par avance l'assignation que le locuteur à en tête : nous
cherchons justement à la reconstituer.  Ainsi, si nous supposons $\Xlo\phi$
vraie dans un modèle {\Modele}, nous pouvons donner une valeur à
$\denote{\Xlo\phi}^{\Modele}$, même (et surtout) si $\Xlo\phi$ contient des
variables libres : $\denote{\Xlo\phi}^{\Modele}$ est l'ensemble de
toutes les fonctions $g$ telles que $\denote{\Xlo\phi}^{\Modele,g}=1$.
C'est la valeur sémantique de $\Xlo\phi$ relativement à {\Modele}, mais ce
n'est pas sa dénotation.  Et qu'ont en commun toutes les assignations,
par exemple, de $\denote{\Xlo\prd{bâiller}(x)}^{\Modele}$ ?  Simplement le
fait qu'elles assignent toutes à $\vrb x$ un individu qui bâille dans le
modèle.  Nous ne savons pas encore qui est exactement cet individu,
mais nous savons que \ref{x:ellebaille} est \emph{pertinente} (\ie\
vraie et conforme à l'état mental du locuteur) par rapport à de telles
assignations.  Nous avons donc là une condition \emph{contextuelle}
qui nous dit sous quelles hypothèses il est convenable d'interpréter
\ref{x:ellebaille} (pour à terme pouvoir efficacement affiner le
modèle).  Pour dire les choses autrement, nous avons une condition qui
nous dit : «attention, avec \ref{x:ellebaille}, il y a quelqu'un
derrière $\vrb x$, et ce quelqu'un n'est pas n'importe qui».  Comme cette
condition est un préalable, elle peut se voir comme définissant un certain
type de \kw{contexte}.  Il ne s'agit pas à proprement parler de ce que
l'on appelle le contexte (ou la situation) d'énonciation (encore que
ce ne soit pas complètement sans rapport, cf. \alien{infra}) ; on parle
plutôt de 
\emph{contexte linguistique}\footnote{Toute la sémantique dynamique\is{semantique@sémantique!\elid\ dynamique} se
fonde sur cette idée ; cf.\ entre autres
\citet{GroeSto:91}\Andexn{Groenendijk, J.}\Andexn{Stokhof, M.}.  Nous
n'aborderons pas explicitement la sémantique dynamique dans cet
ouvrage, mais nous venons de l'effleurer ici.} lorsque l'on cherche à
«incarner» les assignations.  

\newpage

Nous reviendrons plus précisément
sur les contextes dans le chapitre~\ref{Ch:contexte} (vol.~2), mais nous
pouvons déjà rattacher cette notion à un phénomène linguistique bien
connu.  En effet, les pronoms de la langue ont normalement un emploi
soit \kwi{anaphorique}{anaphore} soit \kwa{déictique}{deictique}.  Les pronoms
anaphoriques ont cette 
propriété d'être interprétables en relation avec une autre expression
qui est intervenue en général
plus tôt\footnote{Lorsque l'expression
intervient après le pronom, on dit que celui-ci a un emploi
cataphorique ; exemple : \sicut{Si elle$_1$ remporte ce tournoi,
  Amélie$_1$ deviendra \No1 mondiale}.  On considérera que dans ces
cas là, la structure syntaxique produit un ordre hiérarchique où
l'antécédent domine d'une certaine manière le pronom. Cela ne résout
pas forcément tous les cas de cataphores, mais nous nous en tiendrons
à cette hypothèse dans le cadre de cet ouvrage.}  dans le discours, autrement dit dans le
contexte.  Cette expression est l'\kwo{antécédent}\is{antecedent@antécédent!\elid\ d'un pronom} du pronom et la
relation qui les unit est une relation de \kwa{coréférence}{coreference},
c'est-à-dire d'identité de dénotation.  Par conséquent, pour
interpréter un pronom anaphorique, il faut savoir quels sont les
antécédents disponibles dans le contexte du discours, en les gardant
d'une certaine manière en mémoire.  Les fonctions d'assignation
(ou des ensembles de fonctions d'assignation) peuvent jouer ce rôle :
si un pronom est représenté par la variable $\vrb x$, on pourra considérer
qu'il est pertinent, ou approprié, de l'interpréter par rapport à une
assignation pas complètement quelconque, mais une assignation dont on
sait déjà qu'elle assigne à $\vrb x$ une valeur valable comme antécédent
pour le pronom.  À cet égard les pronoms déictiques ne sont pas si
différents des pronoms anaphoriques : ce qui les distingue
essentiellement c'est que leur référence (dénotation) n'est pas donnée
par le discours qui précède (via un antécédent), mais par le contexte
d'énonciation, à savoir la situation dans laquelle a lieu l'acte de
communication.  Dans un telle situation il peut y avoir des individus
que l'on désignera par les pronoms \sicut{il} ou \sicut{elle}.  Il est
parfois considéré que lorsqu'il ne semble pas y avoir d'antécédents
satisfaisant dans le contexte linguistique, la fonction d'assignation
va alors «piocher»   des valeurs pour les variables pronominales
dans le contexte d'énonciation, \ie\ parmi les individus présents physiquement.  
\is{pronom|)}



\subsection{Restriction du domaine de quantification}
%-------------------------------------
\is{domaine!\elid\ de quantification}
\is{restriction!\elid\ du domaine de quantification}
\label{ss:RestrDQuant}


Les règles d'interprétation des quantifications (\RSem\ref{RIQg2})
«regardent» tout le domaine d'individus \Unv{A} du modèle,
c'est-à-dire la population entière du monde décrit par le modèle.
Mais en français, lorsque l'on emploie des expressions qui s'analysent
par une quantification (des déterminants) comme \sicut{tous les} ou
\sicut{un}, on ne se base pas sur ce domaine en entier, on le
restreint implicitement.  Cela semble tout à fait évident et
ordinaire, mais il faut savoir ce que le système d'interprétation de
{\LO} a à dire sur ce phénomène.



Imaginons par exemple une conversation où il est fait le récit d'un
dîner qui s'est déroulé quelques jours auparavant, on raconte qu'il y
a eu de la tarte à la rhubarbe au dessert, et quelqu'un ajoute alors%
\footnote{Cet exemple est emprunté à \citet{vFintel:94}.\Andexn{von Fintel, K.}} :  %% ou C. Roberts

\ex. \label{x:urtic}
Et tout le monde a fait une crise d'urticaire.


Le locuteur qui énonce \ref{x:urtic} n'est pas en train de signaler
une  pandémie mondiale ;  il ne parle que des convives du dîner en question.
Or si on considère que \sicut{tout le monde} correspond à une
quantification universelle restreinte aux êtres humains, alors
\ref{x:urtic} se traduit par \ref{x:urticf1} (sans entrer dans le
détail de l'analyse du prédicat verbal) :

\ex. \label{x:urticf1}
\(\Xlo\forall x [\prd{humain}(x) \implq \prd{crise.urtic}(x)]\)


Les conditions de vérité de \ref{x:urticf1} sont que tout individu
du modèle (donc du monde) qui est humain a fait une crise
d'urticaire.  Par conséquent, si \ref{x:urticf1} était exactement
le sens de la phrase \ref{x:urtic}, alors le locuteur qui l'a
prononcée dirait quelque chose de certainement faux dans le contexte
que nous imaginons ici.  Ce n'est évidemment pas ce qui se passe en
réalité.  Pourtant en \ref{x:urtic} il s'agit grammaticalement du
même \sicut{tout le monde} que dans \ref{x:meurt1j}, où là il est bien
question de tous les individus du modèle qui sont dans la dénotation
de \prd{humain}.

\ex. \label{x:meurt1j}
Tout le monde meurt un jour.



Mais les restrictions implicites du domaine de quantification sont
plutôt extrêmement courantes dans l'usage%
\footnote{Il est possible que toutes les expressions de la
  quantification en français n'exigent pas le même type de
  restriction.  Ainsi il semble plus naturel de dire \sicut{tout
  homme est mortel} pour énoncer une grande généralité (\ie\ non
  restreinte) que de dire \sicut{chaque homme est mortel} --~\sicut{chaque} donnant l'impression de devoir quantifier sur un
  ensemble d'hommes bien particulier.}.  C'est ce que montrent encore les
  exemples suivants :

\ex.
\a. 
Tous les étudiants ont eu la moyenne. \label{x:Amoy}
\b.
Un étudiant a eu la moyenne. \label{x:Emoy}


Sans restreindre la quantification à un ensemble particulier (et
certainement petit) d'étudiants, \ref{x:Amoy} sera trivialement
fausse par rapport à tout modèle réaliste et normal (car dans le monde
on trouvera toujours un étudiant qui n'a pas la moyenne) et
\ref{x:Emoy} sera trivialement vraie (car on en trouvera toujours un
qui a eu la moyenne).  Autrement dit, si l'on prend au pied de la
lettre les quantifications de ces exemples, on obtient de mauvaises
conditions de vérité.

On pourrait suggérer que des phrases comme celles-ci «travaillent»
en fait sur des mini-modèles comme les modèles-jouets vus précédemment
pour illustrer les règles d'interprétation de {\LO}.  Mais de tels
modèles ne sont pas vraiment 
réalistes ; nous les employions à des fins purement pédagogiques.  Dans un
cadre de communication standard (et réaliste) on tient compte d'un
modèle qui est supposé décrire un état du monde entier.  
L'exemple suivant illustre ce fait (l'exemple
est de D.~Westerst{\aa}hl, cité par \citet{vFintel:94}) :
\ia{Westerst{\aa}hl, Dag}


\ex. \label{x:Westerst}
La Suède est un pays étonnant. Tous les joueurs de tennis ressemblent
à Bj\"orn Borg, et il y a plus d'hommes qui jouent au tennis que de
femmes.  Bien sûr, les hommes comme les femmes détestent les joueurs
de tennis étrangers.

\largerpage[-1]

Sans entrer dans les détails de l'analyse sémantique de cet exemple,
nous pouvons cependant constater que si nous décidions d'interpréter cette
suite de phrases par rapport à un sous-modèle qui ne contient dans son
domaine que des habitants de la Suède, cela permettrait d'avoir les
bonnes conditions de vérité allant avec \sicut{tous les joueurs de
tennis}, \sicut{les hommes}, \sicut{les femmes}, mais ensuite nous
serions alors bloqués, car dans ce modèle il n'y aurait pas de
dénotation pour le {\GN} \sicut{les joueurs de tennis étrangers} puisque
son domaine ne comprend que des suédois.

La restriction implicite qui s'ajoute à une quantification est en fait
dépendante du \emph{contexte}\is{contexte} de la conversation ou, si l'on veut,
elle est sous-entendue par le locuteur.  C'est pourquoi une telle
restriction n'est pas codée par le modèle, car un modèle décrit l'état
du monde et un locuteur ne décide pas de comment est le monde.  Par
contre il décide de ce dont il veut parler, de ce à quoi il fait
référence.  C'est comme si par exemple les phrases \ref{x:Amoy} et
\ref{x:Emoy} contenaient une sorte d'expression anaphorique ou
déictique invisible mais plus ou moins équivalente à ce qu'on retrouve
dans \sicut{tous les étudiants \emph{du groupe en question}}, \sicut{un
étudiant \emph{du groupe en question}}.  Et cela tend à montrer que
la restriction est non seulement contextuelle mais aussi \emph{locale}
à la phrase qui contient la quantification.  À tout moment dans la
conversation (y compris à l'intérieur d'une même phrase), le locuteur
peut modifier la restriction (la réduire ou l'étendre) ; c'est ce qui
se passe dans la dernière phrase de  \ref{x:Westerst}.

Une manière de sauver les conditions de vérité d'une phrase quantifiée
usuelle consiste donc à faire intervenir un prédicat supplémentaire
dans la représentation sémantique de la phrase en {\LO}.  Ce prédicat
traduit la condition de restriction qui est «invisible», \ie\
implicite dans la phrase.  
%Par conséquent le prédicat à ajouter 
%Il faut en tenir compte.
Pour l'instant, nous n'avons pas encore les moyens techniques de
traduire cela  exactement dans {\LO}, mais l'idée est la suivante : on
ajoute dans la formule un «pseudo-prédicat» ou un prédicat
«anonyme» au côté du prédicat nominal  qui correspond au {\GN}
quantifié de la phrase.  Appelons par exemple $C$ ce pseudo-prédicat,
les phrases \ref{x:urtic}, \ref{x:Amoy} et
\ref{x:Emoy} seront alors correctement traduites respectivement par :

\ex. %[\ref{x:urtic}$''$]
\a. \(\Xlo\forall x [[\prd{humain}(x) \wedge C(x)] \implq \prd{crise.urtic}(x)]\)
% \ex. 
\b. \label{x:Amoyf1}%[\ref{x:Amoy}$'$]
\(\Xlo\forall x [[\prd{étudiant}(x) \wedge C(x)] \implq \prd{avoir.moyenne}(x)]\)
%\ex. 
%[\ref{x:Emoy}$'$]
\b. \(\Xlo\exists x [[\prd{étudiant}(x) \wedge C(x)] \wedge \prd{avoir.moyenne}(x)]\)


De manière générale, on ne sait pas forcément ce que dénote exactement
ce prédicat $\vrb C$, mais on sait que c'est un prédicat à une place et
qu'il dénote donc un certain ensemble d'individus.  Et c'est
précisément ce dont nous avons besoin : un sous-ensemble de \Unv{A} auquel se
restreint la quantification.  
Les conditions de vérité de \ref{x:Amoyf1} par exemple sont : pour
tout individu qui est un étudiant \emph{et} qui fait partie de
l'ensemble $\vrb C$ (ou plus exactement de l'ensemble
\(\denote{\vrb C}^{\Modele,g}\)) cet individu a eu la moyenne.
En fait $\vrb C$ est une \emph{variable de prédicat} --~ce qui n'existe pas
pour le moment dans {\LO} tel que nous l'avons défini%
\footnote{Mais ce n'est pas une hérésie formelle : nous verrons au
chapitre~\ref{ch:types} comment introduire très rigoureusement dans 
{\LO} toute sorte de variables, dont des variables de prédicat.},
mais disons simplement que $\vrb C$ est aux ensembles d'individus ce que
les pronoms sont aux individus.  C'est pour cela que j'ai parlé de
prédicat anonyme, et cela rend également compte de la dépendance
contextuelle de la restriction.  Comme $\vrb C$ est une variable, elle est
en fait interprétée par une fonction d'assignation $g$, qui encode des
informations du contexte.   Ainsi  \ref{x:urtic} s'interprétera le
plus naturellement si on l'évalue par rapport à une fonction  $g$
qui assigne à $\vrb C$ l'ensemble des convives  du dîner dont on parle dans
la  conversation.
De même pour \ref{x:Amoy} et \ref{x:Emoy} qui seront normalement
évaluées par rapport à une assignation qui fait de $\vrb C$ 
la classe d'étudiants dont parle le locuteur\footnote{Remarque : un
  autre traitement formel envisageable de la restriction serait de
  l'implémenter directement au niveau des assignations, sans faire
intervenir de prédicat $\vrb C$ dans la traduction en {\LO}.  Il suffirait
simplement de dire qu'une phrase doit normalement s'interpréter par
rapport à des fonctions $g$ qui ne projettent pas les variables sur
tout \Unv{A} mais seulement un sous-ensemble pertinent pour le
contexte.  Cela reviendrait à peu de chose près à la solution
présentée \alien{supra} sauf qu'il serait techniquement moins aisé de
rendre compte des changements locaux et multiples de la restrictions
comme en \ref{x:Westerst}.}.

%\newpage

\section{Groupes nominaux et portées}
%====================================
\label{s:GNportée}
\is{portee@portée|(}
\is{groupe nominal|sqq}

\subsection{La notion de portée dans la langue}
%----------------------------------------------
%\label{s:GNportée}

%\subsubsection{Quelques définitions}
%'''''''''''''''''''''''''''''''''''

Nous avons vu au chapitre précédent que, par nature, un quantificateur
possède une \kwa{portée}{portee} ; c'est une propriété structurelle,
c'est-à-dire 
syntaxique, de {\LO}.  Rappelons (définition~\ref{d:portee},
p.~\pageref{d:portee}) que la portée d'un quantificateur
$\Xlo\exists x$ ou $\Xlo\forall x$ dans une formule est la sous-formule qui
suit immédiatement le quantificateur ; c'est-à-dire $\vrb\phi$ dans
$\Xlo\exists x \phi$ ou $\Xlo\forall x \phi$.

Mais la portée est aussi (et avant tout) un phénomène interprétatif,
dont a déjà pu se faire une idée dans les pages qui précèdent.  Nous
allons ici examiner de près ce phénomène, pour voir comment au juste
il se manifeste dans les données linguistiques, c'est-à-dire dans des
phrases du français. En effet, la démarche de l'analyse sémantique
n'est pas exactement de se dire qu'un {\GN} se reflète par un
quantificateur dans {\LO}, qu'un quantificateur a une
portée dans {\LO}, et donc qu'un {\GN} a une portée.  Au contraire, il
s'agit plutôt de constater que sémantiquement 
tel {\GN} induit ou s'accompagne d'un phénomène de portée, et d'en
conclure alors que ce {\GN} doit (probablement) se traduire par une
quantification dans {\LO}.   Il s'agit donc d'abord de voir comment
détecter des portées dans des phrases.

%\fixme{***} 
Cela va nous amener à constater que la notion de portée dite
logique que nous avons vue ne coïncide pas exactement avec ce
que l'on appelle la portée d'un constituant dans une langue comme le
français ; cependant les deux sont corrélées et nous essaierons de voir comment.
De plus, les observations que nous allons faire vont finir par nous amener, en sous-texte, à soulever la question (non triviale) de ce que dénote au juste un {\GN}.

\subsubsection{Portées des {\GN} et covariation} %: le phénomène}
%'''''''''''''''''''''''''''''
\is{portee@portée}

\sloppy
Nous savons maintenant, au moins intuitivement, que la traduction de
{\GN} construits avec certains déterminants du français fait
intervenir des quantificateurs : ainsi \sicut{tout(e) $N$}, \sicut{chaque $N$},
\sicut{tou(te)s les $N$} et dans une certaine mesure \sicut{les $N$}
correspondent à une quantification universelle ($\forall$), et
l'indéfini\is{indefini@indéfini} \sicut{un(e) $N$} correspond à une quantification
existentielle ($\exists$)\footnote{L'indéfini pluriel \sicut{des $N$}
  peut aussi, dans certaines conditions, se traduire par une
  quantification existentielle.}.
Regardons ainsi une phrase qui combine ces deux types de déterminants :

%Regardons comment peut-on traduire en {\LO} le plus naturellement et
%le plus intuitivement possible les phrases suivantes:
\fussy

\ex. \label{x:sdb}
%\(\left\{\begin{array}{l}\text{}\\ \text{Tous les  appartements possèdent}\end{array}\right\}\) 
Chaque appartement possède une salle de bain. 


Le phénomène sémantique à observer ici est que, le {\GN} indéfini
\sicut{une salle de bain} a beau être singulier, on constate que cette
phrase ne fait pas référence à \emph{une} salle de bain.  En fait en
\ref{x:sdb}, il est question d'autant de salles de bain qu'il y a
d'appartements ; c'est donc un effet de \emph{multiplication}
dénotationnel qui est à l'\oe uvre ici : le {\GN} existentiel
\sicut{une salle de bain} (ou plus exactement sa dénotation) semble
être multiplié par le {\GN} universel \sicut{chaque appartement}.  
Cela peut s'illustrer facilement : il suffit d'imaginer des modèles par
rapport auxquels \ref{x:sdb} est vraie.  Par exemple, un modèle
\tuple{\Unv{A}, \FI} où \Unv{A} contient, disons, 138 objets qui sont
des appartements et 138 autres objets qui sont des salles de bains et tels
que chacune des salles de bains se trouve dans un des appartements
(l'important pour que \ref{x:sdb} soit vraie, c'est qu'il n'y ait
pas dans \Unv{A} d'appartement sans salle de bain, et donc il y aura
au moins autant de salles de bain que d'appartements).  

On dit également, pour exprimer cette même idée,
que le {\GN}   \sicut{une salle de bain} \kwi{covarie}{covariation}
avec le {\GN} 
\sicut{chaque appartement}.  
En effet \sicut{chaque appartement} correspond à une quantification
universelle et provoque donc une variation (\ie\ la variation de valeurs
assignées à une variable) et \sicut{une salle de bain} se retrouve comme
entraîné ou embarqué par la variation due au premier {\GN}. 
Cette \kw{covariation}, ou cet effet de
multiplication, est précisément  une manifestation interprétative du
phénomène de 
portée des {\GN} : si le second {\GN} covarie avec le premier {\GN}
c'est que le second se trouve \emph{dans la portée} du premier.  Et
bien sûr, cela apparaît clairement dans la traduction de \ref{x:sdb}
en {\LO} :

\ex. \label{x:sdb'}
\(\Xlo\forall x [\prd{appart}(x) \implq \exists y [\prd{sdb}(y) \wedge \prd{possède}(x,y)]]\)

\sloppy

Dans cette formule \ref{x:sdb'}, la portée de $\Xlo\forall x$ est
$\Xlo[\prd{appart}(x) 
\implq \exists y [\prd{sdb}(y) \wedge \prd{possède}(x,y)]]$ où l'on
retrouve bien le quantificateur existentiel $\Xlo\exists y$ qui provient 
du {\GN} objet de \ref{x:sdb}.  Et le phénomène de multiplication
dénotationnelle observée en \ref{x:sdb} est proprement prédit par les règles d'interprétation de
{\LO}, en l'occurrence (\RSem\ref{RIQg}), si on les applique à
\ref{x:sdb'}.  On s'en est déjà rendu compte avec les exemples
\ref{x:Qex3} et \ref{x:Qex4} vus en \S\ref{ss:IFQ} et les
figures~\ref{fig:EA} et \ref{fig:AE} (pp.~\pageref{fig:EA} et
\pageref{fig:AE}). \ref{x:sdb'} est de la forme \(\Xlo\forall x [\phi
\implq \exists y \psi]\) et on doit interpréter cette formule par
rapport à une assignation quelconque $g$ ; et pour ce faire, la règle
(\RSem\ref{RIQg}) nous demande d'interpréter $\Xlo[\phi \implq \exists y
\psi]$ pour toutes les variantes en $x$ de $g$, par conséquent on
devra ensuite recommencer l'interprétation de $\Xlo\exists y \psi$ autant
de fois qu'il y a de ces variantes ; et comme l'interprétation de
$\Xlo\exists y \psi$ consiste à trouver une variante en $y$ de
l'assignation courante, on obtient bien l'effet de multiplication attendu
sur la valeur de $y$ (assignée par les variantes successives de $g$). 

\fussy

\largerpage[-2]

Il est important de remarquer ici que nous faisons un double usage du
terme (et de la notion) de \kwa{portée}{portee}. Nous avons parlé de
la portée d'un 
{\GN} et ce faisant nous nous situions par rapport au français (ou
tout autre langue naturelle) ; et nous avons également parlé de la
portée d'un quantificateur, en nous situant alors dans {\LO}.
Techniquement, ce n'est pas exactement la même chose ; simplement la
notion formelle de portée définie dans {\LO} \emph{explique} le
phénomène sémantique de portée observé dans la langue.  Et d'ailleurs,
un quantificateur de {\LO}, c'est-à-dire une séquence $\Xlo\forall x$ ou
$\Xlo\exists y$, ne traduit pas complètement la contribution d'un {\GN}
puisqu'il faut évidemment tenir compte du matériau lexical (nominal)
du {\GN} qui correspond à des prédicats (ici \prd{appart} et \prd{sdb}
pour \sicut{appartement} et \sicut{salle de bain} respectivement)% 
\footnote{La traduction compositionnelle et méthodique des {\GN} dans
{\LO} est d'ailleurs une tâche qui est loin d'être triviale ; nous y
reviendrons précisément dans le chapitre~\ref{ch:types}.}.

Pour dire les choses encore autrement, nous constatons que certains
groupes nominaux induisent de la covariation, et par analogie avec
les quantificateurs de {\LO}, nous en concluons qu'ils ont une portée
et donc qu'il peut être raisonnable de les analyser au moyen des
quantifications logiques définies en (\RSem\ref{RIQg}).


\subsubsection{Portée large et portée étroite}
%'''''''''''''''''''''''''''''''''''''''''''''
\label{ss:PorteeLE}
Ce parallélisme observé entre les {\GN} de la langue et les
quantificateurs de {\LO} va nous permettre de définir les termes de
\kwo{portée large}\is{portee@portée!\elid\ large} et \kwo{portée étroite}\is{portee@portée!\elid\ etroite@\elid\ étroite}, que nous aurons souvent 
l'occasion d'employer par la suite.  Par la syntaxe de {\LO}, nous
savons repérer (dans une formule donnée) qu'un quantificateur se
trouve dans la portée d'un autre quantificateur.  Par exemple en
\ref{x:sdb'}, $\Xlo\exists y$ est dans la portée de $\Xlo\forall x$.  En
retournant au français, nous en avons déduit que \sicut{une salle de
bain} s'interprète dans la portée de \sicut{chaque appartement}.  Dans
ce cas, nous dirons aussi que le {\GN} \sicut{chaque appartement} a
portée large par rapport (ou sur) le {\GN} \sicut{une salle de bain},
dont on dira qu'il a portée étroite.

\begin{defi}[Portée large {\vs} portée étroite]
On dit que dans une phrase, un {\GN} $\alpha$ s'interprète avec portée
large par rapport à un {\GN} $\beta$, et que $\beta$ s'interprète avec
portée étroite par rapport à $\alpha$, si en traduisant la phrase dans
{\LO}, $\beta$ correspond à un quantificateur situé dans la portée du
quantificateur qui représente $\alpha$.
\\
Par extension on dira qu'un {\GN} a portée étroite lorsqu'il n'y a
aucun autre {\GN} dans sa portée, et qu'un {\GN} a portée large
lorsque tous les autres {\GN} sont dans sa portée.
\end{defi}%

Pour résumer que $\alpha$ a portée large sur $\beta$, on utilise parfois la notation $\alpha > \beta$\footnote{Cette notation ne fait, bien sûr, pas partie de \LO, c'est juste un raccourci paresseux pour indiquer que $\alpha$ a portée large sur $\beta$.}.

\smallskip

À ce stade, il est légitime de se demander jusqu'à quel point  le
parallélisme français--{\LO} est robuste et de voir s'il peut faire
les prédictions attendues au sujet des {\GN} de la langue.  Cela
va nous amener à nous interroger sur les règles (ou les contraintes)
qui déterminent la portée effective d'un {\GN} dans la langue.

Nous avons vu que la portée d'un {\GN}, comme celle d'un
quantificateur de {\LO},  est en quelque sorte la
zone d'influence du {\GN} (ou du quantificateur), en ce sens qu'une
variation induite par un {\GN} (comme un quantificateur universel) se
répercute sur un {\GN} qui en soi ne produit pas de variation (comme
un quantificateur existentiel).  Matériellement, dans une langue comme
le français, cette zone d'influence semble être délimitée d'une
manière qui «imite» ce qu'on observe dans la structure de {\LO} :
en \ref{x:sdb} le {\GN} qui est à droite d'un autre {\GN} se
retrouve dans sa portée, de même qu'en {\LO} la portée d'un
quantificateur est une zone qui se situe sur sa droite.
On pourrait donc en déduire une règle très simple pour interpréter les
{\GN} en français :  tout {\GN} $\alpha$ qui se trouve à droite d'un
{\GN} $\beta$ se trouve dans la portée de $\beta$.

Cependant, comme on s'en doute bien, les choses ne se passent du tout
ainsi dans les faits, comme en témoigne la phrase \ref{x:flica} qui
se traduit naturellement en \ref{x:flicb}.

\ex.  \label{x:flic}
\a. Tous les policiers dépendent d'un ministre. \label{x:flica}
\b. \(\Xlo\exists x [\prd{ministre}(x) \wedge \forall y
  [\prd{policier}(y) \implq \prd{dépendre}(y,x)]]\) \label{x:flicb}


Bien que situé à droite du {\GN} \sicut{tous les policiers}, le {\GN}
\sicut{un ministre} ne covarie pas avec le premier, et c'est pourquoi
\(\Xlo\forall y\) apparaît dans la portée de $\Xlo\exists x$ en
\ref{x:flicb}.   Et on dira donc que \sicut{tous les policiers} est
interprété dans la portée de \sicut{un ministre}.  

Ce que cet exemple fait apparaître, en premier lieu, c'est que la
portée d'un {\GN} n'est pas déterminée par sa position dans la phrase.
C'est particulièrement visible avec \ref{x:tableau} où là (du moins
dans l'interprétation la plus naturelle de la phrase) le {\GN}
complément circonstanciel \sicut{chaque pièce} a portée sur le {\GN}
complément d'objet \sicut{un tableau} (il y a bien covariation de ce
\GN). 

\ex. \label{x:tableau}
Annie a accroché un tableau dans chaque pièce.



Les exemples \ref{x:flic} et \ref{x:tableau} illustrent ce que l'on
nomme le phénomène de \kwo{portée inversée}\is{portee@portée!\elid\ inversée} : un groupe nominal qui se
trouve à droite d'un autre est néanmoins interprété avec portée large
sur ce dernier.



Or nos règles sémantiques (\RSem\ref{RIQg2}) n'expliquent pas
(et pour cause, car ce n'est
pas leur rôle)  pourquoi l'ordre des quantificateurs dans les
traductions de \ref{x:sdb} et \ref{x:flic} est différent, alors que les
structures syntaxiques des deux phrases sont assez similaires.


L'explication (si c'en est une...) est qu'en fait, par défaut, les phrases contenant plusieurs quantificateurs
(différents) sont \kwo{ambiguës}. Le quantificateur qui représente un
{\GN} peut se placer à «divers» endroits de la formule qui traduit
la phrase où il apparaît.
C'est ce qu'illustrent \ref{x:doss1} et
\ref{x:fresk}, qui sont effectivement ambiguës. 

\ex.\label{x:doss1}
\a. 
\(\left.\begin{array}{@{}l@{\,}}\text{Chaque dossier sera examiné}\\ \text{Tous les
  dossiers seront examinés}\end{array}\right\}\) par un
  relecteur.
\b. 
\(\Xlo\forall x [\prd{dossier}(x) \implq \exists y [\prd{relecteur}(y)
    \wedge \prd{examiner}(y,x)]]\)
\c. 
\(\Xlo\exists y [\prd{relecteur}(y) \wedge \forall x [\prd{dossier}(x)
    \implq  \prd{examiner}(y,x)]]\) 

\ex. \label{x:fresk}
\a. (Tous) les élèves ont dessiné une fresque.
\b. \(\Xlo\forall x [\prd{élève}(x) \implq \exists y [\prd{fresque}(y)
    \wedge \prd{dessiner}(x,y)]]\)
\c. 
\(\Xlo\exists y [\prd{fresque}(y) \wedge \forall x [\prd{élève}(x)
    \implq  \prd{dessiner}(x,y)]]\) 

%\ex.
%Cette année, tous les élèves ont lu un roman russe.


Et \ref{x:sdb} et \ref{x:flic} sont également
ambiguës dans leurs structures sémantiques, au moins initialement.
Simplement, dans ces exemples, nous excluons immédiatement, sans y
penser, une des deux lectures, car elle est à chaque fois très décalée
de notre vision habituelle du monde : une même salle de bain pour
plusieurs appartement et un ministre par policier sont des idées qui
nous semblent saugrenues ou inattendues.  Mais ce sont nos
connaissances du monde qui nous font porter ces jugements, et pas la
sémantique (du français) en soi.  Donc linguistiquement, \ref{x:sdb}
et \ref{x:flic} sont très probablement similaires à \ref{x:doss1}
et \ref{x:fresk}, du point de vue de leurs structures.

La conclusion que nous pouvons tirer maintenant mérite d'être posément
explicitée ; elle n'est pas triviale, car elle malmène un peu notre
principe de compositionnalité.  D'abord l'interprétation des {\GN}
n'est pas nécessairement locale, au contraire elle peut se faire
«distance», elle est déplaceable ou extraposable vers la gauche de
la structure sémantique.  Ce phénomène, qui nous est objectivement
signalé par les données linguistiques, devra être intégré dans le
système formel par lequel nous décrivons le fonctionnement de la
langue.  Mais pour l'instant nous n'en avons pas les moyens.  En effet
ce déplacement interprétatif des {\GN} est d'une part optionnel, mais
aussi toujours possible, ce qui cause les ambiguïtés systématiques
observées ci-dessus.  C'est ce qui met en péril le principe de
compositionnalité : nous trouvons des phrases qui {a priori} possèdent
\emph{une} structure syntaxique mais donnent lieu à plusieurs
interprétations ; leur sens n'est donc pas simplement déterminé par le
sens de leurs parties et leurs modes de combinaison.  Il nous faudra
trouver une solution à cette carence du système, et nous y reviendrons
dans le chapitre~\ref{ch:ISS}, \S\ref{ss:iss:Qu}, lorsque nous
aborderons la question de l'interface syntaxe--sémantique.  Pour
l'instant, nous devons nous contenter d'admettre cette «mobilité
sémantique» des {\GN}.

\sloppy 
Terminons cette sous-section en examinant quelques exemples
supplémentaires concernant le phénomène.  D'abord il a été postulé que
l'ambiguïté de portée des {\GN} est systématique (du moins tant qu'on
ne fait pas intervenir de considérations pragmatiques), et elle se
manifeste par un effet de covariation.  Cependant --~on s'en doute
bien~-- on peut trouver des phrases comportant plusieurs {\GN} mais
qui ne font pas apparaître de covariation.  Ce sont les cas où les
{\GN} en question correspondent au même type de quantification, comme
en \ref{x:QAAa} et \ref{x:QEEa}.  

\fussy

\ex. \label{x:QAA}
\a. Tout le monde connaît toutes les réponses.\label{x:QAAa}
\b. \(\Xlo\forall x [\prd{humain}(x) \implq \forall y [\prd{réponse}(y)
    \implq \prd{connaître}(x,y)]]\) \label{x:QAAb}
\c. \(\Xlo\forall y [\prd{réponse}(y) \implq \forall x [\prd{humain}(x)
    \implq \prd{connaître}(x,y)]]\) \label{x:QAAc}
\d. \(\left\{\begin{array}{@{}c@{}}\Xlo\forall x \forall y\\\Xlo\forall y \forall x\end{array}\right\} \Xlo[[\prd{humain}(x) \wedge \prd{réponse}(y)]
    \implq \prd{connaître}(x,y)]\) \label{x:QAAd}

\ex. \label{x:QEE}
\a. Quelqu'un a trouvé une clé.\label{x:QEEa}
\b. \(\Xlo\exists x [\prd{humain}(x) \wedge \exists y [\prd{clé}(y)
    \wedge \prd{trouver}(x,y)]]\) \label{x:QEEb}
\c. \(\Xlo\exists y [\prd{clé}(y) \wedge \exists x [\prd{humain}(x)
    \wedge \prd{trouver}(x,y)]]\) \label{x:QEEc}
\d. \(\left\{\begin{array}{@{}c@{}}\Xlo\exists x \exists
y\\\Xlo\exists y \exists x\end{array}\right\} \Xlo [\prd{humain}(x) \wedge \prd{clé}(y)
    \wedge \prd{trouver}(x,y)]\) \label{x:QEEd}


Dans ces exemples, quel que soit le
{\GN} auquel on attribue une portée large par rapport à l'autre (ce
qui est illustré dans les formules (b) et (c) ci-dessus), la
phrase en français a toujours la même interprétation.  La raison est
que les formules (b) et (c) de  \ref{x:QAA} et \ref{x:QEE} sont
sémantiquement équivalentes, et peuvent d'ailleurs se ramener aussi
aux variantes (d).  Par exemple \ref{x:QAAb} dit qu'un individu $x$,
quel qu'il soit, connaît l'ensemble des réponses, et \ref{x:QAAc} dit
que pour chacune des réponses $y$, celle-ci est connue de l'ensemble
des individus.  À l'arrivée, il est demandé dans les deux cas de
vérifier que \(\Xlo[[\prd{humain}(x) \wedge \prd{réponse}(y)]
    \implq \prd{connaître}(x,y)]\) est vrai pour toutes les valeurs
qu'on peut assigner à $\vrb x$ et à $\vrb y$.  Les conditions de
vérité sont bien 
identiques.  C'est similaire pour \ref{x:QEE}.  On peut déduire alors
que le mécanisme de déplacement interprétatif des {\GN} est
possiblement à l'\oe uvre ici, mais il produit des ambiguïtés
artificielles et invisibles.



Cependant il est important de noter que les ambiguïtés de portée
peuvent néanmoins se faire sentir entre deux {\GN}  qui font
intervenir des quantifications assez similaires.  Ainsi les phrases
qui comportent deux indéfinis pluriels et numéraux comme en
\ref{x:3sx2la}.  Bien sûr, pour le moment, {\LO} ne nous permet pas de
rendre compte précisément de la sémantique des déterminant cardinaux
comme \sicut{deux} et \sicut{trois} ; {\LO} reste assez grossier à cet
égard : il ne sait expliciter directement que la contribution de {\GN}
correspondant à des quantifications existentielles simples (\ie\
singulières) et à des universelles.  Mais les gloses données en
\ref{x:3sx2lb} et \ref{x:3sx2lc} nous livrent des conditions de
vérités qui distinguent deux attributions différentes de portées
larges et étroites :


\ex. \label{x:3sx2l}
\a.
 Trois stagiaires ont appris deux langues étrangères. \label{x:3sx2la}
\b. \textit{Il existe trois individus stagiaires tels que pour chacun
  de ces stagiaires  il
existe deux langues étrangères qu'il a apprises.} \label{x:3sx2lb}
\c. \textit{Il existe deux langues étrangères telles que pour chacune
  de ces langues il existe trois stagiaires qui l'ont apprise.} \label{x:3sx2lc}

En \ref{x:3sx2lb} il est question possiblement de six langues
différentes : \sicut{deux langues étrangères} subit la covariation
de \sicut{trois stagiaires} du fait de sa portée étroite.  En
\ref{x:3sx2lc} on n'évoque que deux langues (et éventuellement jusqu'à
six stagiaires), du fait de l'inversion de portée.  Notons que par
analogie avec le déterminant \sicut{un}, qui est l'indéfini singulier mais
aussi le cardinal unitaire, il s'avère raisonnable d'interpréter
\sicut{deux} et \sicut{trois} par des quantifications existentielles
comme le suggèrent les gloses \ref{x:3sx2lb} et \ref{x:3sx2lc} en
\textit{il existe...}.  Mais on remarquera aussi que ces gloses
révèlent une quantification universelle, via \textit{chacun(e)}.
C'est vraisemblablement un effet du pluriel\footnote{Nous reviendrons
plus précisément sur cette question en étudiant la pluralité au
chapitre~\ref{GN++} (vol.~2).  Indiquons seulement que la difficulté qui se
pose à {\LO} ici est qu'il n'est pas (encore) capable d'interpréter la
relation exprimée par la préposition \sicut{de} dans les parties de
gloses comme : \textit{pour chacun \emph{de} ces stagiaires}.} et aussi
ce qui peut expliquer l'ambiguïté effective de la phrase :  les portées
larges et étroites confrontent ici des quantificateurs existentiels et
un quantificateur
universel «caché».

Enfin, il est important de signaler que les {\GN} et les
quantificateurs ne sont pas les seuls éléments susceptibles d'avoir
une portée et d'entrer ainsi dans la ronde des ambiguïtés observées
ici.  C'est par exemple le cas de la négation. 
\is{negation@négation}\is{portee@portée!\elid\ de la negation@\elid\ de la négation}%
Dans {\LO}, on peut définir la portée de l'opérateur $\Xlo\neg$ : c'est la
(sous-)formule à laquelle il s'adjoint par la règle syntaxique
(\RSyn\ref{SynPNeg}).  Posons donc une définition générique concernant
tout opérateur unaire de {\LO} : 

\begin{defi}[Portée d'un opérateur de \LO]  \label{d:portéeOp}
%'''''''''''
Soit $\Xlo *$ un opérateur de {\LO} que la syntaxe introduit par une règle
de la forme : 
\begin{itemize}
\item[] si $\Xlo\phi$ est une formule, alors $\Xlo *\phi$ est aussi une
formule.  
\end{itemize}
On dira alors que $\Xlo\phi$ est la portée de cette occurrence
de l'opérateur $\Xlo *$ dans la formule plus grande où il s'insère.
\end{defi}

Pour le moment nous ne connaissons que l'opérateur $\Xlo\neg$ qui remplisse ces
conditions,  mais nous en verrons d'autres plus tard ; c'est pourquoi
la définition est formulée sur un mode générique, au moyen du
méta-opérateur $\Xlo *$. 

Puisque $\Xlo\neg$ a une portée, celle-ci peut être large ou étroite
relativement à celle d'un quantificateur (ou d'un autre opérateur).
Une ambiguïté est alors envisageable et elle se répercute
effectivement en français, comme l'illustre \ref{x:neg+Q} où dans
chaque traduction la portée de la négation est soulignée :


\ex. \label{x:neg+Q}
\a.  Marie n'aime pas tous ses camarades. \label{x:neg+Qa}
\b. \(\xlo{\forall x [\prd{camarade}(x,\cns m) \implq \neg\,} 
\underline{\Xlo\prd{aimer}(\cns m, x)}\Xlo]\) \label{x:neg+Qb}
\c. \(\xlo{\neg} \underline{\Xlo\forall x [\prd{camarade}(x,\cns m) \implq
    \prd{aimer}(\cns m, x)]}\) \label{x:neg+Qc}

Selon \ref{x:neg+Qb}, le {\GN} \sicut{tous ses camarades} a porté
large sur la négation.  La phrase dit alors que tous les camarades de
Marie font l'objet de sa «non-affection» : elle n'en aime aucun.
Selon \ref{x:neg+Qc}, le {\GN} a portée étroite, et c'est la
quantification universelle qui est niée par $\Xlo\neg$ : il n'est pas vrai
que Marie les aime tous, donc il  en suffit  d'un qu'elle n'aime pas.


Remarquons qu'il n'est pas indispensable ici de supposer que la
négation s'interprète à distance, comme nous l'avons postulé pour les
{\GN}.  Le déplacement interprétatifs des {\GN} seuls est suffisant
pour rendre compte de l'ambiguïté en \ref{x:neg+Q} : on peut se
contenter de dire que c'est \sicut{tous ses camarades} qui s'est
«déplacé» en \ref{x:neg+Qb} (alors qu'il n'a pas bougé en
\ref{x:neg+Qc}).

La négation n'induit pas en soi une covariation\footnote{À moins d'y
  voir éventuellement une multiplication par zéro.}, simplement sa
  sémantique interagit, ou non, avec celle d'un quantificateur, en
  fonction de leurs portées relatives.  Mais il y a  d'autres
  expressions de la langue qui, sans être  des {\GN} sujets ou
  compléments, interagissent également avec des \GN/quantificateurs et
  qui par elles-mêmes induisent de la covariation.  Ce sont des expressions
«adverbiales» comme celles soulignées en \ref{x:AdvQ} : 

\ex.\label{x:AdvQ}
\a. En vacances, Bill visite \emph{souvent} un musée. \label{x:AdvQa}
\b. Ces jours-ci, Bill visite  un musée \emph{quotidiennement}. \label{x:AdvQb}
\c. \emph{Tous les jours}, Bill visite un musée.\label{x:AdvQc}

Dans ces exemples, l'indéfini singulier \sicut{un musée} peut, sous
une certaine lecture, être multiplié.  C'est la lecture où Bill visite
un musée différent à chaque fois.  Si le {\GN} est multiplié c'est
qu'il covarie avec un autre élément de la phrase.  Et cet élément,
responsable de la variation, c'est justement l'adverbe ou l'expression
adverbiale.  À cet égard, on les nomme habituellement \kwi{adverbes de
quantification}{adverbe de quantification}\label{AdvQ1l}, 
car ils expriment eux aussi une quantification et ils
ont une portée qui peut être large ou étroite par rapport à des
{\GN}.

%\footnote{
Nous reviendrons sur ce genre d'adverbes en
  \S\ref{AdvQ+Gen}. 
Pour l'instant nous pouvons simplement remarquer que
là
%Là 
encore {\LO}, en l'état, manque d'expressivité
  pour traduire précisément les phrases \ref{x:AdvQ}.  Au mieux, nous
  pouvons suggérer les pseudo-formules suivantes, où l'adverbe
  interviendrait sous la forme d'un opérateur unaire :
\ex.
\a.
\(\Xlo\textit{\color{black}Souvent}\ \exists x [\prd{musée}(x) \wedge
  \prd{visiter}(\cns b,x)]\)
\b.
\(\Xlo\exists x [\prd{musée}(x) \wedge \textit{\color{black}Souvent}\ 
  \prd{visiter}(\cns b,x)]\)

Mais attention, il n'agit pas de vraies formules de {\LO}, donc cela
ne peut pas constituer pour l'instant une véritable analyse sémantique
du phénomène (nous ne saurions expliciter leurs conditions de
vérité).  Prenons les seulement comme un moyen de subodorer une
analyse possible.
Et pour une approche plus approfondie de ce genre d'analyse, on peut
se reporter à l'article important de \citet{Lewis:75}\Andex{Lewis, D.}.
%}.







\subsubsection{Limite de portée}
%'''''''''''''''''''''''''''''''
\label{sss:limiteportée}

Les exemples que nous venons de voir, et notamment ceux qui présentent
des ambiguïtés, nous montrent
que les {\GN} sont sémantiquement mobiles en ce sens qu'ils peuvent
s'interpréter «plus à gauche» que là où ils apparaissent dans la
structure de la phrase.  Et j'ai même indiqué que cette mobilité était
systématique, c'est-à-dire que ce déplacement interprétatif à gauche
était toujours possible.  Mais en fait c'est inexact : les données
linguistiques nous montrent clairement qu'il y a des contraintes et
des limites à  la mobilité sémantique des {\GN}/quantificateurs.  Même
si, comme nous allons le voir, ces contraintes semblent relever de la
structure syntaxique de la phrase, il est très important de les
connaître pour l'analyse sémantique, par exemple lorsqu'il s'agit de
caractériser les propriétés sémantiques des différents types de \GN.


Pour mettre en évidence cette limite de portée mobile des {\GN}, le
plus simple est de comparer les deux exemples
\ref{x:femporta} et \ref{x:femportb}.

\ex.\label{x:femport}
 Jean a une femme dans chaque port. \label{x:femporta}


L'interprétation la plus naturelle pour \ref{x:femporta} est aussi la
moins morale : c'est celle où \sicut{une femme} covarie avec
\sicut{chaque port}, c'est-à-dire celle où \sicut{chaque port}
s'interprète avec une portée large.  Quant à l'autre lecture, celle où
\sicut{une femme} a portée large et ne covarie pas, nous l'excluons
immédiatement, pour des raisons pragmatiques : elle implique qu'une
même femme habite dans tous les ports que Jean fréquente, et c'est
assez peu plausible.  Or c'est
précisément pour cette raison que \ref{x:femportb} est bizarre. 

\largerpage

\ex.
\juge{\urgh} Jean a une femme qui vit dans chaque port. \label{x:femportb}

Contrairement à \ref{x:femporta}, \ref{x:femportb} ne présente qu'une
seule lecture, celle où \sicut{chaque port} a portée étroite,
c'est-à-dire où les {\GN} sont interprétés \alien{in situ}.  En
\ref{x:femportb} le déplacement interprétatif n'est donc pas possible,
sinon \ref{x:femportb} serait tout simplement synonyme de
\ref{x:femporta}. 

Ce blocage du déplacement interprétatif de {\GN} en \ref{x:femportb}
et le contraste que cela provoque entre \ref{x:femporta} et
\ref{x:femportb} (qui pourtant sont très proches) s'expliquent par 
%la
une
contrainte% suivante%
\footnote{Cette explication a été amorcée notamment par
  \citet{Rodman:76}\Andexn{Rodman, R.}, puis révisée par
  \citet{Farkas:81}\Andexn{Farkas, D.}.} 
qui dit que les
%%\begin{prop}[Contrainte sur la portée des {\GN} et quantificateurs] \label{pt:Portee}
%%Les 
{\GN} ou \label{ct:Portee}
les quantificateurs ne peuvent pas prendre portée large
au-delà de la proposition syntaxique où ils interviennent en surface.
%%\end{prop}



%%Cette proposition~\ref{pt:Portee} 
Cette contrainte constitue une règle empirique qui
discipline un peu la mobilité de portée des {\GN} et qui fait quelques
bonnes prédictions.  On peut ainsi expliquer pourquoi \ref{x:femportb}
n'a pas la lecture de \ref{x:femporta} : en \ref{x:femportb} le {\GN}
\sicut{chaque port} est situé à l'intérieur de la proposition relative
\sicut{qui vit dans chaque port} et la contrainte ci-dessus exclut que
sa portée s'étende à l'extérieur de cette proposition, donc il ne peut
pas avoir portée large sur \sicut{une femme}.

La portée des {\GN} et des quantificateurs semble donc enfermée dans
les propositions syntaxiques ; mais bien sûr au sein même d'une
proposition, la portée des {\GN} reste variable.  C'est ce que montre
par exemple \ref{x:QinP} : s'il n'est question que d'une seule femme,
celle-ci connaît cependant plusieurs banquiers, car \sicut{chaque
  capitale} s'interprète avec 
portée large sur (c'est-à-dire «à gauche» de) \sicut{un banquier}
à l'intérieur de la relative.


\ex. \label{x:QinP}
 Jean a une femme qui connaît un banquier dans chaque capitale.


En \ref{x:femportb} et \ref{x:QinP}, la proposition syntaxique qui
circonscrit la portée des {\GN} est une relative ; mais la
contrainte s'applique pour tout type de proposition, comme par exemple
une subordonnée complétive.  C'est ce qu'illustre cet exemple de 
\citet{Farkas:81}\Andex{Farkas, D.} :

\ex. 
\juge{\urgh} Jean a raconté à un journaliste [que Pierre habite dans chaque ville de
France]. \label{x:Farkas}

Là encore la phrase nous frappe par son absurdité, où du moins parce
qu'elle prête à Jean un discours absurde ou fumeux.  C'est  que
nous l'interprétons forcément avec une portée large de \sicut{un
  journaliste} sur \sicut{chaque ville de France}.  
Si ce dernier {\GN} pouvait étendre sa portée hors de la complétive
indiquée entre crochets et prendre ainsi portée  large sur \sicut{un
  journaliste}, celui-ci pourrait  covarier, et on aurait alors une
lecture un peu plus cohérente,  équivalente à :

\ex.  Pour chaque ville de France, Jean a raconté à un journaliste que
Pierre habite dans cette ville. 

Mais en aucun cas \ref{x:Farkas} ne peut s'interpréter de la sorte.
La portée de \sicut{chaque ville de France} ne peut pas «sortir»
de la complétive. 

Cette limitation de portée peut également s'illustrer dans les
subordonnées conditionnelles, introduites par \sicut{si}, comme  par
exemple en \ref{x:QinSi}.



\ex. \label{x:QinSi}
Si tous les étudiants sont recalés, le prof sera renvoyé.


Le principe est toujours le même : nous comprenons la phrase d'une
seule manière, et l'interprétation putative où le {\GN} (ici
\sicut{tous les étudiants}) aurait une portée large, hors de sa
proposition syntaxique, n'est pas disponible.  \ref{x:QinSi} signifie
simplement que le prof sera renvoyé si \emph{la totalité} des
étudiants se trouvent recalés.  Cette interprétation se traduit comme
il se doit en {\LO} par la formule \ref{x:QinSiO} où la quantification
universelle de \sicut{tous les étudiants} est bien localisée dans
l'antécédent de la conditionnelle, c'est-à-dire la sous-formule qui
correspond à la proposition subordonnée en \sicut{si} (pour faciliter la
lecture j'ai marqué par $\stackrel{\sicut{si}}{\implq}$ l'implication
qui traduit la structure conditionnelle, à ne pas confondre avec
celle qui accompagne la quantification universelle) :

\ex. \label{x:QinSiO}
 \(\Xlo[\forall x [\prd{étudiant}(x) \implq \prd{recalé}(x)] 
\stackrel{\sicut{si}}{\implq}  \prd{renvoyé}(\cns p)]\)


Ce qui est intéressant avec un tel exemple c'est que nous pouvons
faire apparaître la portée large putative  du {\GN}
universel sans avoir besoin de le confronter à un {\GN} 
existentiel ailleurs dans la phrase.  En effet si \sicut{tous les
  étudiants} avait une portée très large débordant de la subordonnée
conditionnelle, cela voudrait dire que
l'implication\is{implication!\elid\ matérielle}
$\stackrel{\sicut{si}}{\implq}$ 
se trouverait dans la portée de la
quantification universelle ; mais nous obtiendrions alors une interprétation
nettement différente, représentée en
\ref{x:QinSiN}\footnote{L'explication de ce phénomène est assez
simple.  Nous avons vu que la portée d'un \GN/quantificateur par
rapport à une négation est significative ; or l'implication contient
logiquement une négation, $\Xlo[\phi \implq \psi]$ équivaut à $\Xlo[\neg\phi
  \vee \psi]$. Donc si l'implication est dans la portée d'un
quantificateur, cela veut dire que ce dernier a portée sur $\Xlo\neg\phi$ ;
inversement s'il est localisé dans l'antécédent $\Xlo\phi$, alors c'est
la négation qui a portée sur lui.} :


\ex. \label{x:QinSiN}
\(\Xlo\forall x [\prd{étudiant}(x) \implq [\prd{recalé}(x) \stackrel{\sicut{si}}{\implq}
  \prd{renvoyé}(\cns p)]]\)

\ref{x:QinSiN} nous dit qu'il suffit d'un étudiant recalé, n'importe
lequel, pour que le prof soit renvoyé.  Par exemple, \ref{x:QinSiN}
est fausse dans un modèle où un seul étudiant est recalé et que le
prof n'est pas renvoyé, alors que dans ce même modèle \ref{x:QinSiO}
sera vraie.  En effet les conditions de vérité de \ref{x:QinSiN}
disent que pour chaque individu assigné à $\vrb x$ et qui appartient à
l'ensemble des étudiants, si cet individu est recalé alors le prof est
renvoyé, et si l'individu n'est pas recalé alors peu importe que le
prof soit renvoyé ou non, $\Xlo[\prd{étudiant}(x) \implq [\prd{recalé}(x)
\stackrel{\sicut{si}}{\implq} \prd{renvoyé}(\cns p)]]$ est vrai, car
dans ce cas l'antécédent de l'implication
$\stackrel{\sicut{si}}{\implq}$ est faux.  Par conséquent,
\ref{x:QinSiN} est nécessairement vraie dès qu'un seul étudiant est
recalé et que le prof est renvoyé.



Tous ces exemples nous montrent la pertinence de la contrainte sur la
limite de portée des {\GN} ; les prédictions qu'elle fait, à savoir
l'impossibilité d'une portée large pour des {\GN} dans certaines
configurations, se sont avérées correctes.  Cependant, il nous faut
maintenant reconnaître que cette contrainte, du moins telle qu'elle
que nous l'avons formulée précédemment (p.~\pageref{ct:Portee}),
%%est formulée dans la proposition~\ref{pt:Portee}, 
\emph{est inexacte}.
Les exemples précédents n'examinaient que des {\GN} correspondant à
des quantifications universelles.  Or les {\GN} qui correspondent à
des quantifications existentielles, les indéfinis,\is{indefini@indéfini} eux échappent à la
contrainte\footnote{Cette importante observation est due entre autres
  à \citet{Farkas:81}\Andexn{Farkas, D.} et \citet{FodorSag:82}\Andexn{Sag, I.}\Andexn{Fodor, J. D.}. Un autre exemple, dû à
  Fodor et Sag, et souvent cité,  est :   \ExNBP
\ex. 
Chaque professeur a eu vent de la rumeur qu'un de mes étudiants a été
convoqué chez le doyen.\par\vspace{-1\baselineskip}} ;  
ce sont de véritables contre-exemples, comme le montrent 
\ref{x:NYpubl}, \ref{x:diplom}, \ref{x:EinSi} :

\ex. 
\a. Jean a acheté tous les livres qui étaient publiés par un éditeur
new-yorkais. 
\label{x:NYpubl}
\b. 
Jon drague chaque fille qui connaît un diplomate à Washington.
\label{x:diplom}
\b. 
Si un étudiant est recalé, le prof sera renvoyé.
\label{x:EinSi}


On peut parfaitement interpréter \ref{x:NYpubl} et  \ref{x:diplom}
avec une portée large des indéfinis, \sicut{un éditeur
  new-yorkais} et \sicut{un diplomate}, par rapport aux {\GN}
universels \sicut{tous les livres} et \sicut{chaque fille}.  Selon ces
lectures, il s'agit respectivement d'un seul et même éditeur et d'un
seul et même diplomate.  Par exemple pour \ref{x:diplom}, cela
correspondra à la glose \sicut{il y a un diplomate dont Jon drague chaque
fille qui le connaît}, ce que nous pouvons rendre à l'aide de
traduction suivante : 

\ex.
\(\Xlo\exists x [\prd{diplomate}(x) \wedge \forall y [[\prd{fille}(y)
      \wedge \prd{connaître}(y,x)] \implq \prd{draguer}(\cns j,y)]]\)

Et comme les {\GN} universels se trouvent hors de la 
relative où apparaît l'indéfini, cela prouve bien que ce dernier a
une portée qui s'étend au-delà de sa proposition syntaxique d'origine.   
Il en va de même avec \ref{x:EinSi}, qui est le pendant indéfini de
\ref{x:QinSi}.  Notons d'abord que l'interprétation de \ref{x:QinSi}
avec une portée étroite (et limitée à la subordonnée conditionnelle)
de \sicut{un étudiant} équivaut à \ref{x:QinSiN}, la lecture qui ne
passait pas pour \ref{x:QinSi} et qui dit qu'il suffit d'un étudiant
recalé, n'importe lequel, pour renvoyer le prof\footnote{\ref{x:EinSi}
se traduira alors par 
\(\Xlo[\exists x [\prd{étudiant}(x) \wedge \prd{recalé}(x)]
  \stackrel{\sicut{si}}{\implq} \prd{renvoyé}(\prd p)]\). Formellement
cette formule n'est pas identique à \ref{x:QinSiN}, mais elle lui est
sémantiquement équivalente. Voir l'équivalence \no \ref{Nex:2implq} de l'exercice~\ref{exo:equivlog}, page~\pageref{exo:equivlog}.}.  Mais la lecture
avec une portée large de l'indéfini est également disponible, bien que
peut-être moins immédiate ;  cette lecture ne parle pas de n'importe
quel étudiant, mais d'un étudiant particulier, et c'est son échec
personnel (et pas forcément celui des autres) qui entraînera  le
renvoi du prof.   La traduction suivante explicite cette lecture :
%
\ex.
\(\Xlo\exists x [\prd{étudiant}(x) \wedge [\prd{recalé}(x)
  \stackrel{\sicut{si}}{\implq} \prd{renvoyé}(\prd p)]]\)


La conclusion à laquelle nous arrivons ici peut s'énoncer en deux
parties.  D'abord nous devons abandonner, en l'état, la 
contrainte sur la limite de portée des {\GN} que nous avons présentée
précédemment. 
%%proposition~\ref{pt:Portee}.  
Nous la remplaçons par une version plus restreinte, mais correcte, de
la contrainte sur la portée des {\GN}, donnée dans la
proposition~\ref{pt:Portee2}.   


\begin{prop}[Contrainte sur la portée des {\GN} universels] \label{pt:Portee2}
Les {\GN} qui correspondent à une quantification universelle ne
peuvent pas prendre portée large 
au-delà de la proposition syntaxique où ils interviennent en surface.
\end{prop}


La seconde partie de la conclusion est que les {\GN} indéfinis\is{indefini@indéfini} ont des
propriétés interprétatives particulières, notamment en ce qui concerne
leur portée.  Nous y reviendrons dans les sections qui suivent et nous
verrons qu'ils constituent une classe de {\GN} bien à part. 

\is{quantificateur|)}
\is{portee@portée|)}

\subsection{Spécifique {\vs} non spécifique}
%-----------------------------------------
\is{specifique@spécifique}\is{specificite@spécificité}
\is{indefini@indéfini|(}
\label{ss:specificite}

Parmi les propriétés interprétatives typiquement associées aux {\GN}
indéfinis et qui peuvent expliquer en partie le comportement singulier
que nous venons de constater, figure la
\kwa{spécificité}{specificite}.  La spécificité a directement trait à
la référence des {\GN}, et il est assez courant d'entendre parler
«d'indéfinis spécifiques», opposés aux «indéfinis non
spécifiques».  Cependant ce genre de formulation est quelque peu
inapproprié, et il est important, je pense, de donner une définition
bien précise de cette propriété pour éviter toute confusion et ainsi
clairement cerner ce dont il s'agit.  En effet, plutôt qu'une
propriété de {\GN}, la spécificité qualifie en fait l'\emph{usage} ou
l'\emph{interprétation} d'un {\GN} indéfini.  C'est pourquoi il
conviendra mieux de parler de l'usage spécifique qu'un locuteur fait
d'un indéfini lorsqu'il l'emploie dans une phrase, ou encore de
l'interprétation spécifique qu'un allocutaire attribue à un indéfini
qu'il rencontre dans une phrase.  À partir de là, nous pouvons nous
donner la définition suivante\footnote{Je m'inspire ici de 
\citet{BFKamp:01e},\Andexn{Kamp, H.}\Andexn{Bende-Farka, Á.}
 véritable somme sur les indéfinis et la
 spécificité, où l'on trouvera des définitions plus rigoureuses encore.
Par ailleurs, il est important d'ajouter qu'il existe en fait plusieurs sortes de spécificités (cf. \citet{Farkas:02spec}\Andexn{Farkas, D.} à cet égard) ; pour être précis ce que nous appelons ici spécificité renvoie à ce que D. Farkas définit comme la \emph{spécificité épistémique},\is{specificite@spécificité!\eli\ epistemique@\elid\ épistémique} et c'est la plus courante.
} :


\begin{defi}[Spécificité]\is{specificite@spécificité}\label{def:spécificité}
Un {\GN} indéfini a un usage ou une interprétation spécifique lorsqu'il
«renvoie» 
à un individu bien \emph{particulier}, identifiable et identifié (au
moins pour le locuteur), bref si le locuteur a un individu bien
particulier en tête lorsqu'il utilise l'indéfini pour le mentionner.
\end{defi}


Illustrons cela avec les exemples suivants :

\ex.\label{x:sonnvamp} 
\a. Quelqu'un a réparé la sonnette.
\label{xsonnette}
\b.  
Un vampire a mordu Alice.
\label{x:vampire}

Dans ces phrases, les indéfinis \sicut{quelqu'un} et \sicut{un
vampire} peuvent être interprétés spécifiquement ou bien non
spécifiquement.  L'interprétation spécifique présume à chaque fois que
le locuteur sait précisément de qui il parle lorsqu'il utilise le
{\GN} indéfini.   Ainsi le contraste avec les interprétations non
spécifiques est assez net.  En \ref{xsonnette}, l'usage non spécifique
de \sicut{quelqu'un} peut correspondre par exemple à une circonstance
où le locuteur constate que la sonnette, qui était en dérangement
depuis un certain temps, est à nouveau en état de marche, et sachant
qu'il est improbable qu'elle se soit réparée toute seule, il signale
ce nouveau fait en énonçant \ref{xsonnette} ; il aurait pu dire aussi
\sicut{la sonnette a été réparée}.  De même, le locuteur qui
fait un usage non spécifique de l'indéfini en \ref{x:vampire} aurait
pu dire simplement quelque chose comme \sicut{il y a des traces de morsure
de vampire sur le cou d'Alice}.

\newpage

Il existe quelques moyens linguistiques de diagnostiquer un usage
spécifique, ou encore de certifier une interprétation spécifique. 
La spécificité se caractérise par un renvoi à une entité
précise et particulière, il est donc alors facile d'apporter des
«compléments d'information» sur des {\GN} indéfinis sans dénaturer
leur interprétation spécifique.  De telles précisions peuvent être
exprimées par l'ajout 
d'explicitations \ref{x:SpecElic}, d'appositions \ref{x:SpecAppo}, 
de propositions relatives descriptives%%
\footnote{Les relatives \emph{descriptives}, dites aussi
  appositives, s'opposent aux relatives dites \emph{restrictives}.  
Un moyen facile de garantir qu'une relative est bien descriptive est d'y
insérer l'adverbe \sicut{d'ailleurs}.}
\ref{x:SpecRelDesc}, etc. :

\ex. 
\a. Quelqu'un a réparé la sonnette : ton frère.\label{x:SpecElic}
%\ex. 
\b. Un vampire, à savoir Dracula en personne, a mordu Alice.\label{x:SpecAppo}
%\ex. 
\b. Un vampire, que je connais très bien d'ailleurs, a mordu Alice.
\label{x:SpecRelDesc}


Dans ces trois variantes, l'ajout de précisions n'est compatible
qu'avec l'interprétation spécifique.  Ce sont donc de bons tests\is{test!\elid\ de spécificité} pour
assurer la spécificité.  

De même, l'adjectif antéposé \sicut{certain}
employé avec l'article indéfini singulier \sicut{un} ou \sicut{une}
force toujours la lecture spécifique du {\GN} (comme en
\ref{x:certainvampire}), car \sicut{certain} a justement une
sémantique qui implique quelque chose comme \sicut{pas n'importe lequel}.

\ex. \label{x:certainvampire}
Un certain vampire a mordu Alice.


Toujours dans le même ordre d'idée, un indéfini interprété
spécifiquement pourra très naturellement être repris anaphoriquement
par un pronom personnel dans une phrase subséquente.  Ainsi dans
\ref{x:coquille1}, les continuations de la première phrase montrent
clairement que le locuteur sait précisément de quelle coquille il est
question. 


\ex. \label{x:coquille1}
Le relecteur a laissé passer une coquille$_1$ dans le manuscrit. Pourtant
\emph{elle}$_1$ est énorme / on ne peut pas \emph{la}$_1$ louper...


Mais il faut faire attention : la reprise pronominale ne constitue pas
un test parfaitement discriminant.  Un indéfini avec une lecture non
spécifique peut, sous certaines conditions, être aussi repris par un
pronom.  Par exemple, il ne serait pas cohérent d'attribuer une
interprétation spécifique à \sicut{un vampire} en \ref{x:vampireNSp} :


\ex. \label{x:vampireNSp}
Un  vampire$_1$ a mordu Alice. Il$_1$ a dû pénétrer dans sa chambre
vers minuit. On n'en sait pas plus...


Enfin la spécificité à une implication très importante pour ce qui
nous intéresse dans cette section : un indéfini interprété spécifiquement a
toujours une portée large, et même une portée maximale.  On peut par
exemple le constater vis-à-vis d'une négation, comme en \ref{xcoquille} :


\ex. \label{xcoquille}
Le relecteur n'a pas vu une coquille dans le manuscrit.


Dans la lecture de \ref{xcoquille} où le {\GN} \sicut{une coquille} a une
portée étroite par rapport à la négation, la phrase signifie que le
relecteur n'a vu \emph{aucune coquille}.  C'est bien ce qu'indique la
traduction en {\LO}\footnote{Là encore, juste par souci de simplification, le
{\GN} défini \sicut{le relecteur} est traduit par une constante, \cns r.} : 
\(\Xlo\neg\exists x [\prd{coquille}(x) \wedge
  \prd{voir}(\cns r, x)]\), en niant l'existence de coquilles vues par
le relecteur.
Et donc dans ce cas, on ne va
évidemment pas envisager qu'il est fait référence à une coquille
particulière ; autrement dit le {\GN} indéfini  est
forcément interprété non spécifiquement.  Inversement, pour
l'interpréter spécifiquement, il faut au contraire que l'existence de
la coquille en question soit bien affirmée dans la structure
sémantique de la phrase ; autrement dit l'indéfini doit alors échapper
à la négation en ayant  portée
large sur elle, comme dans la traduction : \(\Xlo\exists x
[\prd{coquille}(x) \wedge \neg\prd{voir}(\cns r,x)]\).
Remarquons que même si ce n'est peut-être pas la première lecture qui
se présente dans \ref{xcoquille}, l'interprétation spécifique de
l'indéfini est parfaitement disponible, comme le montrent les exemples 
\ref{xcoquille'}.

\ex. \label{xcoquille'}
\a.
\label{xcoquille''}
 Le relecteur n'a pas vu une coquille$_1$ dans le manuscrit. Pourtant
{elle}$_1$ est énorme / on ne peut pas {la}$_1$ louper...
\b.
\label{xcoquille'''}
 Le relecteur n'a pas vu une coquille, qui d'ailleurs se trouvait à
 la première ligne de la première page.


La portée large due à la spécificité apparaît également vis à vis d'un
{\GN} universel, comme en \ref{x:vampA} :

\ex. \label{x:vampA}
Un vampire, que je connais très bien d'ailleurs, a mordu chacune de
mes cousines.

Ici l'interprétation spécifique de \sicut{un vampire} est forcée par
la relative descriptive.  Or \ref{x:vampA} ne peut s'interpréter qu'en
faisant référence à un seul et même vampire.  Cela veut dire que
l'indéfini ne covarie pas avec l'universel \sicut{chacune de
mes cousines} ; et par conséquent c'est bien l'indéfini qui a portée
large. 

Maintenant, étant donné que la spécificité est un phénomène qui a de
réelles répercussions sur l'interprétation, en admettant par exemple
que des phrases comme \ref{xsonnette} et \ref{x:vampire} sont
clairement ambiguës, nous devons alors nous demander comment intégrer
précisément ce phénomène dans l'analyse sémantique, et en
l'occurrence, chercher s'il y a un moyen de le représenter proprement
dans {\LO}. 

Or il faut constater que cette question ne se résout pas si
simplement.  En effet, l'analyse sémantique que nous menons ici et les
traductions en {\LO} rendent compte du sens en termes de conditions de
vérité.  La contribution vériconditionnelle d'un indéfini consiste à
poser l'existence d'un individu ; et il semble bien que cela vaut
aussi bien pour une interprétation non spécifique que pour une
interprétation spécifique.
Par conséquent les conditions de vérité à attribuer à une phrase comme
\ref{x:vampire} sont (au moins) celles exprimées par la formule
suivante :

\ex. \label{x:vampirecv}
Un vampire a mordu Alice.\\
\(\Xlo\exists x [\prd{vampire}(x) \wedge \prd{mordre}(x,\cns{a})]\)


En fait, si \ref{x:vampirecv} nous donne les conditions de vérité
\emph{communes} aux deux interprétations, ce que nous pouvons
raisonnablement supposer c'est qu'il y a «quelque chose en plus»
dans l'interprétation spécifique.  Et la question qui se pose revient
alors à savoir si ce quelque chose en plus est vraiment de nature
vériconditionnelle.  
C'est là une question que nous n'allons pas trancher ici (mais
l'exercice \ref{exo:speci} \alien{infra} sera une occasion d'y
réfléchir un peu) car elle semble encore faire débat dans la
littérature (cf. à cet égard \citealt{BFKamp:01e}).\Andexn{Kamp, H.}\Andexn{Bende-Farka, Á.} 


Ce qui fait le propre de l'usage spécifique d'un {\GN} indéfini, c'est
que le locuteur a en tête un individu bien particulier lorsqu'il
emploie ce {\GN}.  De même, l'interprétation spécifique correspond à
une hypothèse que fait l'allocutaire sur ce que le locuteur a en tête.
C'est pourquoi, on peut avoir le droit de conclure que, d'une certaine
manière, l'emploi ou la lecture spécifique d'un indéfini, c'est
l'affaire du locuteur. Celui-ci pense à un individu particulier, et
cette «pensée» est localisée dans son état cognitif.  Les
conditions de vérité d'un énoncé décrivent partiellement un état du
monde objectif, et habituellement on considère que les états cognitifs
des interlocuteurs ne font pas partie de telles descriptions.  Entre
autres parce qu'un état cognitif est quelque chose de privé et de
relativement libre (chacun peut penser à ce qu'il veut, ça ne change
pas en soi l'état du monde).  En d'autres termes, les états cognitifs
seraient des composantes du contexte plutôt que du modèle.  Et selon
cette optique, la spécificité serait un phénomène cognitif ou, disons,
pragmatique plutôt que sémantique.


Il est certainement prudent de voir les choses ainsi, mais ce n'est
pas pour autant complètement satisfaisant.  La spécificité (en tant
qu'attitude et usage mentaux du locuteur) ne fait pas partie du monde
décrit par la phrase, %extralinguistique, 
mais ce n'est peut-être pas une raison pour
l'exclure de la structure linguistique, et donc sémantique, des
énoncés.  Car d'abord nous avons vu que la spécificité a des effets
sur cette structure :  les
{\GN} spécifiques ont toujours portée large (et pas une portée
variable comme les autres {\GN} indéfinis).  Par ailleurs il est
parfois possible d'exprimer explicitement l'usage spécifique d'un
indéfini.  En français l'adjectif \sicut{certain} a ce pouvoir.
D'autres langues grammaticalisent encore plus nettement la
spécificité ; par exemple le latin dispose de deux pronoms signifiant
\sicut{quelqu'un} : \alien{quidam}, pour l'usage spécifique, et
\alien{aliquis}, pour l'usage non spécifique ; en roumain, le préfixe
\alien{pe-} rend spécifique les {\GN} indéfinis qu'il précède.  La
langue a donc les moyens de \emph{marquer} la spécificité et cela devrait
 apparaître dans l'analyse sémantique.


\sloppy

Or il se trouve que notre système {\LO} permet des représentations
qui, en quelque sorte, rendent vériconditionnelles les interprétations
spécifiques.  Cette solution s'inspire de la proposition de
\citet{FodorSag:82}.  \Andex{Fodor, J. D.}\Andex{Sag, I.}
\citeauthor{FodorSag:82} dans un article très fameux proposaient
d'analyser les indéfinis à lecture spécifique au moyen de \kwo{constantes}\is{constante!\elid\ d'individu}.
Ainsi \ref{x:vampire} pouvait se traduire par \ref{x:vampirecst} (en
plus de la traduction \ref{x:vampirecv} pour l'interprétation non
spécifique) : 

\fussy

\ex. \label{x:vampirecst} %[]
Un vampire a mordu Alice.\\
\(\Xlo[\prd{vampire}(\cns{v}) \wedge \prd{mordre}(\cns{v},\cns{a})]\) \quad (lecture spécifique)


\largerpage

Une constante, par nature, ne covarie pas avec une quantification,
 elle se comporte donc comme si elle avait une portée maximalement
 large\footnote{À proprement parler, une constante n'a pas de portée,
 ce qui compte ici c'est qu'elle échappe à tout effet de quantification.},
ce qui est bien une propriété attendue pour l'interprétation spécifique
 des indéfinis. On obtient  dès lors des  
conditions de vérité correctes pour une lecture de \ref{x:diplom},
 celle où \sicut{un diplomate} est interprété spécifiquement :

\ex. \label{x:diplomcst} %%[\ref{x:diplom}]
Jon drague chaque fille qui connaît un diplomate à Washington.\\
\(\Xlo\forall x [[\prd{fille}(x) \wedge \prd{connaître}(x,\cns{d}) \wedge
    \prd{diplomate}(\cns{d})] \implq \prd{draguer}(\cns{j},x)]\)

Ici il n'est question que d'un seul diplomate, représenté par \cns d.
Le {\GN} indéfini n'est pas traduit par un quantificateur existentiel
ayant une portée large, mais au bout du compte l'interprétation
revient au même : l'indéfini échappe à la portée de tout autre
opérateur de la phrase. 

Cette analyse implique qu'un {\GN} indéfini ou plus exactement que
l'article indéfini \sicut{un} est ambigu : il y aurait en français
l'indéfini existentiel (traduit au moyen de $\exists$) et l'indéfini
spécifique (traduit au moyen d'une constante).  On pourrait penser que
cette solution est un peu gênante «esthétiquement», du fait
qu'elle postule deux articles indéfinis différents (mais homonymes) ;
ce n'est pas spécialement intuitif, ni économique.  Mais bien entendu
cela ne constitue pas un argument scientifique à l'encontre de
l'analyse.  En revanche, nous verrons par la suite 
%(en \S\ref{ss:indef}) 
un argument qui condamne plus sérieusement la
proposition de \citeauthor{FodorSag:82} sur un critère véritablement
sémantique.   

En attendant, nous pouvons commencer à l'aménager, en y apportant deux
légères modifications.  On peut d'abord imaginer une variante qui
«fusionne» l'emploi habituel de la quantification existentielle et
celui des constantes de \citeauthor{FodorSag:82}.  Pour ce faire, on 
considère que l'article indéfini correspond toujours à une
quantification existentielle sur une variable, mais qu'en plus,
l'interprétation spécifique est représentée par l'ajout
d'une information qui relie, par une égalité, la variable quantifiée à
une constante\footnote{Cette stratégie s'inspire, en simplifiant beaucoup, de l'analyse proposée par \citet{BFKamp:01e} notamment.\Andexn{Kamp, H.}\Andexn{Bende-Farka, Á.}}.  Illustrons cela en reformulant les traductions de
\ref{x:vampire} et \ref{x:diplom} respectivement en \ref{x:vampirev=c}
et \ref{x:diplomv=c} :




\ex. \label{x:vampirev=c}
Un vampire a mordu Alice.\\
\(\Xlo\exists x [\prd{vampire}(x) \wedge x=\cns{v} \wedge \prd{mordre}(x,\cns{a})]\)

\ex. \label{x:diplomv=c}
Jon drague chaque fille qui connaît un diplomate à Washington.\\
\(\Xlo\forall x [[\prd{fille}(x) \wedge \exists y [\prd{connaître}(x,y) \wedge
    \prd{diplomate}(y) \wedge y=\cns{d}]] \implq \prd{draguer}(\cns{j},x)]\)


En \ref{x:diplomv=c}, la variable quantifiée existentiellement $\vrb y$
peut 
{a priori}  
covarier avec le quantificateur $\Xlo\forall x$, mais «à condition» que
la valeur de $\vrb y$ soit toujours identique à la dénotation de \cns d,
comme stipulé par la sous-formule \(\Xlo y=\cns d\).  À l'arrivée on n'a
donc pas le choix : la valeur de $\vrb y$ se retrouve  scotchée à
celle d'une constante, ce qui finalement l'empêche de varier.

Ainsi \ref{x:vampirev=c} et \ref{x:diplomv=c} ont les mêmes conditions
de vérité que \ref{x:vampirecst} et \ref{x:diplomcst} (elles sont
respectivement logiquement équivalentes), mais ces
variantes de traductions sont plus conformes à une vision économique de
la spécificité. Ici un indéfini n'est pas en soi ambigu : il se traduit
toujours par une quantification existentielle ; simplement si de
surcroît il reçoit une interprétation spécifique, on ajoute une
information du type \(\Xlo y=\cns d\).  Cela permet de représenter la
spécificité comme un phénomène interprétatif additionnel.

\sloppy

La seconde modification que nous pouvons apporter à l'analyse de
\citeauthor{FodorSag:82} concerne l'emploi de constantes pour
contrecarrer les effets de covariation.  Les constantes employées ici
sont d'un type un peu particulier car on sait qu'elles dénotent
fixement un individu précis, mais on ne sait pas directement lequel
(contrairement aux constantes qui traduisent des noms propres).  Cela
n'est pas exactement le fonctionnement standard des constantes de
{\LO}, car dès qu'un modèle {\Modele} et donc une fonction
d'interprétation {\FI} nous sont donnés, pour toute constante \cns c,
on connaît directement sa dénotation dans {\Modele}, à savoir
l'individu \(\FI(\cns c)\).  Or l'allocutaire qui donne une
interprétation spécifique à un indéfini fait l'hypothèse que le
locuteur pense à un individu précis, mais il peut tout à fait ne pas
savoir de qui il s'agit.  C'est pourquoi, pour rendre compte du
caractère possiblement inconnu d'une interprétation spécifique, une
solution consiste à utiliser, en place des constantes, des
\emph{variables libres}.  

\fussy

\ex. 
\a.
%%Un vampire a mordu Alice.\\
\(\Xlo\exists x [\prd{vampire}(x) \wedge x=z \wedge \prd{mordre}(x,\cns{a})]\)
\label{x:vampirev=v}
\b. 
%%Jon drague chaque fille qui connaît un diplomate à Washington.\\
\(\Xlo\forall x [[\prd{fille}(x) \wedge \exists y [\prd{connaître}(x,y) \wedge
    \prd{diplomate}(y) \wedge y=z]] \implq \prd{draguer}(\cns{j},x)]\)
\label{x:diplomv=v}

En \ref{x:vampirev=v} et \ref{x:diplomv=v} la variable $\vrb z$ joue le
même rôle de point fixe que les constantes \cns v et \cns d en
\ref{x:vampirev=c} et \ref{x:diplomv=c}. En effet, la valeur de $\vrb z$
dépend  seulement de l'assignation $g$ par rapport à laquelle on
évalue globalement la formule ; et l'interprétation de quantificateurs
comme $\Xlo\exists x$ ou $\Xlo\forall x$ peuvent changer la valeur de $g(\vrb x)$ 
mais jamais celle de $g(\vrb z)$.  Dans ces formules, $\vrb z$ aura donc
toujours une valeur fixe, qui en fait ne dépend que de l'état cognitif
du locuteur dans le contexte d'interprétation, et nous avions
justement suggéré (\S\ref{ss:statutg}) que les fonctions
d'assignations pouvaient servir à 
modéliser une partie des états cognitifs des interlocuteurs.

\sloppy

L'avantage de cette solution
est qu'elle pourrait nous autoriser à abandonner la contrainte révisée
énoncée dans la proposition~\ref{pt:Portee2} (p.~\pageref{pt:Portee2}) et à revenir à la version
originale et plus générale donnée p.~\pageref{ct:Portee} qui disait
qu'\emph{aucun} {\GN} ne peut s'interpréter hors de la proposition
syntaxique où il apparaît. C'est en effet ce qu'illustre
\ref{x:diplomv=v} où le quantificateur $\Xlo\exists y$ est «physiquement» dans la portée de $\Xlo\forall x$. Autrement dit
\sicut{un diplomate} s'interprète dans sa proposition, et donc dans la
portée de \sicut{chaque fille} ; mais il ne covarie pas, à cause de
$\Xlo y=z$. Cependant, comme annoncé précédemment, 
nous allons voir 
%nous verrons en \S\ref{ss:indef} 
que cette approche n'est pas complètement
satisfaisante pour rendre compte de la sémantique des \GN\ indéfinis.


\fussy

L'hypothèse de \citeauthor{FodorSag:82} est que lorsqu'un indéfini semble s'interpréter en dehors de sa proposition syntaxique, c'est parce qu'il est spécifique.  Dans ce cas, alors, l'indéfini doit avoir (ou donner l'impression d'avoir) une portée \emph{maximale} --~puisqu'il est spécifique.  Par conséquent, un moyen de falsifier cette hypothèse consiste simplement à chercher des phrases où un indéfini s'interpréterait en dehors de sa proposition \emph{sans} avoir une portée maximale dans la phrase --~par exemple en covariant avec un autre \GN. Et il se trouve que de tels phrases existent, par exemple \ref{x:profF2} (inspiré de \citet{Abusch:94}\footnote{L'exemple original
  (traduit en français) est : \sicut{Chaque enseignant a récompensé
    chaque étudiant qui a lu un livre qu'il a recommandé}.}) :\Andex{Abusch, D.}


\ex.\label{x:profF2}
Chaque professeur a récompensé chaque étudiant [qui a lu un roman de
Flaubert]. 


Par le jeu des trois {\GN} employés ici, \ref{x:profF2} a trois interprétations.
La première est la simple lecture linéaire : chaque professeur a récompensé chaque étudiant ayant lu un roman de
Flaubert, n'importe lequel ; c'est-à-dire \sicut{chaque professeur} $>$ \sicut{chaque étudiant} $>$  \sicut{un roman de Flaubert}, autrement dit la 
\emph{portée étroite} de l'indéfini.  
Elle se représente par la traduction \ref{x:profF2a} :

\ex.\label{x:profF2a}  \raggedright
\(\Xlo\forall x [\prd{prof}(x) \implq \forall y [[\prd{étudiant}(y) \wedge
    \exists z [\prd{roman-F}(z) \wedge \prd{lire}(y,z)]] \implq \coupe%\allowbreak 
\prd{récompenser}(x,y)]]\)


La deuxième correspond à la portée large et \emph{maximale} de l'indéfini, 
\sicut{un roman de Flaubert} $>$  \sicut{chaque professeur} $>$ \sicut{chaque étudiant} : 
il y a un roman de Flaubert particulier (par exemple Salambô) tel que chaque
professeur a récompensé chaque étudiant qui l'a lu, c'est le même roman pour tout le monde.  Selon  \citeauthor{FodorSag:82}, il s'agit de la lecture spécifique de l'indéfini, ce qui est illustré par \ref{x:profF2b2}\footnote{Mais j'indique également \ref{x:profF2b1}, car on peut également défendre l'idée qu'ici l'indéfini n'est pas nécessaire spécifique pour le locuteur : il peut ignorer de quel roman précis il est question.} :

\ex.\label{x:profF2b} \raggedright
\a. \(\Xlo\exists z [\prd{roman-F}(z) \wedge \forall x [\prd{prof}(x) \implq
    \forall y [[\prd{étudiant}(y) \wedge \prd{lire}(y,z)] \implq\coupe 
\prd{récompenser}(x,y)]]]\)\label{x:profF2b1}
%\\ou
\b.
\(\Xlo\forall x [\prd{prof}(x) \implq \forall y [[\prd{étudiant}(y) \wedge
    \exists z [\prd{roman-F}(z) \wedge z=u \wedge \prd{lire}(y,z)]] \implq\coupe \prd{récompenser}(x,y)]]\)\label{x:profF2b2}


Et la troisième interprétation, celle qui nous intéresse ici, correspond à ce que l'on appelle la \kwo{portée intermédiaire}\is{portee@portée!\elid\ intermédiaire} de l'indéfini, \sicut{chaque professeur} $>$ \sicut{un roman de Flaubert} $>$  \sicut{chaque étudiant}. 
C'est la lecture qui dit que pour chaque professeur, il y a un roman de Flaubert (possiblement propre à chaque professeur) tel qu'il a
récompensé chaque étudiant qui l'a lu.  \sicut{Un roman de Flaubert} covarie donc avec \sicut{chaque professeur} mais pas avec \sicut{chaque étudiant} ; ce qui fait que l'analyse \ref{x:profF2b2} ne tient pas ici, car la variable libre \vrb u ne peut pas covarier. 
Nous sommes donc obligés d'admettre l'analyse \ref{x:profF2c} où l'interprétation de l'indéfini est intercalée entre celles des deux autres \GN, et donc \emph{à l'extérieur} de la proposition relative :

\ex.\label{x:profF2c} \raggedright
\(\Xlo\forall x [\prd{prof}(x) \implq \exists z [\prd{roman-F}(z) \wedge
    \forall y [[\prd{étudiant}(y) \wedge \prd{lire}(y,z)] \implq\coupe 
\prd{récompenser}(x,y)]]]\)


\largerpage

En conclusion, la spécificité n'explique pas tout. En tout cas, nous venons de le voir, elle n'explique pas la portée extra-propositionnelle des indéfinis\footnote{Notons cependant qu'il est possible de récupérer la lecture à portée intermédiaire de l'indéfini de \ref{x:profF2} tout en conservant (localement) notre mécanisme de la spécificité tel qu'il apparaît en \ref{x:profF2b2}. Pour cela, il «suffit» de proposer, par exemple, la traduction suivante :\ExNBP
\ex. \raggedright
\(\Xlo\forall x [\prd{prof}(x) \implq \exists u \forall y [[\prd{étudiant}(y) \wedge
    \exists z [\prd{roman-F}(z) \wedge z=u \wedge \prd{lire}(y,z)]] \implq\allowbreak \prd{récompenser}(x,y)]]\)

Cette formule est similaire à \ref{x:profF2b2} sauf que la variable \vrb u n'est plus libre, elle a été astucieusement liée par un $\Xlo\exists u$ correctement placé. Évidemment toute la difficulté de cette analyse réside dans le pourquoi et le comment de l'introduction de ce $\Xlo\exists u$ ; c'est un point qui est loin d'être trivial et qui ne sera pas développé ici, mais je le mentionne car il n'est pas complètement étranger à certaines analyses qui ont été proposées pour traiter les portées extra-propositionnelles des indéfinis.}. 
Et nous devons donc admettre 1) que la proposition \ref{pt:Portee2} sur l'interprétation mobile des {\GN} est toujours valable, et 2) que les indéfinis ont un comportement sémantique bien particulier. Nous verrons d'ailleurs en \S\ref{s:CatGN} plusieurs propriétés qui confirment la particularité des indéfinis. 

\is{indefini@indéfini|)}

\begin{exo}\label{exo:3portee}
Déterminez, en les glosant, 
\pagesolution{crg:3portee}
toutes les lectures possibles de chacune des phrases suivantes -- en prenant bien soin de repérer notamment la lecture avec portée intermédiaire de l'indéfini singulier.
\begin{enumerate}
\item
Chaque convive a raconté plusieurs histoires qui impliquaient un
membre de la famille royale.
\item
Chaque sénateur a raconté à plusieurs journalistes qu'un membre du
cabinet était corrompu.
\item 
Un professeur croit que chaque étudiant a lu un roman de Flaubert.
\end{enumerate}
\begin{solu}(p.~\pageref{exo:3portee})\label{crg:3portee}
\begin{enumerate}
\item
Chaque convive a raconté plusieurs histoires qui impliquaient un
membre de la famille royale.

\begin{enumerate}
\item Chaque convive $>$ plusieurs histoires $>$ un membre : 
pour chaque convive \Obj x, \Obj x a raconté plusieurs histoires, et chacune de ces histoires implique un membre (possiblement différent) de la famille royale.

\item Chaque convive $>$ un membre $>$ plusieurs histoires : 
pour chaque convive \Obj x, il y a un membre (particulier) de la famille royale au sujet duquel \Obj x a raconté plusieurs histoires.

\item Un membre $>$ chaque convive $>$ plusieurs histoires : 
il y a un membre (particulier) de la famille royale au sujet duquel tous les convives ont raconté plusieurs histoires (possiblement différentes). 

\item Un membre $>$ plusieurs histoires $>$ chaque convive : 
il y a un membre (particulier) de la famille royale au sujet duquel plusieurs histoires ont été racontées par tous les convives (et chacun racontait les mêmes histoires que les autres).

\item Plusieurs histoires $>$ un membre $>$ chaque convive : 
il y a plusieurs histoires impliquant divers membres de la famille royale qui ont été racontées par tous les convives (et chacun racontait les mêmes histoires que les autres).

\sloppy
\item Il y a une dernière lecture théoriquement possible mais pragmatiquement étrange et qui correspond à : plusieurs histoires $>$ chaque convive $>$ un membre.  Cette lecture apparaît spécifiquement dans un scénario comme le suivant : il y a une certaine série d'histoires et tous les convives racontent cette même série, mais chacun change le personnage sur lequel portent les histoires. 
  À la rigueur cette lecture peut passer si on comprend \sicut{histoire} comme dénotant des types ou modèles d'histoires mais pas d'anecdotes précises.

\fussy
\end{enumerate}

\item
Chaque sénateur a raconté à plusieurs journalistes qu'un membre du
cabinet était corrompu.

\begin{enumerate}
\item Chaque sénateur $>$ plusieurs journalistes $>$ un membre :
chaque sénateur \Obj x s'adresse à plusieurs journalistes et à chaque journaliste, \Obj x parle d'un membre corrompu (possiblement différent pour chaque journaliste).

\item Chaque sénateur $>$ un membre $>$ plusieurs journalistes :
pour chaque sénateur \Obj x il y a un membre \Obj y du cabinet dont \Obj x parle à plusieurs journalistes.

\item  Un membre $>$ chaque sénateur $>$ plusieurs journalistes : 
il y a un membre corrompu \Obj y et chaque sénateur parle de \Obj y à plusieurs journalistes (les journalistes peuvent être différents selon les sénateurs).

\item  Un membre $>$ plusieurs journalistes $>$ chaque sénateur :
il y a un membre corrompu \Obj y et un groupe particulier de journaliste et tous les sénateurs parlent de \Obj y à ce groupe de journalistes.

\item  Plusieurs journalistes $>$ un membre $>$ chaque sénateur :
il y a un groupe de journalistes, pour chacun de ces journalistes il y a un membre corrompu (possiblement différent d'un journaliste à l'autre) et chaque sénateur parle de ce membre au journaliste qui lui est «associé».

\item  Plusieurs journalistes $>$ chaque sénateur $>$ un membre :
il y a un groupe donné de journalistes, tous les sénateurs s'adressent à ce même groupe et chacun parle d'un membre corrompu différent.

\end{enumerate}

NB : il existe encore d'autres lectures possibles (et subtiles) de cette phrase qui mettent en jeu un phénomène que nous aborderons plus précisément en \S\ref{ss:re/dicto}, mais la troisième phrase ci-dessous nous en donne un petit aperçu (avec notamment la lecture~3c).

\item 
Un professeur croit que chaque étudiant a lu un roman de Flaubert.

\begin{enumerate}
\item Un professeur $>$ (que) chaque étudiant $>$ un roman :
il y a un professeur qui pense que chaque étudiant a choisi librement un roman de Flaubert et l'a lu.

\item Un roman $>$ un professeur $>$ (que) chaque étudiant :
il y a un roman écrit par Flaubert et un professeur qui pense que tous les étudiants ont lu ce roman.

\item Un professeur $>$ (que) un roman $>$ chaque étudiant :
il y a un professeur qui pense à un roman particulier, par exemple \emph{Le rouge et le noir}, il pense que ce roman est de Flaubert (mais il se trompe, en l'occurrence) et pense que tous les étudiants l'ont lu.
Notons cependant que cette lecture n'implique nécessairement que le professeur se trompe sur l'auteur du roman, mais elle est compatible avec un tel cas de figure.

\end{enumerate}

\end{enumerate}
\end{solu}
\end{exo}


% -*- coding: utf-8 -*-
\begin{exo}\label{exo:speci}
%
\begin{solu}(p.~\pageref{exo:speci})\label{crg:speci}

Si nous voulons montrer que \ref{x:vampire} est ambiguë selon la définition \ref{d:ambig}, nous devons construire un modèle par rapport auquel la phrase sera jugée vraie et fausse selon que le \GN\ a une lecture spécifique ou non.
Un tel modèle doit comporter minimalement l'individu \Obj{Alice}, une jeune fille,  un autre individu, appelons-le \Obj{Lestat}, qui est un vampire et \Obj{Lestat} a mordu \Obj{Alice} pendant la nuit.  Mais dans ce modèle \ref{x:vampire} sera toujours vraie, que \sicut{un vampire} soit spécifique ou non.  La lecture (ou l'usage) non spécifique émerge quand le locuteur ne connaît pas l'identité de \Obj{Lestat}, mais cette information n'est pas codée dans le modèle (tels que les modèles sont définis dans notre système).  Nous devons donc conclure que \ref{x:vampire} n'est pas sémantiquement ambiguë (si elle l'est, il s'agira plutôt d'une ambiguïté pragmatique).
\end{solu}
En reprenant la définition \ref{d:ambig} 
\pagesolution{crg:speci}%
donnée en \S\ref{s:Ambiguïté}
(p.~\pageref{d:ambig}), essayez de montrer si \ref{x:vampire} est, ou
non, un cas de véritable ambiguïté sémantique.

\ex.[\ref{x:sonnvamp}]
\a.[b.] Un vampire a mordu Alice.

\end{exo}


% -*- coding: utf-8 -*-
\begin{exo} \label{exo:specdicto}
\begin{solu}(p.~\pageref{exo:specdicto})\label{crg:specdicto}

Dans la phrase (\ref{x:specpourqui}), \sicut{un vampire} à forcément un usage spécifique du fait de l'apposition.  D'après la définition de la spécificité (p.~\pageref{def:spécificité}), nous conclurons que le locuteur pense à un certain vampire, et plus précisément Dracula, et affirme qu'Alice pense que ce vampire l'a mordue.  Mais il y a, par ailleurs, une façon très naturelle de comprendre (\ref{x:specpourqui}) selon laquelle c'est avant tout Alice qui pense à un vampire particulier, Dracula, et pense qu'il l'a mordue.  Autrement dit \sicut{un vampire} s'avère d'abord spécifique pour Alice avant de l'être pour locuteur.  Foncièrement cela ne change probablement pas grand chose pour l'analyse sémantique de la phrase (si le \GN\ est spécifique pour Alice, il l'est a fortiori aussi pour le locuteur), mais du point de vue de son adéquation descriptive,   l'analyse passe un peu à côté de cette caractéristique qui est que la source de la spécificité n'est pas toujours le fait du locuteur.
\end{solu}
Quel problème peut éventuellement 
\pagesolution{crg:specdicto}%
poser la phrase ci-dessous vis-à-vis de la définition de la
spécificité vue dans cette section ?
%%Spécifique pour qui ?

\begin{enumerate}
\item \label{x:specpourqui}
Alice croit qu'un vampire, à savoir Dracula en personne, l'a mordue
pendant la nuit.
\end{enumerate}
\end{exo}


% -*- coding: utf-8 -*-
\begin{exo}\label{exo:Spec3}
Essayez d'imaginer un contexte dans lequel la phrase ci-dessous
\pagesolution{crg:Spec3}%
peut être prononcée avec un usage \emph{non spécifique} du {\GN}
\sicut{un acteur américain}.  

\begin{enumerate}
\item Hier j'ai rencontré un acteur américain.
\end{enumerate}
%
\begin{solu} (p.~\pageref{exo:Spec3}) \label{crg:Spec3}

Évidemment, la lecture la plus naturelle de la phrase est celle avec un usage spécifique du \GN\ puisque le locuteur raconte une expérience vécue personnellement (il a donc une idée précise de l'acteur en question).   Un contexte permettant cet usage serait par exemple une situation où le locuteur est invité à une soirée, il y rencontre Bryan Cranston, qui est un acteur américain, il le reconnaît pour l'avoir vu dans des films (sans forcément se souvenir de son nom) et il nous rapporte cet épisode avec la phrase (1). 

Pour que le \GN\ ait un usage non spécifique, il faudrait que le locuteur sache qu'il a rencontré un individu qui est un acteur américain tout en ignorant de quel individu il s'agit (c'est-à-dire de qui il parle précisément).  C'est là que réside la difficulté.  Supposons d'abord que le locuteur ait rencontré de nombreuses personnes au cours de la soirée, dont Bryan Cranston (que le locuteur ne connaît pas et n'a jamais vu en film ou à la télévision).  Dans ce contexte, le locuteur peut prononcer la phrase (1) sans avoir une idée précise de qui il parle (puisqu'il a rencontré plusieurs personnes ce qui laisse le choix dans l'identité de l'individu).  Mais comment sait-il alors qu'il s'agit d'un acteur américain ?  Une possibilité est qu'il l'apprenne par une tierce personne : un ami du locuteur présent à la soirée l'a vu discuter avec Bryan Cranston et l'informe le lendemain en lui disant «hier tu as parlé avec un acteur américain» (sans plus de précision). À partir de là, le locuteur peut alors prononcer la phrase (1) avec un usage non spécifique du \GN, du moment qu'il ne sait toujours pas lequel des invités était l'acteur.
\end{solu}
\end{exo}





\subsection{Générique {\vs} non générique}
%---------------------------------------
\label{AdvQ+Gen}%\is{generique@générique}% changer en "voir à"...
\is{generique@générique|see{généricité}}
\is{genericite@généricité|(} 

À ce stade de l'étude, il n'est pas inutile d'aborder (sommairement)
la notion de généricité, ne serait-ce que, dans un premier temps, pour
tordre le cou à une idée reçue, malheureusement fort répandue. En
effet, il arrive assez souvent de voir opposer \kwi{générique}{genericite@généricité} à \kwi{spécifique}{specificite@spécificité},
selon une relation de complémentarité.
C'est une erreur. Il s'agit en fait de deux catégories
orthogonales\footnote{Ce qui implique que l'on peut rencontrer des
interprétations qui ne sont ni génériques ni spécifiques, ou au
contraire, à la fois génériques et spécifiques. Voici des exemples de
phrases qui expriment à la fois la généricité et la spécificité :
\ExNBP
\ex. 
\a. Un hominidé a vécu au Sahel il y a 7 millions d'années : le \emph{Sahelanthropus tchadensis}.
\b. En Australie, on rencontre une petite araignée rouge extrêmement
venimeuse. 

Nous y reviendrons
au chapitre~\ref{GN++} (vol.~2).},
qui sont définies indépendamment l'une de l'autre.
Il faudra donc bien se garder de nommer spécifique ce qui n'est pas
générique et vice versa.
Si la spécificité a trait à la manière dont le locuteur conçoit la
référence à un objet, la généricité est à mettre en rapport avec un
certain mode de quantification. 
De plus, il faut préciser que la notion de généricité est assez difficile
à définir sémantiquement, c'est-à-dire en termes de conditions de
vérité précises. Ici je ne ferai qu'esquisser l'idée générale qui
sous-tend cette sémantique.
À cela s'ajoute une autre complication qui est qu'aujourd'hui, il est
couramment admis que ce que l'on appelle généricité recouvre en fait
deux phénomènes sémantiques distincts. Dans cette section, nous allons
surtout observer celui qui apparaît essentiellement avec les 
\GN\ indéfinis\is{indefini@indéfini} en français\footnote{Le second phénomène sera abordé
  plus tard dans le chapitre~\ref{GN++} (vol.~2).}.

Les {\GN} indéfinis singuliers que nous avons vu s'interprètent
sémantiquement en posant l'existence d'(au moins) un individu comme
une condition nécessaire et suffisante.  
%Et ça ne va pas plus loin.
Mais parfois un {\GN} indéfini n'exprime pas une simple existence, ou
du moins, la simple existence d'un individu ne suffit pas à déterminer
correctement le sens de la phrase qui contient l'indéfini.
C'est ce que l'on peut constater en \ref{xlion1}, par contraste avec \ref{xlion2} :

\ex.  \label{xlion1}
\a. Un lion est un mammifère. \label{xlion1a}\hfill  (générique)
\b. Un lion est paresseux.  \label{xlion1b} \hfill (idem)
\c. Un lion mange de la viande. \label{xlion1c} \hfill (idem)

\ex.  \label{xlion2}
\a. Un lion a saccagé le canapé. \label{xlion2a} \hfill (spécifique ou non)
\b. Un lion a mangé de la viande (qui était dans le
frigo). \label{xlion2b} \hfill (idem)


Les phrases en \ref{xlion1} illustrent le phénomène de généricité.
Pour le présenter assez simplement, disons que la
généricité\is{genericite@généricité} apparaît lorsque dans une
phrase, quelque chose de général ou de typique est affirmé au sujet de
\emph{tous} les membres d'une classe, d'une espèce, d'une
catégorie... C'est donc, au moins en partie, une affaire de
quantification universelle (même si, comme nous le verrons ci-dessous,
il s'agit d'une universalité un peu particulière).  Et c'est bien ce
que l'on observe en \ref{xlion1} : ces exemples expriment ce qui
semble être une quantification universelle, malgré le fait qu'ils
contiennent un indéfini singulier.  En effet une bonne manière de
rendre le sens de \ref{xlion1a} avec ce dont on dispose dans {\LO},
c'est d'écrire tout simplement :

\ex.[\ref{xlion1}]
\a. Un lion est un mammifère.\\
\(\Xlo\forall x [\prd{lion}(x)\implq \prd{mammifère}(x)]\)


Et c'est aussi pour exactement la même raison qu'il est légitime de considérer
que \ref{x:3anglea} et \ref{x:3angleb} sont de très bonnes paraphrases
l'une de l'autre :

\ex.
\a.
Un triangle a trois côtés. \label{x:3anglea}
\b.
 Tout triangle a trois côtés. \label{x:3angleb}


Évidemment en \ref{xlion2}, nous n'observons aucune quantification
universelle ; la généricité n'est pas à l'\oe uvre dans ces phrases. Et
il importe de bien constater que, pour autant, les \GN\ indéfinis en
\ref{xlion2} n'ont pas forcément un usage spécifique : le locuteur peut
très bien  prononcer \ref{xlion2} sans avoir aucune idée de quel lion
particulier est l'auteur des méfaits relatés.

Il existe un test très efficace en français pour diagnostiquer une
interprétation générique dans une phrase qui contient un indéfini
singulier en position sujet.  C'est le test\is{test!\elid\ de la dislocation en \sicut{ça}} de la
dislocation\is{dislocation} en \sicut{ça}% 
\footnote{\label{fn:dislo}La dislocation %\is{dislocation} 
est une construction  syntaxique dans laquelle 
les \GN\ pleins qui correspondent aux arguments (sujet ou compléments)
du verbe ne sont pas réalisés dans leur position syntaxique canonique,
mais dans une position «périphérique» initiale (dislocation à
gauche) ou finale (dislocation à droite) de la phrase. La position
canonique des éléments disloqués est alors occupée par un pronom
personnel. Exemples : \sicut{Paul, il est parti à midi} ; \sicut{Je ne
  l'ai pas vu, Paul}. Le type de dislocation mentionné ci-dessus est
particulier du fait que le pronom sujet utilisé n'est pas \sicut{il} ou
\sicut{elle}, mais \sicut{ça} (ou \sicut{ce}).}
du
\GN\ indéfini, illustré en \ref{xlionça}.  Ce type de construction
très particulière n'est compatible qu'avec une interprétation
générique, comme le montre le contraste avec les exemples
\ref{xlionça*}. 

\ex. \label{xlionça}
\a. Un lion, c'est paresseux
\b. Un lion, ça mange de la viande.

\ex. \label{xlionça*}
\a.* Un lion, ça a saccagé le canapé.
\b. * Un lion, ça a mangé de la viande.


La question qui se présente à nous alors est : serait-on (encore une
fois !) en présence d'une ambiguïté  de l'article indéfini\is{indefini@indéfini} singulier en
français ?  Aurait-on deux homonymes :  le \sicut{un} existentiel non
générique et le \sicut{un} universel générique ? La réponse est non ;
cette option n'est pas retenue, et nous allons voir quelques arguments
qui indiquent que l'on peut rendre compte de ce phénomène de
généricité sans postuler une telle ambiguïté. %de l'article indéfini.

On peut d'abord remarquer un parallélisme intéressant avec les
\GN\ définis singuliers comme en \ref{xlion3} et \ref{xlion4} (à
comparer avec \ref{xlion1} et \ref{xlion2}).

\ex. \label{xlion3}
\a. Le lion est un mammifère.  \label{xlion3a}
\b. Le lion  est paresseux.  
\b. Le lion mange de la viande.

\ex.  \label{xlion4}
\a. Le lion a saccagé le canapé. \label{xlion4a}
\b. Le lion a mangé de la viande.


Nous n'avons pas encore regardé de près la sémantique des {\GN}
définis comme \sicut{le lion}, mais on peut dores et déjà 
constater que dans \ref{xlion3a}, il n'est pas fait référence à un lion
particulier et singulier, contrairement à \ref{xlion4a}. On pourrait
en conclure que le phénomène de généricité qu'on avait en \ref{xlion1} se
reproduit ici en \ref{xlion3} avec l'article défini. En réalité, les
choses ne sont pas aussi simples : ce n'est pas exactement le même type de
généricité qui se manifeste en \ref{xlion3}\footnote{Rendez-vous au
  chapitre \ref{GN++} (vol.~2) pour plus détails sur ce point.}.  Mais cette
symétrie (au moins apparente) entre \ref{xlion1}--\ref{xlion2} et
\ref{xlion3}--\ref{xlion4} a le mérite de suggérer que ce n'est
peut-être pas sur le \GN\ sujet qu'est localisée la source de la
généricité. 

En effet si l'hypothèse de l'ambiguïté de l'article indéfini était
correcte, on pourrait s'attendre à retrouver assez systématiquement
cette ambiguïté au niveau des phrases qui contiennent cet article, et
qui pourraient être  ainsi interprétées soit génériquement soit non
génériquement. De telles phrases existent, mais il est d'abord
intéressant de noter que les phrases \ref{xlion2}, elles, ne sont en
aucun cas ambiguës à cet égard (c'est d'ailleurs ce que montre
\ref{xlionça*}). 
En revanche, on peut admettre que les phrases
\ref{xlion1c} et \ref{xchatm} sont ambiguës : elles peuvent
aussi avoir une interprétation non générique (même si celle-ci est {a
priori} moins naturelle ou moins immédiate).

\ex. \label{xchatm}
Un chat miaule.


La dislocation en \sicut{ça} permet de faire apparaître la
lecture générique dans ce genre de phrase (\sicut{un chat, ça
miaule}). À l'inverse, il y a une autre construction qui ne peut
exprimer que la lecture existentielle non générique.  Il s'agit de la
tournure dite justement... existentielle, de la forme 
\sicut{il y a} suivi du \GN\ et d'une proposition relative, comme
illustré en \ref{xilya}. Cela nous fournit un nouveau test :
lorsqu'une phrase est sémantiquement équivalente à sa variante en
\sicut{il y a}, c'est que la lecture existentielle est disponible et
que c'est celle-ci qui est prise en compte. 

\ex. \label{xilya}
\a. Il y a un chat qui miaule. \label{xilyaa}
\b. \juge\zarb Il y a un lion qui est un mammifère. \label{xilyab}
\b. \juge{\urgh} Il y a un lion qui est \label{xilyac}paresseux.
\b. Il y a un lion qui mange de la viande.
\b. Il y a un lion qui a saccagé le canapé. \label{xilyae}
\b. Il y a un lion qui a mangé de la viande.



Cette construction existentielle n'est pas compatible avec la
généricité, car, comme on peut l'observer, la relative y exprime
toujours quelque chose de particulier : souvent un événement ou une
action qui a lieu à un moment donné (un miaulement \ref{xilyaa}, un
saccage \ref{xilyae}, etc.),
ou simplement un fait singulier.  
Et avec la généricité, d'une certaine manière, il ne se passe rien ;
on se contente d'assigner une propriété.
%\ref{xilyab} et \ref{xilyac} sont bizarres justement parce qu'elles
%ne restituent pas l'interprétation générique de \ref{xlion1a} et
%\ref{xlion1b} : ex 
Ainsi en \ref{xilyac}, au mieux, on comprend que
parmi un groupe de lions, un en particulier se distingue des autres
par sa paresse --~ce qui n'est pas du tout l'interprétation générique
naturelle de \ref{xlion1b}.  Et pour \ref{xilyab}, on ne peut
carrément pas obtenir une interprétation similaire : notre
connaissance du monde exclut de considérer \sicut{être un mammifère}
comme une propriété particulière d'un individu qui l'opposerait à ses
congénères.


Ces observations tendent à montrer qu'en fait, ce n'est pas le
déterminant indéfini qui cause la lecture générique, mais un autre
élément de la phrase.  C'est d'ailleurs pourquoi il est préférable de
ne pas parler de {\GN} génériques pour \ref{xlion1}, mais plutôt de
\kwi{phrases génériques}{generique@générique!phrase \elid}.  Cet
élément qui déclenche la lecture générique est localisé sur le groupe
verbal et est associé au temps qui le conjugue.  En effet, les phrases
\ref{xlion1} sont conjuguées au présent avec cette valeur que, dans
les grammaires traditionnelles, on appelle justement «de
vérité... \emph{générale}~». Les phrases existentielles de \ref{xlion2}
et celles qui fonctionnent bien en \ref{xilya} sont conjuguées au
passé composé ou au présent dit actuel, qui ont une valeur épisodique
et donc particulière%
\footnote{Le passé simple a aussi cette valeur
  épisodique qui induit l'interprétation existentielle. L'imparfait et
le futur sont un peu plus neutres à cet égard : il peuvent parfois
donner une lecture générique, même si leur ancrage temporel favorise
souvent la lecture existentielle.}.
D'ailleurs le rôle du temps verbal et ces deux valeurs distinctes
du présent apparaissent très 
clairement dès que l'on traduit nos phrases en anglais : le présent
«générique» se traduit par le présent simple \ref{xAcata}, et le
présent «épisodique» par une forme progressive \ref{xAcatb}.

\largerpage

\ex.
\a. A cat meows. (ou Cats meow.)\label{xAcata}\\ 
Un chat miaule. \qquad (générique)
\b. A cat is meowing.\label{xAcatb}\\ 
Un chat miaule. \qquad (existentielle)\


Mais maintenant que nous avons débusqué la source de la généricité
dans le groupe verbal, que pouvons nous conclure sur l'analyse
sémantique des {\GN} indéfinis dans les phrases génériques ?  Pour
cela, nous devons faire le rapprochement avec le phénomène de
covariation causé par les \kwi{adverbes de quantification}{adverbe de quantification} qui a été abordé en
\S\ref{ss:PorteeLE} (p.~\pageref{AdvQ1l}).  Il s'agit d'adverbes
(habituellement qualifiés de temporels dans les grammaires) comme
\sicut{toujours}, \sicut{souvent}, \sicut{parfois}, \sicut{rarement},
\sicut{jamais}..., ainsi que de syntagmes adverbiaux  qui contiennent
une idée de quantification (\sicut{quelques fois}, \sicut{de temps en
  temps}, \sicut{la plupart du temps}...).
Nous avions vu que ces expressions ont le pouvoir de faire
covarier un indéfini singulier qui se trouve dans leur portée.
% ,qui ainsi pouvait toujours s'analyser
% au moyen d'un quantificateur existentiel. 
Le résultat n'est pas forcément une phrase
générique, mais l'indéfini perd, comme il se doit, sa valeur de
singulier (puisqu'il covarie).  C'est ce qu'illustrent les phrases de 
\ref{xAdvQindef2} :


\ex. \label{xAdvQindef2}
\a. Le soir, à la fermeture du magasin, \emph{un client} vient
\emph{souvent} me  déranger.
\b. \emph{De temps en temps}, Pierre fume \emph{un cigare} après le
dîner. 
\b. \emph{Un roman de Stephen King} fait \emph{rarement} moins de 600 pages.
%\b. Une équation quadratique a souvent deux solutions différentes. 
\b. \emph{Une mère} aime \emph{toujours} son enfant.%Un prof est toujours fatigué.
\label{xprof0} 


Deux choses importantes sont à noter ici. D'abord, dans ces
constructions, l'indéfini s'analyse de façon ordinaire, au moyen d'une
quantification existentielle. C'est précisément ce qui lui permet de
covarier (sous la portée du quantificateur contenu dans l'adverbe).
Ensuite il faut bien remarquer que, bien que ces adverbes soient
formellement liés à la temporalité, ils ne quantifient pas
systématiquement sur des instants, des moments, des périodes ou des
occasions.  Ainsi dans \ref{xprof0}, \sicut{toujours} ne signifie pas
vraiment \sicut{perpétuellement} ou \sicut{à tout moment} ou \sicut{éternellement}. 
En fait l'interprétation de \ref{xprof0} nous fait comprendre qu'ici
\sicut{toujours} sert simplement à «fabriquer» une
quantification universelle sur \sicut{une mère} : \ref{xprof0} signifie
\sicut{toute mère aime son enfant}.  Il est d'ailleurs remarquable de
constater que, dans ce type d'interprétation, à tout adverbe de
quantification correspond, en fonction de sa valeur
quantificationnelle propre, un quantificateur nominal (\ie\ un
certain déterminant). C'est ce
qu'illustrent les paires d'équivalences en \ref{xAdvQ-QNom} : 

\ex.  \label{xAdvQ-QNom}
\begin{tabular}[t]{@{}llll}
a. & Un prof est toujours sévère. & = & Tous les profs sont sévères.\\
b. & Un prof est souvent sévère. & = & {Beaucoup\footnotemark} de profs sont sévères.\\
c. & Un prof est parfois sévère. & = & Quelques profs sont sévères.\\
d. & Un prof est rarement sévère. & = & Peu de profs sont sévères.\\
e. & Un prof n'est jamais sévère. & = & Aucun prof n'est sévère.\\
\end{tabular}%
\footnotetext{Dans certains cas, le déterminant \sicut{la plupart de} peut aussi valoir comme équivalent de \sicut{souvent}, même si \sicut{la plupart} et \sicut{beaucoup} n'ont exactement le même sens (voir \S\ref{s:QG}).}

Deux questions restent ouvertes ici. D'abord nous savons que ces
adverbes introduisent un quantificateur, mais, de façon générale, nous
ne savons pas exactement sur quoi il quantifie. C'est une question
assez complexe, et nous la laisserons en suspens ici\footnote{Mais là
  encore, je renvoie à \citet{Lewis:75}\Andexn{Lewis, D.} pour l'approche originale de
  l'analyse sémantique de ce phénomène.}.  Ensuite nous ne disposons
pas dans {\LO} de symbole de quantification pour traduire
\sicut{souvent}/\sicut{beaucoup}/\sicut{la plupart} et
\sicut{rarement}/\sicut{peu}.  Mais cela va venir ; nous verrons
comment leur attribuer une sémantique assez simple en \S\ref{s:QG}.  Ce qui nous intéresse pour le moment
c'est seulement que l'adverbe fait covarier le quantificateur
existentiel introduit par l'indéfini qui se trouve dans sa portée.

Et maintenant nous pouvons revenir à nos phrases génériques, car nous
avons en main les éléments nécessaires pour  expliquer leur
fonctionnement sémantique. 
Leur analyse repose sur l'hypothèse suivante : dans la structure
sémantique d'une phrase générique, le temps verbal (en l'occurrence le
présent générique) s'accompagne
d'un quantificateur dont la contribution sémantique est similaire à
celle d'un adverbe de quantification avec une
portée maximale.
Ou pour dire les choses plus simplement, les phrases génériques
contiennent un adverbe de 
quantification implicite\footnote{En fait cet adverbe n'est pas
  forcément implicite ; nous allons tout de suite voir des
exemples où il est explicitement exprimé dans la phrase.}.
De quel adverbe s'agit-il ? A priori, on pourrait penser qu'il s'agit
de \sicut{toujours}, puisque les phrases génériques expriment une
quantification universelle. Cela serait d'ailleurs confirmé par
l'exemple \ref{x:3anglea} : \sicut{un triangle a \emph{toujours} trois
côtés}\footnote{Mais ici on peut envisager un autre adverbe qui, à sa
manière, est aussi un adverbe de quantification :
\sicut{nécessairement}. Il convient pour \ref{x:3anglea} (un triangle
a \emph{nécessairement} trois côtés) et pour \ref{xlion1a} (un lion
est \emph{nécessairement} un mammifère). \sicut{Nécessairement} est un adverbe
  modal qui contient une quantification universelle. Nous aborderons
  les modalités le chapitre~\ref{Ch:t+m}. Ici il faut comprendre
  \sicut{nécessairement} au sens de 
  \sicut{par définition}.}.
Mais nous devons constater que la généricité ne correspond pas
forcément à une quantification telle que celle exprimée par
\sicut{toujours}. En effet, comparez
\ref{x:4roues1}, qui est typiquement une phrase générique, à
\ref{x:4roues2}, qui équivaut à \ref{x:4roues3}.

\ex.
\a. Une voiture a quatre roues. \label{x:4roues1}
\b. Une voiture a toujours quatre roues. \label{x:4roues2}
\b. Toute voiture a quatre roues. \label{x:4roues3}


Intuitivement, on perçoit assez bien une nuance qui distingue les
conditions de vérité de \ref{x:4roues1} de celles de
\ref{x:4roues2}. Dans notre monde, il y a des voitures qui n'ont pas
quatre roues\footnote{Celle de Mr. Bean en a trois, par exemple.} ;
donc on peut assez légitimement estimer que, dans notre monde, la
phrase de \ref{x:4roues2} est fausse.  En revanche, on s'autorise
aussi facilement à juger \ref{x:4roues1} vraie : l'existence de
quelques voitures qui n'ont pas quatre roues ne suffit pas,
semble-t-il, à falsifier la phrase. C'est que \ref{x:4roues1} parle
des voitures \emph{en général}, des voitures typiques ou prototypiques, des
voitures normales. Par conséquent, sémantiquement, la généricité
correspond à une quantification universelle un peu spéciale, qui
peut tolérer des exceptions (au moins dans certains cas). 

C'est pourquoi les adverbes  de quantification qui, par défaut, sont
sous-jacents dans les phrases génériques standards sont plutôt
\sicut{généralement},  
\sicut{en général}, \sicut{normalement}, \sicut{typiquement}... 


\ex. \label{x:AdvGen}
\(\left.\begin{array}{@{}l}
\text{Généralement} \\ \text{En général} \\ \text{Normalement} 
\\ \text{Typiquement}
\end{array}\right\}\)
une voiture a quatre roues.


On remarquera d'ailleurs que cette forme de généricité n'est pas sans
rapport avec ce que l'on pourrait appeler l'habitualité : dans une
certaine mesure (et dans certains cas), l'adverbe
\sicut{habituellement} peut aussi fonctionner dans le paradigme
\ref{x:AdvGen}. 
Et ce que tendent à montrer ces exemples, finalement, c'est que la
généricité phrastique n'est probablement qu'un cas particulier d'un
phénomène sémantique plus large, celui de la quantification induite
par ce type d'adverbes% 
\footnote{Ainsi on peut même être tenté d'envisager une échelle de
  gradation qui ordonnerait les différentes forces
  quantificationnelles exprimées par les adverbes : à une extrémité de
  l'échelle on aurait l'universalité stricte (\sicut{toujours},
  \sicut{nécessairement}), ensuite une universalité un peu «molle» (\sicut{généralement}, \sicut{normalement}), puis une haute
  fréquence (\sicut{souvent}, \sicut{habituellement}), puis une
    fréquence moyenne (\sicut{parfois}), une fréquence faible
    (\sicut{rarement}), jusqu'à  l'universalité négative
    (\sicut{jamais}).  L'idée d'une telle échelle n'est pas absurde,
    mais elle n'est probablement pas suffisante, et l'histoire est un
    peu plus compliquée que cela. Il a été suggéré (par ex.\ \citealt{Generic:Intro})\Andexn{Krifka, M.}\nocite{Generic} que certains de ces
    adverbes (et en particulier les adverbes génériques) ont des
    propriétés modales qui les distinguent singulièrement des autres.}.
À ce moment-là, la particularité de la généricité est avant tout qu'elle 
peut apparaître par défaut, c'est-à-dire sans adverbe explicite.


Nous nous arrêtons là pour le moment sur la généricité\footnote{Les lecteurs intéressés par le phénomène peuvent se reporter, entre autres, à \citet{Generic:Intro},\Andexn{Krifka, M.} \citet{Cohen:02},\Andexn{Cohen, A.} \citet{Sorin:06g}\Andexn{Dobrovie-Sorin, C.} pour un panorama plus approfondi.} ; 
nous avons atteint ce que nous visions, à savoir, je le répète ici, que dans les phrases génériques, il n'y a pas de raison d'analyser les {\GN} indéfinis singuliers différemment que dans les phrases ordinaires.
Ceux-ci correspondent toujours à des quantificateurs existentiels ; simplement, par l'effet d'un adverbial implicite ou explicite, ils se retrouvent multipliés sur un «mode de généralité».


\is{genericite@généricité|)} 


\section{Catégories de groupes nominaux}
%=======================================
\label{s:CatGN}

Dans la section précédente (\S\ref{s:GNportée}), nous avons vu que
certains {\GN} avaient une portée, que cette portée pouvait
être variable et que l'interprétation de ces {\GN}
était mobile.  Nous avons vu aussi que l'interprétation de certains
{\GN} pouvait covarier avec d'autres expressions de la
phrase. Ces phénomènes, du moins dans les cas simples, sont
correctement restitués dans les traductions en {\LO} grâce à la
sémantique que nous avons attribuée aux quantificateurs. La question
que nous pouvons nous poser à présent est : est-ce que ces
observations et l'analyse sémantique formelle qui les accompagne
s'appliquent à tous les {\GN} ? Autrement dit, est-ce que
tous les {\GN} du français peuvent s'analyser de la même
façon, au moyen de quantificateurs ? On se doute bien que ce n'est pas
le cas : ne serait-ce que pour les noms propres et les pronoms, nous avons  déjà vu qu'ils se traduisaient respectivement par des
constantes et  par des variables. Cette question en
engendre alors une seconde qui est : existe-t-il différentes
catégories sémantiques de {\GN} et si oui, lesquelles ?  Et ce qui
pourrait être intéressant pour nous, c'est de voir si ces catégories
sémantiques sont corrélées à, disons, des catégories grammaticales de
\GN. Cela nous fournirait en effet un guide intéressant pour traduire
les phrases en {\LO} : on saurait que telle catégorie grammaticale
bien identifiée de
{\GN} correspond à une catégorie sémantique particulière et donne
ainsi lieu à un type de traduction précis.


\subsection{Critères logico-sémantiques}
%---------------------------------------


\subsubsection{Effets de portée}
%'''''''''''''''''''''''''''''''
\label{sss:effportee}

Une première chose que nous pouvons regarder c'est quels sont les
{\GN} qui induisent de la covariation. Mais il faut être assez
vigilant sur les conclusions que nous pouvons tirer de ces
observations. Si un {\GN} induit de la covariation, nous pourrons en
conclure qu'il a probablement une portée et que son interprétation met
en jeu, d'une manière ou d'une autre, un calcul répété de la valeur
sémantique de ce qui se trouve dans sa portée. Mais si un {\GN}
n'induit pas de covariation, nous ne devrons pas nécessairement en conclure qu'il n'a pas de portée. 
On s'en doute bien, pour induire de la covariation par lui-même, un {\GN} doit être associé à une idée de pluralité\footnote{Même s'il est grammaticalement singulier, comme avec \sicut{chaque} ou \sicut{tout}.}, et donc les {\GN} fondamentalement singuliers, se ramenant au mieux à «une multiplication par 1», ne risquent pas vraiment de provoquer de la covariation. Pourtant, nous l'avons vu, les indéfinis singuliers ont tout de même une portée.
Donc si ce qui nous intéresse est de détecter les {\GN} qui ont une portée, il y a certainement des stratégies plus robustes ; pour autant la propriété de pouvoir induire de la covariation n'est pas complètement sans intérêt et sémantiquement, et nous reviendrons sur ce test via l'exercice~\ref{exoInduCovar}, p.~\pageref{exoInduCovar} \alien{infra}.

Nous pouvons également tester les {\GN} qui sont susceptibles ou non de \emph{subir} de la covariation. Mais là encore il faut bien faire attention aux conclusions à tirer. D'abord qu'un {\GN} puisse ou non covarier, cela ne nous dit rien au sujet de sa portée potentielle.  Mais ça n'est pas grave, car ce test peut commencer à faire apparaître des catégories pertinentes de {\GN}, en particulier ceux qui semblent ne jamais covarier. Dans un premier temps nous ne pourrons donc que constater l'émergence de ces catégories, sans tirer de conclusion plus avancée, et nous devrons par la suite essayer d'expliquer ce phénomène au moyen d'une analyse sémantique satisfaisante.
Il faut également prendre quelques précautions sur l'application du test, car en fait il se trouve qu'il existe deux mécanismes sémantiques distincts qui causent un effet de covariation, ou de multiplication référentielle (nous y reviendrons en fin de chapitre) ; et cela nous induirait un peu en erreur si nous confondions les deux.  À cet égard, le plus sûr est d'observer les covariations possibles de {\GN} par rapport à un adverbe (ou adverbial) de quantification comme \sicut{souvent}, \sicut{parfois}, \sicut{toujours}, \sicut{à plusieurs reprises}, etc. 
Le principe du test\is{test!\elid\ de covariation} est assez simple, il suffit d'abord de placer le {\GN} à examiner dans une phrase où on s'attend qu'il puisse se retrouver dans la portée d'un adverbe de quantification. C'est ce que présente la série \ref{xGNsubitcv} en plaçant l'adverbial \sicut{à plusieurs reprises} en tête de phrases. Ensuite, pour chacune des phrases, on se demande si le {\GN} en italique renvoie toujours à la même chose parmi les différentes «reprises» évoquées. Autrement dit, est-ce que chaque phrase, prise une par une, nous parle du même prisonnier ou du même groupe de prisonnier ? Si c'est le cas, alors c'est que le {\GN} échappe à la covariation ; si au contraire il peut s'agir de prisonniers différents, c'est que le {\GN} covarie (au moins dans une des interprétations de la phrase ; mais cela nous suffit). 
Dans la liste ci-dessous j'utilise exceptionnellement la marque {\mcovar} pour 
indiquer les phrases qui font apparaître de la covariation.


\ex. \label{xGNsubitcv}
\a. À plusieurs reprises, \emph{Joe} a tenté de s'évader. 
\b. À plusieurs reprises, \emph{il} a tenté de s'évader.\label{xpriso-b}  
\b. À plusieurs reprises, \emph{le prisonnier} a tenté de s'évader.\label{xpriso-c} 
\b. À plusieurs reprises, \emph{ce prisonnier} a tenté de s'évader. 
\b. À plusieurs reprises, \emph{mon prisonnier} a tenté de s'évader.%
\footnote{\label{fnmonpriso}On peut voir ici un effet de covariation si on se place dans une situation où, par exemple, le locuteur est un geôlier qui a toujours eu la charge de ne surveiller qu'un seul prisonnier à la fois, mais que, au long de sa carrière, il ait eu différents prisonniers à surveiller. Avec ce genre de situation, on peut d'ailleurs même trouver un effet de covariation aussi pour \ref{xpriso-c} (ainsi que pour \ref{xpriso-f} \alien{mutatis mutandis}). Dans ce cas, ce n'est pas nécessairement le {\GN} lui-même qui covarie, mais peut-être un autre paramètre de la structure sémantique de la phrase (nous aurons l'occasion de voir plus tard de quoi il peut s'agir). %***
Pour le moment, pour ne pas fausser le test, faisons l'hypothèse que chacune des phrases se réfère implicitement à, disons, une certaine situation de captivité qui reste toujours la même.} 
\b. \jcovar À plusieurs reprises, \emph{un prisonnier} a tenté de s'évader. 
\b. \jcovar À plusieurs reprises, \emph{trois prisonniers} ont tenté de s'évader. 
\b. \jcovar À plusieurs reprises, \emph{des prisonniers} ont tenté de s'évader. 
\b. \jcovar À plusieurs reprises, \emph{plusieurs prisonniers} ont tenté de s'évader. 
\b. \jcovar À plusieurs reprises, \emph{quelques prisonniers} ont tenté de s'évader. 
%\b. À plusieurs reprises, \emph{certains prisonniers} ont tenté de s'évader. 
\b. \jcovar À plusieurs reprises, \emph{la plupart des prisonniers} ont tenté de s'évader.\label{xpriso-l}   
\b. À plusieurs reprises, \emph{tous les prisonniers} ont tenté de s'évader.\label{xpriso-m}


On voit se dessiner deux grands groupes de {\GN} (et donc de déterminants) \Last[a--e] et \Last[g--l].  Il semble assez clair (en tenant compte de la note de bas de page \ref{fnmonpriso}) que les {\GN} du premier groupe dénotent fixement un individu %(ou un groupe d'individus) 
et ne donnent ainsi pas prise à la covariation. Par exemple dans \ref{xpriso-b}, bien qu'on ne sache pas qui est exactement ce \sicut{il}, on est sûr qu'il s'agit toujours de la même personne. Au contraire, dans le deuxième groupe, il ne s'agit pas forcément du même ou des mêmes prisonnier(s) qui tente(nt) de s'évader à chaque fois ; les {\GN} covarient. Y compris pour \ref{xpriso-l}, où il est tout à fait possible qu'à chaque tentative d'évasion, les prisonniers impliqués ne soient pas toujours \emph{tous} les mêmes.
Quant au {\GN} \sicut{tous les prisonniers} en \ref{xpriso-m}, à cause de sa force universelle, il est justement trop fort pour covarier : quoi qu'il arrive, il englobe toujours la totalité des prisonniers
(et c'est un peu ce que nous avions déjà observé avec l'exemple \ref{x:QAAa}, \S\ref{ss:PorteeLE} p.~\pageref{x:QAAa}). Notre test le rattache donc au premier groupe de {\GN} ; et nous verrons qu'il y a d'autres bonnes raisons de faire ce rattachement. Cependant nous allons immédiatement voir qu'il existe également de bonnes raisons de le tenir un peu à l'écart, ou plus exactement de procéder à des regroupements un peu différents de ceux de \ref{xGNsubitcv}.

%\smallskip
\largerpage

Nous avions constaté, en \S\ref{ss:PorteeLE} (p.~\pageref{x:neg+Q}), que les {\GN} qui correspondent à des quantificateurs, du fait de leur interprétation mobile, induisent une ambiguïté vis-à-vis de la négation. C'est une propriété qu'il est intéressant de tester pour d'autres {\GN}, car cela constitue un test\is{test!\elid\ de la négation} qui renforce et affine le précédent. De cette façon, ce que nous diagnostiquons est à la fois un cas limite, mais radical, de covariation (une sorte de multiplication par 0) et la possible mobilité interprétative des {\GN}. Le principe du test est simple : nous plaçons des {\GN} dans une phrase négative\footnote{Le plus sûr est de le mettre dans une position objet ou oblique, car la position sujet peut parfois, pour d'autres raisons, exclure la portée inversée du {\GN} et de la négation.}, et nous nous demandons si la phrase est ambiguë. Marquons ci-dessous par {\mambig} les phrases qui présentent une ambiguïté :

\ex. \label{xGNambigNeg}
\a. Jean n'a pas lu \emph{Guerre et Paix}.
\b. Jean ne \emph{l'}a pas lu.
\b. Jean n'a pas lu \emph{le dossier}.
\b. Jean n'a pas lu \emph{ce dossier}.
\b. Jean n'a pas lu \emph{mon dossier}.
%\b. Jean n'a pas lu \emph{les dossiers}.
\b. \jambig Jean n'a pas lu \emph{un dossier}.
\b. \jambig Jean n'a pas lu \emph{trois dossiers}.
\b. \jambig Jean n'a pas lu \emph{des dossiers}.
\b. \jambig Jean n'a pas lu \emph{plusieurs dossiers}.
\b. \jambig Jean n'a pas lu \emph{quelques dossiers}.
\b. \jambig Jean n'a pas lu \emph{la plupart des dossiers}.
\b. \jambig Jean n'a pas lu \emph{tous les dossiers}.


Le test reprend \emph{presque} la même subdivision que \ref{xGNsubitcv}, avec l'exception notable de \sicut{tous les N} qui est passé dans le second groupe. 
Mais comme les jugements de {\Last} ne sont pas tous entièrement triviaux, faisons quelques commentaires. 

Pour \Last[a--e], il semble assez clair que les phrases ne sont pas ambiguës. 

Pour \Last[f], la phrase peut signifier soit \sicut{il y a un dossier que Jean n'a pas lu} (mais il peut en avoir lu d'autres), soit \sicut{il n'a lu aucun dossier}.  Cette deuxième lecture peut sembler moins naturelle ou moins fréquente, certainement pour des raisons pragmatiques, en particulier parce qu'il existe des manières moins équivoques de faire passer ces conditions de vérité. Par exemple, cette lecture apparaît, sans ambiguïté, si on ajoute \sicut{seul} dans le {\GN} : \sicut{Jean n'a pas lu un seul dossier} ; ou si l'on  accentue  le déterminant : \sicut{Jean n'a pas lu \textsc{un} dossier}.  Mais on ne peut pas exclure que \Last[f] en lui-même possède cette interprétation. 

Pour \Last[g], c'est un peu similaire, la phrase peut se comprendre comme \sicut{il y a trois dossiers que Jean n'a pas lus} (mais il peut en avoir lu une dizaine d'autres par ailleurs), ou comme \sicut{Jean a lu moins de trois dossiers}. 

L'exemple \Last[h] est, quant à lui, différent ; d'abord la phrase semble, à première vue\footnote{Ou à première ouïe...}, sonner un peu bizarrement d'un point de vue grammatical, mais il existe deux types de circonstances où elle peut s'énoncer. Soit pour signifier qu'il y a des dossiers que Jean n'a pas lu ; soit pour dire quelque chose que l'on pourrait paraphraser un peu grossièrement en \sicut{ce n'est pas des dossiers que Jean a lu} (mais par exemple des bandes dessinées). Cette deuxième lecture implique, de fait, que Jean n'a lu aucun dossier\footnote{Ajoutons que, comme on le sait, il existe également en français la formulation \sicut{Jean n'a pas lu de dossier(s)}, mais elle n'est pas très pertinente pour notre test car cette forme \sicut{de N} est précisément celle que prend ordinairement l'indéfini pluriel lorsqu'il doit s'interpréter sous la portée d'une négation. Il n'y a alors pas d'ambiguïté possible.}. 

\Last[i], avec \sicut{plusieurs}, fonctionne un peu comme \Last[g], sa deuxième lecture étant \sicut{Jean a lu au plus un dossier} ; et \Last[j], avec \sicut{quelques}, lui, fonctionne plutôt comme \Last[h]. 

Le cas de \Last[k], avec \sicut{la plupart}, est un peu plus difficile, car l'ambiguïté est assez fine. Mais elle existe bel et bien. Pour la mettre au jour, il nous faut d'abord faire une hypothèse sur le sens de \sicut{la plupart} ; cela ne va pas absolument de soi, même (ou surtout) si nous nous fions à notre intuition, et c'est peut-être une question assez difficile à trancher. Accordons-nous simplement à considérer que nous ne prendrons pas trop de risque en posant que \sicut{la plupart} signifie \sicut{au moins} plus de la moitié ; cela nous suffira ici.
Supposons maintenant, par exemple, que Jean avait douze dossiers à lire. S'il n'en a lu que cinq, ou quatre, ou trois, etc.\ alors nous jugerons que la phrase \Last[k] est vraie. C'est que nous l'interprétons comme signifiant \sicut{plus de la moitié des dossiers sont non lus}, autrement dit \sicut{la plupart des dossiers sont non lus}. L'autre lecture est moins saillante, elle se glose en \sicut{il est faux que Jean a lu la plupart des dossiers}, ce qui revient à \sicut{il n'en a pas lu plus de la moitié}. Cette lecture sera vraie si, dans notre exemple, Jean a lu exactement six dossiers, car six ce n'est pas \emph{plus} de la moitié de douze.  Il s'agit bien d'une véritable ambiguïté, car dans ce dernier cas de figure, la phrase sera en même temps fausse avec la première lecture. En effet si Jean a lu exactement six dossiers, alors il y a exactement six dossiers qu'il n'a pas lus, et donc il n'y a pas plus de la moitié des dossiers qui sont non lus.
Certes l'ambiguïté est assez subtile, car il n'y a qu'un seul type de cas de figure qui discrimine les deux lectures, ceux où le partage se fait exactement à 50-50.
Dans les autres cas, les deux lectures ne se distinguent pas\footnote{Notons que si nous avions fait l'hypothèse plus audacieuse (mais probablement incorrecte) que \sicut{la plupart} signifie plus des deux tiers ou plus des trois quarts, alors nous aurions plus de configurations discriminantes.}. Pour autant cela suffit à prouver l'ambiguïté.

Enfin l'ambiguïté de \Last[l], avec \sicut{tous les}, a déjà été observée en 
\S\ref{ss:PorteeLE}
(exemple \ref{x:neg+Q}, p.~\pageref{x:neg+Q}). Là encore une des deux lectures s'impose plus immédiatement que l'autre.  C'est celle qui signifie qu'il est faux que Jean ait lu tous les dossiers, autrement dit qu'il y a au moins un dossier qu'il n'a pas lu.  L'autre équivaut à \sicut{Jean n'a lu aucun dossier}.



Pour récapituler, nous voyons émerger deux grands groupes de {\GN}, et donc de déterminants. 
Cela n'a rien de bien surprenant ; cela confirme, d'une certaine manière, ce que nous avons dans {\LO}. Le premier  groupe contient ce qui correspond aux constantes (N propres), aux variables (pronoms), et à des {\GN} que nous ne savons pas (encore) traduire (définis, démonstratifs, possessifs) ; le second groupe contient ce qui correspond aux quantificateurs ($\exists$ et $\forall$) et à d'autres {\GN} (\sicut{la plupart des...}), que nous ne savons pas traduire non plus.  
Mais, justement, nous commençons à nous faire une idée sur où ranger sémantiquement ces {\GN} que nous ne savons pas encore traduire. Et nous allons encore affiner notre classification.


\subsubsection{Tests de consistance et de complétude}
%''''''''''''''''''''''''''''''''''''''''''''''''''''
\is{test!\elid\ de consistance}\is{test!\elid\ de complétude}

Il existe une autre paire de tests souvent utilisés pour classer les {\GN}.
Ils exploitent deux lois logiques que nous avons vues au 
chapitre~\ref{LCP} dans l'exercice \ref{exotcc}, la loi de contradiction \ref{loi:contra} et la loi du tiers exclu \ref{loi:tiersx} :%
\is{loi!\elid\ de contradiction}\is{loi!\elid\ du tiers exclu}

\ex. \label{lois:c+te}
\a. \(\satisf \Xlo \neg[\phi\wedge\neg\phi]\)\label{loi:contra}
\b. \(\satisf \Xlo [\phi\vee\neg\phi]\)\label{loi:tiersx}


\newcommand{\Neg}{Neg}
La loi de contradiction dit qu'une formule de la forme $\Xlo[\phi\wedge\neg\phi]$ est toujours fausse et la loi du tiers-exclu qu'une formule de la forme $\Xlo[\phi\vee\neg\phi]$ est toujours vraie. Ces deux lois concernent {\LO}, mais il est assez facile de les importer dans une langue naturelle comme le français. Les tests que nous allons regarder ici consistent justement à observer l'application de ces lois dans des phrases du français... en trichant un peu. Mais qu'on se rassure, cette «triche» n'est pas déloyale, elle fait partie du principe même des tests (voir note~\ref{fn:triche} \alien{infra}). 
L'idée est la suivante. 
On considère d'abord des phrases à la structure syntaxique simple, de la forme [\GN\ \GV], où \GV\ représente un groupe verbal simple. Ensuite on les coordonne, avec \sicut{et} puis \sicut{ou}, à leur négation. Comme en français la négation ordinaire d'une phrase se positionne au niveau du {\GV} (avec \sicut{ne... pas} par exemple), la version négative des phrases de départ sera de la forme [\GN\ \Neg-\GV], où {\Neg} représente le  constituant qui apporte la négation.  À partir de là, on se demande si les phrases coordonnées, i.e.\ [\GN\ \GV\ \sicut{et} \GN\ \Neg-\GV] et [\GN\ \GV\ \sicut{ou} \GN\ \Neg-\GV], sont bien respectivement des contradictions et des tautologies. 
Et pour que les tests fonctionnent adéquatement, nous allons faire en sorte que {\Neg} s'interprète toujours localement, sur le prédicat verbal%
\footnote{\label{fn:triche}%
C'est là que l'entorse logique peut apparaître. Car les lois de \ref{lois:c+te} utilisent la négation $\Xlo\neg$ de {\LO}, qui porte sur des \emph{formules}, autrement dit des phrases complètes.  Or {\Neg} se présente plutôt comme une négation de «verbes». 
Comme nous le savons bien maintenant, certaines phrases ont des traductions sémantiques à la structure relativement complexe, notamment lorsqu'elles font intervenir des quantificateurs, et donc si la négation, via {\Neg}, s'interprète «collée» au prédicat verbal, nous nous doutons qu'elle n'aura pas le même effet sémantique que si elle s'interprète sur l'ensemble de la phrase comme en \ref{lois:c+te}. Il est donc facile de prédire, \emph{a priori}, que les deux lois ne s'appliqueront pas pour certains {\GN}. Mais c'est précisément ce que guettent les tests : si les lois ne s'appliquent pas c'est que les {\GN} en jeu ont très probablement une certaine mobilité interprétative vis-à-vis de la négation en donnant aux phrases une structure sémantique particulière.}.
Nous venons de le voir, la négation usuelle (\sicut{ne... pas}) interagit avec les {\GN} de manière parfois complexe (les phrases peuvent être ambiguës, les portées peuvent être inversées, etc.) ; pour garantir que {\Neg} porte seulement sur le {\GV}, il est très préférable de passer par une négation de constituant, comme avec l'adverbe \sicut{non} antéposé à un prédicat ou carrément une négation lexicale.
Je propose à cet égard que nous utilisions la paire d'antonymes \sicut{barbu} et \sicut{imberbe}, en faisait l'hypothèse que ces deux antonymes sont complémentaires, autrement dit qu'ils sont la négation l'un de l'autre%
\footnote{Il y a certainement des arguments légitimes pour contester cette hypothèse de sémantique lexicale, en pointant sur des cas intermédiaires. Mais pour les besoins du tests, autorisons-nous cette simplification (qui n'est pas entièrement déraisonnable) en considérant qu'un moustachu est imberbe et qu'un homme qui porte un collier de barbe ou un bouc est barbu. %Je ne sais pas dans quelle catégorie ranger Wolverine, donc nous ne le ferons pas intervenir dans le test.
}. Techniquement, cela peut se formaliser simplement, il suffit de contraindre tous nos modèles {\Modele} à satisfaire la condition suivante : pour toute assignation $g$, \(\denote{\Xlo\prd{imberbe}(x)}^{\Modele,g}=\denote{\Xlo\neg\prd{barbu}(x)}^{\Modele,g}\)\footnote{Nous pouvons, de façon équivalente, formaliser cela en posant que tout modèle {\Modele} doit être tel que \(\Modele \satisf \Xlo\forall x [\prd{imberbe}(x)\ssi \neg\prd{barbu}(x)]\). C'est qu'on appelle un \emph{postulat de signification} ; nous y reviendrons au chapitre \ref{Ch:t+m}.}.  Et on pourrait, de la même façon, faire le test avec \sicut{présent}/\sicut{absent}, \sicut{vivant}/\sicut{mort}, \sicut{malade}/\sicut{en bonne santé}, \sicut{qualifié}/\sicut{non qualifié}...


\largerpage

Dernière précaution à prendre pour se lancer dans les tests : 
chaque jugement de phrase doit se faire \emph{à contexte constant}. 
Nous allons voir dans un instant ce que cela veut dire et ce que cela implique en pratique.
La raison de cette précaution est que le contexte est un paramètre qui peut influer sur le sens des expressions ; or ce que nous visons ici c'est précisément d'examiner les propriétés du sens de certains {\GN} ; nous ne devons donc pas prendre le risque de faire  changer ce sens en cours du test, sans quoi nous ne saurions plus ce que nous observons exactement.



Le premier test est le \kwo{test de consistance}\is{test!\elid\ de consistance}, qui s'appuie sur la loi de contradiction ($\Xlo[\phi\wedge \neg\phi]$). Nous construisons des phrases de la forme [\GN\ \GV\ \sicut{et} \GN\ \Neg-\GV] comme dans la série \ref{test:contra}, et pour chacune nous nous demandons s'il s'agit bien d'une contradiction. 
La méthode est assez simple : il suffit à chaque fois de chercher s'il existe ne serait-ce qu'un modèle, ou un type de modèle, par rapport auquel on peut juger la phrase \emph{vraie}. Si l'on y parvient, c'est que la phrase n'est pas contradictoire ; si cela s'avère impossible, c'est qu'elle est bien une contradiction et donc qu'elle réussit le test. Dans ce dernier cas, nous le signalons en posant la marque ${}^{\bot}$ devant la phrase.

\ex. \label{test:contra}
\a. \juge{${}^{\bot}$} Pierre est barbu et Pierre est imberbe.
\b. \juge{${}^{\bot}$} Il est barbu et il est imberbe.
\b. \juge{${}^{\bot}$} Le candidat est barbu et le candidat est imberbe.
\b. \juge{${}^{\bot}$} Ce candidat est barbu et ce candidat est imberbe.\label{test:contrad}
\b. \juge{${}^{\bot}$} Mon candidat est barbu et mon candidat est imberbe.
%%\b. \juge{${}^{\bot}$} Les candidats sont barbus et les candidats sont imberbes.\label{test:contraf}
\b. Un candidat est barbu et un candidat est imberbe.
\b. Trois candidats sont barbus et trois candidats sont imberbes.
\b. Des candidats sont barbus et des candidats sont imberbes.
\b. Plusieurs candidats sont barbus et plusieurs candidats sont imberbes.
\b. Quelques candidats sont barbus et quelques candidats sont imberbes.
%\b. Certains candidats sont barbus et certains candidats sont imberbes.
\b. %\juge{${}^{\bot?}$} 
Aucun candidat n'est barbu et aucun candidat n'est imberbe.\label{test:contral}
\b. \juge{${}^{\bot}$} Chaque candidat est barbu et chaque candidat est imberbe.
\b. \juge{${}^{\bot}$} La plupart des candidats sont barbus et la plupart candidats sont imberbes.
\b. \juge{${}^{\bot}$} Tous les candidats sont barbus et tous les candidats sont imberbes.


La première chose à remarquer est que le test ne nous donne pas les mêmes regroupements que les précédents. Cette fois, nous isolons le groupe \Last[f--k]
qui ne contient ni \sicut{la plupart des N} ni \sicut{tous les N}.  

Le test est relativement simple à appliquer, mais pour certaines phrases néanmoins les jugements de (non-)contradictions ne sont pas entièrement triviaux et quelques remarques s'imposent. 
D'abord pour \ref{test:contrad}, avec le démonstratif, on peut a priori très facilement concevoir une situation qui rende la phrase vraie : une situation où les deux occurrences du {\GN} \sicut{ce candidat} sont utilisées par le locuteur pour désigner des candidats différents.  On le sait, les {\GN} démonstratifs s'accompagnent généralement d'une attitude qui, de la part du locuteur, établit une référence univoque vers un objet ; ce peut être un geste de pointage, un regard ou tout autre rapport à la situation d'énonciation qui rende un objet suffisamment saillant pour l'isoler. Il s'agit là d'une caractéristique certainement cruciale des démonstratifs, et on doit en tenir compte lorsque l'on est amené à spécifier le fonctionnement sémantique de ces expressions. Mais pour ce qui nous intéresse ici --~le test de consistance~-- nous allons nous contenter de remarquer que cette attitude «monstrative» du locuteur est un paramètre constitutif du contexte, puisque c'est une attitude qui caractérise le locuteur au moment où il produit son énoncé.  Et comme nous avons posé que le contexte doit être constant, il en résulte que \sicut{ce candidat} dénote à chaque fois le même individu. 
Et \ref{test:contrad} est alors bien contradictoire.

\newpage

En \ref{test:contral}, \sicut{aucun} a été ajouté au paradigme (il s'appliquait assez mal aux tests précédents), et le jugement indiqué peut sembler contre-intuitif. Mais il existe un type de situations qui rende la phrase vraie. Imaginons une audition à laquelle sont convoqués quatre candidats, mais que finalement aucun des quatre ne se présente ; il n'y alors pas de candidat dans le domaine de quantification. Dans ce cas, par la force des choses, les deux phrases connectées de  \ref{test:contral} sont vraies. 

%\paragraph{Test de complétude.} Basé sur la loi du tiers exclu $\satisf  [\phi\vee \neg\phi]$

\smallskip

Le second test est le \kwo{test de complétude},\is{test!\elid\ de complétude} qui utilise la loi du tiers exclu ($\Xlo[\phi\vee\neg\phi]$).
Cette fois, nous construisons des phrases de la forme [\GN\ \GV\ \sicut{ou} \GN\ \Neg-\GV] et nous vérifions si ce sont des tautologies ou non. 
La méthode est inversée : il faut chercher s'il existe des modèles qui peuvent rendre chaque phrase fausse. 


\ex. \label{test:compl}
\a. \juge{${}^{\top}$} Pierre est barbu ou Pierre est imberbe.
\b. \juge{${}^{\top}$} Il est barbu ou il est imberbe.
\b. \juge{${}^{\top}$} Le candidat est barbu ou le candidat est imberbe.
\b. \juge{${}^{\top}$} Ce candidat est barbu ou ce candidat est imberbe.
\b. \juge{${}^{\top}$} Mon candidat est barbu ou mon candidat est imberbe.
%\b. \juge{${}^{?}$} Les candidats sont barbus ou les candidats sont imberbes.
\b. Un candidat est barbu ou un candidat est imberbe.\label{test:complg}
\b. Trois candidats sont barbus ou trois candidats sont imberbes.
\b. Des candidats sont barbus ou des candidats sont imberbes.
\b. Plusieurs candidats sont barbus ou plusieurs candidats sont imberbes.
\b. Quelques candidats sont barbus ou quelques candidats sont imberbes.
%\b. Certains candidats sont barbus ou certains candidats sont imberbes.
\b. Aucun candidat n'est barbu ou aucun candidat n'est imberbe.
\b. Chaque candidat est barbu ou chaque candidat est imberbe.
\b. La plupart des candidats sont barbus ou la plupart candidats sont imberbes.\label{test:complm}
\b. Tous les candidats sont barbus ou tous les candidats sont imberbes.\label{test:complo}


Ici nous retombons \emph{presque} sur la subdivision obtenue en \ref{xGNsubitcv}. La différence notable est le classement de \sicut{tous les N}.
En général il est assez facile d'appliquer le test sur ces phrases. Par exemple, \ref{test:complo} est fausse dans un modèle qui contient des candidats barbus et des candidats imberbes ; c'est le cas aussi pour \ref{test:complm} mais en ajoutant la condition qu'il y ait exactement autant de barbus que d'imberbes. 

Pour \ref{test:complg}, cela saute un peu moins aux yeux, mais le cas de figure qui rend la phrase fausse est le même que pour \ref{test:contral} : une situation où il n'y a pas de candidat. 



Les tests de consistance et de complétude font donc apparaître trois grandes catégories de {\GN} : ceux qui satisfont les deux tests (comme les noms propres, \sicut{le N}...), ceux qui satisfont le premier mais pas le second (\sicut{tous les N}, \sicut{la plupart des N}...) et ceux qui ne satisfont aucun des deux (\sicut{un N}, \sicut{quelques N}...). 
Et il se trouve que cette tripartition correspond aux trois grandes classes de {\GN} traditionnellement identifiées en sémantique. 
La première classe est celle de ce que l'on appelle les \kwo{expressions référentielles}\is{expression!\elid\ referentielle@\elid\ référentielle} (ou {\GN} référentiels), la deuxième est celle des {\GN} dits \kwi{quantificationnels}{quantificationnel} et la troisième celle des \kwi{indéfinis}{indefini@indéfini} (ou {\GN} indéfinis).  
Ces dénominations montrent que les indéfinis constituent une classe à part, dissociée de celles de {\GN} quantificationnels, même si les indéfinis (au moins certains d'entre eux) s'analysent au moyen de la quantification existentielle $\Xlo\exists$.  Cela semble confirmer que la quantification existentielle a des propriétés interprétatives particulières que la quantification universelle n'a pas (et vice versa) 
--~ce que nous avions déjà pu observer en 
\S\ref{sss:limiteportée} sur les limites de portée large.
Les {\GN} quantificationnels ne correspondent, bien sûr, pas tous à de la quantification universelle, et nous verrons en \S\ref{s:QG}, comment les analyser formellement.


% -*- coding: utf-8 -*-
\begin{exo}\label{exo:DefPlur}
Dans le travail de classification des {\GN} que nous venons d'opérer, 
\pagesolution{crg:DefPlur}%
le défini pluriel, \sicut{les N}, a été volontairement omis. 
C'est qu'il manifeste des propriétés interprétatives assez particulières.
Appliquez les quatre tests présentés ci-dessus (covariation, ambiguïté avec la négation, consistance et complétude) sur un défini pluriel pour tenter de le rattacher à l'une de nos trois classes de {\GN}.  Qu'observe-t-on ? 
%
\begin{solu} (p.~\pageref{exo:DefPlur})\label{crg:DefPlur}

\noindent\emph{Test de covariation} (cf. \ref{xGNsubitcv}, p.~\pageref{xGNsubitcv}): on place le GN défini pluriel dans une phrase qui commence, par exemple, par \sicut{à plusieurs reprises}.

\begin{enumerate}[label=(\arabic*)]
\item À plusieurs reprises, \emph{les prisonniers} ont tenté de \label{xpriso-f}s'évader.
\end{enumerate}

Cette phrase ne semble pas manifester une covariation du défini pluriel : il est question d'une certaine compagnies de prisonniers et c'est toujours à celle-ci que sont attribuées les tentatives d'évasion.
Il est très important de noter ici que l'on peut voir dans cette phrase ce que l'on appelle parfois la lecture de groupe, ou encore lecture solidaire (en anglais on parle de \alien{team credit}), du défini pluriel. Par cette lecture, il n'est pas nécessaire, dans les faits, que \emph{tous} les prisonniers aient été impliqués dans la tentative d'évasion pour imputer l'action à l'ensemble du groupe. Les responsables réels de la tentative peuvent donc ne pas être toujours les mêmes. Mais pour nous, cela ne change rien, il n'y a pas covariation.  Car le propre de cette lecture est justement d'assigner (à tort ou à raison) le prédicat verbal  à tout le groupe, et ce groupe reste toujours le même (sous encore l'hypothèse de la note \ref{fnmonpriso} p.~\pageref{fnmonpriso} du chapitre).  

\smallskip

\noindent\emph{Test de la négation} (cf. \ref{xGNambigNeg}, p.~\pageref{xGNambigNeg}): on place le GN défini pluriel en position d'objet dans une phrase négative.

\begin{enumerate}[label=(\arabic*),resume]
\item Jean n'a pas lu \emph{les dossiers}.\label{jnapaslulesd}
\end{enumerate}

Nous voyons apparaître ici une distinction sémantique subtile et non triviale entre \sicut{tous les} et \sicut{les}.
Il est généralement admis que \ref{jnapaslulesd} n'est pas ambiguë et qu'elle signifie uniquement que Jean n'a lu aucun des dossiers ; une situation où il aurait lu certains dossiers mais pas tous est le plus souvent perçue comme rendant la phrase ni vraie ni fausse. C'est donc qu'il y a probablement une affaire de présupposition ; nous ne la développerons pas ici mais elle sera abordée au chapitre~\ref{GN++} (vol.~2).

\smallskip

\noindent\emph{Test de consistance} (cf. \ref{test:contra}, p.~\pageref{test:contra}): on vérifie si la conjonction \sicut{les $N$ GV et les $N$ non-GV} est une contradiction.

\begin{enumerate}[label=(\arabic*),resume]
\item Les candidats sont barbus et les candidats sont imberbes.\label{test:contraf}
\end{enumerate}


On constate que si l'une des deux phrases connectées est vraie, l'autre est forcément fausse.  La phrase \ref{test:contraf} ne peut donc jamais être vraie. 
Et si l'une des deux est fausse, la phrase \ref{test:contraf} est bien sûr immédiatement fausse.  Le défini pluriel semble donc se comporter ici comme \sicut{tous les N}.   Mais que se passe-t-il dans 
 un modèle où il y a, par exemple, 50\% de barbus et 50\% d'imberbes ?
Ici les jugements sont un peu délicats ; cependant il se dégage généralement  une tendance  qui est que les deux propositions seront jugées ni vraies ni fausses, mais plutôt inappropriées. Cela est, encore une fois,  lié à l'effet présuppositionnel mentionné \alien{supra} et qui fait que le test de consistance réussit, mais seulement «à moitié» (la phrase n'est jamais vraie, mais elle n'est pas exactement toujours fausse).   Cet effet présuppositionnel se retrouve aussi dans le test de complétude.

\smallskip

\noindent\emph{Test de complétude} (cf. \ref{test:compl}, p.~\pageref{test:compl}) : on vérifie si la disjonction \sicut{les $N$ GV ou les $N$ non-GV} est une tautologie.

\begin{enumerate}[label=(\arabic*),resume]
\item Les candidats sont barbus ou les candidats sont imberbes.\label{test:complf}
\end{enumerate}

Certes si les candidats sont tous barbus ou s'ils sont tous imberbes, la phrase \ref{test:complf} sera globalement vraie.  Inversement, s'il y a, par exemple, 50\% de barbus et 50\% d'imberbes, aucune des deux phrases connectées ne sera vraie, mais, comme précédemment, il sera difficile de les tenir pour fausses pour autant.  Et donc là encore, le test réussit seulement à moitié.

Ces tests, et notamment les deux derniers, montrent que les définis pluriels ont un comportement un peu à part, comparés aux autres \GN, même si, globalement, ils se rapprochent plus de la catégorie de \sicut{le N} que de \sicut{tous les N}.
\end{solu}
\end{exo}



\subsection{Critères syntactico-sémantiques}
%-------------------------------------------

Nous allons ici voir des corrélats (et tests) grammaticaux qui confirment assez nettement ce classement des {\GN} en trois grandes catégories. %, et nous ferons une synthèse en \S\ref{sss:SyntheseGN}. \fixme{*** reprendre ***}


\subsubsection{Dislocation à droite}
%'''''''''''''''''''''''''''''''''''
\is{dislocation}


L'exercice \ref{exo:DefPlur} ci-dessus vise à montrer que les définis pluriels, comme \sicut{les candidats}, s'insèrent apparemment mal dans la triple classification que nous venons de dégager : ils semblent hybrides entre référentiels et quantificationnels.  Mais il s'avère que, par ailleurs, il y a de bonnes raisons pour les classer dans les expressions référentielles.  
Parmi ces raisons on trouve le test de la dislocation à droite :\is{test!\elid\ de la dislocation à droite} on peut constater que seules les expressions référentielles peuvent facilement apparaître en position disloquée à droite, comme le montrent les exemples \ref{x:dislodr1}.
Et \sicut{les candidats} passe très bien dans cette construction. 
En revanche les indéfinis et les quantificationnels, en  \ref{x:dislodr2}, sont soit exclus soit pas entièrement appropriés. 


\ex.  \label{x:dislodr1}
\a. Il était assez faible, Pierre.
\b. Il était assez faible, lui.
\b. Il était assez faible, le candidat.
\c. Il était assez faible, ce candidat.
\c. Il était assez faible, mon candidat.
\d. Ils étaient assez faibles, les candidats.

\ex.  \label{x:dislodr2}
\a. * Il était assez faible, un candidat.
\b. * Ils étaient assez faibles, trois candidats.
\c. * Ils étaient assez faibles, quelques candidats.
\d. * Ils étaient assez faibles, plusieurs candidats.
\d. * Il était assez faible, chaque candidat.
\d. \juge{\urgh} Ils étaient assez faibles, la plupart des candidats.
\d. \juge{\urgh} Ils étaient assez faibles, tous les candidats.


En \ref{x:dislodr2}, j'utilise la marque * car on est en droit de juger que les exemples sont grammaticalement (\ie\ syntaxiquement) inacceptables. Pour \Last[f--g], certes, les jugements sont beaucoup moins nets (et donc marqués par \urgh).  Les locuteurs francophones peuvent être tentés (à juste titre) de ne pas rejeter ces phrases comme inacceptables  (on peut les entendre et les produire sans trop de problème) ; mais à bien y regarder, elles ne semblent pas aussi naturelles et canoniques que celles de \ref{x:dislodr1}.  
Cette ambivalence de \sicut{la plupart des N} et \sicut{tous les N} vis-à-vis de ce test s'explique  de plusieurs manières, et en particulier par le fait que ces {\GN} peuvent dans certaines conditions être réinterprétés comme des expressions référentielles\footnote{Sans entrer dans les détails, disons simplement que \sicut{la plupart des N} peut parfois s'analyser comme un superlatif (comme \sicut{les meilleurs N}, \sicut{les N les plus jeunes}...). Quant à \sicut{tous les N}, nous verrons au chapitre~\ref{GN++} (vol.~2) une analyse formelle qui en fait une véritable expression référentielle, proche de \sicut{les N}, en concurrence avec l'analyse quantificationnelle par $\Xlo\forall$.}.




\subsubsection{Déterminants forts {\vs} déterminants faibles}
%''''''''''''''''''''''''''''''''''''''''''''''''''''''''''
\label{sss:Dfortfaible}

Le test de la dislocation à droite confirme la subdivision entre les expressions référentielles et les autres {\GN} ; celui que nous allons voir à présent, lui, isole la classe des indéfinis des deux autres.\is{indefini@indéfini} 
Il repose sur la distinction entre déterminants (ou {\GN}) forts et faibles, introduite par \citet{Milsark:77}.\Andex{Milsark, G.}
Ce test\is{test!\elid\ des existentielles} met en jeu les constructions existentielles,\is{existentielle (phrase \elid)} 
c'est-à-dire les phrases en \sicut{il y a} ou \sicut{il existe}, en remarquant que seuls les {\GN} dits faibles peuvent apparaître dans ces constructions.  Et les {\GN} faibles coïncident avec nos indéfinis.

Mais pour appliquer ce test, il est très important de prendre deux précautions.  
Il existe en français (au moins) trois types de constructions existentielles ; elle n'ont pas les mêmes fonctions, et il est nécessaire de savoir les distinguer car notre test concerne uniquement le premier type.  Il s'agit :

\begin{enumerate}
\item des constructions existentielles proprement dites ; elles servent à indiquer la présence d'individus à certains endroits ou dans certains environnements ; elles sont de la forme \sicut{il y a {\GN} (W)}, où \sicut{W}  représente un complément circonstanciel, généralement de lieu, qui peut être omis sans nuire à la bonne formation de la phrase (il est alors récupéré via le contexte) ;
\item des constructions existentielles événementielles ; elles servent à décrire un événement particulier et ce sont celles que nous avons vues en \S\ref{AdvQ+Gen}, p.~\pageref{xilya} ; elles sont de la forme \sicut{il y a {\GN} Rel}, où \sicut{Rel} est une proposition relative obligatoire, comme par exemple dans \sicut{il y a le bébé qui pleure}, \sicut{il y a les flics qui sont passés ce matin}, etc. ;
\item des constructions événementielles énumératives, qui, comme leur nom l'indique, n'interviennent que dans des énumérations ; elles sont normalement de la forme \sicut{il y a \GN$_1$, \GN$_2$, \GN$_3$...} (même si parfois l'énumération peut s'interrompre très vite et s'arrêter au premier élément) ; par exemple : \sicut{dans ce film, il y a John Travolta, Samuel L.\ Jackson, Bruce Willis, Uma Thurman...}
\end{enumerate}

\largerpage[-1]

En prenant bien soin d'exclure les éventuelles lectures événementielles et énumératives (et pour ce faire, je place \sicut{dans la cour} entre parenthèses pour insister sur son optionalité), nous pouvons observer le contraste entre \ref{x:existlf} (les indéfinis, donc les faibles) et \ref{x:existlF} (les autres, les forts) : 

\ex.  \label{x:existlf}
\a. Il y a un éléphant (dans la cour).
\b. Il y a trois éléphants (dans la cour).
\b. Il y a des éléphants (dans la cour).
\b. Il y a quelques éléphants (dans la cour).
\b. Il y a plusieurs éléphants (dans la cour).
%\b. Il y a beaucoup d'éléphants dans la cour.

\ex.  \label{x:existlF}
\a. \juge{\uurgh} Il y a Dumbo (dans la cour).
\b. \juge{\uurgh} Il y a  l'éléphant (dans la cour).
\b. \juge{\uurgh} Il y a cet éléphant (dans la cour).
\b. \juge{\uurgh} Il y a mon éléphant (dans la cour).
\b. \juge{\uurgh} Il y a les éléphants (dans la cour).
\b. \juge{\uurgh} Il y a chaque éléphant (dans la cour).
\b. \juge{\uurgh} Il y a la plupart des éléphants (dans la cour).
\b. \juge{\uurgh} Il y a tous les éléphants (dans la cour).


Les phrases \ref{x:existlF} sont marquées {\uurgh}, car on pourrait juger qu'elles ne sont pas inacceptables au point de mériter une *. 
C'est là la seconde précaution à prendre. 
Selon les variantes dialectales, régionales, de registre, etc., ces phrases peuvent sembler assez ordinaires\footnote{Et c'est notamment dû au fait que certaines de ces phrases peuvent être réinterprétées comme des existentielles événementielles elliptiques équivalant à \sicut{il y a l'éléphant (qui est) dans la cour} ou \sicut{il y a l'éléphant (qui fait quelque chose) dans la cours}.}, cependant si nous nous en tenons au français (très) standard, elles apparaissent malgré tout comme moins naturelles et moins correctes que celles de \ref{x:existlf}, surtout lorsque le circonstanciel de lieu est omis.

Ce test des existentielles confirme la singularité des indéfinis par rapport aux autres {\GN} et suggère également une possible affinité sémantique entre les expressions référentielles (notamment les définis) et les {\GN} quantificationnels forts.  


\subsection{Synthèse}
%--------------------
\label{sss:SyntheseGN}

Le tableau~\ref{f:synthGN} (page suivante) résume les résultats de nos tests et montre les trois classes de {\GN} qui s'en dégagent : 
les expressions référentielles, les \GN\ indéfinis\is{indefini@indéfini} et les \GN\ quantificationnels.

\begin{table}[h!]
\begin{bigcenter}\small
\begin{tabular}{rcccccl}\lsptoprule
%\rowcolor{gray!40}
\multicolumn{1}{c}{\rule{0pt}{8em}Classes de \GN}
& %\rotatebox{90}
\begin{rotate}{45}{\small Test de consistance}\end{rotate}
& %\rotatebox{90}
\begin{rotate}{45}{\small  Test de complétude}\end{rotate}
& %\rotatebox{90}
\begin{rotate}{45}{\small Ambiguïté avec la négation}\end{rotate}
 & %\rotatebox{90}
\begin{rotate}{45}{\small Dislocation à droite}\end{rotate}
& \begin{rotate}{45}{\small Phrases existentielles}\end{rotate}
  %\rotatebox{90}{\small Phrases existentielles} 
& \multicolumn{1}{c}{\small Exemples}
\\\midrule
expressions référentielles & $+$ & $+$ & $-$ & $+$ & $-$ & \small noms propres, définis, démonstratifs...\\%\hline
\GN\ indéfinis & $-$ & $-$ & $+$ & $-$ & $+$ & \small \sicut{un} N, \sicut{des} N, \sicut{quelques} N, \sicut{plusieurs} N...\\ %\hline
\GN\ quantificationnels & $+$ &$-$ & $+$ & $-$ & $-$ & \small \sicut{tous les} N, \sicut{chaque} N, \sicut{la plupart des} N...\\
\lspbottomrule
\end{tabular}\normalsize
\end{bigcenter}
\caption{Propriétés des différentes classes de \GN}\label{f:synthGN}
\end{table}


Les expressions référentielles sont nommées ainsi car elles ont la particularité, à l'instar des constantes de \LO, de dénoter (ou de référer à) un individu précis du modèle. Les {\GN} des deux autres classes, eux, n'ont pas ce même pouvoir référentiel.  C'est en accord avec la sémantique de la quantification que nous avons vue au chapitre précédent et en \S\ref{ss:IFQ} : l'interprétation des formules quantifiées met en jeu un ou plusieurs calculs, soit en \emph{choisissant} (plus ou moins librement) une valeur pour la variable quantifiée (pour $\Xlo\exists$), soit en \emph{parcourant} l'ensemble de ses valeurs possibles (pour $\Xlo\forall$). 
Nous pouvons également nous rendre compte de cette absence de pouvoir référentiel si nous reformulons les conditions de vérité de phrases incluant une quantification sans passer par {\LO} mais en allant consulter directement dans le modèle la dénotation des prédicats.  Ainsi, par exemple, \ref{x:QEns1b} et \ref{x:QEns2b} --~qui ne contiennent pas de variable~-- expriment tout à fait correctement les conditions de vérité de \ref{x:QEns1} et \ref{x:QEns2} :



\ex. \label{x:QEns1}
Un homme est entré.
\a. \(\Xlo\exists x [\prd{homme}(x) \wedge \prd{entrer}(x)]\)
\b. \( \denote{\prd{homme}}^{\Modele,g} \cap \denote{\prd{entrer}}^{\Modele,g}
\neq \Evide\) \label{x:QEns1b}


\ex. \label{x:QEns2}
Tous les hommes sont mortels.
\a. \(\Xlo\forall x [\prd{homme}(x) \implq \prd{mortel}(x)]\)
\b. \(\denote{\prd{homme}}^{\Modele,g} \inclus \denote{\prd{mortel}}^{\Modele,g}\) \label{x:QEns2b}


\ref{x:QEns1b} dit que l'intersection de l'ensemble des hommes et de l'ensemble de ceux qui sont entrés doit être non vide ; finalement peu importe quel homme précis est entré\footnote{Sauf, bien sûr, comme nous l'avons vu, si l'indéfini a une interprétation spécifique, auquel cas il devient, en quelque sorte, référentiel.}, il peut même y en avoir plusieurs, du moment qu'il y en a au moins un, la phrase sera vraie. 
\ref{x:QEns2b} dit que l'ensemble des hommes doit être inclus dans l'ensemble des mortels ; là encore il n'est pas fait référence à un ou plusieurs hommes particuliers, c'est tout l'ensemble qui est considéré.
D'une certaine manière donc, l'interprétation des {\GN} quantificationnels et indéfinis est plus une histoire d'ensembles que de référence.

\sloppy

Cette approche ensembliste de l'analyse de la quantification est %même 
tout à fait pertinente, en particulier pour le traitement des {\GN} quantificationnels forts (et nous y reviendrons en \S\ref{s:QG}).
Elle permet également de mettre le doigt sur une curiosité, et même un problème, qui a pu nous apparaître et qui nous suit depuis le chapitre~\ref{LCP} et qui concerne la \emph{dénotation} des \GN.  Les expressions référentielles dénotent des entités du modèle (comme ce que nous avions vu en introduisant la notion fregéenne de dénotation en \S\ref{ss:SuB}) ; cela veut dire que nous n'aurons pas de problème à déterminer et calculer des valeurs pour \(\denote{\sicut{Alice}}^{\Modele,g}\) ou \(\denote{\sicut{la salle de bain}}^{\Modele,g}\) via leur traduction dans {\LO} (voir \S\ref{s:GNdef} pour la traduction des \GN\ définis). 
En revanche nous ne pouvons pas le faire pour \(\denote{\sicut{tous les hommes}}^{\Modele,g}\)
et \(\denote{\sicut{un homme}}^{\Modele,g}\), car leur interprétation ne renvoie pas directement à un élément du domaine, c'est plutôt une instruction qui procède du calcul interprétatif \emph{global} de la phrase. Et cela se reflète dans {\LO} : compositionnellement, ces \GN\ ne se traduisent pas par des expressions bien formées de {\LO} ; au mieux, ils correspondent à des «morceaux» de formules ($\Xlo\forall x [\prd{homme}(x) \implq \dots$, $\Xlo\exists x [\prd{homme}(x) \wedge \dots$) mais qui ne sont pas interprétables en soi.
Autrement dit, les {\GN} quantificationnels et indéfinis ne semblent pas avoir de dénotation, ce qui est très handicapant pour leur définir formellement un sens.
Nous résoudrons ce problème au chapitre~\ref{ch:ISS}.

\fussy

\smallskip

Nous pouvons également généraliser ici des observations faites en \S\ref{s:GNportée}.  Les {\GN} quantificationnels et les indéfinis ont tous une interprétation mobile et, de ce fait, une portée variable. Ce n'est pas le cas des expressions référentielles. Comme l'indique la proposition \ref{pt:Portee2} (p.~\pageref{pt:Portee2}), la mobilité interprétative des {\GN} quantificationnels est limitée à leur proposition syntaxique (même ceux qui ne correspondent pas à une quantification universelle, cf.\ \ref{xFarkasvar}), mais pas celle des indéfinis \ref{xIndefPLarge}.

\ex. 
\a. 
\juge{\urgh} Jean a raconté à un journaliste [que Pierre habite dans la plupart des  villes de France].\label{xFarkasvar}
\b. Chaque ministre a confié à chaque journaliste [que  plusieurs membres du gouvernement étaient corrompus].\label{xIndefPLarge}



Comme nous l'avons vu, les indéfinis peuvent avoir un usage ou une interprétation spécifiques --~y compris lorsqu'ils sont pluriels.  
Et la spécificité a pour effet, en quelque sorte, d'accorder aux  indéfinis un certain pouvoir référentiel, puisqu'ils servent ainsi à désigner un individu particulier (ou un groupe particulier d'individus) auquel pense le locuteur.   
Mais cela ne veut pas dire que, par la spécificité, les indéfinis \emph{deviendraient} des expressions référentiels ; ils restent des indéfinis (la spécificité ne les fait pas réussir les tests de consistance et de complétude), mais avec une  propriété sémantique supplémentaire.

D'ailleurs, cette affinité que les indéfinis entretiennent avec la référentialité se retrouve dans un phénomène que nous avons déjà observé en \S\ref{ss:specificite} (p.~\pageref{x:coquille1}), à savoir la reprise par un pronom anaphorique.
Par nature, les expressions référentielles peuvent être reprises sans problème par un pronom \ref{x:reprER} : en effet elles dénotent une entité particulière et rien n'empêche un pronom, par la suite, de partager cette dénotation (c'est le phénomène de coréférence).\is{coreference@coréférence}
Les {\GN} quantificationnels singuliers ne peuvent normalement pas être repris par un pronom singulier\footnote{Pour la reprise par des pronoms pluriels de quantificationnels «pluriels» en \sicut{tous les}, \sicut{la plupart}... c'est  plus complexe, et nous y reviendrons brièvement au chapitre~\ref{Ch:contexte} (vol.~2).} \ref{x:reprQu}, du fait précisément de la variation de valeurs qu'ils imposent sur une variable liée. 
Mais les indéfinis \emph{peuvent} (dans certaines conditions) être repris très naturellement par des pronoms, même lorsqu'ils n'ont pas un emploi spécifique \ref{x:reprIn}. 


\ex. 
\a. Le majordome$_1$ s'est absenté pendant une semaine. Il$_1$ était souffrant et alité.\label{x:reprER}
\b.  Marie a examiné chaque dossier$_1$. \zarb Il$_1$ est sur le bureau.\label{x:reprQu}
\b. Un espion$_1$ s'est introduit dans le quartier général et il$_1$ a dérobé des dossiers compromettants. Il$_1$ n'a laissé aucune empreinte.\label{x:reprIn}


Bien que la reprise anaphorique d'un indéfini par un pronom soit un phénomène linguistique somme toute très banal en soi, il pose de \emph{très} épineux problèmes pour le système sémantique formel --~nous y reviendrons au chapitre~\ref{Ch:contexte} et dans le chapitre conclusif~\ref{ch:conclu} (vol.~2).  Pour en donner un simple aperçu, considérons une des principales interrogations qu'ils soulèvent :
si les indéfinis ne sont pas référentiels, comment peut-on formaliser leur «coréférence» avec des pronoms ? Et s'ils sont référentiels, pourquoi n'ont-ils pas les propriétés que nous avons vues des expressions référentielles ? 

%****
Les indéfinis manifestent beaucoup d'autres particularités, et ils sont encore aujourd'hui un
sujet d'étude inépuisable en analyse sémantique du domaine nominal\footnote{Voir par exemple \citet{Corblin:87}, \citet{Haspelmath:97},\Andexn{Haspelmath, M.} \citet{Farkas:02salt},\Andexn{Farkas, D.} \citet{SorinBeyssade:05} pour d'amples panoramas.\Andexn{Dobrovie-Sorin, C.}\Andexn{Beyssade, C.} 
Sur le rapport entre indéfinis et pronoms, la meilleure introduction reste le chapitre~1 de \citet{Heim:82}.\Andexn{Heim, I.}}. 
Même s'ils semblent partager certaines caractéristiques avec les expressions référentielles et les {\GN} quantificationnels, ils constituent une catégorie bien à part, qui peut certes présenter de l'hétérogénéité, ou du moins de la variété, à la fois dans leurs propriétés sémantiques intrinsèques et dans leurs emplois possibles.




Rappelons enfin qu'il est  extrêmement important de bien avoir en tête que les appellations de \emph{quantificationnels} et \emph{indéfinis} sont pour nous ici des \emph{termes techniques} qui dénomment des catégories de {\GN} que nous avons définies ici à l'aide de nos tests logiques et grammaticaux (cf.\ le tableau de synthèse \ref{f:synthGN}). % en \S\ref{sss:SyntheseGN}). 
Il ne faudrait donc surtout pas commettre l'erreur de se fier simplement à notre compréhension informelle et quotidienne de ces termes pour en déduire l'appartenance de tel ou tel {\GN} à telle ou telle catégorie. En particulier, les {\GN} qui véhiculent une information sur une quantité ne sont pas nécessairement quantificationnels : c'est par exemple le cas des pluriels de la forme \sicut{trois N}, \sicut{un million de N}, \sicut{quelques N} ou \sicut{plusieurs N}, qui sont, comme nous l'avons vu, des indéfinis. 




\medskip


% -*- coding: utf-8 -*-
\begin{exo}\label{exoInduCovar}
En vous inspirant des tests \ref{xGNsubitcv} (p.~\pageref{xGNsubitcv}) et \ref{xGNambigNeg} (p.~\pageref{xGNambigNeg}), 
\pagesolution{crg:InduCovar}
identifiez les
 \GN\ qui induisent de la covariation (vous établirez donc à cet effet un test approprié). 
\begin{solu}(p.~\pageref{exoInduCovar})\label{crg:InduCovar}

Une façon de définir un test de la covariation induite par un \GN\ consiste à placer ce \GN\ dans une position sujet d'une phrase avec un indéfini singulier en position de complément d'objet.  Nous observons ensuite si cet indéfini peut ou non être multiplié par le sujet. 

\ex.[]
\a.  \emph{Julie} a corrigé une page wikipédia.
\b.  \emph{Elle} a corrigé une page wikipédia.
\b.  \emph{L'étudiante} a corrigé une page wikipédia.
\b.  \emph{Cette étudiante} a corrigé une page wikipédia.
\b.  \emph{Mon étudiante} a corrigé une page wikipédia.
\b.  \jcovar \emph{Elles} ont corrigé une page wikipédia.
\b.  \jcovar \emph{Les étudiants} ont corrigé une page wikipédia.
\b.  \jcovar \emph{Ces étudiants} ont corrigé une page wikipédia.
\b.  \jcovar \emph{Mes étudiants} ont corrigé une page wikipédia.
\b.  \emph{Une étudiante} a corrigé une page wikipédia.
\b.  \jcovar \emph{Trois étudiants} ont corrigé une page wikipédia.
\b.  \jcovar \emph{Des étudiants} ont corrigé une page wikipédia.
\b.  \jcovar \emph{Plusieurs étudiants} ont corrigé une page wikipédia.
\b.  \jcovar \emph{Quelques étudiants} ont corrigé une page wikipédia.
\b.  \jcovar \emph{Aucun étudiant} n'a corrigé une page wikipédia.
\b.  \jcovar \emph{La plupart des étudiants} ont corrigé une page wikipédia.
\b.  \jcovar \emph{Tous les étudiants} ont corrigé une page wikipédia.
\b.  \jcovar \emph{Chaque étudiant} a corrigé une page wikipédia.
\b. etc.


Globalement, nous remarquons que les \GN\ pluriels induisent de la covariation (ainsi que \sicut{chaque $N$}).  Faisons tout de même deux remarques.  Pour les définis pluriels (g--i), la covariation de l'objet semble peut-être moins naturelle et moins courante, mais a priori nous ne pouvons pas complètement en exclure la possibilité\footnote{Bien sûr, la covariation devient nette si l'on ajoute le quantificateur dit flottant \sicut{tous} (comme dans \sicut{ces étudiants ont tous corrigé une page wikipédia}) mais cela ne montre rien puisque ça peut très probablement être \sicut{tous} qui est responsable du phénomène et pas le \GN\ en soi.}.  Le jugement porté sur \sicut{aucun étudiant} (o) est discutable ; nous pouvons certes y observer une multiplication par $0$, mais celle-ci peut être induite par la négation (\sicut{ne}) qui accompagne \sicut{aucun}.
\end{solu}
\end{exo}
\is{test!\elid\ de covariation}

% -*- coding: utf-8 -*-
\begin{exo}\label{exo:CatGN}
En appliquant les tests que nous avons à notre disposition, 
\pagesolution{crg:CatGN}%
déterminez à quelles classes appartiennent les {\GN} suivants : \sicut{un tiers des candidats}, \sicut{trois quarts des candidats}, \sicut{beaucoup de candidats}.
\begin{solu}(p.~\pageref{exo:CatGN})\label{crg:CatGN}

\sicut{Un tiers des candidats} est un indéfini car, d'un point de vue strictement sémantique, il ne satisfait pas le test de consistance (p.~\pageref{test:contra}) : \sicut{un tiers des candidats sont barbus et un tiers des candidats sont imberbes} peut être jugé vrai dans une situation qui comporte, par exemple, 50\% de barbus et 50\% d'imberbes, si nous considérons qu'\sicut{un tiers} signifie \sicut{au moins un tiers} (cf. \S\ref{ss:implicatures}).

\sicut{Trois quarts des candidats} est quantificationnel, car il satisfait le test de consistance (\sicut{trois quarts des candidats sont barbus et trois quarts des candidats sont imberbes} ne peut pas être vrai), mais il ne satisfait pas le test de complétude (p.~\pageref{test:compl}): \sicut{trois quarts des candidats sont barbus ou trois quarts des candidats sont imberbes} est faux par exemple dans une situation avec 50\% de barbus et 50\% d'imberbes.

Pour \sicut{beaucoup de candidats}, la classification est moins simple, car en fait le déterminant est ambigu.  Commençons avec le test de consistance : \sicut{beaucoup de candidats sont barbus et beaucoup de candidats sont imberbes}.  Supposons que nous sommes dans une situation où il y a 200 candidats barbus et 200 candidats imberbes et que nous estimons que 200 est un nombre important de candidats (par exemple parce que nous en attendions seulement une soixantaine), dans ce cas la phrase pourra être jugée vraie.  Et cela suffit à prouver que le GN\ est un indéfini.  

Mais on peut comprendre le déterminant \sicut{beaucoup de} d'une
autre manière. Dans les phrases ci-dessus, \sicut{beaucoup de} est
interprété comme signifiant «une grande quantité de» et la
«grandeur» de cette quantité dépend du contexte. L'autre interprétation de
\sicut{beaucoup de} est celle qui signifie quelque chose comme «une grande proportion de» ; là encore la grandeur de cette proportion dépend habituellement du contexte.  Selon le seuil que fixe le contexte pour estimer qu'on est en présence d'une \emph{grande} proportion,  \GN\ apparaîtra alors comme quantificationnel (si le seuil est supérieur à 50\%) ou indéfini (si le seuil est inférieur à 50\%) (cf. \S\ref{ss:QGDet}).   Dans certains cas, il est possible cependant d'associer à \sicut{beaucoup de} un sens qui ne dépend pas du contexte en l'interprétant comme signifiant «la plus grande proportion de».  Dans ce cas le \GN\ sera quantificationnel.  En 
effet dans ce cas \sicut{beaucoup de candidats sont barbus et beaucoup de candidats sont imberbes} est contradictoire, car quel que soit le
nombre de candidats au total, ce sont soit les barbus, soit les
imberbes qui représentent la plus grande proportion, mais pas les deux
à la fois.  Quant à \sicut{beaucoup de candidats sont barbus et beaucoup de candidats sont imberbes}, elle est fausse dans le cas où il y a
autant de barbus que d'imberbes dans le groupe de candidats.
\end{solu}
\end{exo}



% -*- coding: utf-8 -*-
\begin{exo}\label{exo:DefCovar}
Trouvez deux exemples (suffisamment différents) de phrases avec un {\GN}
\pagesolution{crg:DefCovar}%
défini singulier (en \sicut{le} ou \sicut{la}) qui (en dépit de ce que nous avons
dit et montré ci-dessus) covarie avec un autre élément de la phrase.
Et tentez une explication.
\begin{solu} (p.~\pageref{exo:DefCovar})\label{crg:DefCovar}

Il y a différentes façons de construire un {\GN} défini qui covarie
nettement avec une autre expression de la phrase, mais toutes se
ramènent généralement au même phénomène.

Premier type de constructions : une expression quantifiée apparaît en
complément ou en modifieur de la tête nominale du {\GN} défini.  Exemple :

\begin{enumerate}[label=(\arabic*)]
\item \label{x:def1}
Paul a contrefait \underline{la signature de \emph{chaque membre
  de sa famille}}.
\end{enumerate}


Dans (\ref{x:def1}), il est bien question d'autant de signatures qu'il
y a de membres de la famille, il y a donc bien une multiplication,
c'est-à-dire un effet de covariation.

Deuxième type de construction : le {\GN} défini contient un pronom (ou un
élément anaphorique) qui est lié d'une manière ou d'une autre (par
exemple par son antécédent) à une expression quantifée.  Exemple :

\begin{enumerate}[label=(\arabic*),resume]
\item \label{x:def2}
\emph{Tout dompteur$_i$} craint \underline{le lion qu'\emph{il$_i$}
est en train de dompter}. 

\item
\emph{Chaque candidat$_i$} doit remplir \underline{le formulaire qui
\emph{lui$_i$} a été remis}.
\end{enumerate}

Là encore, dans (\ref{x:def2}) par exemple, il y a autant de lions que
de dompteurs, bien que le {\GN} soit singulier.
Remarque : l'élément anaphorique en question peut être implicite :

\begin{enumerate}[label=(\arabic*),resume]
\item \emph{Dans chaque appartement que nous avons visité}, \underline{la salle de bain} était minuscule.
\end{enumerate}

Ici il s'agit de la salle de bain «de lui», où \sicut{lui}
est l'appartement dont il est question à chaque fois\footnote{Ce phénomène est usuellement désigné par le terme d'\emph{anaphore associative}.}.

Pour les éléments d'explication, voir \S\ref{sss:defdep} p. \pageref{sss:defdep}.

\end{solu}
\end{exo}





\subsection{Les descriptions définies}  %{Les groupes nominaux définis}
%----------------------------------------
\label{s:GNdef}
\is{defini@défini!groupe nominal \elid|(}
\is{description définie|(}


Répétons-le, notre langage {\LO} ne nous permet, \emph{pour l'instant}, de traduire que très peu de {\GN} parmi la variété que nous avons examinée précédemment : les indéfinis singuliers (\sicut{un(e) N}), les quantifications universelles (\sicut{tous les N}, \sicut{chaque N}) et les noms propres. 
Cela peut nous suffire dans un premier temps, cependant il y a des {\GN} qu'il nous serait particulièrement utile de savoir traiter formellement dès à présent, ne serait-ce que parce qu'ils sont extrêmement communs : les {\GN} définis. 
%Nous verrons comment traiter la plupart des autres {\GN} au chapitre~\ref{ch:ISS}.  
C'est ce que nous allons aborder dans cette dernière section du chapitre, en les examinant sous l'angle de ce que les philosophes et les sémanticiens appellent des \emph{descriptions définies}. 
Une description définie est tout simplement un {\GN} défini singulier, donc de la forme \sicut{le $N$} ou \sicut{la $N$} (où $N$ est un substantif ou un prédicat nominal complexe), dont on considère qu'il renvoie à un \emph{unique} individu.\is{defini@défini!article \elid}

Dans cette présentation, je vais prendre la peine de retracer, brièvement, l'historique du débat qui a opposé \citet{Rus:05fr}\Andex{Russell, B.} et \citet{Strawson:50fr}\Andex{Strawson, P.}, afin de  replacer dans leur  contexte épistémologique les tenants et les aboutissants de la vision sémantique que nous pouvons avoir des {\GN} définis.

\subsubsection{Existence et unicité} 
%'''''''''''''''''''''''''''''''''''
\label{sss:Russell}

\citet{Rus:05fr} aborde le sujet, entre autres, en soulevant un paradoxe qui semble mettre à mal les propositions de \citet{Frege:SuB}.\Andex{Frege, G.} Pour illustrer ce paradoxe, 
prenons tout de suite un exemple,  la description définie
\sicut{le président des États-Unis en 2015}.  Et
considérons qu'elle possède une dénotation.  Celle-ci est bien unique,
et, à ce titre, cela rappelle l'interprétation des noms
propres.  D'ailleurs, les dénotations de \sicut{le président des
  États-Unis en 2015} et de \sicut{Barack H. Obama} 
coïncident exactement dans notre monde.  Traduire cette description
définie dans {\LO} sous la forme d'un terme, et plus exactement d'une
constante, disons \cns{p}, semble donc une option raisonnable et
judicieuse.  Nous saurons alors que, par rapport à un modèle conforme à
notre monde réel (en 2015), la formule \(\Xlo\cns{p}=\cns{o}\) est vraie,
où \cns{o} traduit \sicut{Barack Obama}.

Mais alors que faire d'une description
définie comme \sicut{l'actuel roi de France} ?  Elle ne dénote
rien dans notre monde ; or chaque constante de {\LO} est  supposée
dénoter un individu du domaine.  On pourrait s'en tirer en admettant
que le domaine peut contenir des individus fictifs, comme Lucky Luke,
Ulysse, Dracula ou Lara Croft, avec dans {\LO} des constantes qui leur
correspondent (nous reviendrons d'ailleurs sur cette idée dans le
chapitre~\ref{Ch:t+m}).  Mais cela ne résoudrait pas le
problème.

Supposons en effet que \sicut{l'actuel roi de France} se
traduise par la constante \cns{r}, dénotant un individu fictif.   
Et souvenons-nous de la loi du tiers-exclu, qui dit que pour une
formule $\Xlo\phi$ donnée, par rapport à une modèle donné, soit $\Xlo\phi$ est
vraie, soit $\Xlo\neg\phi$ est vraie (\ie\ $\Xlo[\phi \vee \neg\phi]$ est vraie
pour tout modèle).  Russell considère que la phrase {\Next} est fausse (dans notre monde).

\ex. \label{x:rdf1}
L'actuel   roi de France est chauve\\
\(\Xlo\prd{chauve}(\cns{r})\)

Donc en vertu de la loi du tiers
exclu, la phrase {\Next} (négation de {\Last})  devrait être vraie (dans notre monde).

\ex. \label{x:rdf2}
L'actuel   roi de France n'est pas chauve\\
\(\Xlo\neg\prd{chauve}(\cns{r})\)

Mais c'est là une conclusion
que nous n'avons pas envie d'accepter : il n'y a pas de raison de
penser que \ref{x:rdf2} soit «moins fausse» que \ref{x:rdf1}. 


Pour résoudre ce problème, Russell défend une analyse qui va à l'encontre de ce que proposait Frege --~et aussi de ce que nous avons vu dans les pages précédentes.
Son idée est que les descriptions
définies ont une contribution sémantique très différente de celle des
noms propres, en ce sens qu'elles \emph{ne  dénotent pas} un
individu ; leur rôle est au contraire d'introduire de l'information dans la phrase où elles interviennent.
Par \emph{information}, ici, il faut comprendre \emph{des conditions de vérité}, c'est-à-dire, en pratique, des formules (ou plus exactement des «morceaux» de formules).
Ces informations sont au nombre de deux :  de manière générale, une
description \sicut{le $N$} permet d'affirmer $i$) qu'il \emph{existe}
dans le domaine un individu qui satisfait le prédicat nominal $N$ et
$ii$) cet individu est \emph{le seul} individu du domaine qui
satisfait $N$.  Ces conditions d'existence\is{existence} et d'unicité\is{unicité} peuvent se
traduire facilement dans {\LO}.  Pour \ref{x:rdf1}, en traduisant
\sicut{actuel roi de France} par le prédicat \prd{rdf}, nous
obtenons : 

\ex. \label{x:rdf2ea}
L'actuel   roi de France est chauve\\
\(\Xlo\exists x [\prd{rdf}(x) \wedge \forall y [\prd{rdf}(y) \ssi y=x] \wedge \prd{chauve}(x)]\)\label{x:rdf1'}


La première partie de la traduction de \ref{x:rdf1'}, \(\Xlo\exists x
[\prd{rdf}(x)\dots\), exprime la condition d'existence : il y a un
individu, $\vrb x$, qui est roi de France.  La deuxième partie, \(\Xlo\forall y
[\prd{rdf}(y) \ssi y=x]\), dit qu'un individu $\vrb y$ peut être roi de France
si et seulement si cet individu est identique à $\vrb x$, autrement dit
qu'il n'existe pas d'individu autre que $\vrb x$ qui soit roi de France :
c'est bien la condition d'unicité attendue.

Et ainsi le problème «du tiers exclu» disparaît, du moment que nous traduisons (\ie\ interprétons) \ref{x:rdf2}  
avec une négation qui porte sur le prédicat verbal :

\ex.
L'actuel roi de France n'est pas chauve\\
\(\Xlo\exists x [\prd{rdf}(x) \wedge \forall y [\prd{rdf}(y) \ssi y=x] \wedge \neg\prd{chauve}(x)]\)\label{x:rdf2'}


\ref{x:rdf2'} est  bien fausse par rapport à notre monde, et
ce pour les mêmes raisons que \ref{x:rdf1'}, car la condition
d'existence n'est pas satisfaite : il n'y a pas d'individu qui soit roi
de France, et cela suffit à falsifier \ref{x:rdf1'} \emph{et}
\ref{x:rdf2'}. 

Enfin, Russell ajoute qu'en fait, \ref{x:rdf2} peut avoir une autre
interprétation, illustrée en \ref{x:rdf2''}, avec une portée large
pour la négation :

\ex.
L'actuel roi de France n'est pas chauve\\
\(\Xlo\neg\exists x [\prd{rdf}(x) \wedge \forall y [\prd{rdf}(y) \ssi y=x] \wedge \prd{chauve}(x)]\)\label{x:rdf2''}


\largerpage

\ref{x:rdf2''} dit que, dans le domaine, il n'existe pas
d'individu qui soit à la fois l'unique roi de France et chauve.  Elle
est donc vraie dans notre monde.  Ce qui semble justifier cette
interprétation, c'est que l'on puisse dire (dans notre monde) quelque
chose comme : 
\ex. \label{x:rdf3}
Et oui, l'actuel roi de France n'est pas chauve, car justement il n'y
a plus de roi de France aujourd'hui.


Notons également  que \ref{x:rdf2''} sera fausse dans un monde où
il existe un 
unique roi de France et qu'il est  chauve (comme pour
\ref{x:rdf2'}).

\subsubsection{Définis et présupposition}
%''''''''''''''''''''''''''''''''''''''''
\label{sss:DefPsp}
\is{presupposition@présupposition}

On remarquera que par l'analyse de Russell, \ref{x:rdf1} s'interprète donc
comme une phrase existentielle.  Cela peut sembler ne pas aller de
soi, mais ce n'est finalement pas incohérent, car c'est précisément ce
que visait Russell.  Partant,  il justifie que \ref{x:rdf2} est
sémantiquement ambiguë, ce que nous avons démontré ci-dessus, via les
traduction \ref{x:rdf2'} et \ref{x:rdf2''}, en trouvant des
valeurs de vérités différentes pour un même modèle, à savoir un modèle
conforme à la réalité de notre monde.  Pourtant, ce point précis peut
être sujet à critique.  En particulier, sommes-nous vraiment prêts à
admettre que \ref{x:rdf2} peut être vrai dans un monde tel que le
nôtre ? ou pour dire les choses différemment, est-ce vraiment le cas
que \ref{x:rdf2}  puisse s'interpréter comme niant l'existence d'un
unique roi de France ?  L'argument que nous avons donné pour cette
hypothèse est illustré en \ref{x:rdf3}, mais justement : ne
pourrions-nous pas considérer, somme toute, qu'en \ref{x:rdf3} ce
n'est pas la phrase \ref{x:rdf2} qui par elle-même pose la négation
de l'existence du roi, mais plutôt la seconde phrase de
\ref{x:rdf3}, et ce de manière «rétroactive» et a posteriori ?  En
considérant cette possibilité, nous réinstaurons la notion de
présupposition que nous avions laissé de côté provisoirement.  En
effet, ce que \ref{x:rdf3}, et surtout sa seconde phrase, illustre
est bien un phénomène d'annulation d'une présupposition.  


Nous touchons là la critique que \citet{Strawson:50fr}\Andex{Strawson,
  P.} avait adressée à 
la proposition de Russell.  Un point de désaccord essentiel que Strawson
soulève est que, par rapport à notre monde, la phrase \ref{x:rdf1}
n'est pas fausse, contrairement à ce que disait Russell.  Et
\ref{x:rdf1} n'est pas vraie non plus. En fait, la question de sa valeur de vérité (par rapport à notre monde) ne se pose tout simplement pas.
Il en va de même pour \ref{x:rdf2}. Et
cela n'est pas étrange, car Strawson 
ajoute que \ref{x:rdf1} ne pose pas (ou n'affirme pas) l'existence
d'un roi de France, elle «l'\emph{implique}~». 
Et ce que Strawson qualifie ainsi d'implication\footnote{Strawson prend bien soin de préciser que cette implication doit être distinguée de l'implication logique (\ie\ la conséquence logique). Il faut rappeler également que dans son article, Strawson n'emploie jamais les termes \emph{présupposition} ou \emph{présupposer} ; c'est plus tard que le rapport sera établi et que l'article de Strawson deviendra l'acte de naissance de l'analyse présuppositionnelle des {\GN} définis dans la tradition contemporaine.}
correspond en fait à ce que nous connaissons aujourd'hui sous la notion de \emph{présupposition}.
Et comme cette
présupposition n'est pas satisfaite dans notre monde (où il n'y a pas
de roi en France), \ref{x:rdf1}  est dépourvue de valeur de vérité,
ce qui revient à dire qu'il est inapproprié de prononcer
\ref{x:rdf1} (cf.  chapitre~\ref{Ch:1},
p.~\pageref{s:presuppositions}).   

De nos jours c'est cette analyse qui est très majoritairement adoptée.
Mais comment l'intégrer dans notre formalisation sémantique ?  Nous
avons vu (ou plutôt stipulé) dans le chapitre~\ref{Ch:1} que les
présuppositions étaient un peu en dehors du champ de la sémantique, se
situant sur un autre plan de la communication.  Dans les termes que
nous utilisons maintenant, nous dirons qu'une présupposition ne fait
pas directement partie des conditions de vérité de la phrase qui la
contient.  Ce que l'on explicite en éclatant le «contenu» de \ref{x:rdf1}
de la manière suivante :

\ex. 
{présupposé} : \emph{il existe un et un seul roi de France}\\
{proféré} : \emph{il (\ie\ ce roi de France) est chauve}


Et les conditions de vérité de \ref{x:rdf1} sont données par
la partie proférée seulement.  Comme sa formulation ci-dessus contient un
pronom, \ref{x:rdf1} devra se traduire simplement~en :

\ex. \label{x:chauvex}
\(\Xlo\prd{chauve}(x)\)

en considérant que l'on sait \emph{par ailleurs} que ce $\vrb x$ fait
référence à cet unique roi de France dont l'existence est
présupposée.  C'est un peu comme si l'information présupposée
fonctionnait comme l'antécédent d'un pronom représenté par $\vrb x$.  Si
cet antécédent n'est pas défini c'est probablement que parmi les
fonctions d'assignation par rapport auxquelles on a le droit
l'interpréter \ref{x:chauvex}, aucune n'est en mesure de proposer
une valeur (acceptable) pour $\vrb x$%
\footnote{Il faut cependant être vigilant sur ce genre de formulation :
dans le cadre de la théorie et du formalisme que nous présentons dans
cette ouvrage, nous avons défini les assignation comme des fonctions
(totale) de 
{\VAR} dans \Unv{A}. Par conséquent tout variable de {\VAR}, y compris
le $\vrb x$ de \ref{x:chauvex} doit recevoir une valeur par toute
assignation $g$.  C'est que l'analyse de la présupposition que nous
esquissons ici relève, là encore, de la sémantique dynamique.\is{semantique@sémantique!\elid\ dynamique}}, 
et dans ce contexte on échouera à
trouver une valeur de vérité pour la formule.  Cette analogie entre
les présuppositions et les anaphores n'a rien de fortuit : elle est au
c\oe ur d'une analyse moderne (et dynamique) de la présupposition,
celle de \citet{vdS:92}.\Andex{van der Sandt, R.} 

%\largerpage[2]

Cette manière de formaliser le sens de \ref{x:rdf1} est assurément
satisfaisante sur le plan théorique, mais, comme nous l'avons dit,
elle transcende la couverture de cet ouvrage, car elle met en jeu des
principes de sémantique dynamique.  Dans les pages qui suivent,  ne
nous interdirons pas  de l'utiliser, mais il également intéressant
d'examiner une autre façon de formaliser le phénomène.  D'abord parce
qu'elle est assez couramment usitée dans les écrits de sémantique
formelle.  Ensuite parce qu'elle présente l'avantage de ne pas
escamoter le contenu «lexical» du {\GN} défini, et aussi parce qu'elle
peut astucieusement concilier les positions de Russell et de
Strawson.  Et enfin parce que cette formalisation est très facilement
implémentable avec les outils formels dont nous disposons.


Pour ce faire, nous ajoutons au vocabulaire de {\LO} un nouvel
opérateur, $\Xlo\atoi$\is{i@\atoi\ (iota)} (la lettre grecque \alien{iota},
$\iota$, mais inversée), que l'on appelle l'\kwi{opérateur de
description définie}{operateur@opérateur!\elid\ de description définie}.  Ce genre d'opérateur est un \kw{lieur}, car il lie une
variable. 
La règle de syntaxe qui l'introduit dans les expressions de {\LO}
ressemble à celle des quantificateurs, mais, attention, elle ne
produit pas des formules mais des \emph{termes}.

\begin{defi}[Syntaxe de $\atoi$]
\begin{enumerate}[resume*=RglSyn1] 
\item Si $\vrb\phi$ est une formule bien formée de {\LO} et si $\vrb v$ est une
  variable de {\VAR}, alors \(\Xlo\atoi v \phi\) est un terme.%
\label{SynPatoi} 
\setcounter{RglSynt}{\value{enumi}}
\end{enumerate}
\end{defi}

\newpage
 
Le terme \(\Xlo\atoi v \phi\) dénote l'unique individu du modèle qui
satisfait $\vrb\phi$.  Ainsi \sicut{le roi de France} se traduira en
$\Xlo\atoi x\, \prd{rdf}(x)$, et comme ceci est un terme, il peut apparaître en position d'argument d'un prédicat. De cette façon,  \ref{x:rdf1} se
traduira en :

\ex.
\(\Xlo\prd{chauve}(\atoi x\, \prd{rdf}(x))\)


La règle d'interprétation sémantique
des «~\atoi-termes» est la suivante :

\begin{defi}[Interprétation de $\atoi$]
\begin{enumerate}[sem,resume=RglSem2]
\item \(\denote{\Xlo\atoi v \phi}^{\Modele,g}=\Obj{d}\) ssi \Obj{d} est
  l'unique individu de \Unv{A} tel que \(\denote{\Xlo\phi}^{\Modele,g_{[\Obj{d}/\vrb v]}}=1\).
\(\denote{\Xlo\atoi v \phi}^{\Modele,g}\) n'est pas défini sinon.
\label{RIatoi}
\setcounter{RglSem}{\value{enumi}}
\end{enumerate}
\end{defi}

Nous pourrions expliciter cette règle en précisant rigoureusement  la
condition d'unicité qu'elle contient.  Cela donnerait :
\(\denote{\Xlo\atoi v \phi}^{\Modele,g}=\Obj{d}\) ssi pour tout individu
\Obj{d}$'$ de \Unv{A},  \(\denote{\Xlo\phi}^{\Modele,g_{[\Obj{d}'/\vrb v]}}=1\)
ssi \(\Obj{d}' = \Obj{d}\). 


%\largerpage

Et que se passe-t-il si dans \Unv{A} il n'existe pas d'individu qui
satisfait $\vrb\phi$ ou s'il en existe plusieurs ?  Eh bien (\RSem\ref{RIatoi}) nous dit que \(\denote{\Xlo\atoi v \phi}^{\Modele,g}\) n'est pas
défini, que son calcul échoue, et que par conséquent on ne pourra pas trouver de valeur
vérité pour les formules qui contiennent cette expression : il y aura
échec global de l'interprétation.  C'est bien
ce que Strawson annonçait.  En même temps la description définie
reprend l'idée d'existence et d'unicité de
Russell, mais cette existence et cette unicité ne sont pas exprimées
dans {\LO}, elles ont été exportées dans le métalangage et elles
apparaissent comme des prérequis pour la définition 
sémantique de la description en $\xlo\atoi$ ; autrement dit, elles sont présupposées\footnote{Petite précision
  épistémologique : j'ai annoncé que cette solution de l'opérateur
  $\Xlo\atoi$ conciliait les approches de Russell et Strawson ; mais
elle le fait de manière un peu inattendue. 
Je choisis ici  d'attribuer à $\Xlo\atoi$ une interprétation quelque peu différente de sa définition originale 
introduite par \citet{PM:1}.\Andexn{Russell, B.}\Andexn{Whitehead, A.}
Pour \citeauthor{PM:1}, lorsque les conditions d'existence et d'unicité ne sont pas satisfaites, la valeur du $\atoi$-terme reste néanmoins définie, en étant identifiée à un individu particulier, abstrait et unique du domaine (que l'on pourrait noter $\bot$ ou \Obj{nil}) et qui, par définition, n'appartient à la dénotation d'aucun prédicat. Ainsi, pour eux, \(\Xlo\prd{chauve}(\atoi x\,\prd{rdf}(x))\) sera fausse dans notre monde. 
Il est préférable pour nous de maintenir notre définition (\RegleSemantique\ref{RIatoi}) pour garantir, comme il se doit, le caractère présuppositionnel des {\GN} définis.}.  



\subsubsection{Familiarité}
%''''''''''''''''''''''''''

L'opérateur $\Xlo\atoi$ va beaucoup nous dépanner pour traduire les {\GN} définis (singuliers) du français. 
Mais gardons à l'esprit qu'il ne permet pas de formaliser 
les propriétés sémantiques de ces {\GN} aussi bien que l'analyse par
présupposition. % ; en ce sens il n'implémente qu'imparfaitement la
%proposition de Strawson (et {\atoi} est clairement plus russellien).
À cet égard, il est important de mentionner ici une approche alternative (mais complémentaire) de l'analyse sémantique des définis. Elle est issue, elle aussi, d'une longue tradition, plus linguistique que philosophique\footnote{Voir par exemple \citet{Heim:82,Heim:83a}\Andexn{Heim, I.} et les références qui y sont mentionnées.}, et qui, plutôt que de mettre en avant l'unicité du référent d'un {\GN} défini, repose sur la notion de \emph{familiarité}\is{familiarité}.  L'idée, très communément admise, est qu'un défini, correctement employé, sert généralement à faire référence à un individu que les interlocuteurs connaissent déjà ou avec lequel ils partagent déjà une certaine accointance, c'est-à-dire avec lequel ils sont, d'une manière ou d'une autre, déjà familiers.  Soit parce que l'individu en question a été mentionné plus tôt dans le discours ou la conversation, soit parce qu'il est particulièrement saillant dans le contexte.  Et cette vision est assez cohérente avec ce que nous avons vu ci-dessus, à savoir que les {\GN} définis fonctionnent comme des pronoms. La familiarité s'accorde donc très bien avec l'analyse présuppositionnelle (en particulier dans son traitement dynamique). 
Mais il y a moyen de sauver, en partie, la situation avec $\Xlo\atoi$.

Par exemple, si des étudiants, en discutant entre eux, déclarent \Next, 
nous comprenons bien qu'ils ne présupposent pas qu'il y a un et un seul professeur de sémantique dans le monde. 
Ils parlent d'un professeur précis, qu'ils connaissent, probablement celui dont ils suivent les cours.

\ex.
Le prof de sémantique est toujours mal habillé.


Cela rappelle le point que nous avons vu sur les restrictions des domaines de quantification (\S\ref{ss:RestrDQuant}). Et il se trouve qu'à sa façon, $\Xlo\atoi$ fait de la quantification (existentielle \emph{et} universelle, cf.\ \ref{x:rdf2ea}).  Nous aurons donc tout intérêt à tirer profit, là aussi, des pseudo-prédicats \vrb C qui délimitent contextuellement les champs d'application des quantifications.  Ainsi, \sicut{le prof de sémantique}  se traduira en : 

\ex.
\(\Xlo\atoi x [\prd{prof-sem}(x) \wedge C(x)]\)

Dans un modèle donné, \Last\ dénote l'unique individu qui est professeur de sémantique \emph{et} qui appartient à l'ensemble représenté par \vrb C.  Il suffit alors que cet ensemble ne contienne qu'un seul professeur de sémantique pour que \Last\ obtienne une valeur sémantique définie et appropriée. 

Cette stratégie pour gérer la familiarité nous suffira ici, mais on peut concevoir, à juste titre, qu'elle n'est pas entièrement satisfaisante, car finalement elle ne fait que reporter le problème sur \vrb C.  
En effet une analyse comme \Last\ ne sera correcte qui si \vrb C renvoie à un ensemble dont la composition est conforme à ce que nous attendons de l'interprétation du \GN.  Cela apparaît plus nettement encore avec un exemple comme {\Next}.


\ex.
Hier à une conférence, j'ai rencontré un allemand et une hollandaise.  La hollandaise est logicienne à l'ILLC d'Amsterdam.

\largerpage[-1]

Non seulement il n'est toujours pas question de présupposer qu'il y a une seule hollandaise dans le modèle, mais nous sommes quasiment certains que la hollandaise dénotée par le {\GN} défini est précisément celle qui est mentionnée par le {\GN} indéfini de la première phrase. 
Il ne suffit donc pas que \vrb C contienne une seule hollandaise, il faut aussi pouvoir s'assurer qu'il s'agit de celle dont on vient juste de parler. Et rien n'indique cela dans la traduction \(\Xlo\atoi x [\prd{hollandaise}(x) \wedge C(x)]\).

Notons enfin que la familiarité fait partie des critères qui  distinguent sémantiquement les définis des indéfinis. Si les définis servent à dénoter un individu déjà connu ou présent dans le contexte, les indéfinis, au contraire, servent à \emph{introduire} un individu nouveau dans l'univers du discours.

\subsubsection{Les définis dépendants}
%'''''''''''''''''''''''''''''''''''''
\label{sss:defdep}\is{defini@défini!\elid\ dependant@\elid\ dépendant}

L'opérateur $\Xlo\atoi$  nous permet bien d'analyser les {\GN} définis comme des expressions référentielles. La règle (\RSem\ref{RIatoi}) explique pourquoi un {\GN} défini ne covarie pas avec un quantificateur, puisqu'il s'agit de trouver un individu unique du domaine.  C'est ce qu'illustre la phrase \Next\ et sa traduction (j'omets à présent les restrictions \vrb C, afin de ne pas surcharger les formules).

\ex.
(Dans cette classe) tous les garçons sont amoureux de la maîtresse.\\
\(\Xlo\forall x [\prd{garçon}(x) \implq \prd{amoureux}(x,\atoi y\,\prd{maîtresse}(y))]\)

Le défini et le $\atoi$-terme correspondant s'interprètent dans la portée du quantificateur universel, mais comme ils dénotent l'unique maîtresse du domaine, ils ne covarieront pas.

Cependant, en réalité, on trouve des {\GN} définis qui covarient avec une expression quantifiée de la phrase (c'était l'objet de l'exercice~\ref{exo:DefCovar} \alien{supra}). C'est ce qui est illustré par exemple en \Next. 

\ex.
Chaque élève présentera \emph{le roman qu'il a lu}.

Ici il est bien question (potentiellement) d'autant de romans qu'il y a d'élèves, bien que le {\GN} soit singulier.  Mais le phénomène à l'\oe uvre  dans cette phrase est en fait très différent des covariations entre quantificateurs que nous avons vues jusqu'ici.  Dans \Last, ce qui importe, nous le voyons, c'est que le {\GN} défini comprend une proposition relative qui elle-même contient un pronom (\sicut{il}) qui est \emph{lié} à la variable quantifiée par \sicut{chaque élève}.  La covariation que nous observons ici est donc fondamentalement due à un phénomène de \kw{liage} de variables, et pas seulement l'interprétation d'un {\GN} dans la portée d'un autre.  Cela se révèle clairement dans la traduction de {\Last} en utilisant {\Xlo\atoi} :


\ex.
\(\Xlo\forall x [\prd{élève}(x)\implq \prd{présenter}(x,\atoi y [\prd{roman}(y)\wedge \prd{lire}(x,y)])]\)

Le défini est traduit par \(\Xlo\atoi y [\prd{roman}(y)\wedge \prd{lire}(x,y)]\), et pour une valeur donnée de \vrb x, il est prévu que ce \atoi-terme dénote un unique roman (celui lu par l'élève \vrb x).  Mais dès que la valeur de \vrb x varie (ce que fait $\Xlo\forall x$), nous obtenons plusieurs «uniques romans» (un par élève), ce qui explique bien la covariation observée en \LLast.

\largerpage[-1]

Un défini dépendant est donc un {\GN}   défini qui contient, dans sa traduction sémantique,  une variable localement libre, mais liable (et liée) par un quantificateur dans la phrase.
Notons que dans un $\atoi$-terme, de la forme $\Xlo\atoi x\phi$, nous pouvons dire que $\Xlo\atoi x$ a une portée : c'est \vrb\phi.
Pour autant il serait impropre de dire que cela constitue la portée du {\GN} correspondant ; les {\GN} définis n'ont pas vraiment de portée.  En revanche, on pourra dire que \vrb\phi\ correspond à la portée du déterminant défini \sicut{le} ou \sicut{la} --~car compositionnellement c'est bien le déterminant que traduit $\Xlo\atoi x$.

De par cette définition, nous constatons qu'il y a une catégorie de {\GN}, dont nous avons montré que ce sont des expressions référentielles, qui sont très souvent des définis dépendants, ce sont les possessifs, et en particulier ceux de 3\ieme\ personne en \sicut{son} et \sicut{sa}.\is{possessif}
Les déterminants possessifs «contiennent» sémantiquement un pronom (d'ailleurs dans beaucoup de langues, la possession s'exprime au moyen d'un pronom génitif\is{genitif@génitif}) : \sicut{son} signifie \sicut{de lui} ou \sicut{d'elle} ou \sicut{à lui}, \sicut{à elle}.  Ainsi un {\GN} comme \sicut{son smartphone} se traduira en \(\Xlo\atoi x[\prd{smartphone}(x)\wedge\prd{posséder}(y,x)]\) : l'objet qui est un smartphone possédé par \vrb y.  La variable \vrb y qui traduit la composante pronominale du possessif rend le {\GN} potentiellement dépendant, comme dans :

\ex.
Tout le monde regarde son smartphone.\\
\(\Xlo\forall y [\prd{humain}(y)\implq \prd{regarder}(y,\atoi x[\prd{smartphone}(x)\wedge\prd{posséder}(y,x)])]\)


Mais il faut remarquer que, malgré l'appellation de \emph{possessif}, la relation qui relie la dénotation globale du {\GN} à celle de la variable-pronom n'est pas nécessairement de la possession. Un exemple typique est celui de \sicut{son portrait} ou \sicut{sa photo}, qui sont particulièrement polysémiques. \sicut{Son portait} peut signifier \sicut{le portrait qu'\emph{il} possède} ou, plus couramment, \sicut{le portrait qui \emph{le} représente} ou \sicut{le portrait qu'\emph{il} a réalisé}. 
\sicut{Portrait} est un nom relationnel ;\is{nom!\elid\ relationnel} il se traduit par un prédicat binaire, sachant que \(\Xlo\prd{portrait}(x,y)\) signifie que \vrb x est un portrait qui représente \vrb y (un portrait est toujours le portrait de quelqu'un). Les trois traductions possibles du {\GN} sont donc :

\ex. son portrait
\a. \(\Xlo\atoi x[\exists z \prd{portrait}(x,z)\wedge \prd{posséder}(y,x)]\)
\b. \(\Xlo\atoi x\,\prd{portrait}(x,y)\)
\b. \(\Xlo\atoi x[\exists z \prd{portrait}(x,z)\wedge \prd{réaliser}(y,x)]\)

Notons ici l'ajout du quantificateur \(\Xlo\exists z\) pour lier \vrb z qui n'a (probablement) pas besoin de rester libre ici. Mais sachons aussi que cet ajout n'est pas du tout une opération triviale d'un point de vue compositionnel. 

De manière générale, avec les noms relationnels, le possessif viendra souvent (mais pas toujours, cf.\ \Last) occuper l'argument-complément du nom.  Ainsi \sicut{son père} se traduit en \(\Xlo\atoi x\, \prd{père}(x,y)\), \ie\ l'unique individu qui est père de {\vrb y}.
La relation \prd{posséder} semble plutôt s'employer par défaut, et bien sûr c'est celle que nous retrouvons habituellement avec les noms simples (non relationnels, \ie\ «unaires»). 
Mais là encore il faut être vigilant, car même dans le cas des noms simples, la relation n'est pas toujours de la possession à proprement parler.  Un étudiant assis dans une salle de cours peut dire \sicut{ma chaise est cassée}, or elle ne lui appartient pas. Tout le monde a le droit de dire \sicut{ma rue}, \sicut{mon quartier}, \sicut{ma ville}, \sicut{mon pays}, etc.\ sans passer pour un mégalomane. 
De même avec \sicut{mon boulanger}, \sicut{mon dentiste}, \sicut{mon équipe}, \sicut{mon champion}, \sicut{ma cible}, \sicut{mon film}, etc. 
Ce sont des exemples de possessifs somme toute assez banals, mais ce qui nous occupe ici c'est comment gérer cela dans notre système formel. Il y a plusieurs possibilités. 
Une première serait de dire que ces emplois expriment la possession mais dans des sens figurés, métaphoriques, que ces {\GN} mentent un peu en faisant passer pour de la possession ce qui n'en est pas exactement.  Une autre possibilité, plus indulgente, serait de dire que la relation en question n'est en fait pas la même que celle exprimée par \prd{posséder}, qu'il s'agit d'une relation plus large, que l'on pourrait représenter par un prédicat \prd{poss} ou \prd{gén} (comme \emph{génitif}\is{genitif@génitif}) et qui regrouperait les sens de \sicut{posséder}, \sicut{occuper}, \sicut{habiter}, \sicut{fréquenter}, etc.  
Enfin on peut considérer que le possessif introduit une relation qui est réellement sémantiquement sous-spécifiée (comme peuvent souvent l'être les contributions des prépositions \sicut{de} et \sicut{à}, ainsi que le verbe \sicut{avoir}), en laissant le soin au contexte de la préciser, par raisonnement pragmatique. 
Nous verrons au chapitre~\ref{ch:types} que {\LO} peut intégrer sans problème de tels prédicats sous-spécifiés (à l'instar de nos restrictions \vrb C).



\is{defini@défini!groupe nominal \elid|)}
\is{description définie|)}

\medskip

% -*- coding: utf-8 -*-
\begin{exo}\label{exo:tradiota}
Traduisez dans {\LO} les phrases suivantes, en utilisant l'opérateur
$\Xlo\atoi$.\pagesolution{crg:tradiota}
\begin{enumerate}
\item Le maire a rencontré le pharmacien.
\item Gontran a perdu son chapeau.
\item Le boulanger a prêté l'échelle au cordonnier.
\item La maîtresse a confisqué le lance-pierres de l'élève.
\item Gontran a attrapé le singe qui avait volé son chapeau.
\item Celui qui a gagné le gros-lot, c'est Fabrice.
\item Celui qui a gagné un cochon, c'est Fabrice.
\item Tout cow-boy aime son cheval.
\item Georges a vendu son portrait de Picasso.
\end{enumerate}
\begin{solu} (p.~\pageref{exo:tradiota})\label{crg:tradiota}

%Traductions avec $\Xlo\atoi$.
D'après la règle (\RSem\ref{RIatoi}) p.~\pageref{RIatoi}, \(\Xlo\atoi x\phi\) dénote l'unique individu \vrb x tel que \vrb\phi\ est vraie.  Donc si $N$ se traduit par \vrb\alpha, \sicut{le $N$} se traduira par $\Xlo\atoi x\,\alpha(x)$.
\begin{enumerate}
\item Le maire a rencontré le pharmacien.
\\$\leadsto$ \(\Xlo\prd{rencontrer}(\atoi x\,\prd{maire}(x),\atoi x\,\prd{pharmacien}(x))\) 

\item Gontran a perdu son chapeau.
%\\$\leadsto$ \(\Xlo\prd{perdre}(\cns g,\atoi x [\prd{chapeau}(x)\wedge \prd{poss}(y,x)]) \wedge y=\cns g\)
%\\ou directement
\\$\leadsto$ \(\Xlo\prd{perdre}(\cns g,\atoi x [\prd{chapeau}(x)\wedge \prd{poss}(\cns g,x)])\)

Le prédicat \prd{poss} exprime la possession : $\Xlo\prd{poss}(x,y)$ signifie que \vrb x possède \vrb y.

\item Le boulanger a prêté l'échelle au cordonnier.
\\$\leadsto$ \(\Xlo\prd{prêter}(\atoi x\,\prd{boulanger}(x),\atoi x\,\prd{échelle}(x),\atoi x \,\prd{cordonnier}(x))\)

\item La maîtresse a confisqué le lance-pierres de l'élève.
\\$\leadsto$ \(\Xlo\prd{confisquer}(\atoi x\,\prd{maîtresse}(x),\atoi x [\prd{lance-pierres}(x)\wedge \prd{poss}(\atoi y \,\prd{élève}(y),x)])\)

Remarque : là encore on peut traduire \sicut{le lance-pierres de
  l'élève} en utilisant toujours la variable $\vrb x$ : \(\Xlo\atoi x
[\prd{lance-pierres}(x)\wedge \prd{poss}(\atoi x
  \,\prd{élève}(x),x)]\), car on peut reconnaître que les différentes
occurrences de $\vrb x$ ne sont pas liées par le même $\Xlo\atoi x$.
Cependant, pour des raisons de lisibilité et de confort, il est
naturel d'utiliser ici deux variables distinctes. 

\item Gontran a attrapé le singe qui avait volé son chapeau.
%\\$\leadsto$ \(\Xlo\prd{attraper}(\cns g, \atoi x [\prd{singe}(x)\wedge
%  \prd{voler}(x,\atoi y [\prd{chapeau}(y)\wedge \prd{poss}(z,y)])])\wedge z=\cns g\)
%\\ou directement
\\$\leadsto$ \(\Xlo\prd{attraper}(\cns g, \atoi x [\prd{singe}(x)\wedge
  \prd{voler}(x,\atoi y [\prd{chapeau}(y)\wedge \prd{poss}(\cns g,y)])])\)

Remarque :  le contenu de la relative, \sicut{qui avait volé son
  chapeau}, se retrouve dans la portée de la première description
définie dès lors que l'on comprend cette relative comme une relative
\emph{restrictive} : on parle ici de l'unique l'individu qui est à la
fois un
singe \emph{et}  a volé le chapeau de Gontran. Dans ce cas, la
phrase pourra être vraie même s'il y a plusieurs singes dans le modèle
(et le contexte), du moment qu'un seul ait volé le chapeau.

Au contraire, si on comprend la relative comme une relative
\emph{descriptive} (on mettrait alors plus volontiers une virgule :
\sicut{Gontran a attrapé le singe, qui avait volé son chapeau}), la
phrase ne peut être vraie que s'il n'y a qu'un seul singe dans le
contexte. La traduction serait alors :
\\$\leadsto$ \(\Xlo\prd{attraper}(\cns g, \atoi x\,\prd{singe}(x)) \wedge
  \prd{voler}(\atoi x\,\prd{singe}(x),\atoi y
      [\prd{chapeau}(y)\wedge \prd{poss}(\cns g,y)])\) 

\item Celui qui a gagné le gros-lot, c'est Fabrice.
\\$\leadsto$ \(\Xlo\atoi x\, \prd{gagner}(x,\atoi y \,\prd{gros-lot}(y)) =
\cns f\)

\item Celui qui a gagné un cochon, c'est Fabrice.
\\$\leadsto$ \(\Xlo\atoi x \exists y [\prd{cochon}(y) \wedge \prd{gagner}(x,y)] =
\cns f\)

\item Tout cow-boy aime son cheval.
%\\$\leadsto$ \(\Xlo\forall x [\prd{cow-boy}(x)\implq [\prd{aimer}(x,\atoi
%    y [\prd{cheval}(y)\wedge\prd{poss}(z,y)]) \wedge z=x]]\)
%\\ ou directement:
\\$\leadsto$ \(\Xlo\forall x [\prd{cow-boy}(x)\implq \prd{aimer}(x,\atoi
    y [\prd{cheval}(y)\wedge\prd{poss}(x,y)])]\)

\item Georges a vendu son portrait de Picasso.
\\ Comme vu dans le chapitre, cette phrase est ambiguë\footnote{NB :  ces traductions incluent un argument \vrb z pour \prd{vendre} en considérant que $\Xlo\prd{vendre}(x,y,z)$ signifie \vrb x vend \vrb y à \vrb z.} :
\begin{enumerate}
\item $\leadsto$ \(\Xlo\exists z\, \prd{vendre}(\cns g,\atoi x[\prd{portrait}(x,\cns p)\wedge\prd{posséder}(\cns g,x)],z)\)\\
le portrait représente Picasso et appartient à Georges
\item $\leadsto$ \(\Xlo\exists z\, \prd{vendre}(\cns g,\atoi x[\prd{portrait}(x,\cns p)\wedge\prd{réaliser}(\cns g,x)],z)\)\\
le portrait représente Picasso et a été réalisé par Georges (bien sûr si Georges le vend, c'est aussi qu'il le possédait, mais dans ce cas, c'est une inférence supplémentaire qui n'est pas directement liée au déterminant possessif)
\item $\leadsto$ \(\Xlo\exists z\, \prd{vendre}(\cns g,\atoi x[\exists y\,\prd{portrait}(x,y)\wedge\prd{réaliser}(\cns p,x) \wedge \prd{posséder}(\cns g,x)],z)\)\\
le portrait a été réalisé par Picasso et appartient à Georges (et on ne sait pas qui il représente)
\item $\leadsto$ \(\Xlo\exists z\, \prd{vendre}(\cns g,\atoi x[\prd{portrait}(x,\cns g)\wedge\prd{réaliser}(\cns p,x)],z)\)\\
le portrait représente Georges et a été réalisé par Picasso (cette interprétation est moins naturelle, en français on dira plutôt \sicut{son portrait par Picasso})
\end{enumerate}

\end{enumerate}

\smallskip

Remarque : en toute rigueur, dans tous ces exemples, la traduction des {\GN} définis devrait toujours comporter une sous-formule supplémentaire qui restreint les valeurs pertinentes de la variable, comme vu en \S\ref{ss:RestrDQuant}.  Autrement dit, la traduction complète de la phrase 1 est en fait :

\begin{enumerate}
\item \(\Xlo\prd{rencontrer}(\atoi x[\prd{maire}(x) \wedge C_1(x)],\atoi x[\prd{pharmacien}(x)\wedge C_2(x)])\) 
\end{enumerate}
Cela permet de faire référence à l'unique individu qui est maire \emph{dans l'ensemble \vrbi C1} et l'unique individu qui est pharmacien \emph{dans l'ensemble \vrbi C2}.  Il est probable que la phrase s'interprète naturellement avec \vrbi C1 $=$ \vrbi C2 (il peut s'agir, par exemple, de l'ensemble des habitants d'un village ou d'un quartier -- où il n'y aura qu'un seul pharmacien), mais par souci de généralité, il est prudent de considérer que chaque {\GN} introduit son propre ensemble de restriction.
\end{solu}
\end{exo}






