% -*- coding: utf-8 -*-
\begin{exo}\label{exo:6Vdit}
Quelles sont les différences sémantiques
\pagesolution{crg:6Vdit}%
 (s'il y en a) entre les \lterme s suivant ?\addtolength{\multicolsep}{-10pt}
\begin{enumerate}
\item \(\Xlo\lambda Y\lambda Z\lambda x[Z(\lambda z[Y(\lambda y\,\prd{donner}(x,y,z))])]\) 
\item \(\Xlo\lambda Y\lambda Z\lambda x[Y(\lambda y[Z(\lambda z\,\prd{donner}(x,y,z))])]\)
\item \(\Xlo\lambda Z\lambda Y\lambda x[Z(\lambda z[Y(\lambda y\,\prd{donner}(x,y,z))])]\) 
\item \(\Xlo\lambda Z\lambda Y\lambda x[Y(\lambda y[Z(\lambda z\,\prd{donner}(x,y,z))])]\)
\end{enumerate}
\begin{solu}(p.~\pageref{exo:6Vdit})\label{crg:6Vdit}

\small\noindent
1. \(\Xlo\lambda Y\lambda Z\lambda x[Z(\lambda z[Y(\lambda y\,\prd{donner}(x,y,z))])]\)
\quad 
2. \(\Xlo\lambda Y\lambda Z\lambda x[Y(\lambda y[Z(\lambda z\,\prd{donner}(x,y,z))])]\)
\\
3. \(\Xlo\lambda Z\lambda Y\lambda x[Z(\lambda z[Y(\lambda y\,\prd{donner}(x,y,z))])]\) 
\quad
4. \(\Xlo\lambda Z\lambda Y\lambda x[Y(\lambda y[Z(\lambda z\,\prd{donner}(x,y,z))])]\)
\normalsize

\sloppy

Ces quatre \lterme s attendent deux quantificateurs généralisés \vrb Y et \vrb Z.  \vrb Y correspond au complément direct du verbe (car il se combine avec une expression qui fait abstraction de \vrb y, le second argument de \prd{donner}) et \vrb Z correspond au complément d'objet indirect (datif).  Ce qui distingue, d'une part, 1 de 3 et, d'autre part, 2 de 4, c'est l'ordre de leurs \labstraction s qui reflète l'ordre dans lequel le verbe rencontre syntaxiquement ses compléments.  Quant à ce qui distingue 1 de 2 (ainsi que 3 de 4) c'est qu'en 1 le quantificateur \vrb Y se trouve dans la portée de \vrb Z, et 2 présente les portées inverses.  Si nous traduisons \sicut{donne un exercice à tous les élèves}, avec 1 (ou 3) nous obtiendrons :
\(\Xlo\lambda x \forall z[\prd{élève}(z)\implq \exists y[\prd{exercice}(y)\wedge \prd{donner}(x,y,z)]]\), et avec 2 (ou 4) : 
\(\Xlo\lambda x \exists y[\prd{exercice}(y)\wedge \forall z[\prd{élève}(z)\implq \prd{donner}(x,y,z)]]\).  Le problème est que ces deux interprétations sont possibles, et donc que nous ne devons pas choisir entre 1 et 2 (ou 3 et 4 selon l'analyse syntaxique) ; en d'autres termes nous devrions postuler une ambiguïté de traduction pour les verbes ditransitifs -- mais ce n'est pas la manière la plus efficace de procéder comme le montre la suite du chapitre.

\fussy

\end{solu}
\end{exo}
