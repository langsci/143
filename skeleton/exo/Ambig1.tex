% -*- coding: utf-8 -*-
\begin{exo}[Ambiguïtés]\label{exo:1Ambig}
Montrez, en utilisant la définition \ref{d:ambig} ci-dessus, que les phrases suivantes
sont ambiguës.
\pagesolution{crg:1Ambig}

\begin{enumerate}
\item J'ai rempli ma bouteille d'eau.
\item Pierre s'est fait arrêter par un policier en pyjama. 
\item Daniel n'est pas venu à la fête parce que le président était là.
\item Alice ne mange que des yaourts au chocolat. 
\item Kevin dessine tous les gens moches.
\end{enumerate}
\begin{solu}(p.~\pageref{exo:1Ambig})\label{crg:1Ambig}

En accord avec la définition \ref{d:ambig} p.~\pageref{d:ambig}, pour chaque phrase, nous construisons un scénario par rapport auquel nous pouvons juger que la phrase est à la fois vraie et fausse (selon le sens retenu).
\begin{enumerate}
\item %J'ai rempli ma bouteille d'eau.\\
Scénario : \emph{j'ai une bouteille d'eau minérale en plastique, initialement vide ; je l'ai remplie de limonade.} %\\
La phrase est vraie, car c'est bien ma bouteille d'eau que j'ai remplie.  Mais elle est aussi fausse, car je ne l'ai pas remplie d'eau.
\item %Pierre s'est fait arrêter par un policier en pyjama. \\
Scénario : \emph{Pierre, très distrait, se promène dans la rue en pyjama ; un agent de police, en uniforme, l'arrête pour lui demander la raison de cette tenue étrange.} %\\
La phrase est vraie car Pierre, alors qu'il était en pyjama, s'est fait arrêter par un policier. Elle est aussi fausse car ce n'est pas un policier en pyjama qui l'a arrêté. 
\item %Daniel n'est pas venu à la fête parce que le président était là.
Scénario : \emph{Daniel est venu à la fête, le président aussi ; Daniel ne savait même pas que le président serait présent et il est venu parce qu'il aime les fêtes et n'en manque aucune.}
La phrase est vraie car ce n'est pas parce que le président était là que Daniel est venu.  Elle est fausse parce que Daniel est venu à la fête.  Les deux sens de la phrase peuvent se paraphraser en : i) la raison pour laquelle Daniel est venu n'est pas que le président était là ; ii) la raison pour laquelle Daniel n'est pas venu est que le président était là.
\item %Alice ne mange que des yaourts au chocolat. 
Scénario : \emph{Alice est une omnivore accomplie et a un régime alimentaire varié et équilibré, mais pour ce qui est des yaourts, elle n'en mange qu'au chocolat.}
La phrase est vraie car les seuls yaourts qu'Alice mange sont au chocolat, et elle est fausse car Alice ne se nourrit pas exclusivement de yaourts au chocolat.
\item %Kevin dessine tous les gens moches.
Scénario : \emph{Kevin aime bien dessiner, et en particulier il aime faire le portrait des gens qui sont plutôt beaux (ils ne dessine jamais les gens qu'il trouve moches) ; mais il a un mauvais coup de crayon et ses dessins enlaidissent toujours les sujets.}
La phrase est vraie car, les gens qu'il dessine, il les dessine moches. Et elle est fausse car il ne dessine pas les gens qui sont moches.  Les deux sens se paraphrasent ainsi : i) si Kevin dessine quelqu'un, il le dessine moche ; ii) si quelqu'un est moche, Kevin le dessine.
\end{enumerate}
\end{solu}
\end{exo}
