% -*- coding: utf-8 -*-
\begin{exo}\label{exo:Spec3}
Essayez d'imaginer un contexte dans lequel la phrase ci-dessous
\pagesolution{crg:Spec3}%
peut être prononcée avec un usage \emph{non spécifique} du {\GN}
\sicut{un acteur américain}.  

\begin{enumerate}
\item Hier j'ai rencontré un acteur américain.
\end{enumerate}
%
\begin{solu} (p.~\pageref{exo:Spec3}) \label{crg:Spec3}

Évidemment, la lecture la plus naturelle de la phrase est celle avec un usage spécifique du \GN\ puisque le locuteur raconte une expérience vécue personnellement (il a donc une idée précise de l'acteur en question).   Un contexte permettant cet usage serait par exemple une situation où le locuteur est invité à une soirée, il y rencontre Bryan Cranston, qui est un acteur américain, il le reconnaît pour l'avoir vu dans des films (sans forcément se souvenir de son nom) et il nous rapporte cet épisode avec la phrase (1). 

Pour que le \GN\ ait un usage non spécifique, il faudrait que le locuteur sache qu'il a rencontré un individu qui est un acteur américain tout en ignorant de quel individu il s'agit (c'est-à-dire de qui il parle précisément).  C'est là que réside la difficulté.  Supposons d'abord que le locuteur ait rencontré de nombreuses personnes au cours de la soirée, dont Bryan Cranston (que le locuteur ne connaît pas et n'a jamais vu en film ou à la télévision).  Dans ce contexte, le locuteur peut prononcer la phrase (1) sans avoir une idée précise de qui il parle (puisqu'il a rencontré plusieurs personnes ce qui laisse le choix dans l'identité de l'individu).  Mais comment sait-il alors qu'il s'agit d'un acteur américain ?  Une possibilité est qu'il l'apprenne par une tierce personne : un ami du locuteur présent à la soirée l'a vu discuter avec Bryan Cranston et l'informe le lendemain en lui disant «hier tu as parlé avec un acteur américain» (sans plus de précision). À partir de là, le locuteur peut alors prononcer la phrase (1) avec un usage non spécifique du \GN, du moment qu'il ne sait toujours pas lequel des invités était l'acteur.
\end{solu}
\end{exo}
