% -*- coding: utf-8 -*-
\begin{exo}\label{exo:6mvt}
Détaillez la composition sémantique
\pagesolution{crg:6mvt}%
 de \ref{x:montéesujet}  \sicut{Alice a appelé Bruno} et du DP \sicut{tous les écrivains que Charles connaît}. 
%%
%%
\begin{solu}(p.~\pageref{exo:6mvt})\label{crg:6mvt}
\begin{enumerate}
\item Nous reprenons l'analyse syntaxique de \ref{x:montéesujet}, p.~\pageref{x:montéesujet}, rappelée ici :\\ {}
[\Stag{TP} [\Stag{DP} \sicut{Alice}]$_1$ [\Stag{T$'$} \sicut{a} [\Stag{VP} $t_1$ [\Stag{V$'$} \sicut{appelé Bruno}]]]], et nous allons traduire tous les DP par des quantificateurs généralisés de type \ett.  
\begin{enumerate}
\item \(\text{[\Stag{V$'$} appelé Bruno]} \leadsto
\Xlo[\lambda Y\lambda x[Y(\lambda y\,\prd{appeler}(x,y))](\lambda P[P(\cns b)])]\)\\
\(=\Xlo\lambda x[\lambda P[P(\cns b)](\lambda y\,\prd{appeler}(x,y))]\)
\hfill{\small(\breduc\ sur \vrb Y)}\\
\(=\Xlo\lambda x[\lambda y\,\prd{appeler}(x,y)(\cns b)]\)
\hfill{\small(\breduc\ sur \vrb P)}\\
\(=\Xlo\lambda x\,\prd{appeler}(x,\cns b)\)
\hfill{\small(\breduc\ sur \vrb y)}

\item \(t_1 \leadsto \Xlo\lambda P[P(x_1)]\) 
\hfill{\small(cf. p.~\pageref{p.trace1})}

\item \(\text{[\Stag{VP} $t_1$ appelé Bruno]} \leadsto 
\Xlo[\lambda P[P(x_1)](\lambda x\,\prd{appeler}(x,\cns b))]\)\\
\(=\Xlo[\lambda x\,\prd{appeler}(x,\cns b)(x_1)]\)
\hfill{\small(\breduc\ sur \vrb P)}\\
\(=\Xlo\prd{appeler}(x_1,\cns b)\)
\hfill{\small(\breduc\ sur \vrb x)}

\item Pour l'instant nous considérons que l'auxiliaire n'a pas de contribution sémantique et donc T$'$ se traduit comme VP.

\item \(\text{[\Stag{TP} Alice$_1$ [\Stag{T$'$} a $t_1$ appelé Bruno]]} \leadsto\)\\
\(\Xlo [\lambda P[P(\cns a)](\lambda x_1\,\prd{appeler}(x_1,\cns b))]\) 
\hfill {\small (montée du sujet, cf. \ref{ri:MontSuj}, p.~\pageref{ri:MontSuj})}\\
\(=\Xlo [\lambda x_1\,\prd{appeler}(x_1,\cns b)(\cns a)]\) 
\hfill{\small(\breduc\ sur \vrb P)}\\
\(=\Xlo \prd{appeler}(\cns a,\cns b)\) 
\hfill{\small(\breduc\ sur \vrbi x1)}
\end{enumerate}

\item En nous inspirant de ce qui a été vu dans le chapitre, nous posons l'analyse syntaxique suivante : %\\{}
[\Stag{DP} \sicut{tous les} [\Stag{NP} [\Stag{NP} \sicut{écrivains}] [\Stag{CP} \sicut{que}$_2$ [\Stag{TP} \sicut{Charles}$_1$ [\Stag{VP} $t_1$ \sicut{connaît} $t_2$]]]]].\\
Les traces $t_1$ et $t_2$ sont traduites respectivement par $\Xlo\lambda P[P(x_1)]$ et $\Xlo\lambda P[P(x_2)]$.
\begin{enumerate}
\item \(\text{[\Stag{V$'$} connaît $t_2$]} \leadsto
\Xlo[\lambda Y\lambda x[Y(\lambda y\,\prd{connaître}(x,y))](\lambda P[P(x_2)])]\)\\
\(=\Xlo\lambda x[\lambda P[P(x_2)](\lambda y\,\prd{connaître}(x,y))]\)
\hfill{\small(\breduc\ sur \vrb Y)}\\
\(=\Xlo\lambda x[\lambda y\,\prd{connaître}(x,y)(x_2)]\)
\hfill{\small(\breduc\ sur \vrb P)}\\
\(=\Xlo\lambda x\,\prd{connaître}(x,x_2)\)
\hfill{\small(\breduc\ sur \vrb y)}

\item \(\text{[\Stag{VP} $t_1$ connaît $t_2$]} \leadsto
\Xlo[\lambda P[P(x_1)](\lambda x\,\prd{connaître}(x,x_2))]\)\\
\(=\Xlo[\lambda x\,\prd{connaître}(x,x_2)(x_1)]\)
\hfill{\small(\breduc\ sur \vrb P)}\\
\(=\Xlo\prd{connaître}(x_1,x_2)\)
\hfill{\small(\breduc\ sur \vrb x)}

\item \(\text{[\Stag{TP} Charles$_1$ $t_1$ connaît $t_2$]} \leadsto\)\\
\(\Xlo[\lambda P[P(\cns c)](\lambda x_1\,\prd{connaître}(x_1,x_2))]
\)
\hfill{\small(montée du sujet, p.~\pageref{ri:MontSuj})}\\
\(=\Xlo[\lambda x_1\,\prd{connaître}(x_1,x_2)(\cns c)]
\)
\hfill{\small(\breduc\ sur \vrb P)}\\
\(=\Xlo\prd{connaître}(\cns c,x_2)
\)
\hfill{\small(\breduc\ sur \vrbi x1)}

\item \(\text{que} \leadsto \Xlo \lambda P\lambda x[P(x)]\)
\hfill{\small(cf. p.~\pageref{p.prorel})}%
\footnote{Nous pourrions aussi choisir de traduire le pronom relatif par $\Xlo\lambda Q\lambda P[[P(x)]\wedge[Q(x)]]$ comme suggéré dans le chapitre. Dans ce cas, nous n'aurons pas à appliquer la règle de modification de prédicat.}

\item \(\text{[\Stag{CP} que$_2$ Charles$_1$ $t_1$ connaît $t_2$]} \leadsto\)\\
\(\Xlo[\lambda P\lambda x[P(x)](\lambda x_2\,\prd{connaître}(\cns c,x_2))]\)
\hfill{\small(montée du pronom, p.~\pageref{p.montprorel})}\\
\(=\Xlo\lambda x[\lambda x_2\,\prd{connaître}(\cns c,x_2)(x)]\)
\hfill{\small(\breduc\ sur \vrb P)}\\
\(=\Xlo\lambda x\,\prd{connaître}(\cns c,x)\)
\hfill{\small(\breduc\ sur \vrbi x2)}\\
CP est de type \et\ comme il se doit, et va se combiner avec [\Stag{NP} \sicut{écrivains}] également de type {\et} ($\Xlo\lambda x\,\prd{écrivain}(x)$) ; nous devons donc appliquer la règle de modification de prédicat vue en \S\ref{ss:ISSmodifieurs} (règle \ref{ri:PM}, p.~\pageref{ri:PM}):

\item \(\text{[\Stag{NP} [\Stag{NP} écrivains] que$_2$ Charles$_1$ $t_1$ connaît $t_2$]} \leadsto \)\\
\(\Xlo\lambda y[[\lambda x\,\prd{écrivain}(x)(y)]\wedge [\lambda x\,\prd{connaître}(\cns c,x)(y)]]\)
\hfill{\small(modification de prédicat)}\\
\(=\Xlo\lambda y[\prd{écrivain}(y)\wedge \prd{connaître}(\cns c,y)]\)
\hfill{\small(\breduc s sur \vrb x)}

\item \(\text{[\Stag{DP} tous les écrivains que$_2$ Charles$_1$ $t_1$ connaît $t_2$]} \leadsto \)\\
\(\Xlo [\lambda Q\lambda P\forall x[[Q(x)]\implq[P(x)]](\lambda y[\prd{écrivain}(y)\wedge \prd{connaître}(\cns c,y)])]\)\\
\(=\Xlo \lambda P\forall x[[\lambda y[\prd{écrivain}(y)\wedge \prd{connaître}(\cns c,y)](x)]\implq[P(x)]]\)
\hfill{\small(\breduc\ sur \vrb Q)}\\
\(=\Xlo \lambda P\forall x[[\prd{écrivain}(x)\wedge \prd{connaître}(\cns c,x)]\implq[P(x)]]\)
\hfill{\small(\breduc\ sur \vrb y)}
\end{enumerate}
\end{enumerate}
\end{solu}
\end{exo}
