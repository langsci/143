\begin{exo}\label{exo:2denot2}
Calculez  par rapport à $\Modele_1$ (cf. p.~\pageref{Modele1}) la
\pagesolution{crg:2denot2}
dénotation des formules suivantes :
\begin{exolist}
\item \(\Xlo\forall x [\prd{elfe}(x) \wedge \prd{farceur}(x)]\)\label{Qex:f1}
\item \(\Xlo\forall x [\prd{elfe}(x) \implq \neg\prd{triste}(x)]\)
\item \(\Xlo\neg\exists x [\prd{âne}(x) \wedge \prd{elfe}(x)]\)
\item \(\Xlo\exists x \forall y \, \prd{aimer}(y,x)\)
\item \(\Xlo\forall y \exists x \, \prd{aimer}(y,x)\)
\end{exolist}
Pour chaque formule, proposer une phrase en français qui peut se
traduire par cette formule.
\\
Comparer la formule \numero \ref{Qex:f1} avec la formule \ref{x:QA1}
\alien{supra}. 
%
\begin{solu} (p.~\pageref{exo:2denot2})\label{crg:2denot2}
\begin{exolist}
\item \(\Xlo\forall x [\prd{elfe}(x) \wedge \prd{farceur}(x)]\)

La règle (\RSem\ref{RIQ}b) nous dit que cette formule est vraie ssi pour \emph{toute} constante $\kappa$ du langage, nous trouvons \(\denote{\Xlo\prd{elfe}(\kappa) \wedge \prd{farceur}(\kappa)}^{\Modele_1}=1\).   Mais évidemment cette sous-formule est fausse pour de nombreuses constantes, par exemple \cnsi t1, puisque \Obj{Thésée} n'est pas un elfe.  La formule \(\Xlo\forall x [\prd{elfe}(x) \wedge \prd{farceur}(x)]\) est donc fausse dans $\Modele_1$.  En français, elle correspondra à \sicut{toute chose est un elfe farceur} ou \sicut{tout ce qui existe est un elfe farceur}. 
Elle se distingue donc crucialement de \ref{x:QA1}, \(\Xlo\forall x [\prd{elfe}(x) \implq \prd{farceur}(x)]\), qui elle correspond à \sicut{tous les elfes sont farceurs} et qui est vraie dans $\Modele_1$ (cf.\ p.~\pageref{x:QA1}).


\item \(\Xlo\forall x [\prd{elfe}(x) \implq \neg\prd{triste}(x)]\)

\sloppy

Comme précédemment, la formule sera vraie ssi pour toute constante $\kappa$, on trouve \(\denote{\Xlo\prd{elfe}(\kappa) \implq \neg\prd{triste}(\kappa)}^{\Modele_1}=1\).  Pour toutes les constantes qui dénotent des individus qui ne sont pas des elfes, nous savons déja que cela sera vrai (puisque $\Xlo\prd{elfe}(\kappa)$ sera faux).  Il reste à examiner les constantes \cns o, \cns p et \cnsi t2.  Comme aucun des individus dénotés par ces trois constantes (\Obj{Obéron}, \Obj{Puck} et \Obj{Tita\-nia}) ne sont dans $\FI_1(\prd{triste})$, nous obtiendrons, dans les trois cas, \(\denote{\Xlo\neg\prd{triste}(\kappa)}^{\Modele_1}=1\).  Cela suffit à montrer que \(\Xlo\Xlo\prd{elfe}(\kappa) \implq \neg\prd{triste}(\kappa)\) est toujours vraie et donc que la formule globale est vraie.
Elle correspond en français à \sicut{aucun elfe n'est triste}.

\fussy

\item \(\Xlo\neg\exists x [\prd{âne}(x) \wedge \prd{elfe}(x)]\)

Étant donné $\FI_1(\prd{âne})$ et $\FI_1(\prd{elfe})$, nous constatons qu'il n'y a pas d'individu du modèle qui est à la fois dans les deux ensembles.  Autrement dit, il n'existe pas de constante $\kappa$ telle que \(\Xlo[\prd{âne}(\kappa) \wedge \prd{elfe}(\kappa)]\) soit vraie.  Nous en concluons donc, par la règle (\RSem\ref{RIQ}a), que \(\Xlo\exists x [\prd{âne}(x) \wedge \prd{elfe}(x)]\) est fausse et que \(\Xlo\neg\exists x [\prd{âne}(x) \wedge \prd{elfe}(x)]\) est vraie dans $\Modele_1$.
En français, la formule correspond à \sicut{il est faux qu'il y a un âne qui est un elfe} (ou \sicut{un elfe qui est un âne}), ce qui plus simplement peut se formuler en \sicut{aucun âne n'est un elfe} ou \sicut{aucun elfe n'est un âne}.

\item \(\Xlo\exists x \forall y \, \prd{aimer}(y,x)\)

Cette formule est vraie ssi il existe une constante $\kappa_1$ telle que $\Xlo\forall y \, \prd{aimer}(y,\kappa_1)$ est vraie dans $\Modele_1$.  Et $\Xlo\forall y \, \prd{aimer}(y,\kappa_1)$ est vraie ssi pour toutes les constantes $\kappa_2$ nous trouvons $\Xlo\prd{aimer}(\kappa_2,\kappa_1)$.
Une telle constante $\kappa_1$ devrait donc dénoter un individu qui se retrouvait 11 fois en seconde position d'un couple $\tuple{\Obj x,\Obj y}$ dans la dénotation de \prd{aimer} (puisqu'il y a 11 constantes possibles pour $\kappa_2$).  Mais en regardant $\FI_1(\prd{aimer})$, nous voyons immédiatement qu'il n'y a pas autant de couples dans l'ensemble et donc qu'une telle constante $\kappa_1$ n'existe pas.  Par conséquent la formule est fausse dans $\Modele_1$.  En français elle correspond à \sicut{il y a quelqu'un que tout le monde aime}.

\item \(\Xlo\forall y \exists x \, \prd{aimer}(y,x)\) \sloppy

Cette formule est vraie ssi pour toute constante $\kappa_1$ \(\Xlo\exists x \, \prd{aimer}(\kappa_1,x)\) est vraie dans $\Modele_1$.  Cela fait donc, \emph{en théorie}, 11 calculs à effectuer, et chaque \(\Xlo\exists x \, \prd{aimer}(\kappa_1,x)\) sera vraie si à chaque fois on trouve une constante $\kappa_2$ telle que \(\Xlo\prd{aimer}(\kappa_1,\kappa_2)\) est vraie.  Nous pouvons rapidement montrer que cela ne se produit pas dans $\Modele_1$ en prenant, par exemple, d'abord \cns p pour $\kappa_1$. Dans $\FI_1(\prd{aimer})$ il n'y a pas de couple de la forme \tuple{\Obj{Puck},\dots}, donc nous ne trouverons pas de constante $\kappa_2$ telle que \(\Xlo\prd{aimer}(\cns p,\kappa_2)\) soit vraie.  Ce qui veut dire qu'il y a au moins une constante $\kappa_1$ telle que \(\Xlo\exists x \, \prd{aimer}(\kappa_1,x)\) est fausse, ce qui suffit à montrer que la formule globale est fausse dans $\Modele_1$.  En français elle correspond à \sicut{tout le monde aime quelqu'un} (dans le sens de \sicut{chaque personne a quelqu'un qu'elle aime}).
\end{exolist}

\fussy

\end{solu}
\end{exo}
