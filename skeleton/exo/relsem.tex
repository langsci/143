\begin{exo}[Relations sémantiques]\label{exo:1RelSem}
Pour chaque paire suivante, indiquez quelle est la \emph{relation de sens}
qui relie la phrase (a) à la phrase (b).

\begin{enumerate}
\item\begin{enumerate}
\item Jack est retourné en Pantagonie.
\item Jack a  déjà été en Patagonie.
\end{enumerate}

\item\begin{enumerate}
\item Charlie a failli réveiller Lucy. 
\item Charlie n'a pas  réveillé Lucy. 
\end{enumerate}

\item\begin{enumerate}
\item J'ai des amis qui aiment le chocolat.
\item J'ai des amis qui n'aiment pas le chocolat.
\end{enumerate}

\item\begin{enumerate}
\item J'ai des amis qui aiment le chocolat.
\item Je n'ai pas d'amis qui aiment le chocolat.
\end{enumerate}

\item\begin{enumerate}
\item Patty est déjà allée en Inde.
\item Il est faux que Patty n'a jamais été en Inde.
\end{enumerate}

\item\begin{enumerate}
\item François est allé à New York.
\item François n'est pas allé à New York à la nage.
\end{enumerate}
\end{enumerate}

\begin{solu}
(p.~\pageref{exo:1RelSem})

\begin{enumerate}
\item Présupposition.
\item Conséquence logique.
\item Implicature conversationnelle (scalaire).
\item Contradiction.
\item Équivalence logique.
\item Implicature conversationnelle (particularisée).
\end{enumerate}
\end{solu}
\end{exo}
