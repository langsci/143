% -*- coding: utf-8 -*-
\begin{exo}\label{exoInduCovar}
En vous inspirant des tests \ref{xGNsubitcv} (p.~\pageref{xGNsubitcv}) et \ref{xGNambigNeg} (p.~\pageref{xGNambigNeg}), 
\pagesolution{crg:InduCovar}
identifiez les
 \GN\ qui induisent de la covariation (vous établirez donc à cet effet un test approprié). 
\begin{solu}(p.~\pageref{exoInduCovar})\label{crg:InduCovar}

Une façon de définir un test de la covariation induite par un \GN\ consiste à placer ce \GN\ dans une position sujet d'une phrase avec un indéfini singulier en position de complément d'objet.  Nous observons ensuite si cet indéfini peut ou non être multiplié par le sujet. 

\ex.[]
\a.  \emph{Julie} a corrigé une page wikipédia.
\b.  \emph{Elle} a corrigé une page wikipédia.
\b.  \emph{L'étudiante} a corrigé une page wikipédia.
\b.  \emph{Cette étudiante} a corrigé une page wikipédia.
\b.  \emph{Mon étudiante} a corrigé une page wikipédia.
\b.  \jcovar \emph{Elles} ont corrigé une page wikipédia.
\b.  \jcovar \emph{Les étudiants} ont corrigé une page wikipédia.
\b.  \jcovar \emph{Ces étudiants} ont corrigé une page wikipédia.
\b.  \jcovar \emph{Mes étudiants} ont corrigé une page wikipédia.
\b.  \emph{Une étudiante} a corrigé une page wikipédia.
\b.  \jcovar \emph{Trois étudiants} ont corrigé une page wikipédia.
\b.  \jcovar \emph{Des étudiants} ont corrigé une page wikipédia.
\b.  \jcovar \emph{Plusieurs étudiants} ont corrigé une page wikipédia.
\b.  \jcovar \emph{Quelques étudiants} ont corrigé une page wikipédia.
\b.  \jcovar \emph{Aucun étudiant} n'a corrigé une page wikipédia.
\b.  \jcovar \emph{La plupart des étudiants} ont corrigé une page wikipédia.
\b.  \jcovar \emph{Tous les étudiants} ont corrigé une page wikipédia.
\b.  \jcovar \emph{Chaque étudiant} a corrigé une page wikipédia.
\b. etc.


Globalement, nous remarquons que les \GN\ pluriels induisent de la covariation (ainsi que \sicut{chaque $N$}).  Faisons tout de même deux remarques.  Pour les définis pluriels (g--i), la covariation de l'objet semble peut-être moins naturelle et moins courante, mais a priori nous ne pouvons pas complètement en exclure la possibilité\footnote{Bien sûr, la covariation devient nette si l'on ajoute le quantificateur dit flottant \sicut{tous} (comme dans \sicut{ces étudiants ont tous corrigé une page wikipédia}) mais cela ne montre rien puisque ça peut très probablement être \sicut{tous} qui est responsable du phénomène et pas le \GN\ en soi.}.  Le jugement porté sur \sicut{aucun étudiant} (o) est discutable ; nous pouvons certes y observer une multiplication par $0$, mais celle-ci peut être induite par la négation (\sicut{ne}) qui accompagne \sicut{aucun}.
\end{solu}
\end{exo}
